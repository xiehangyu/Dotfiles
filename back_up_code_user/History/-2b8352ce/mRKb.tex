\input{Format/praambel.tex}
%\numberwithin{equation}{section} %Gleichung werden nach Kapiteln nummeriert

\usepackage{upgreek}
\usepackage{siunitx}

\begin{document}
\input{Format/formatierung}

%◼◼◼◼◼◼◼◼◼◼◼◼◼◼◼◼◼◼◼◼◼◼◼◼
%\input{content.tex}
\section{Introduction}
\input{Section/intro}
\section{Question}
\input{Section/Question}
\section{Evaluation}
\input{Section/Evaluation}
%◼◼◼◼◼◼◼◼◼◼◼◼◼◼◼◼◼◼◼◼◼◼◼◼
%%% Ende des Fließtextes
\input{Format/formatierung_unten.tex}


% \newcommand{\initAnhang}{
%     \renewcommand{\thepage}{\Alph{chapter}\ \arabic{page}}
%     \newpage
% }
% \newcommand{\anhang}[1]{
%     \setcounter{page}{1}
%     \input{#1}
%     \newpage
% }

% \initAnhang
%  \anhang{Anhang.tex}


%\clearpage

%\printbibliography[
%heading=MyCollection,
%title={References}
%]
\label{Ende Literatur}


%\pagenumbering{roman}
%\setcounter{page}{1}
%\fancyfoot[C]{\thepage}

\label{Ende Anhang}

\end{document}
