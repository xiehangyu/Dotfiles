\subsection{Four-Point Measure:}

It can eliminate the contact resistance of measurement setup. It is
used to measure very small resistance, such as in superconductor.

\subsection{Radiation frequency}

The lowest order harmonic has the equation $fh=2eU$ or 
\begin{equation}
f=\frac{2eU}{h},
\end{equation}
 which will give the value $U$ as $U=2\times10^{-5}V$.

For the second order harmonic, $f=\frac{4eU}{h}$ and $U=1\times10^{-5}V$.

\subsection{2nd Josephson equation}

Denote the scalar potential by $\Phi$. From the Schrodinger equation,
we know that the time dependent of $\varphi$ is $\varphi(t)=-\frac{2e}{\hbar}\Phi t$.
Therefore, from Eq. (2.8), we arrive at 
\begin{align}
\dot{\gamma} & =-\frac{2e}{\hbar}(\Phi_{2}-\Phi_{1})+\frac{2e}{\hbar}\int_{1}^{2}\frac{\dot{\overrightarrow{A}}}{c}\cdot\overrightarrow{ds}\nonumber \\
 & =-\frac{2e}{\hbar}\int_{1}^{2}(\nabla\Phi-\frac{\dot{\overrightarrow{A}}}{c})\cdot\overrightarrow{ds}\nonumber \\
 & =-\frac{2e}{\hbar}\int_{1}^{2}\overrightarrow{E}\cdot\overrightarrow{ds}=\frac{2e}{\hbar}U
\end{align}
Therefore, 
\begin{equation}
\dot{\gamma}-\frac{2e}{\hbar}U=0
\end{equation}


\subsection{Josephson effect and pendulum}
From
\begin{equation}
I_{tot}=I_{c}\sin\gamma+\frac{U}{R_{n}}+C\dot{U}
\end{equation}
with $U=\frac{\hbar\dot{\gamma}}{2e}$, we have 
\begin{equation}
I_{tot}=I_{c}\sin\gamma+\frac{\hbar}{2eR_{n}}\dot{\gamma}+\frac{C\hbar}{2e}\ddot{\gamma}.
\end{equation}
We can also write it as 
\begin{equation}
\frac{I_{tot}}{I_{c}}=\sin\gamma+\frac{\Phi_{0}}{2\pi R_{n}I_{c}}\dot{\gamma}+\frac{C\Phi_{0}}{2\pi I_{c}}\ddot{\gamma}.
\end{equation}
We further introduce $t'=\sqrt{\frac{2\pi I_{c}}{C\Phi_{0}}}t$ to
make the equation dimensionless:
\begin{equation}
\frac{I_{tot}}{I_{c}}=\sin\gamma+\frac{d\gamma}{dt'}\sqrt{\frac{\Phi_{0}}{2\pi R_{n}^{2}CI_{c}}}+\frac{d\gamma}{dt''},
\end{equation}
or 
\begin{equation}
a_{d}=\sin\gamma+\frac{1}{\beta_{c}^{2}}\frac{d\gamma}{dt'}+\frac{d\gamma}{dt''}.
\end{equation}
In classical mechanics, the pendulum has the equation
\begin{equation}
ml^{2}\ddot{\theta}+mgl\sin\theta+f\dot{\theta}=L_{d},
\end{equation}
where $f$ is the damping moment caused by friction, $L_{d}$ is the
external driven moment. We can see that 
\begin{equation}
\begin{cases}
\gamma & =\theta;\\
I_{c} & =mgl;\\
\frac{\Phi_{0}}{2\pi R_{n}} & =f;\\
\frac{C\Phi_{0}}{2\pi} & =ml^{2};\\
I_{tot} & =L_{d}.
\end{cases}
\end{equation}


\subsection{Gauge invariant}

It is well known that the $\overrightarrow{E}$ is gauge invariant.
We only need to show that $\gamma$ is gauge invariant. Actually,
under gauge transformation, 
\begin{align}
\frac{2e}{\hbar c}\int_{1}^{2}\overrightarrow{A}\cdot\overrightarrow{s} & \to\frac{2e}{\hbar c}\int_{1}^{2}\overrightarrow{A}\cdot\overrightarrow{s}+\frac{2e}{\hbar c}\int_{1}^{2}\nabla\chi\overrightarrow{ds}\nonumber \\
 & =\frac{2e}{\hbar c}\int_{1}^{2}\overrightarrow{A}\cdot\overrightarrow{s}+\frac{2e}{\hbar c}(\chi_{2}-\chi_{1})
\end{align}
and 
\begin{equation}
\varphi_{2}-\varphi_{1}\to\varphi_{2}-\varphi_{1}-\frac{2e}{\hbar c}(\chi_{2}-\chi_{1})
\end{equation}
Therefore, $\gamma$ is gauge invariant. 
