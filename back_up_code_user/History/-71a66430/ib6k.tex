\documentclass[twocolumn,english,prl,aps,superscriptaddress,amsmath,amssymb,floatfix]{revtex4-2}
\usepackage[T1]{fontenc}
\usepackage{verbatim}
\setcounter{secnumdepth}{2}
\setcounter{tocdepth}{2}
\usepackage{amsmath}
\usepackage{amssymb}
\usepackage{graphicx}

\makeatletter
\usepackage{times}
\usepackage{textcomp}
\usepackage{epstopdf}
\usepackage{braket}
\usepackage{tikz}
\usepackage[colorlinks,linkcolor=blue,citecolor=blue]{hyperref}
\usepackage{tikz-network}
\usepackage{amsfonts}
\newtheorem{theorem}{Theorem}



\pdfpageheight\paperheight
\pdfpagewidth\paperwidth

\providecommand{\tabularnewline}{\\}


\@ifundefined{textcolor}{}{%
 \definecolor{BLACK}{gray}{0}
 \definecolor{WHITE}{gray}{1}
 \definecolor{RED}{rgb}{1,0,0}
 \definecolor{GREEN}{rgb}{0,1,0}
 \definecolor{BLUE}{rgb}{0,0,1}
 \definecolor{CYAN}{cmyk}{1,0,0,0}
 \definecolor{MAGENTA}{cmyk}{0,1,0,0}
 \definecolor{YELLOW}{cmyk}{0,0,1,0}
}

\usepackage{xcolor}\usepackage{soul}
\setcounter{MaxMatrixCols}{10}

\newcommand{\dg}{$^\circ$ }
\newcommand{\dgc}{$^\circ\mathrm{C}$}
\def \Tr {\mathrm{Tr}}
\definecolor{blue}{rgb}{0,0,1}
\definecolor{red}{rgb}{1,0,0}
\definecolor{green}{rgb}{0,1,0}
\newcommand{\red}[1]{\textcolor{red}{ #1}}
\newcommand{\blue}[1]{\textcolor{blue}{ #1}}
\newcommand{\green}[1]{\textcolor{green}{ #1}}


\usepackage{babel}
\begin{document}
\title{Free-fermion %Gaussian 
Page Curve: Canonical Typicality and Dynamical Emergence}
\author{Xie-Hang Yu}
\affiliation{Max-Planck-Institut f\"ur Quantenoptik, Hans-Kopfermann-Stra{\ss}e 1, D-85748 Garching, Germany}
\affiliation{Munich Center for Quantum Science and Technology, Schellingstra{\ss}e 4, 80799 M\"unchen, Germany}
\author{Zongping Gong}
\affiliation{Max-Planck-Institut f\"ur Quantenoptik, Hans-Kopfermann-Stra{\ss}e 1, D-85748 Garching, Germany}
\affiliation{Munich Center for Quantum Science and Technology, Schellingstra{\ss}e 4, 80799 M\"unchen, Germany}
\author{J. Ignacio Cirac}
\affiliation{Max-Planck-Institut f\"ur Quantenoptik, Hans-Kopfermann-Stra{\ss}e 1, D-85748 Garching, Germany}
\affiliation{Munich Center for Quantum Science and Technology, Schellingstra{\ss}e 4, 80799 M\"unchen, Germany}
\begin{abstract}
We provide further analytical insights into the newly established noninteracting (free-fermion) Page curve, focusing on both the kinematic and dynamical aspects. First, we unveil the underlying canonical typicality and atypicality for random free-fermion states%, which
. The former appears for a small subsystem and is exponentially weaker than the well-known result in the interacting case. The latter explains why the free-fermion Page curve differs remarkably from the interacting one when the subsystem is macroscopically large, i.e., comparable with the entire system. Second, we find that the free-fermion Page curve emerges with unexpectedly high accuracy in some simple tight binding %local free-fermion 
models in long-time quench dynamics. This contributes a rare analytical result concerning quantum thermalization on a macroscopic scale, where conventional paradigms such as the generalized Gibbs ensemble and quasi-particle picture are not applicable. 

\end{abstract}
\maketitle


\emph{Introduction.--}As
a central concept in quantum information science \citep{nielsen00},
entanglement has been recognized to play vital roles in describing
and understanding quantum many-body systems in and out of equilibrium
\citep{Luigi2008,relation_entropy_Phase,Eisert2015,Abanin2019}. 
For example, entanglement area laws for ground states
of gapped local Hamiltonians enable their efficient descriptions based
on tensor networks \citep{RevModPhys.93.045003}, 
while their violations may signature quantum phase transitions \citep{Vidal2003,Calabrese2004}. 
The emergence of thermal ensemble from unitary evolution, a process
known as quantum thermalization \cite{Srednicki1994}, is ultimately attributed to the entanglement
generated between a subsystem and the complement \citep{Nandkishore2015}. 

Almost thirty years ago, Page considered the fundamental
problem of bipartite entanglement in a fully random many-body system
and found a maximal entanglement entropy (EE) up to finite-size corrections \cite{Page1993}.
This seminal work was originally motivated by the black-hole information
problem \cite{Page1993blackhole}. 
Remarkably, it
has been attracted increasing and much broader interest in the past
decade, due not only to the new theoretical insights from quantum thermalization \citep{PhysRevLett.115.267206,PhysRevLett.119.220603,Nakagawa2018,PageCurve_Thermal,Lu2019,PageCurve_Thermal2,PhysRevLett.125.021601,Kaneko2020,PhysRevB.91.081110} and 
quantum chaos \citep{Sekino_2008, Chaos_Scrambl,Nahum2018},
but also to the practical relevance in light of the rapid experimental development in quantum simulations 
\citep{entropy_measure1,entropy_measure2,entropy_measure3,long_range_accessible,long_range_accessible2,Semeghini2021,
Yang2020, Liu2022}. 
In particular, the saturation of maximal entropy has been found to
be a consequence of canonical typicality \citep{Popescu2006,Goldstein2006,Reimann2007},
which means most random states behave locally like the canonical ensemble. %Also, 
This typicality 
behavior
has been argued to emerge in generic interacting many-body
systems satisfying the eigenstate thermalization hypothesis \citep{Srednicki1994,QuantumChaosETH,Abanin2019,
Rigol2008,Nandkishore2015,Moessner2017} 
and can even be rigorously established or ruled out in specific situations \cite{%PhysRevLett.124.200604,
ETH_CT_proof3,Hamazaki2018}. 


In this Letter, we provide analogous insights into
the noninteracting counterpart of Page's problem.  
That is, we focus on free fermions or (fermionic) Gaussian states, which are of their own interest in quantum many-body physics, quantum information and computation \citep{matchgates,PhysRevA.65.032325,Bravyi2005,Wolf2006,Banuls2007,Fidkowski2010,PhysRevLett.116.030401,Shi2018,PhysRevLett.120.190501,PhysRevLett.121.200501,Circuit_complexity_free_fermion,fermionicTomograph1,Oszmaniec2022,Matos2022,PhysRevLett.119.020601,PhysRevB.100.165135,PhysRevB.106.035143}. Somehow surprisingly, in the seemingly simpler noninteracting
case, the subsystem-size dependence of averaged EE,
which is described by the Page curve pictorially, was not solved until
very recently \citep{Bianchi2021,Bianchi2021a,PhysRevB.104.214306}. It turns out to be similar to the interacting case for a small subsystem, but differ significantly otherwise. See Fig.~\ref{Setup_of_CCRFG_ensemble}(b) for an illustration. With the measure concentration results on compact-group manifolds,  we establish the corresponding 
canonical typicality (atypicality) in microscopic (macroscopic) regions for the free-fermion ensemble. Thus we explicitly explain the similarity and difference from the kinematic aspect. In addition, we show that the free-fermion Page curve can be relevant to extremely simple tight-binding models via long-time quench dynamics. By classifying the systems according to their conserved (eigen) mode occupation numbers, we construct two classes of Hamiltonians which can/cannot give rise to a highly similar Page curve. Our finding concerning macroscopic properties which cannot be captured by the generalized Gibbs ensemble or quasi-particle picture and thus goes beyond the conventional paradigm of local thermalization. 



\emph{Canonical typicality and atypicality.--}
We start by generalizing %generalize
the main result in \citep{Popescu2006} to the random fermionic Gaussian (RFG) ensemble. While \citep{Popescu2006} already considers possible restrictions, we stress that Gaussianity is inadequate since Gaussian states do not constitute a Hilbert subspace. 
For simplicity, we 
consider number-conserving systems with totally $N$ modes occupied by $N/2$ fermions, i.e., the half-filling case. Compared to the fully random case without number conservation, this setting appears to be more physically comprehensible and experimentally relevant, while displaying exactly the same Page curve 
\citep{Bianchi2021a,Bianchi2021}. 
More general ensembles are discussed 
in Supplemental Material \cite{SM}. 

\begin{figure}
\includegraphics[width=1\columnwidth]{figs/new_set_up_fig}\caption{(a) The entire 
free-fermion system has $N$ sites with half filling. The 
subsystem of interest 
has $N_{A}$ ($N_{A}\le N$) sites. 
The RFG ensemble is generated by Haar-random Gaussian unitaries with number conservation. 
(b) 
The Page curves of the RFG 
and interacting ensemble in the thermodynamic limit $N\to\infty$. 
It is obvious that these two Page curves agree with each other in the microscopic region but show a $\mathcal{O}(1)$ deviation in the macroscopic region. The interacting Page curve in the thermodynamic limit is always saturated. (c) The table summarizes the typicality/atypicality property for the RFG and interacting ensembles. Here ``Poly." and ``Exp." indicates polynomial and exponential scalings, respectively.}
\label{Setup_of_CCRFG_ensemble}
\end{figure}

A pictorial illustration of our setup is shown 
in Fig.~\ref{Setup_of_CCRFG_ensemble}(a). 
Due to Wick's theorem \citep{Hackl2021}, a fermionic Gaussian 
state $\rho$ is 
fully captured 
by its covariance matrix $C_{j,j'}=\mathrm{Tr}(\rho a_{j}^{\dagger}a_{j'})$ \cite{Peschel2003}.
Here $a_{j}$ is the annihilation operator for mode $j$, which may label, e.g., a lattice site. 
As the covariance matrix for any RFG-pure state can be related to each
other by a unitary transformation, 
the uniform distribution over this ensemble can be generated at the
level of the covariance matrix $\{C=UC_{0}U^{\dagger}\}$. Here 
$U$ is taken Haar-randomly over the unitary group $\mathbb{U}(N)$ 
\citep{Bianchi2021,Bianchi2021a} and
$C_{0}$ is an arbitrary reference %is a specific 
covariance matrix in the ensemble satisfying $C_0^2=C_0$ and $\Tr C_0=N/2$. 

An important property of Gaussian states is that their subsystems remain Gaussian. 
We denote $C_{A}$ as the $N_A\times N_A$ covariance matrix restricted
to subsystem $A$ with $N_A$ modes. The EE $S_A=-\Tr(\rho_A\log_2\rho_A)$ of the reduced state $\rho_A=\Tr_{\bar A}\rho$ ($\bar A$: complement of $A$) then reads:
\begin{equation}
\begin{split}
    S_{A}=&-\Tr(C_A\log_2 C_A)\\
    &-\Tr((I_A-C_A)\log_2 (I_A-C_A)), 
    \end{split}
    \label{eq:SA}
\end{equation}
where $I_A$ is the identity matrix with dimension $N_A$.

Our first result is the measure concentration property of the covariance
matrix for RFG ensemble:

\begin{theorem}
For arbitrary $\epsilon>0$ and subsystem $A$, the probability that the reduced covariance matrix of a state in the RFG ensemble deviates from the ensemble average satisfies 
\begin{equation}
\mathbb{P}(d_{\rm HS}(C_{A},
I_{A}/2)\geq\eta+2\epsilon)\leq2e^{-\frac{\epsilon^{2}}{\eta'}}
\label{eq:Canonicality_result_in_CCRFG}
\end{equation}
and
\begin{equation}
\mathbb{P}(d_{\rm HS}^2(C_{A},
I_{A}/2)\leq\eta_{\rm a}-2\epsilon)\le 2e^{-\frac{\epsilon^2}{\eta'_{\rm a}}}    
\label{eq:atypicality}
\end{equation}
with $\eta=\sqrt{\frac{N_{A}^{2}}{2(N-1)}}$, 
$\eta'=\frac{12}{N}$, $\eta_{\rm a}=\frac{N_A^2}{4(N+1)}$, $\eta'_{\rm a}=\frac{12N_A}{N}$ and $d_{\rm HS}(C,C')=\sqrt{\Tr(C-C')^2}$ being the Hilbert-Schmidt distance. 
\end{theorem}
The proof largely relies on the generalized Levy's lemma
for Riemann manifolds with positive curvature \citep{measure_concentration1,measure_concentration2,Meckes2019},
which allows us to turn the upper bound on 
the distance average $\langle d_{\mathrm{HS}}(C_{A},I_{A}/2)\rangle\leq\sqrt{\frac{N_A^2}{2(N-1)}}$ or the lower bound $\langle d^2_{\mathrm{HS}}(C_{A},I_{A}/2)\rangle\geq\frac{N_A^2}{4(N+1)}$
into a probability inequality \citep{SM}. 
From Eq. (\ref{eq:Canonicality_result_in_CCRFG}) we can easily see for
infinite environments $N\to\infty$, the local microscopic system will have maximal entropy
$S_{A}\to N_{A}$.

We emphasize that in Eq. (\ref{eq:Canonicality_result_in_CCRFG}),
$\eta$ and $\eta'$ only scales polynomially with the (sub)system size. 
This contrasts starkly with the exponential scaling canonical typicality 
for random interacting 
ensemble \citep{Popescu2006}. 
Intuitively, this is because in the interacting 
case, the 
Hilbert-space dimension scales exponentially with the (sub)system size, 
which, however, simply equals to the size 
of the covariance matrix in the free-fermion case. 
Physically, the Gaussian constraint 
makes the ensemble
only explore a very limited sub-manifold in the entire 
Hilbert space.
This polynomial scaling means that, for a fixed subsystem size $N_A$, the reduced state still exhibits canonical typicality, while the atypicality is only polynomially suppressed by the environment size. Accordingly, the averaged EE should achieve the maximal value but with a polynomial finite-size correction. In fact, such an exponentially weaker canonical typicality (\ref{eq:Canonicality_result_in_CCRFG}) can also be exploited to explain the qualitatively larger variance of the EE for the RFG ensemble, which is $\mathcal{O}(N^{-2})$ 
in comparison to  
$e^{-\mathcal{O}(N)}$ in the interacting case 
\citep{Bianchi2021a,Bianchi2021,SM}.  

On the other hand, 
if the subsystem is macroscopically large, meaning that $f=N_A/N$ is $\mathcal{O}(1)$, the concentration inequality (\ref{eq:Canonicality_result_in_CCRFG}) becomes meaningless since $\eta$ is $\mathcal{O}(\sqrt{N})$, the same order as the Hilbert-Schmidt norm of $C_A$. Instead, we may take $\epsilon=\mathcal{O}(N^\alpha)$ with $\alpha\in(0,1)$ in Eq.~(\ref{eq:atypicality}), finding that the majority of the reduced covariance matrix differs significantly from the ensemble average. In other words, the RFG ensemble exhibits canonical atypicality in this case. In particular, this result implies an $\mathcal{O}(1)$ deviation in the EE density from the maximal value. We recall that, in stark contrast, the canonical typicality for interacting states is exponentially stronger and persists even on any macroscopic scale with $f<1/2$. 



The above discussions can be made more straightforward 
by considering the measure concentration property of 
$S_{A}$. By bounding $S_A$ using $d_{\rm HS}(C_A,I_A/2)$ from both sides, we obtain \cite{SM} 
\begin{equation}
\mathbb{P}(S_{A}\leq N_{A}-\epsilon)\le2e^{-\frac{(\sqrt{\epsilon}-\xi)^{2}}{\xi'}},\;\;\;\;\forall\epsilon>\xi^2
\label{eq:main_text_typicality_for_entropy}
\end{equation}
in the microscopic region and
\begin{equation}
\mathbb{P}(S_{A}\geq N_{A}-\xi_{a}+\epsilon) \leq2e^{-\frac{\epsilon^{2}}{\xi'_{a}}},\;\;\;\;
\forall\epsilon>0
\label{eq:main_text_atypicality_for_entropy}
\end{equation}
in the macroscopic region. Here $\xi=\sqrt{\frac{2N_{A}^{2}}{N-1}}$, $\xi'=\frac{192}{N}$, $\xi_{a}=\frac{N_{A}^{2}}{2\ln2(N+1)}$ and $\xi'_{a}=\frac{192N_{A}}{\ln^22N}$. 
Note that Eq.~(\ref{eq:main_text_typicality_for_entropy}) also becomes meaningless in the macroscopic region since $\xi^2$ will be comparable with $N_A$.
Choosing $\epsilon=(\xi+\mathcal{O}(N^{-\alpha/2}))^2$ for Eq.~(\ref{eq:main_text_typicality_for_entropy}) and $\epsilon=\mathcal{O}(N^\alpha)$ in Eq.~(\ref{eq:main_text_atypicality_for_entropy}) with $\alpha\in(0,1)$, we fully explain the microscopic similarity and macroscopic difference 
between 
the Page curves for the 
RFG and interacting ensembles. 
See Fig.~\ref{Setup_of_CCRFG_ensemble}(b) and (c).



\begin{figure*}
\includegraphics[width=0.9\textwidth]{figs/dynamical_page_curve_total.png}
\caption{(a) and (b) show the tight binding Hamiltonians $H_0+H_1$ with period 2. 
$H_1$ in (a) only includes the odd-range hopping, while (b) includes even-range hopping. (c) Dynamical Page curve for the minimal model (\ref{eq:NNH_Hamiltonian}) 
(blue) as a representative of (a) 
and its comparison with the Page curve for the RFG ensemble
(red) 
as well as our theoretic result up to order $\mathcal{O}(f^5)$ 
(green). 
Here $N=200$. 
These three lines are very close to each other, with a difference $\sim 10^{-3}$ which agrees with our analysis. 
This figure can also represent the general dynamical Page curve for Hamiltonians in (a). 
(d) Dynamical Page curve for Hamiltonian
$H=(\sum_{j=1}^{N}a_{j}^{\dagger}a_{j+1}+0.3\sum_{j:\mathrm{even}}a_{j}^{\dagger}a_{j+2}-0.3\sum_{j:\mathrm{odd}}a_{j}^{\dagger}a_{j+2})+{\rm H.c.}$ as a representative of (b). Here also $N=200$. The dynamical Page curve is obviously different
from the Page curve for the RFG ensemble. The considerable 
deviation between the
theoretical result and the dynamical Page curve near $f=\frac{1}{2}$, where higher-order terms become least negligible, is because we only calculate up to the third term in Eq. (\ref{eq:Taylor_expansion_for_entropy}) \cite{SM}. 
}
\label{dynamical_page_curve_total}
\end{figure*}


\emph{Dynamically emergent Page curve.--}
We recall that a particularly intriguing point of the (interacting) Page curve is its emergence in physical many-body systems with local interactions \cite{PhysRevLett.115.267206,PhysRevLett.119.220603,Nakagawa2018,PageCurve_Thermal,Lu2019,PageCurve_Thermal2}, which are typically chaotic but yet far from fully random. Indeed, a popular phenomenological theory for describing generic entanglement dynamics on the macroscopic level, the so-called entanglement membrane theory \cite{Nahum2018}, explicitly assumes that the entanglement profile of the thermalized system follows the Page curve. The intuition is that a long-time evolution can generate highly non-local correlations in a state and roughly exhaust the whole Hilbert (sub)space, provided the dynamics is ergodic. It is thus natural to ask whether the free-fermion Page curve could be relevant to thermalization in real physical systems without interactions. Note that this question is complementary to the aforementioned (a)typicality results, which are kinematic, i.e., irrelevant to dynamics, as in the interacting case \cite{Popescu2006}. 


We try to address the above question by 
analytically investigating the long-time averaged EE in the quench dynamics governed by 
some simple local quadratic Hamiltonians with number conservation. Hereafter, we use the term ``dynamical Page curve" to refer to this long-time averaged entanglement profile. 
Unlike \citep{PhysRevE.104.014146, Dias2021} which deal with models with strong spatiotemporal disorder so the emergence of the RFG Page curve is somehow expectable, we assume the Hamiltonian $H$ to be time-independent, translation-invariant (under the periodic boundary condition) and specify our initial state $|\Psi_0\rangle$ to be %is 
a period-2 density wave with half filling. Our simple setup thus appears to be far-from-random and highly experimentally accessible. See Fig.~\ref{dynamical_page_curve_total}(a-b) for a schematic illustration. The dynamical Page curve is %thus 
formally given by $\overline{S(\rho_A(t))}$, where $\rho_A=\Tr_{\bar A}[e^{-iHt}|\Psi_0\rangle\langle\Psi_0|e^{iHt}]$ and $\overline{f(t)}=\lim_{T\to\infty}T^{-1}\int^T_0 dt f(t)$ denotes the long-time average. It is worth mentioning that the dynamical Page curve is ensured to be concave by translation invariance, as a result of the strong subadditivity of quantum entropy \cite{Wolf2008}. 

We primarily focus on the minimal 
model, i.e., a one-dimensional lattice with nearest-neighbor hopping:
\begin{equation}
H_{0}=\sum_{j}a_{j}^{\dagger}a_{j+1}+{\rm H.c.},
\label{eq:NNH_Hamiltonian}
\end{equation}
whose 
band dispersion reads $E_k=2\cos k$. We believe that the exact results for the large (spatiotemporal) scale dynamical behaviors of this fundamental model are interesting on their own.  Moreover, our method and results actually apply to much broader situations, as will soon become clear below. 


Surprisingly, despite the additional translation-invariant and energy-conserving constraints compared to the RFG ensemble, this minimal model (\ref{eq:NNH_Hamiltonian}) turns out to give rise to 
a dynamical Page curve extremely close to that for the 
RFG ensemble 
(see blue and red curves in Fig.~\ref{dynamical_page_curve_total}(c)). 
To gain some analytic insights, we perturbatively expand 
the entropy expression (\ref{eq:SA}) 
around $C_{A}=\frac{I_{A}}{2}$, obtaining 
\begin{equation}
S_{A}(t)=N_{A}-\sum_{n=1}^{\infty}\frac{\mathrm{Tr}(2C_{A}(t)-I_{A}){}^{2n}}{2n(2n-1)\ln2}.
\label{eq:Taylor_expansion_for_entropy}
\end{equation}
Thanks to the translational invariance, $C_{A}(t)$ can be related to the block-diagonal momentum-space covariance matrix $\tilde C(t)=\bigoplus_k\tilde C_k(t)$ via 
$C_{A}(t)=\Pi_{A}U_{\rm F}\tilde{C}(t)U^{\dagger}_{\rm F}\Pi_{A}^{\dagger}$. Here
$U_{\rm F}$ and $\Pi_{A}$ are the Fourier transformation matrix and projector 
to subsystem $A$, respectively. 
The off-diagonal
elements of a $2\times2$ block $\tilde{C}_k(t)$ involve 
a time-dependent 
phase
$e^{i\theta_{k}(t)}$ with 
$\theta_{k}(t)=t(E_{k}-E_{k+\pi})$.
When calculating $\overline{\mathrm{Tr}(2C_{A}(t)-I_{A})^{2n}}$, 
we will encounter %meet the 
terms like $\overline{e^{i\theta_{k}(t)}e^{i\theta_{k'}(t)}}$,
which equals to $\delta_{k,k'+\pi}$ in the thermodynamic limit.
This contraction allows us to establish a set of Feynman rules for 
systematically calculating Eq.~(\ref{eq:Taylor_expansion_for_entropy})
order by order \cite{SM}. 


Since the bipartite EE is identical for either of the subsystems, the Page curve is reflection-symmetric with respect to $f=\frac{1}{2}$ and thus 
 it suffices to 
focus on $f=N_A/N\leq\frac{1}{2}$. In the thermodynamic limit, the dynamical Page curve turns out to be \cite{SM} 
%is:
\begin{equation}
\frac{\overline{S_{A}}}{N}=f-\frac{1}{\ln{2}}\left(\frac{1}{2}f^{2}+\frac{1}{6}f^{3}+\frac{1}{10}f^{4}\right)+\mathcal{O}(f^{5}).\label{eq:the_Page_curve_for_NNH}
\end{equation}
 On the other hand, the 
 Page curve for RFG ensemble
is \citep{Bianchi2021}
\begin{equation}
\frac{\langle S_{A}\rangle}{N}=f-\frac{1}{\ln{2}}\left(\frac{1}{2}f^{2}+\frac{1}{6}f^{3}+\frac{1}{12}f^{4}\right)+\mathcal{O}(f^{5}).\label{eq:the_Page_curve_result_for_CCRFG}
\end{equation}
The above two equations differ 
only by $\frac{1}{60\ln{2}}f^{4}+\mathcal{O}(f^5)$, 
which is as small as about $
10^{-3}$ even for $f$ near $1/2$.


Interestingly, if we add a perturbation $H_{1}$ to Eq.~(\ref{eq:NNH_Hamiltonian}),
as long as $H_{1}$ is period-2 
and only includes odd-range hopping, as represented by Fig.~\ref{dynamical_page_curve_total}(a),  
the dynamical Page curve can be analytically demonstrated 
to be the same as Eq. (\ref{eq:the_Page_curve_for_NNH}) in the thermodynamic 
limit, as the same Feynman rules apply \cite{SM}. 
One example 
is $H_{1}=J(\sum_{j:\mathrm{even}}a_{j}^{\dagger}a_{j+2m+1}-\sum_{j:\mathrm{odd}}a_{j}^{\dagger}a_{j+2m+1})+{\rm H.c.}$
for arbitrary $J$ and integer $m$. Thus, we have defined another ensemble
of fermionic Gaussian states by dynamical evolution, which covers a
wide class of Hamiltonians and this ensemble has remarkably similar
Page curve as the RFG ensemble. 

However, if $H_{1}$ includes even-range 
hopping, as represented by Fig.~\ref{dynamical_page_curve_total}(b) 
the dynamical Page curve will be very different, 
as shown in Fig.~\ref{dynamical_page_curve_total}(d).
This can be easily explained with the canonical typicality property
proved above: for this class of Hamiltonians, their conserved (eigen) mode
occupation number $n_{k}$ deviates from the average value of RFG ensemble,
which is 
$\frac{1}{2}$. 
Thus, the dynamical ensemble is naturally 
``atypical" even for microscopic scale because the local conserved observable is constructed from mode occupation numbers \citep{Ishii2019}. This result implies the reduced state on a small subsystem deviates considerably from being maximally mixed so that the tangent slope of the dynamical Page curve at $f=0$ is well below that for the RFG ensemble.
In contrast, one can show that all the conserved mode occupation number for the class of Hamiltonians mentioned in the last paragraph are $\frac{1}{2}$. 

All the observations above constitute our second main result: 
\begin{theorem}
The RFG ensemble-like dynamical Page curve (\ref{eq:the_Page_curve_for_NNH}) emerges for a period-2 short-range free-fermion Hamiltonian if and only if the conserved mode occupation numbers are $1/2$. 
\end{theorem}




\emph{Discussions.--}
It is well-known that the generalized Gibbs ensemble (GGE) 
characterizes 
the local thermalization of integrable systems including free fermions \citep{PhysRevLett.98.050405, Cassidy2011,doi:10.1126/science.1257026, Essler2016,Ishii2019}. 
However, in principle,
GGE only predicts the expectation values of observables, which do not include the entropy. Note that the former (latter) is linear (nonlinear) in $\rho_A$ and thus commmutes (does not commute) with time average. 
Moreover, 
we also study the macroscopic scale, which can not be captured by GGE as well as its recently proposed refined version \cite{Lucas2022} concerning the purified subsystem by measuring the complement 
\cite{Ho2022}. In this sense, our study goes well beyond the conventional paradigm of quantum thermalization in integrable systems, pointing out especially the highly nontrivial behaviors on the macroscopic level, where typicality may completely break down. 

Finally, let us mention the relation between our strategy
and the quasi-particle picture, which is widely 
used to calculate EE 
growth \citep{PhysRevLett.127.060404,Jurcevic2014,Castro2016_,Essler2016,Calabrese2005,Fagotti2008,Bertini2018,BertiniB2018_2}. 
It turns out this picture fails to 
reproduce the 
dynamical Page curve. Under the 
periodical boundary condition, the quasi-particle
picture predicts 
$\overline{S_{A}}=N-\frac{N_{A}^{2}}{N}$
for the Hamiltonian satisfying the conditions in Theorem 2 \cite{SM}. This result is obtained by counting the steady number of entangled pairs shared by $A$ and $\bar A$. 
On the other hand, noting that $(2C_{A}-I_{A})^{2n}\leq(2C_{A}-I_{A})^{2}$,
if we replace all the higher-order terms of $(2C_{A}-I_{A})$ in Eq. (\ref{eq:Taylor_expansion_for_entropy})
with $(2C_{A}-I_{A})^{2}$, we will get a lower entropy 
bound, which coincides with the prediction 
by the quasi-particle picture: $S_{A}\geq N_{A}-\frac{\mathrm{Tr}(2C_{A}-I_{A})^{2}}{\ln2}\sum_{n}\frac{1}{2n(2n-1)}=N_{A}-\frac{N_{A}^{2}}{N}$.
It is thus plausible 
to argue that the quasi-particle picture ignores
possible higher-order correlations beyond 
quasi-particle pairs. 


\emph{Conclusion and outlook.--}
We have derived the canonical
(a)typicality for the RFG ensemble 
and pointed out the quantitative 
scaling difference in atypicality suppression 
from interacting systems. This 
explains the very different behaviors of the Page curves. 
We have also explored the relevance to long-time quench dynamics of 
free-fermion systems. 
To our surprise, some simple time-independent
Hamiltonians are enough to 
make the free-fermion Page curve 
emerge to a very high accuracy. 
We analytically prove a necessary and sufficient condition about this behavior. The breakdown 
of the quasi-particle picture was also discussed.

Strictly speaking, we define a new 
ensemble arising from 
a wide class of free-fermion Hamiltonians, whose dynamical Page curve resembles a lot but yet differs from 
the fully random one. 
The properties of this new ensemble and its corresponding Page curve 
merit further study. Another interesting question is how the 
dynamical Page curves will be enriched upon imposing 
additional symmetries (such as the Altland-Zirnbauer symmetries \cite{Altland1997}), in which case one may naturally consider the symmetry-resolved EE 
\citep{PhysRevD.106.046015,Lau2022}. 
Our work proposes a methodology to study this question. Besides, whether
or not the fully random Page curve can emerge exactly for a time-independent free Hamiltonian 
also remains open.

We thank L. Piroli for valuable communications. Z.G. is supported by the Max-Planck-Harvard Research Center for Quantum Optics (MPHQ). J.I.C. acknowledges support by the EU Horizon 2020 program through the ERC Advanced Grant QENOCOBA No. 742102.

\emph{Note added.---}While finalizing this manuscript, a related
work by Isoue \emph{et al}. appeared in Ref.~\cite{Iosue2022}, which reported the %Page curve and 
typicality for random bosonic Gaussian states.

\bibliographystyle{utphys}

\bibliographystyle{apsrev4-2}
\addcontentsline{toc}{section}{\refname}
\bibliography{MyCollection}
\end{document}