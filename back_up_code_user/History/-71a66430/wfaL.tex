\documentclass[twocolumn,english,prl,aps,superscriptaddress,amsmath,amssymb,floatfix]{revtex4-2}
\usepackage[T1]{fontenc}
\usepackage{verbatim}
\setcounter{secnumdepth}{2}
\setcounter{tocdepth}{2}
\usepackage{amsmath}
\usepackage{amssymb}
\usepackage{graphicx}

\makeatletter
\usepackage{times}
\usepackage{textcomp}
\usepackage{epstopdf}
\usepackage{braket}
\usepackage{tikz}
\usepackage[colorlinks,linkcolor=blue,citecolor=blue]{hyperref}
\usepackage{tikz-network}
\usepackage{amsfonts}
\newtheorem{theorem}{Theorem}



\pdfpageheight\paperheight
\pdfpagewidth\paperwidth

\providecommand{\tabularnewline}{\\}


\@ifundefined{textcolor}{}{%
 \definecolor{BLACK}{gray}{0}
 \definecolor{WHITE}{gray}{1}
 \definecolor{RED}{rgb}{1,0,0}
 \definecolor{GREEN}{rgb}{0,1,0}
 \definecolor{BLUE}{rgb}{0,0,1}
 \definecolor{CYAN}{cmyk}{1,0,0,0}
 \definecolor{MAGENTA}{cmyk}{0,1,0,0}
 \definecolor{YELLOW}{cmyk}{0,0,1,0}
}

\usepackage{xcolor}\usepackage{soul}
\setcounter{MaxMatrixCols}{10}

\newcommand{\dg}{$^\circ$ }
\newcommand{\dgc}{$^\circ\mathrm{C}$}
\def \Tr {\mathrm{Tr}}
\definecolor{blue}{rgb}{0,0,1}
\definecolor{red}{rgb}{1,0,0}
\definecolor{green}{rgb}{0,1,0}
\newcommand{\red}[1]{\textcolor{red}{ #1}}
\newcommand{\blue}[1]{\textcolor{blue}{ #1}}
\newcommand{\green}[1]{\textcolor{green}{ #1}}


\usepackage{babel}
\begin{document}
\title{Free-fermion %Gaussian 
Page Curve: Canonical Typicality and Dynamical Emergence}
\author{Xie-Hang Yu}
\affiliation{Max-Planck-Institut f\"ur Quantenoptik, Hans-Kopfermann-Stra{\ss}e 1, D-85748 Garching, Germany}
\affiliation{Munich Center for Quantum Science and Technology, Schellingstra{\ss}e 4, 80799 M\"unchen, Germany}
\author{Zongping Gong}
\affiliation{Max-Planck-Institut f\"ur Quantenoptik, Hans-Kopfermann-Stra{\ss}e 1, D-85748 Garching, Germany}
\affiliation{Munich Center for Quantum Science and Technology, Schellingstra{\ss}e 4, 80799 M\"unchen, Germany}
\author{J. Ignacio Cirac}
\affiliation{Max-Planck-Institut f\"ur Quantenoptik, Hans-Kopfermann-Stra{\ss}e 1, D-85748 Garching, Germany}
\affiliation{Munich Center for Quantum Science and Technology, Schellingstra{\ss}e 4, 80799 M\"unchen, Germany}
\begin{abstract}
We provide further analytical insights into the newly established noninteracting (free-fermion) Page curve, focusing on both the kinematic and dynamical aspects. First, we unveil the underlying canonical typicality and atypicality for random free-fermion states%, which
. The former appears for a small subsystem and is exponentially weaker than the well-known result in the interacting case. The latter explains why the free-fermion Page curve differs remarkably from the interacting one when the subsystem is macroscopically large, i.e., comparable with the entire system. Second, we find that the free-fermion Page curve emerges with unexpectedly high accuracy in some simple tight binding %local free-fermion 
models in long-time quench dynamics. This contributes a rare analytical result concerning quantum thermalization on a macroscopic scale, where conventional paradigms such as the generalized Gibbs ensemble and quasi-particle picture are not applicable. 

\end{abstract}
\maketitle


\emph{Introduction.--}As
a central concept in quantum information science \citep{nielsen00},
entanglement has been recognized to play vital roles in describing
and understanding quantum many-body systems in and out of equilibrium
\citep{Luigi2008,relation_entropy_Phase,Eisert2015,Abanin2019}. 
For example, entanglement area laws for ground states
of gapped local Hamiltonians enable their efficient descriptions based
on tensor networks \citep{RevModPhys.93.045003}, 
while their violations may signature quantum phase transitions \citep{Vidal2003,Calabrese2004}. 
The emergence of thermal ensemble from unitary evolution, a process
known as quantum thermalization \cite{Srednicki1994}, is ultimately attributed to the entanglement
generated between a subsystem and the complement \citep{Nandkishore2015}. 

Almost thirty years ago, Page considered the fundamental
problem of bipartite entanglement in a fully random many-body system
and found a maximal entanglement entropy (EE) up to finite-size corrections \cite{Page1993}.
This seminal work was originally motivated by the black-hole information
problem \cite{Page1993blackhole}. 
Remarkably, it
has been attracted increasing and much broader interest in the past
decade, due not only to the new theoretical insights from quantum thermalization \citep{PhysRevLett.115.267206,PhysRevLett.119.220603,Nakagawa2018,PageCurve_Thermal,Lu2019,PageCurve_Thermal2,PhysRevLett.125.021601,Kaneko2020,PhysRevB.91.081110} and 
quantum chaos \citep{Sekino_2008, Chaos_Scrambl,Nahum2018},
but also to the practical relevance in light of the rapid experimental development in quantum simulations 
\citep{entropy_measure1,entropy_measure2,entropy_measure3,long_range_accessible,long_range_accessible2,Semeghini2021,
Yang2020, Liu2022}. 
In particular, the saturation of maximal entropy has been found to
be a consequence of canonical typicality \citep{Popescu2006,Goldstein2006,Reimann2007},
which means most random states behave locally like the canonical ensemble. %Also, 
This typicality 
behavior
has been argued to emerge in generic interacting many-body
systems satisfying the eigenstate thermalization hypothesis \citep{Srednicki1994,QuantumChaosETH,Abanin2019,
Rigol2008,Nandkishore2015,Moessner2017} 
and can even be rigorously established or ruled out in specific situations \cite{%PhysRevLett.124.200604,
ETH_CT_proof3,Hamazaki2018}. 


In this Letter, we provide analogous insights into
the noninteracting counterpart of Page's problem.  
That is, we focus on free fermions or (fermionic) Gaussian states, which are of their own interest in quantum many-body physics, quantum information and computation \citep{matchgates,PhysRevA.65.032325,Bravyi2005,Wolf2006,Banuls2007,Fidkowski2010,PhysRevLett.116.030401,Shi2018,PhysRevLett.120.190501,PhysRevLett.121.200501,Circuit_complexity_free_fermion,fermionicTomograph1,Oszmaniec2022,Matos2022,PhysRevLett.119.020601,PhysRevB.100.165135,PhysRevB.106.035143}. Somehow surprisingly, in the seemingly simpler noninteracting
case, the subsystem-size dependence of averaged EE,
which is described by the Page curve pictorially, was not solved until
very recently \citep{Bianchi2021,Bianchi2021a,PhysRevB.104.214306}. It turns out to be similar to the interacting case for a small subsystem, but differ significantly otherwise. See Fig.~\ref{Setup_of_CCRFG_ensemble}(b) for an illustration. With the measure concentration results on compact-group manifolds,  we establish the corresponding 
canonical typicality (atypicality) in microscopic (macroscopic) regions for the free-fermion ensemble. Thus we explicitly explain the similarity and difference from the kinematic aspect. In addition, we show that the free-fermion Page curve can be relevant to extremely simple tight-binding models via long-time quench dynamics. By classifying the systems according to their conserved (eigen) mode occupation numbers, we construct two classes of Hamiltonians which can/cannot give rise to a highly similar Page curve. Our finding concerning macroscopic properties which cannot be captured by the generalized Gibbs ensemble or quasi-particle picture and thus goes beyond the conventional paradigm of local thermalization. 



\emph{Canonical typicality and atypicality.--}
We start by generalizing %generalize
the main result in \citep{Popescu2006} to the random fermionic Gaussian (RFG) ensemble. While \citep{Popescu2006} already considers possible restrictions, we stress that Gaussianity is inadequate since Gaussian states do not constitute a Hilbert subspace. 
For simplicity, we 
consider number-conserving systems with totally $N$ modes occupied by $N/2$ fermions, i.e., the half-filling case. Compared to the fully random case without number conservation, this setting appears to be more physically comprehensible and experimentally relevant, while displaying exactly the same Page curve 
\citep{Bianchi2021a,Bianchi2021}. 
More general ensembles are discussed 
in Supplemental Material \cite{SM}. 

\begin{figure}
\includegraphics[width=1\columnwidth]{figs/new_set_up_fig}\caption{(a) The entire 
free-fermion system has $N$ sites with half filling. The 
subsystem of interest 
has $N_{A}$ ($N_{A}\le N$) sites. 
The RFG ensemble is generated by Haar-random Gaussian unitaries with number conservation. 
(b) 
The Page curves of the RFG 
and interacting ensemble in the thermodynamic limit $N\to\infty$. 
It is obvious that these two Page curves agree with each other in the microscopic region but show a $\mathcal{O}(1)$ deviation in the macroscopic region. The interacting Page curve in the thermodynamic limit is always saturated. (c) The table summarizes the typicality/atypicality property for the RFG and interacting ensembles. Here ``Poly." and ``Exp." indicates polynomial and exponential scalings, respectively.}
\label{Setup_of_CCRFG_ensemble}
\end{figure}

A pictorial illustration of our setup is shown 
in Fig.~\ref{Setup_of_CCRFG_ensemble}(a). 
Due to Wick's theorem \citep{Hackl2021}, a fermionic Gaussian 
state $\rho$ is 
fully captured 
by its covariance matrix $C_{j,j'}=\mathrm{Tr}(\rho a_{j}^{\dagger}a_{j'})$ \cite{Peschel2003}.
Here $a_{j}$ is the annihilation operator for mode $j$, which may label, e.g., a lattice site. 
As the covariance matrix for any RFG-pure state can be related to each
other by a unitary transformation, 
the uniform distribution over this ensemble can be generated at the
level of the covariance matrix $\{C=UC_{0}U^{\dagger}\}$. Here 
$U$ is taken Haar-randomly over the unitary group $\mathbb{U}(N)$ 
\citep{Bianchi2021,Bianchi2021a} and
$C_{0}$ is an arbitrary reference %is a specific 
covariance matrix in the ensemble satisfying $C_0^2=C_0$ and $\Tr C_0=N/2$. 

An important property of Gaussian states is that their subsystems remain Gaussian. 
We denote $C_{A}$ as the $N_A\times N_A$ covariance matrix restricted
to subsystem $A$ with $N_A$ modes. The EE $S_A=-\Tr(\rho_A\log_2\rho_A)$ of the reduced state $\rho_A=\Tr_{\bar A}\rho$ ($\bar A$: complement of $A$) then reads:
\begin{equation}
\begin{split}
    S_{A}=&-\Tr(C_A\log_2 C_A)\\
    &-\Tr((I_A-C_A)\log_2 (I_A-C_A)), %\mathrm{Tr}\begin{bmatrix}C_{A}\\ & I_{A}-C_{A} \end{bmatrix}\log_{2}\begin{bmatrix}C_{A}\\& I_{A}-C_{A}\end{bmatrix}
    \end{split}
    \label{eq:SA}
\end{equation}%, the distance between two covariance matrix is calculated with Hilbert-Schmidt distance $d_{\mathrm{H-S}}(C_{1},C_{2})=\sqrt{(C_{2}-C_{1})^{\dagger}(C_{2}-C_{1})}.
where $I_A$ is the identity matrix with dimension $N_A$.

Our first result is the measure concentration property of the covariance
matrix for RFG ensemble:

\begin{theorem}
%\textbf{Theorem 1}: 
For arbitrary $\epsilon>0$ and subsystem $A$, the probability that the reduced covariance matrix of a state in the RFG ensemble deviates from the ensemble average satisfies %for each arbitrary state in RFG ensemble with half filling, its covariance matrix restricted in subsystem $A$ is almost identity with high probability $\mathbb{P}$ as long as the environment system size is large enough
\begin{equation}
\mathbb{P}(d_{\rm HS}(C_{A},%\frac{1}{2}
I_{A}/2)\geq\eta+2\epsilon)\leq2e^{-\frac{\epsilon^{2}}{\eta'}}
\label{eq:Canonicality_result_in_CCRFG}
\end{equation}
and
\begin{equation}
\mathbb{P}(d_{\rm HS}^2(C_{A},%\frac{1}{2}
I_{A}/2)\leq\eta_{\rm a}-2\epsilon)\le 2e^{-\frac{\epsilon^2}{\eta'_{\rm a}}}    
\label{eq:atypicality}
\end{equation}
with $\eta=\sqrt{\frac{N_{A}^{2}}{2(N-1)}}$, %and 
$\eta'=\frac{12}{N}$, $\eta_{\rm a}=\frac{N_A^2}{4(N+1)}$, $\eta'_{\rm a}=\frac{12N_A}{N}$ and $d_{\rm HS}(C,C')=\sqrt{\Tr(C-C')^2}$ being the Hilbert-Schmidt distance. 
\end{theorem}
%One of the skills we used in the proof is
The proof largely relies on the generalized Levy's lemma
%in 
for Riemann manifolds with positive curvature \citep{measure_concentration1,measure_concentration2,Meckes2019},
which allows us to turn the upper bound on %of
the distance average $\langle d_{\mathrm{HS}}(C_{A},I_{A}/2)\rangle\leq\sqrt{\frac{N_A^2}{2(N-1)}}$ or the lower bound $\langle d^2_{\mathrm{HS}}(C_{A},I_{A}/2)\rangle\geq\frac{N_A^2}{4(N+1)}$
into a probability inequality \citep{SM}. %\textcolor{red}{See} %The details of the proof can be seen in 
%Supplemental Material \textcolor{red}{for detail}. 
%Although we are using the Hilbert-Schmidt distance here, one can easily obtain the result for other distance measures, since all measures of distance in finite space is equivalent. 
From Eq. (\ref{eq:Canonicality_result_in_CCRFG}) we can easily see for
infinite environments $N\to\infty$, the local microscopic system will have maximal entropy
$S_{A}\to N_{A}$.

%It is also worthy noting 
We emphasize that in Eq. (\ref{eq:Canonicality_result_in_CCRFG}),
$\eta$ and $\eta'$ only scales polynomially with the (sub)system size. %sites.
%It is a big contradictory to 
This contrasts starkly with the exponential scaling canonical typicality %result 
for random interacting %interaction 
ensemble \citep{Popescu2006}. %in which they scales exponentially.
Intuitively, this is because in the interacting %interaction 
case, the %dimension of the 
Hilbert-space dimension scales exponentially with the (sub)system size, %sites, while 
which, however, simply equals to the size %degrees of} freedom 
of the covariance matrix in the free-fermion case. %only scales polynomially with the system sites. 
Physically, the Gaussian constraint %constriction 
makes the ensemble
%can 
only explore a very limited sub-manifold in the entire %total 
Hilbert space.
%\textcolor{red}{(ZG: Try to explain in further detail: for a macroscopically large subsystem, $\eta/N_A$ can be exponentially small in the interacting case, but is $\mathcal{O}(1)$)}
This polynomial scaling means that, for a fixed subsystem size $N_A$, the reduced state still exhibits canonical typicality, while the atypicality is only polynomially suppressed by the environment size. Accordingly, the averaged EE should achieve the maximal value but with a polynomial finite-size correction. In fact, such an exponentially weaker canonical typicality (\ref{eq:Canonicality_result_in_CCRFG}) can also be exploited to explain the qualitatively larger variance of the EE for the RFG ensemble, which is $\mathcal{O}(N^{-2})$ %$\frac{1}{N^{2}}$ 
in comparison to  
%for RFG ensemble while 
$e^{-\mathcal{O}(N)}$ in the interacting case %for interacting %interaction one
\citep{Bianchi2021a,Bianchi2021,SM}.  

On the other hand, %Moreover, 
if the subsystem is macroscopically large, meaning that $f=N_A/N$ is $\mathcal{O}(1)$, the concentration inequality (\ref{eq:Canonicality_result_in_CCRFG}) becomes meaningless since $\eta$ is $\mathcal{O}(\sqrt{N})$, the same order as the Hilbert-Schmidt norm of $C_A$. Instead, we may take $\epsilon=\mathcal{O}(N^\alpha)$ with $\alpha\in(0,1)$ in Eq.~(\ref{eq:atypicality}), finding that the majority of the reduced covariance matrix differs significantly from the ensemble average. In other words, the RFG ensemble exhibits canonical atypicality in this case. In particular, this result implies an $\mathcal{O}(1)$ deviation in the EE density from the maximal value. We recall that, in stark contrast, the canonical typicality for interacting states is exponentially stronger and persists even on any macroscopic scale with $f<1/2$. %\textcolor{blue}{except when $N_{A}\geq\frac{N}{2}$.


%\textcolor{blue}{More specifically, for RFG ensemble with macroscopical subsystem,
%the atypicality is lower bounded by 
%\begin{equation*}
%\begin{split}
%\mathbb{P}\{d_{\mathrm{HS}}^{2}(C_{A},I_{A}/2) & \leq\frac{N_{A}^{2}}{4(N+1)}-2\sqrt{N_{A}}/N^{\frac{1}{3}}\}\\
% & \leq2e^{-\frac{N^{\frac{1}{3}}}{4}}
%\end{split}
%\end{equation*}
%This means that when $f\sim \mathcal{O}(1)$, the Gaussian state is anti-concentrated
%by an $\mathcal{O}(N)$ factor with probability very close to $1$.} 
%although the free fermions can still be kinematically thermalized, its thermalization rate is much slower than the \textcolor{red}{interacting} %interaction one. 

The above discussions can be made more straightforward %in \citep{SM}, where we 
by considering the measure concentration property of %the subsystem entropy 
$S_{A}$. By bounding $S_A$ using $d_{\rm HS}(C_A,I_A/2)$ from both sides, we obtain \cite{SM} %obtaining}
\begin{equation}
\mathbb{P}(S_{A}\leq N_{A}-\epsilon)\le2e^{-\frac{(\sqrt{\epsilon}-\xi)^{2}}{\xi'}},\;\;\;\;\forall\epsilon>\xi^2
%\begin{split} & \mathbb{P}(S_{A}\leq N_{A}-\epsilon)\\
 %& \leq\begin{cases}
%2e^{-\frac{(\sqrt{\epsilon}-\xi)^{2}}{\xi'}}, & \epsilon>\xi^{2};\\
%1, & \epsilon\leq\xi^{2}
%\end{cases}
%\end{split}
\label{eq:main_text_typicality_for_entropy}
\end{equation}
in the microscopic region and
\begin{equation}
\mathbb{P}(S_{A}\geq N_{A}-\xi_{a}+\epsilon) \leq2e^{-\frac{\epsilon^{2}}{\xi'_{a}}},\;\;\;\;
\forall\epsilon>0
\label{eq:main_text_atypicality_for_entropy}
\end{equation}
in the macroscopic region. Here $\xi=\sqrt{\frac{2N_{A}^{2}}{N-1}}$, $\xi'=\frac{192}{N}$, $\xi_{a}=\frac{N_{A}^{2}}{2\ln2(N+1)}$ and $\xi'_{a}=\frac{192N_{A}}{\ln^22N}$. 
%It is worthy emphasizing 
Note that Eq.~(\ref{eq:main_text_typicality_for_entropy}) also becomes meaningless in the macroscopic region since $\xi^2$ will be comparable with $N_A$.
%slower convergence rate can 
Choosing $\epsilon=(\xi+\mathcal{O}(N^{-\alpha/2}))^2$ for Eq.~(\ref{eq:main_text_typicality_for_entropy}) and $\epsilon=\mathcal{O}(N^\alpha)$ in Eq.~(\ref{eq:main_text_atypicality_for_entropy}) with $\alpha\in(0,1)$, we fully explain the microscopic similarity and macroscopic difference %, which occur on the microscopic and macroscopic scales, respectively, 
between % obvious deviation of 
the Page curves for the %Gaussian states 
RFG and interacting ensembles. %with 
%the one for interacting %interaction 
%states, 
See Fig.~\ref{Setup_of_CCRFG_ensemble}(b) and (c).
%This rate will also lead to very different scaling behavior of mutual information. In \citep{Shapourian2021}, the mutual information will almost vanish if $N_{A}<\frac{N}{2}$. However, for free fermions, the mutual information never vanishes even if $N_{A}<\frac{N}{2}$, see Supplemental Material.



\begin{figure*}
\includegraphics[width=0.9\textwidth]{figs/dynamical_page_curve_total.png}
\caption{(a) and (b) show the tight binding Hamiltonians $H_0+H_1$ with period 2. %in our dynamical model. 
$H_1$ in (a) only includes the odd-range hopping, while (b) includes even-range hopping. (c) Dynamical Page curve for the minimal model (\ref{eq:NNH_Hamiltonian}) %nearest neighbor hoping Hamiltonian
(blue) as a representative of (a) %line) 
and its comparison with the Page curve for the RFG ensemble
(red) %line) 
as well as our theoretic result up to order $\mathcal{O}(f^5)$ 
(green). %line). 
Here $N=200$. %We can see 
These three lines are very close to each other, with a difference $\sim%\mathcal{O}()
10^{-3}$ which agrees with our analysis. %A relatively larger deviation of our theoretical result from the other two curves at $f=\frac{1}{2}$ is due to the truncation. 
This figure can also represent the general dynamical Page curve for Hamiltonians in (a). %argument. 
(d) Dynamical Page curve for Hamiltonian
$H=(\sum_{j=1}^{N}a_{j}^{\dagger}a_{j+1}+0.3\sum_{j:\mathrm{even}}a_{j}^{\dagger}a_{j+2}-0.3\sum_{j:\mathrm{odd}}a_{j}^{\dagger}a_{j+2})+{\rm H.c.}$ as a representative of (b). Here also $N=200$. The dynamical Page curve is obviously different
from the Page curve for the RFG ensemble. The considerable %big
deviation between the
theoretical result and the dynamical Page curve near $f=\frac{1}{2}$, where higher-order terms become least negligible, is because we only calculate up to the third term in Eq. (\ref{eq:Taylor_expansion_for_entropy}) \cite{SM}. %\textcolor{blue}{This figure also illustrates the property of dynamical Page curve for general Hamiltonians in (b).} 
%\textcolor{red}{[ZG: make the subfigure labels ``(a)" and ``(b)" smaller and lower.]}
%, see Supplemental Material. 
%The range where $f=\frac{1}{2}$ will receive large corrections from higher order terms. 
}
\label{dynamical_page_curve_total}
\end{figure*}


\emph{Dynamically emergent Page curve.--}%On top of the above kinematic results, 
We recall that a particularly intriguing point of the (interacting) Page curve is its emergence in physical many-body systems with local interactions \cite{PhysRevLett.115.267206,PhysRevLett.119.220603,Nakagawa2018,PageCurve_Thermal,Lu2019,PageCurve_Thermal2}, which are typically chaotic but yet far from fully random. Indeed, a popular phenomenological theory for describing generic entanglement dynamics on the macroscopic level, the so-called entanglement membrane theory \cite{Nahum2018}, explicitly assumes that the entanglement profile of the thermalized system follows the Page curve. The intuition is that a long-time evolution can generate highly non-local correlations in a state and roughly exhaust the whole Hilbert (sub)space, provided the dynamics is ergodic. It is thus natural to ask whether the free-fermion Page curve could be relevant to thermalization in real physical systems without interactions. Note that this question is complementary to the aforementioned (a)typicality results, which are kinematic, i.e., irrelevant to dynamics, as in the interacting case \cite{Popescu2006}. %Here, 


We try to address the above question by %will 
analytically investigating the long-time averaged EE in the quench dynamics governed by %that the free-fermion Page curve \textcolor{red}{does emerge, although not perfectly,} in the long-time dynamics of
some simple local quadratic Hamiltonians with number conservation. Hereafter, we use the term ``dynamical Page curve" to refer to this long-time averaged entanglement profile. 
%can dynamically emerge the Page curve to a very high accuracy with long time average. 
%For simplicity, the periodic boundary condition is assumed.
%To simplify the analytic calculations,
Unlike \citep{PhysRevE.104.014146, Dias2021} which deal with models with strong spatiotemporal disorder so the emergence of the RFG Page curve is somehow expectable, we assume the Hamiltonian $H$ to be time-independent, translation-invariant (under the periodic boundary condition) and specify our initial state $|\Psi_0\rangle$ to be %is 
a period-2 density wave with half filling. Our simple setup thus appears to be far-from-random and highly experimentally accessible. See Fig.~\ref{dynamical_page_curve_total}(a-b) for a schematic illustration. The dynamical Page curve is %thus 
formally given by $\overline{S(\rho_A(t))}$, where $\rho_A=\Tr_{\bar A}[e^{-iHt}|\Psi_0\rangle\langle\Psi_0|e^{iHt}]$ and $\overline{f(t)}=\lim_{T\to\infty}T^{-1}\int^T_0 dt f(t)$ denotes the long-time average. It is worth mentioning that the dynamical Page curve is ensured to be concave by translation invariance, as a result of the strong subadditivity of quantum entropy \cite{Wolf2008}. %\textcolor{blue}{ Different than the stochastic analysis in \citep{PhysRevE.104.014146, Dias2021} where the randomness comes from highly spatiotemporal disorders, at first glance our system displays no randomness at all.}
%The Hamiltonian is a \textcolor{red}{noninteracting} %non-interaction
%spinless one which conserves the total charge number $\frac{N}{2}$, see Fig.~\ref{Setup_of_CCRFG_ensemble}(b). First, let us consider 

We primarily focus on the minimal %a rather simple 
model, i.e., a one-dimensional lattice with nearest-neighbor hopping:
\begin{equation}
H_{0}=\sum_{j}a_{j}^{\dagger}a_{j+1}+{\rm H.c.},
\label{eq:NNH_Hamiltonian}
\end{equation}
whose %Assuming the periodic boundary condition, one immediately obtains the 
band dispersion reads $E_k=2\cos k$. We believe that the exact results for the large (spatiotemporal) scale dynamical behaviors of this fundamental model are interesting on their own.  Moreover, our method and results actually apply to much broader situations, as will soon become clear below. %Later we will use $\overline{O}$ to denote the long time average of the quantity $O$.


Surprisingly, despite the additional translation-invariant and energy-conserving constraints compared to the RFG ensemble, this minimal model (\ref{eq:NNH_Hamiltonian}) turns out to give rise to %simple nearest neighbor hopping Hamiltonian can dynamically emerge 
a dynamical Page curve extremely close to that for the %the Page curve of 
RFG ensemble %in the long time average 
%to a very high accuracy, 
(see blue and red curves in Fig.~\ref{dynamical_page_curve_total}(c)). 
%We %will 
%use the term dynamical Page curve to refer to this long-time average behavior. 
%Our calculation strategy is by 
To gain some analytic insights, we perturbatively expand %ing
the entropy expression (\ref{eq:SA}) %, which can be calculated as $S_{A}=-\mathrm{Tr}\begin{pmatrix}C_{A}\\ & I_{A}-C_{A}
%\end{pmatrix}\log_{2}\begin{pmatrix}C_{A}\\ & I_{A}-C_{A}\end{pmatrix}$. 
%Expanding it 
around $C_{A}=\frac{I_{A}}{2}$, obtaining %we will get 
\begin{equation}
S_{A}(t)=N_{A}-%\frac{1}{\ln2}
\sum_{n=1}^{\infty}\frac{\mathrm{Tr}(2C_{A}(t)-I_{A}){}^{2n}}{2n(2n-1)\ln2}.
\label{eq:Taylor_expansion_for_entropy}
\end{equation}
Thanks to the translational invariance, $C_{A}(t)$ can be related to the block-diagonal momentum-space covariance matrix $\tilde C(t)=\bigoplus_k\tilde C_k(t)$ via %in Fourier space 
$C_{A}(t)=\Pi_{A}U_{\rm F}\tilde{C}(t)U^{\dagger}_{\rm F}\Pi_{A}^{\dagger}$. Here
$U_{\rm F}$ and $\Pi_{A}$ are the Fourier transformation matrix and projector %projection operator 
to subsystem $A$, respectively. %$\tilde{C}$ is the covariance matrix in momentum space with block-diagonal structure. 
The off-diagonal
elements of a $2\times2$ block $\tilde{C}_k(t)$ involve %will have 
a time-dependent %dynamical 
phase
$e^{i\theta_{k}(t)}$ with %which relates to eigenenergies 
$\theta_{k}(t)=t(E_{k}-E_{k+\pi})$.
When calculating $\overline{\mathrm{Tr}(2C_{A}(t)-I_{A})^{2n}}$, 
we will encounter %meet the 
terms like $\overline{e^{i\theta_{k}(t)}e^{i\theta_{k'}(t)}}$,
which equals to $\delta_{k,k'+\pi}$ in the thermodynamic limit.
This contraction allows us to establish a set of Feynman rules for %with which we can 
systematically calculating Eq.~(\ref{eq:Taylor_expansion_for_entropy})
order by order \cite{SM}. %Details can be found in Supplemental Material. 

%Denoting $f=\frac{N_{A}}{N}$. 
Since the bipartite EE is identical for either of the subsystems, the Page curve is reflection-symmetric with respect to $f=\frac{1}{2}$ and thus %Due to the symmetry of bipartition entropy
 it suffices to %we can only 
focus on $f=N_A/N\leq\frac{1}{2}$. In the thermodynamic limit, the dynamical Page curve turns out to be \cite{SM} 
%is:
\begin{equation}
\frac{\overline{S_{A}}}{N}=f-\frac{1}{\ln{2}}\left(\frac{1}{2}f^{2}+\frac{1}{6}f^{3}+\frac{1}{10}f^{4}\right)+\mathcal{O}(f^{5}).\label{eq:the_Page_curve_for_NNH}
\end{equation}
 On the other hand, the %volume-law coefficient in 
 Page curve for RFG ensemble
is \citep{Bianchi2021}
\begin{equation}
\frac{\langle S_{A}\rangle}{N}=f-\frac{1}{\ln{2}}\left(\frac{1}{2}f^{2}+\frac{1}{6}f^{3}+\frac{1}{12}f^{4}\right)+\mathcal{O}(f^{5}).\label{eq:the_Page_curve_result_for_CCRFG}
\end{equation}
%the difference between 
The above two equations differ %are 
only by $\frac{1}{60\ln{2}}f^{4}+\mathcal{O}(f^5)$, %\leq\frac{1}{480}$, 
which is as small as about $%\mathcal{O}()
10^{-3}$ even for $f$ near $1/2$.
%indicating the dynamical process of nearest neighbor hopping Hamiltonian can emerge the Page curve for RFG ensemble to a very high accuracy!


%Even 
Interestingly, if we add a perturbation $H_{1}$ to Eq.~(\ref{eq:NNH_Hamiltonian}),
as long as $H_{1}$ is period-2 %fulfills the symmetry of the initial state (namely it is symmetric when translating 2 sites) 
and %it 
only includes %the 
odd-range hopping, as represented by Fig.~\ref{dynamical_page_curve_total}(a),  %term with odd range, 
the dynamical Page curve can be analytically demonstrated %proved 
to be the same as Eq. (\ref{eq:the_Page_curve_for_NNH}) in the thermodynamic %thermal dynamical 
limit, as the same Feynman rules apply \cite{SM}. %see the discussion below Theorem 3 and the Supplemental Material. 
One example %of $H_{1}$ 
is $H_{1}=J(\sum_{j:\mathrm{even}}a_{j}^{\dagger}a_{j+2m+1}-\sum_{j:\mathrm{odd}}a_{j}^{\dagger}a_{j+2m+1})+{\rm H.c.}$
for arbitrary $J$ and integer $m$. Thus, we have defined another ensemble
of fermionic Gaussian states by dynamical evolution, which covers a
wide class of Hamiltonians and this ensemble has remarkably similar
Page curve as the RFG ensemble. 

However, if $H_{1}$ includes even-range %the 
hopping, as represented by Fig.~\ref{dynamical_page_curve_total}(b) %term with even range, 
the dynamical Page curve will be very different, %from the one for RFG ensemble, 
as shown in Fig.~\ref{dynamical_page_curve_total}(d).
This can be easily explained with the canonical typicality property
proved above: for this class of Hamiltonians, their conserved (eigen) mode
occupation number $n_{k}$ deviates from the average value of RFG ensemble,
which is %will be 
$\frac{1}{2}$. %This deviation is a strong evidence that this ensemble for the dynamical process corresponding to this class of Hamiltonians must be very different from RFG ensemble. 
Thus, %its Page curve 
the dynamical ensemble is naturally %reasonable to be 
``atypical" even for microscopic scale because the local conserved observable is constructed from mode occupation numbers \citep{Ishii2019}. This result implies the reduced state on a small subsystem deviates considerably from being maximally mixed so that the tangent slope of the dynamical Page curve at $f=0$ is well below that for the RFG ensemble.
%\textcolor{red}{Accordingly, the Feynman rules for perturbatively calculating the Page curve become more complicated.} 
In contrast, one can show that all the conserved mode occupation number for the class of Hamiltonians mentioned in the last paragraph are $\frac{1}{2}$. 

%Now we can introduce our 
All the observations above constitute our second main result: %third 
%theorem: 
\begin{theorem}
%\textbf{Theorem 2}: 
The RFG ensemble-like dynamical Page curve (\ref{eq:the_Page_curve_for_NNH}) emerges for a period-2 short-range free-fermion Hamiltonian if and only if the conserved mode occupation numbers are $1/2$. %In order for a Non-interaction charge-conserved fermionic system with periodic-2 symmetry in space to dynamically emerge the Page curve for RFG ensemble, a necessary condition is that all the conserved mode occupation number must be $\frac{1}{2}$. 
\end{theorem}

%The necessary part of the theorem can be argued similarly as above. For sufficient part, in Supplemental Material, we will show that those Hamiltonians satisfying the conditions above will have the same Feynman rule as for the nearest neighbor hopping Hamiltonian, thus the same dynamical Page curve.

%Our techniques in calculating Eq. (\ref{eq:Taylor_expansion_for_entropy}) can also be generalized to the class of Hamiltonians in Fig. (\ref{The_figure_of_dynamical_page_curve_for_range2}) with a different and much more complicated Feynman rules.


\emph{Discussions.--}%Sometimes 
It is well-known that the generalized Gibbs ensemble (GGE) %is also used to 
characterizes %argue
the local thermalization of integrable systems including free fermions \citep{PhysRevLett.98.050405, Cassidy2011,doi:10.1126/science.1257026, Essler2016,Ishii2019}. %non-interaction states. 
However, in principle,
GGE only predicts the expectation values of observables, which do not include the entropy. Note that the former (latter) is linear (nonlinear) in $\rho_A$ and thus commmutes (does not commute) with time average. 
Moreover, %In addition, 
we also study the macroscopic scale, which can not be captured by GGE as well as its recently proposed refined version \cite{Lucas2022} concerning the purified subsystem by measuring the complement %following the interacting counterpart
\cite{Ho2022}. In this sense, our study goes well beyond the conventional paradigm of quantum thermalization in integrable systems, pointing out especially the highly nontrivial behaviors on the macroscopic level, where typicality may completely break down. %compared to GGE.
%The canonical (a)typicality also upper bounds the thermalization rate for finite systems. This can be used to predict the $\mathcal{O}(\frac{1}{N})$ finite size effect in Page curve as well as the correct variance scaling \cite{SM}. %The details are shown in Supplemental Material. 

Finally, let us mention the relation between our %calculation 
strategy
and the quasi-particle picture, which is widely %can also be 
used to calculate %dynamical 
EE %entropy 
growth \citep{PhysRevLett.127.060404,Jurcevic2014,Castro2016_,Essler2016,Calabrese2005,Fagotti2008,Bertini2018,BertiniB2018_2}. %Alba2018,}.
%However, 
It turns out this picture fails to %can not 
reproduce the %give the exact result for 
dynamical Page curve. Under the %When considering 
periodical boundary condition, the quasi-particle
picture predicts %the subsystem entropy is 
$\overline{S_{A}}=N-\frac{N_{A}^{2}}{N}$
for the Hamiltonian satisfying the conditions in Theorem 2 \cite{SM}. This result is obtained by counting the steady number of entangled pairs shared by $A$ and $\bar A$. %and the discussions there. 
On the other hand, noting that $(2C_{A}-I_{A})^{2n}\leq(2C_{A}-I_{A})^{2}$,
if we replace all the higher-order terms of $(2C_{A}-I_{A})$ in Eq. (\ref{eq:Taylor_expansion_for_entropy})
with $(2C_{A}-I_{A})^{2}$, we will get a lower entropy 
bound, %of the entropy,
which coincides with the prediction %is just the result predicted 
by the quasi-particle picture: $S_{A}\geq N_{A}-\frac{\mathrm{Tr}(2C_{A}-I_{A})^{2}}{\ln2}\sum_{n}\frac{1}{2n(2n-1)}=N_{A}-\frac{N_{A}^{2}}{N}$.
It is thus plausible %reasonable 
to argue that the quasi-particle picture ignores
possible higher-order correlations beyond %among multiple 
quasi-particle pairs. %which are generated simultaneously. 
%Therefore, the quasi-particle picture can only give a lower bound of the dynamical Page curve.


\emph{Conclusion and outlook.--}%In this letter, 
We have derived the canonical
(a)typicality for the RFG ensemble %This canonical typicality makes the concept of ergodicity and thermalization clearer in free-fermions system.
%What's more, we also 
and pointed out the quantitative %\textcolor{red}{the qualitative} %that there is obvious 
scaling difference in atypicality suppression %of the thermalization rate 
from %the interaction 
interacting systems. This %slower thermalization rate which is due to the property of Gaussian ensemble can 
explains the very different behaviors of the Page curves. %, even for some local quantities such as variance, as discovered in \citep{Bianchi2021a}.
We have also explored the relevance to long-time quench dynamics of %also focus on the dynamical Page curve of 
free-fermion systems. %, which is a global property and can not be fully characterized by the canonical typicality. 
To our surprise, some simple time-independent
Hamiltonians are enough to 
%\textcolor{red}{already accommodate a dynamical Page curve that highly resemble the one for the RFG ensemble.} 
make %dynamically emerge 
the free-fermion Page curve %of RFG ensemble 
emerge to a very high accuracy. 
We %study this problem further and 
analytically prove a necessary and sufficient condition about this behavior. The breakdown %correctness 
of the quasi-particle picture was also discussed.

Strictly speaking, we define a new %\textcolor{red}{RFG} 
ensemble arising from %covering 
a wide class of free-fermion Hamiltonians, whose dynamical Page curve resembles a lot but yet differs from %to yet different from 
the fully random one. %for RFG ensemble. 
%However, there is a tiny difference from any page curve we know so far, such as the fully random Page curve and the Page curve of averaging all eigenstates \citep{Bianchi2021,Bianchi2021a,Vidmar2017}.
The properties of this new ensemble and its corresponding Page curve %It is a new kind of Page curve whose connection to others 
merit further study. Another interesting question is how the %there will be more kinds of 
dynamical Page curves will be enriched upon imposing %regarding 
additional symmetries (such as the Altland-Zirnbauer symmetries \cite{Altland1997}), in which case one may naturally consider %known as 
the symmetry-resolved EE %Page curve 
\citep{PhysRevD.106.046015,Lau2022}. %to the Hamiltonians.
Our work proposes a methodology to study this question. Besides, whether
or not the fully random Page curve can emerge exactly for a time-independent free Hamiltonian %can exactly dynamically emerge  
also remains open.

We thank L. Piroli for valuable communications. Z.G. is supported by the Max-Planck-Harvard Research Center for Quantum Optics (MPHQ). J.I.C. acknowledges support by the EU Horizon 2020 program through the ERC Advanced Grant QENOCOBA No. 742102.

\emph{Note added.---}While finalizing this manuscript, a related
work by Isoue \emph{et al}. appeared in Ref.~\cite{Iosue2022}, which reported the %Page curve and 
typicality for random bosonic Gaussian states.

\bibliographystyle{utphys}

\bibliographystyle{apsrev4-2}
\addcontentsline{toc}{section}{\refname}%\nocite{*}
\bibliography{MyCollection}
\input{SupplementalMaterial.tex}
\end{document}