\documentclass[english]{ctexart}
\usepackage[T1]{fontenc}
\usepackage{geometry}
\geometry{verbose}
\setcounter{secnumdepth}{2}
\setcounter{tocdepth}{2}
\usepackage{graphicx}
\usepackage[authoryear]{natbib}

\makeatletter
%%%%%%%%%%%%%%%%%%%%%%%%%%%%%% User specified LaTeX commands.
\usepackage{braket}
\usepackage{tikz}
%\usepackage{braket}
%\usepackage{braket}
\usepackage{listings}
\usepackage{xcolor}
\usepackage{color}
\usepackage{diagbox}
\usepackage{chngcntr}
\lstset{
numbers=left,
framexleftmargin=10mm,
frame=none,
keywordstyle=\bf\color{blue},
identifierstyle=\bf,
numberstyle=\color[RGB]{0,192,192},
commentstyle=\it\color[RGB]{0,96,96},
stringstyle=\rmfamily\slshape\color[RGB]{128,0,0}
}

%\usetheme{Darmstadt}
%\usetheme{Frankfurt}
% or ...

%\setbeamercovered{transparent}
\lstdefinelanguage
   [x64]{Assembler}     % add a "x64" dialect of Assembler
   [x86masm]{Assembler} % based on the "x86masm" dialect
   % with these extra keywords:
   {morekeywords={CDQE,CQO,CMPSQ,CMPXCHG16B,JRCXZ,LODSQ,MOVSXD, %
                  POPFQ,PUSHFQ,SCASQ,STOSQ,IRETQ,RDTSCP,SWAPGS, %
                  rax,rdx,rcx,rbx,rsi,rdi,rsp,rbp, %
                  r8,r8d,r8w,r8b,r9,r9d,r9w,r9b, %
                  r10,r10d,r10w,r10b,r11,r11d,r11w,r11b, %
                  r12,r12d,r12w,r12b,r13,r13d,r13w,r13b, %
                  r14,r14d,r14w,r14b,r15,r15d,r15w,r15b,retq,callq,cmpl}} % etc.

\lstset{language=[x64]Assembler}
\counterwithin*{section}{part}

\makeatother

\usepackage{babel}
\begin{document}
\title{对电流随光强增加而减小的文献总结}
\author{戴梦佳\ \ PB20511879}

\maketitle
总体来说,光电流的产生包括两个步骤:
\begin{enumerate}
\item 光子将电子从价带打到导带,形成电子空穴对。
\item 电子和空穴被输运到电极产生光电流,这其中包括电子与空穴的重新组合(recombination)。一旦电子与空穴重新组合,光电流就将减小(相当于载流子浓度减小了)。
\end{enumerate}
增加光强,第一个过程肯定是会被加强的。这将导致光电流的增加,但第二个过程随光强的变化关系未知,很可能是复杂的非线性关系,也不一定是单调的。由于最终测得的光电流是两个过程共同作用的结果,而由于第二个过程对光强的复杂依赖关系,最终的光电流的确可能随光强的增加不是单调的。这一现象也在其他文献不同体系中被观测到了。不过,这些文献大多也只给了定性的原理解释,很难给出定量的计算。

\part{文献一:\emph{Light intensity-induced photocurrent switching effect(https://doi.org/10.1038/s41467-020-14675-5)}}

\section{主要结论:}

这篇文章观察到了光电流随光强的非单调变化,如图 (\ref{fig1})所示。

\begin{figure}
\includegraphics[width=0.7\textwidth]{\string"Figure/Screenshot from 2023-04-01 23-38-52\string".png}

\caption{光电流随光强变化}

\label{fig1}
\end{figure}
在这幅图中,文章作者提出,净光电流随光强的增加先呈增加趋势,然后到达最大值后会减小,最终改变方向反向增加。更进一步,我们可以将净光电流分成阴极光电流和阳极光电流,净光电流为两者之差。阴极光电流先随光强增大迅速增大,之后达到饱和,而阳极光电流则始终增大。从而,在低光强下阴极电流占主导,高光强下阳极电流占主导,最终导致了光电流随光强的非单调变化。

\section{实验设施:}

\textbf{实验材料:} $ZnO$, $CA$和$C_{60}(OH)_{30-36}$的混合物。

\textbf{实验要求:}
\begin{enumerate}
\item 需要有氧气(氧化剂)的参与,这里氧化剂或氧气的作用是在电解质中参与氧化还原反应,从而起到吸引电子,将电子从材料输运到电解质溶液(产生阴极光电流)的作用。在我们的实验体系中,虽然观察到的现象是真空中电流的非单调性,但考虑到我们很多实验结果的真空、空气对比都与大多数光电材料的真空、空气对比相反。所以,这并不和我们的结果构成排斥。
\item 材料需要提前经历一段时间的光照(约为10分钟),称为预处理阶段。
\end{enumerate}
\textbf{实验方法: }包括UV-Vis spectral, differential pulse voltammogram
(DPV)等多种辅助方法,测定反应过程中及反应前后物质的化学结构,推断出变化的原因。

\section{实验解释:}

\textbf{简单版本:}

经过预处理的光照之后,$ZnO$与$C_{60}(OH)_{30-36}$形成了新的共价键,这个共价键的性质(例如可以容纳的电子数量)与光强有关。这个共价键可以trap住电子,使得电子不能产于输运。而一旦电子被trap住,它就有可能重新和正电荷组合,或者被输运到基底(而不是电极)上。由于提升光强影响了共价键的性质,使得电子被trap的几率和时间增强,从而更容易和正电荷组合,降低载流子浓度,从而使得整体光电流下降。

我们可以近似认为,电子的寿命就是电子被共价键trap的时间。通过分析电子的寿命,见图(\ref{fig2}),我们发现,电子被trap的时间的确随着光强的增强而增大,证明提升光强后,电子更容易被trap。

\begin{figure}
\includegraphics[width=0.3\textwidth]{\string"Figure/Screenshot from 2023-04-02 00-04-31\string".png}

\caption{电子寿命随光强变化}

\label{fig2}
\end{figure}

\textbf{复杂版本:}

在电子被trap的过程中,发生的动力学过程是很复杂的,其中还涉及到动力学与统计力学的相互作用,以及电子不同输运pathway的竞争。大体来说,主要包括三个pathway:
\begin{enumerate}
\item 与正电荷重新结合。这将导致光电流的减小。
\item 被输运到基底中,这实际上对应贡献了阳极光电流,见图(\ref{fig1})。
\item 被输运到电解质中与氧化物质作用,这对应着贡献了阴极光电流。
\end{enumerate}
对于新形成的共价键,可以理解为对应着电子的一个新的束缚态,这些束缚态只能捕获激发状态的电子。当光强很小时,只有非常少的一部分电子被激发,从而落入到束缚态(共价键)中的电子非常少,即该束缚态大部分都是空的,密度很低。这时候,上述的第三个pathway占主导,束缚态中的电子可以很容易与氧化物质作用,贡献阴极光电流。并且初始时随着光强的提升,束缚态中电子密度增大,从而有更多电子与氧化物质作用,对应着初始时阴极光电流的快速增加。然后,随着光强的进一步提升,束缚态逐渐饱和,从而电子和氧化物质作用的速率也达到饱和,对应着图(\ref{fig1})中阴极光电流的饱和。而此时,阳极光电流不仅可以由上述第二个pathway贡献,还可以由电子直接从导带输运到基底中,从而阳极光电流始终随着被激发的电子数目(也就是光强)增加。这就解释了图(\ref{fig1})中阳极光电流和阴极光电流随光强的不同变化曲线。由于净光电流是两者之差,所以净光电流会有一个拐点。

\part{文献二:\emph{High intensity induced photocurrent polarity switching
in lead sulfide nanowire field effect transistors (doi:10.1088/0957-4484/25/19/195202)}}

\section{主要结论}

文章作者实验并模拟计算了光电流随光强的变化,发现光电流将会在大于$\sim100W/cm^{2}$的光强量级下改变方向, 见图(\ref{fig3})。他们将这归咎与光子激发的载流子(photo-injected
carriers)改变了半导体触面(contact)处的电场大小。

\begin{figure}
\includegraphics[viewport=0bp 0bp 401bp 370bp,clip,width=0.6\textwidth]{\string"Figure/Screenshot from 2023-04-02 01-13-55\string".png}

\caption{模拟结果}

\label{fig3}
\end{figure}


\section{实验设施}

\textbf{实验材料:} lead sulfide (PbS) nanowires (NW) field effect transistors
(FETs) ,为一种红外photodetector.

\textbf{实验方法:}scanning photocurrent microscopy (SPCM)

\textbf{有无预处理:}无。

\section{实验解释}

通过有限元模拟计算在光照射下,半导体的表面电场强度与电荷密度(使用有源麦克斯韦方程组),得到光电流的强度。在麦克斯韦方程组中,他们做了如下假定:
\begin{enumerate}
\item 电子的寿命是恒定的(此假定与文献一不同)。
\item 将光对电子的激发过程用一个(均匀的)电子激发率代替,同时将正电荷与电子的重新组合过程(recombination)用一个(均匀的)电子消失率代替。
\item 考虑半导体中掺杂,例如受主原子(吸收电子,提供正电荷,例如第三主族元素)的密度对求解电场过程的影响。
\end{enumerate}
通过有限元模拟,他们发现,如果掺杂的密度是均匀的(在空间分布中是一个常数),那么光电流将始终随着光强单调变化。所以,光电流对光强的非单调变化来源于掺杂的密度是不均匀的。而产生这种不均匀的原因则是
photo-injected carriers。他们的解释是:
\begin{itemize}
\item photo-injected carriers会显著改变半导体能带的结构形状,从而产生极强的非线性;
\item 这种非线性的一个体现就是,在触面附近,有限电荷密度会远远低于体内的电荷密度,从而使得触面附近有很强的边界效应,即边界处的电场被弯曲,见图(\ref{fig4})。
\item 这种触面处的边界效应与门电压和栅极电压一起,产生了一种有效的掺杂。而且这种有效的掺杂在触面附近浓度低,体内浓度高。从而产生了一种等效的不均匀掺杂密度。
\item 这种等效的不均匀掺杂密度最终导致了光电流不再随光强线性变化。
\end{itemize}
所以,他们将这种光电流不随光强线性变化的原因归结于photo-injected carriers引起的强烈非线性。

\begin{figure}
\includegraphics[width=0.6\textwidth]{\string"Figure/Screenshot from 2023-04-02 01-35-22\string".png}

\caption{边界效应}

\label{fig4}
\end{figure}

\end{document}
