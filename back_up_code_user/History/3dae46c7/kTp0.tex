%% LyX 2.3.6.1 created this file.  For more info, see http://www.lyx.org/.
%% Do not edit unless you really know what you are doing.
\documentclass[english]{revtex4-2}
\usepackage[T1]{fontenc}
\usepackage{geometry}
\geometry{verbose}
\setcounter{secnumdepth}{2}
\setcounter{tocdepth}{2}
\usepackage{amsmath}
\usepackage{amssymb}
\usepackage{graphicx}

\makeatletter
%%%%%%%%%%%%%%%%%%%%%%%%%%%%%% User specified LaTeX commands.
\usepackage{braket}
\usepackage{tikz}
%\usepackage{braket}
%\usepackage{braket}
\usepackage{listings}
\usepackage{xcolor}
\usepackage{color}
\usepackage{diagbox}
\lstset{
numbers=left,
framexleftmargin=10mm,
frame=none,
keywordstyle=\bf\color{blue},
identifierstyle=\bf,
numberstyle=\color[RGB]{0,192,192},
commentstyle=\it\color[RGB]{0,96,96},
stringstyle=\rmfamily\slshape\color[RGB]{128,0,0}
}

%\usetheme{Darmstadt}
%\usetheme{Frankfurt}
% or ...

%\setbeamercovered{transparent}
\lstdefinelanguage
   [x64]{Assembler}     % add a "x64" dialect of Assembler
   [x86masm]{Assembler} % based on the "x86masm" dialect
   % with these extra keywords:
   {morekeywords={CDQE,CQO,CMPSQ,CMPXCHG16B,JRCXZ,LODSQ,MOVSXD, %
                  POPFQ,PUSHFQ,SCASQ,STOSQ,IRETQ,RDTSCP,SWAPGS, %
                  rax,rdx,rcx,rbx,rsi,rdi,rsp,rbp, %
                  r8,r8d,r8w,r8b,r9,r9d,r9w,r9b, %
                  r10,r10d,r10w,r10b,r11,r11d,r11w,r11b, %
                  r12,r12d,r12w,r12b,r13,r13d,r13w,r13b, %
                  r14,r14d,r14w,r14b,r15,r15d,r15w,r15b,retq,callq,cmpl}} % etc.

\lstset{language=[x64]Assembler}

\makeatother

\usepackage{babel}
\begin{document}

\section{Hierarchy of solvability for two-point function}

The illustration of hierarchy of solvability can be seen in Fig. (\ref{Figure_1_illustration_hierarchy}).

\begin{figure}
\includegraphics[width=1\textwidth]{Figs/dual_unitary_hierarchy}

\caption{Illustration of hierarchy structure for solvability.}

\label{Figure_1_illustration_hierarchy}
\end{figure}

\textbf{\emph{1st:}}

Dual-unitary gates

\textbf{\emph{2nd:}}

CNOT gates, with single site operator satisfying $Re\{U_{11}^{*}U_{10}\}=0$.

\textbf{\emph{3rd:}}

CZ gates. 

\section{SFF and transformation matrix}

\textbf{\emph{with randomness:}}

\emph{Dual-unitary gates:} chaotic.

\emph{CNOT:}
\begin{enumerate}
\item It self along do not have this property.
\item With single site randomness, it is chaotic.
\item One CNOT with one dual-unitary and single site randomness is still
chaotic.
\end{enumerate}
\emph{CZ:}
\begin{enumerate}
\item With single site randomness, it is integrable.
\item One CZ with one dual-unitary and single site randomness is still integrable.
\end{enumerate}
\textbf{\emph{without randomness:}}

A gate whose Schmidt decomposition is $U=\sum_{i}O_{i}\otimes P_{j}$
and satisfying 
\[
[O_{i},P_{j}]=0\ \text{for}\ \text{all}\ i,j.
\]

Here we prove that the CNOT with one dual-unitary and single site
randomness is chaotic. To prove that, we will show that the operator
\begin{equation}
Q_{k}=\frac{1}{t}\sum_{\tau=0}^{t-1}e^{i\frac{2\pi k\tau}{t}}\Pi_{2t}^{2\tau}
\end{equation}
is still nearly the eigenvector of $\widetilde{CNOT}\otimes\widetilde{CNOT}^{\dagger}$.
Actually, since $Q_{k}$ is made up with $2$-site translation operator,
we have
\begin{equation}
\widetilde{CNOT}^{\otimes t}Q_{k}(\widetilde{CNOT}^{\dagger})^{\otimes t}=Q_{k}\widetilde{CNOT}^{\otimes t}(\widetilde{CNOT}^{\dagger})^{\otimes t}=Q_{k}(I_{4}+\eta)^{\otimes t},
\end{equation}
where 
\begin{equation}
\eta=\begin{pmatrix} &  &  & 1\\
 &  & 1\\
 & 1\\
1
\end{pmatrix}.
\end{equation}

For the operator $Q_{k}(I_{4}+\eta)^{\otimes t}$, the components
outside the space $\mathrm{span}\{Q_{0},Q_{1}\cdots Q_{t-1}\}$ will
be projected out. Next we will calculate the components inside the
eigenspace. Since $Q_{k'}Q_{k}=\delta_{k,k'}Q_{k}$, we have 
\begin{equation}
\mathrm{Tr}\{Q_{k'}Q_{k}(I_{4}+\eta)^{\otimes t}\}=0\ \mathrm{for\ }k\neq k'.
\end{equation}
The non-vanishing overlap is only with $Q_{k}$. Introduce the normalization
factor, we have $|Q_{k}\rangle\leftrightarrow\frac{Q_{k}}{\sqrt{\mathrm{Tr}Q_{k}}}$
and
\begin{equation}
\langle Q_{k}|(\widetilde{CNOT}\otimes\widetilde{CNOT}^{\dagger})^{\otimes t}|Q_{k}\rangle=\frac{\mathrm{Tr}\{Q_{k}^{2}(I_{4}+\eta)^{\otimes t}\}}{\mathrm{Tr}Q_{k}}=\frac{\mathrm{Tr}\{Q_{k}(I_{4}+\eta)^{\otimes t}\}}{\mathrm{Tr}Q_{k}}.
\end{equation}

We fuse two sites into a big site with dimension $4$ and write $\Pi_{2t}^{2\tau}\rightarrow\Pi_{t}^{\tau}$,
where $\Pi_{t}$ is the translation operator for the big site.

Let estimate the trace of $\Pi_{t}^{\tau}$. Suppose $\theta_{\tau}$
is the minimum number that satisfies
\begin{equation}
\tau\theta_{\tau}=0\ \mathrm{mod}\ t
\end{equation}
or
\begin{equation}
\tau\theta_{\tau}=kt\ \mathrm{for}\ k\in\mathbb{N}^{+}
\end{equation}
and $\phi_{\tau}=\frac{t}{\theta_{\tau}}=\frac{\tau}{k}\leq\begin{cases}
\tau & \mathrm{if}\ \tau\leq\frac{t}{2}\\
\frac{\tau}{2} & \mathrm{if}\ \tau>\frac{t}{2}
\end{cases}\leq\frac{t}{2}$. With this notation, the following inequality holds
\begin{equation}
\mathrm{Tr}\Pi_{t}^{\tau}=4^{\phi_{\tau}}\leq2^{t}.
\end{equation}
The trace of $Q_{k}$ can be lower bounded
\begin{equation}
\mathrm{Tr}Q_{k}\geq\frac{\mathrm{Tr}\Pi_{t}^{0}}{t}-\frac{\sum_{\tau=1}^{t-1}\mathrm{Tr}\Pi_{t}^{\tau}}{t}\geq\frac{4^{t}-(t-1)2^{t}}{t}.
\end{equation}
In other direction, 
\begin{equation}
\mathrm{Tr}Q_{k}\leq\frac{\mathrm{Tr}\Pi_{t}^{0}}{t}+\frac{\sum_{\tau=1}^{t-1}\mathrm{Tr}\Pi_{t}^{\tau}}{t}\leq\frac{4^{t}+(t-1)2^{t}}{t}.
\end{equation}
The $\mathrm{|Tr}\{Q_{k}(I_{4}+\eta)^{\otimes n}-Q_{k}I_{4}\}|$ can
also be bounded 
\begin{align}
 & |\mathrm{Tr}\{Q_{k}(I_{4}+\eta)^{\otimes n}-Q_{k}I_{4}\}|\nonumber \\
\leq & \mathrm{\sum_{\tau=1}^{t-1}\mathrm{Tr}\{\Pi_{t}^{\tau}}[(I_{4}+\eta)^{\otimes n}-I_{4}]\}\nonumber \\
= & \sum_{\tau=1}^{t-1}4^{\phi_{\tau}}(\begin{pmatrix}\theta_{\tau}\\
2
\end{pmatrix}+\begin{pmatrix}\theta_{\tau}\\
4
\end{pmatrix}+\cdots+\begin{pmatrix}\theta_{\tau}\\
2\lceil\frac{\theta_{\tau}-1}{2}\rceil
\end{pmatrix})^{\phi_{\tau}}\nonumber \\
\leq & \sum_{\tau=1}^{t-1}4^{\phi_{\tau}}(2^{\theta_{\tau}-1})^{\phi_{\tau}}\nonumber \\
= & \sum_{\tau=1}^{t-1}4^{\phi_{\tau}}2^{\theta_{\tau}\phi_{\tau}-\phi_{\tau}}\nonumber \\
= & 2^{t}\sum_{\tau=1}^{t-1}2^{\phi_{\tau}}\nonumber \\
\leq & t(2\sqrt{2})^{t}.
\end{align}
The first equality can be seen in Fig. (\ref{illustration_eta}).

\begin{figure}
\includegraphics[width=0.6\textwidth]{Figs/illustration_of_eta}

\caption{Illustration of the effect of $\eta$ and $\Pi_{t}^{\tau}$.}

\label{illustration_eta}
\end{figure}

Therefore, we finally get 
\begin{equation}
\frac{4^{t}-t(2\sqrt{2})^{t}}{\frac{4^{t}+(t-1)2^{t}}{t}}\leq\langle Q_{k}|(\widetilde{CNOT}\otimes\widetilde{CNOT}^{\dagger})^{\otimes t}|Q_{k}\rangle\leq\frac{4^{t}+t(2\sqrt{2})^{t}}{\frac{4^{t}-(t-1)2^{t}}{t}}
\end{equation}
which approaches $1$ when $t\to\infty$. 
\end{document}
