%Definition des Papierformats und der Art des Dokuments
%\documentclass[12pt,a4paper]{report}
\documentclass[12pt,a4paper]{article}
\usepackage{tabto} %festgelegter Abstand
\usepackage{textpos} %festgelegte Position

\usepackage{float} %vertikale Tabelleneinträge ?
%\usepackage[doublespacing]{setspace} %doppelter Zeilenabstand
\usepackage{esvect} %Vektorpfeil mit \vv{}
 
\usepackage{lmodern}
%\usepackage{setspace}
 
%\usepackage[version=4]{mhchem} %Chemie
 

 
\usepackage{autobreak} %Zeilenumbruch in align
 
\usepackage{bbold} %identity matrix
\usepackage[utf8]{inputenc}
\usepackage{mathtools}
\usepackage[english]{babel}
\usepackage{graphicx}
\usepackage[separate-uncertainty = true,multi-part-units=single]{physics,siunitx}
\usepackage[nosumlimits]{amsmath}
\usepackage{gensymb}
\usepackage{subfigure}
\usepackage{subcaption}
\usepackage{amssymb}
\usepackage[thinc]{esdiff}
\usepackage{float}
\usepackage{listings}
\usepackage{xcolor}
\usepackage{color}
\usepackage{braket}
\usepackage{minted}
 
\usepackage{tikz} % Graphiken zeichnen ?
\usetikzlibrary{decorations.pathreplacing}% Graphiken zeichnen ?

%Für in deutsch geschriebene Dokumente 


%\usepackage[scaled]{uarial}
\usepackage{csquotes}
\usepackage{helvet}   %Schriftart Helvetica
\renewcommand{\familydefault}{\sfdefault}
\sloppy % reduziert Silbentrennung stark auf ein Minimum


\usepackage{epstopdf}   % Für .eps files
\usepackage{tabularx}
\usepackage{comment}
\usepackage{lastpage}

\usepackage{fancyhdr}   % Seitenlayout mit Kopf/Fußzeile
\usepackage[hidelinks]{hyperref} % Setzt Verweise und Links innerhalb des PDF-Dokuments
\usepackage{url}        % Ermöglicht Erstellen von URL-Links
\usepackage{booktabs}
\clubpenalty  = 10000 % Schusterjungen verhindern
\widowpenalty = 10000 % Hurenkinder verhindern


% Literatur
\usepackage[
    backend=biber, % biber statt bibtex kann besser mit sonderzeichen umgehen...
    language=english,
    bibstyle=ieee,
    citestyle=ieee,
    maxbibnames=1000,
    url=true,
    sorting=none]
    {biblatex}
%\bibliography{MyCollection.bib}
\addbibresource{MyCollection.bib}


\usepackage{newunicodechar}
\newunicodechar{́}{\'} %definiere Unicode-Zeichen (wg. Fehlermeldung bei .bib)



%\def \Titelshort{ }
%\def \Autor{Tim Kaiser}

%%% Seitenlayout
\usepackage[a4paper,left=3cm,right=3cm,top=3cm,bottom=3cm]{geometry}
\setlength{\parindent}{0cm} % Einrücken nach \\ wird auf 0 gesetzt 
\setlength{\parskip}{1.5ex plus0.5ex minus0.2ex} % Absatzabstand

%%Kopfzeile und Fußzeile
\pagestyle{fancy}
\fancyhf{}                  %alle Kopf- und Fußzeilenfelder bereinigen
\fancyhead[L]{}             %Kopfzeile links
%\fancyhead[R]{\Autor}      %Kopfzeile rechts
%\renewcommand{\headrulewidth}{0.4pt} %obere Trennlinie
\fancyfoot[C]{\thepage\ }        %Seitennummer
%\fancyfoot[C]{Seite \thepage\ von \pageref{LastPage}}  
\renewcommand{\footrulewidth}{0.4pt} %untere Trennlinie



\usepackage{blindtext}
\usepackage[T1]{fontenc}


\def \Datum{Date of Experiment: 
June 9, 2022}