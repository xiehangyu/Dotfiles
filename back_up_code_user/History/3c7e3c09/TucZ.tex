\subsection{Four-Point Measure:}

It can eliminate the contact resistance of measurement setup. It is
used to measure very small resistance, such as in superconductor.

\subsection{Radiation frequency}

The lowest order harmonic has the equation $fh=2eU$ or 
\begin{equation}
f=\frac{2eU}{h},
\end{equation}
 which will give the value $U$ as $U=2\times10^{-5}V$.

For the second order harmonic, $f=\frac{4eU}{h}$ and $U=1\times10^{-5}V$.

\subsection{2nd Josephson equation}

For a simple derivation, let's us omit the magnetic vector potential
$\overrightarrow{A}$ first. In this case, the phase difference $\gamma$
is just defined as $\gamma=\varphi_{2}-\varphi_{1}$. According to
Schrodinger equation $i\hbar\partial_{t}\psi_{1,2}=E_{1,2}\psi_{1,2}$
and $\psi_{1,2}=\exp(i\varphi_{1,2})$, we arrive at
\begin{equation}
i\hbar\exp(i\varphi_{1,2})\partial_{t}(i\varphi_{1,2})=E_{1,2}\exp(i\varphi_{1,2}),
\end{equation}
or equivalently 
\begin{equation}
\partial_{t}\varphi_{1,2}=-\frac{E_{1,2}}{\hbar}.
\end{equation}
It is worthy noting that here $E_{1,2}$ denotes the eigenenergies
of superconductor $1,2$, respectively, which should not be mixed
up with electric field $\overrightarrow{\bm{E}}$ appearing later.
Since the electro-magnetic potential $\Phi$ between two superconductors
are different and each copper pair takes $-2e$ charge, we obtain
\begin{equation}
\begin{split}\partial_{t}\gamma= & \partial_{t}\varphi_{2}-\partial_{t}\varphi_{1}\\
= & -\frac{(E_{2}-E_{1})}{\hbar}\\
= & \frac{2e}{\hbar}(\Phi_{2}-\Phi_{1})\\
= & \frac{2e}{\hbar}U.
\end{split}
\label{eq:partial_derivation_for_J2}
\end{equation}
Hence, 
\begin{equation}
\dot{\gamma}-\frac{2e}{\hbar}U=0.
\end{equation}
For a complete derivation, we need to consider the magnetic vector
potential $\overrightarrow{A}$. In this case, $\gamma$ should be
defined as a gauge-independent way $\gamma=\varphi_{2}-\varphi_{1}+\frac{2e}{\hbar c}\int_{1}^{2}\overrightarrow{A}\cdot d\overrightarrow{s}$.
From Eq. (\ref{eq:partial_derivation_for_J2}), we obtain
\begin{equation}
\begin{split}\partial_{t}\gamma= & \partial_{t}(\varphi_{2}-\varphi_{1})+\frac{2e}{\hbar c}\int_{1}^{2}\dot{\overrightarrow{A}}\cdot d\overrightarrow{s}\\
= & \frac{2e}{\hbar}(\Phi_{2}-\Phi_{1})+\frac{2e}{\hbar c}\int_{1}^{2}\dot{\overrightarrow{A}}\cdot d\overrightarrow{s}\\
= & \frac{2e}{\hbar}\int_{1}^{2}(\nabla\Phi+\frac{1}{c}\dot{\overrightarrow{A}})\cdot d\overrightarrow{s}\\
= & -\frac{2e}{\hbar}\int_{1}^{2}\overrightarrow{\bm{E}}\cdot d\overrightarrow{s},
\end{split}
\label{eq:complete_proof_of_J2}
\end{equation}
where the electric field $\overrightarrow{\bm{E}}=-(\nabla\Phi+\frac{1}{c}\dot{\overrightarrow{A}})$
according to the Maxwell's equation. From the electromagnetism class,
we can define the gauge-independent electric potential $U$ as $U=-\int_{1}^{2}\overrightarrow{\bm{E}}\cdot d\overrightarrow{s}$
and the relation
\begin{equation}
\dot{\gamma}-\frac{2e}{\hbar}U=0
\end{equation}
still holds.

\subsection{Josephson effect and pendulum}
From
\begin{equation}
I_{tot}=I_{c}\sin\gamma+\frac{U}{R_{n}}+C\dot{U}
\end{equation}
with $U=\frac{\hbar\dot{\gamma}}{2e}$, we have 
\begin{equation}
I_{tot}=I_{c}\sin\gamma+\frac{\hbar}{2eR_{n}}\dot{\gamma}+\frac{C\hbar}{2e}\ddot{\gamma}.
\end{equation}
We can also write it as 
\begin{equation}
\frac{I_{tot}}{I_{c}}=\sin\gamma+\frac{\Phi_{0}}{2\pi R_{n}I_{c}}\dot{\gamma}+\frac{C\Phi_{0}}{2\pi I_{c}}\ddot{\gamma}.
\end{equation}
We further introduce $t'=\sqrt{\frac{2\pi I_{c}}{C\Phi_{0}}}t$ to
make the equation dimensionless:
\begin{equation}
\frac{I_{tot}}{I_{c}}=\sin\gamma+\frac{d\gamma}{dt'}\sqrt{\frac{\Phi_{0}}{2\pi R_{n}^{2}CI_{c}}}+\frac{d\gamma}{dt''},
\end{equation}
or 
\begin{equation}
a_{d}=\sin\gamma+\frac{1}{\beta_{c}^{2}}\frac{d\gamma}{dt'}+\frac{d\gamma}{dt''}.
\end{equation}
In classical mechanics, the pendulum has the equation
\begin{equation}
ml^{2}\ddot{\theta}+mgl\sin\theta+f\dot{\theta}=L_{d},
\end{equation}
where $f$ is the damping moment caused by friction, $L_{d}$ is the
external driven moment. We can see that 
\begin{equation}
\begin{cases}
\gamma & =\theta;\\
I_{c} & =mgl;\\
\frac{\Phi_{0}}{2\pi R_{n}} & =f;\\
\frac{C\Phi_{0}}{2\pi} & =ml^{2};\\
I_{tot} & =L_{d}.
\end{cases}
\end{equation}


\subsection{Gauge invariant}

It is well known that the $\overrightarrow{E}$ is gauge invariant.
We only need to show that $\gamma$ is gauge invariant. Actually,
under gauge transformation, 
\begin{align}
\frac{2e}{\hbar c}\int_{1}^{2}\overrightarrow{A}\cdot\overrightarrow{s} & \to\frac{2e}{\hbar c}\int_{1}^{2}\overrightarrow{A}\cdot\overrightarrow{s}+\frac{2e}{\hbar c}\int_{1}^{2}\nabla\chi\overrightarrow{ds}\nonumber \\
 & =\frac{2e}{\hbar c}\int_{1}^{2}\overrightarrow{A}\cdot\overrightarrow{s}+\frac{2e}{\hbar c}(\chi_{2}-\chi_{1})
\end{align}
and 
\begin{equation}
\varphi_{2}-\varphi_{1}\to\varphi_{2}-\varphi_{1}-\frac{2e}{\hbar c}(\chi_{2}-\chi_{1})
\end{equation}
Therefore, $\gamma$ is gauge invariant. 
