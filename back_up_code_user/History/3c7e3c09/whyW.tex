\subsection{Four-Point Measure:}

It can eliminate the contact resistance of measurement setup. It is
used to measure very small resistance, such as in superconductor.

\subsection{Radiation frequency}

The lowest order harmonic has the equation $fh=2eU$ or 
\begin{equation}
f=\frac{2eU}{h},
\end{equation}
 which will give the value $U$ as $U=2\times10^{-5}V$.

For the second order harmonic, $f=\frac{4eU}{h}$ and $U=1\times10^{-5}V$.

\subsection{2nd Josephson equation}

For a simple derivation, let's us omit the magnetic vector potential
$\overrightarrow{A}$ first. In this case, the phase difference $\gamma$
is just defined as $\gamma=\varphi_{2}-\varphi_{1}$. According to
Schrodinger equation $i\hbar\partial_{t}\psi_{1,2}=E_{1,2}\psi_{1,2}$
and $\psi_{1,2}=\exp(i\varphi_{1,2})$, we arrive at
\begin{equation}
i\hbar\exp(i\varphi_{1,2})\partial_{t}(i\varphi_{1,2})=E_{1,2}\exp(i\varphi_{1,2}),
\end{equation}
or equivalently 
\begin{equation}
\partial_{t}\varphi_{1,2}=-\frac{E_{1,2}}{\hbar}.
\end{equation}
It is worthy noting that here $E_{1,2}$ denotes the eigenenergies
of superconductor $1,2$, respectively, which should not be mixed
up with electric field $\overrightarrow{\bm{E}}$ appearing later.
Since the electro-magnetic potential $\Phi$ between two superconductors
are different and each copper pair takes $-2e$ charge, we obtain
\begin{equation}
\begin{split}\partial_{t}\gamma= & \partial_{t}\varphi_{2}-\partial_{t}\varphi_{1}\\
= & -\frac{(E_{2}-E_{1})}{\hbar}\\
= & \frac{2e}{\hbar}(\Phi_{2}-\Phi_{1})\\
= & \frac{2e}{\hbar}U.
\end{split}
\label{eq:partial_derivation_for_J2}
\end{equation}
Hence, 
\begin{equation}
\dot{\gamma}-\frac{2e}{\hbar}U=0.
\end{equation}
For a complete derivation, we need to consider the magnetic vector
potential $\overrightarrow{A}$. In this case, $\gamma$ should be
defined as a gauge-independent way $\gamma=\varphi_{2}-\varphi_{1}+\frac{2e}{\hbar c}\int_{1}^{2}\overrightarrow{A}\cdot d\overrightarrow{s}$.
From Eq. (\ref{eq:partial_derivation_for_J2}), we obtain
\begin{equation}
\begin{split}\partial_{t}\gamma= & \partial_{t}(\varphi_{2}-\varphi_{1})+\frac{2e}{\hbar c}\int_{1}^{2}\dot{\overrightarrow{A}}\cdot d\overrightarrow{s}\\
= & \frac{2e}{\hbar}(\Phi_{2}-\Phi_{1})+\frac{2e}{\hbar c}\int_{1}^{2}\dot{\overrightarrow{A}}\cdot d\overrightarrow{s}\\
= & \frac{2e}{\hbar}\int_{1}^{2}(\nabla\Phi+\frac{1}{c}\dot{\overrightarrow{A}})\cdot d\overrightarrow{s}\\
= & -\frac{2e}{\hbar}\int_{1}^{2}\overrightarrow{\bm{E}}\cdot d\overrightarrow{s},
\end{split}
\label{eq:complete_proof_of_J2}
\end{equation}
where the electric field $\overrightarrow{\bm{E}}=-(\nabla\Phi+\frac{1}{c}\dot{\overrightarrow{A}})$
according to the Maxwell's equation. From the electromagnetism class,
we can define the gauge-independent electric potential $U$ as $U=-\int_{1}^{2}\overrightarrow{\bm{E}}\cdot d\overrightarrow{s}$
and the relation
\begin{equation}
\dot{\gamma}-\frac{2e}{\hbar}U=0
\end{equation}
still holds.

\subsection{Josephson effect and pendulum}
From
\begin{equation}
I_{tot}=I_{c}\sin\gamma+\frac{U}{R_{n}}+C\dot{U}
\end{equation}
with $U=\frac{\hbar\dot{\gamma}}{2e}$, we have 
\begin{equation}
I_{tot}=I_{c}\sin\gamma+\frac{\hbar}{2eR_{n}}\dot{\gamma}+\frac{C\hbar}{2e}\ddot{\gamma}.
\end{equation}
We can also write it as 
\begin{equation}
\frac{I_{tot}}{I_{c}}=\sin\gamma+\frac{\Phi_{0}}{2\pi R_{n}I_{c}}\dot{\gamma}+\frac{C\Phi_{0}}{2\pi I_{c}}\ddot{\gamma}.
\end{equation}
We further introduce $t'=\sqrt{\frac{2\pi I_{c}}{C\Phi_{0}}}t$ to
make the equation dimensionless:
\begin{equation}
\frac{I_{tot}}{I_{c}}=\sin\gamma+\frac{d\gamma}{dt'}\sqrt{\frac{\Phi_{0}}{2\pi R_{n}^{2}CI_{c}}}+\frac{d\gamma}{dt''},
\end{equation}
or 
\begin{equation}
a_{d}=\sin\gamma+\frac{1}{\beta_{c}^{2}}\frac{d\gamma}{dt'}+\frac{d\gamma}{dt''}.
\end{equation}
In classical mechanics, the pendulum has the equation
\begin{equation}
ml^{2}\ddot{\theta}+mgl\sin\theta+f\dot{\theta}=L_{d},
\end{equation}
where $f$ is the damping moment caused by friction, $L_{d}$ is the
external driven moment. We can see that 
\begin{equation}
\begin{cases}
\gamma & =\theta;\\
I_{c} & =mgl;\\
\frac{\Phi_{0}}{2\pi R_{n}} & =f;\\
\frac{C\Phi_{0}}{2\pi} & =ml^{2};\\
I_{tot} & =L_{d}.
\end{cases}
\end{equation}


\subsection{Gauge invariant}

In the simple case where we omit the magnetic vector potential $\overrightarrow{A}$,
$\gamma$ is defined as $\gamma=\varphi_{2}-\varphi_{1}$ and $U$
is defined as $U=\Phi_{2}-\Phi_{1}$, where $\Phi$ is the gauge-dependent
electric potential. Under a gauge transformation $\varphi\to\varphi-\frac{2e}{\hbar c}\chi$
and $\Phi\to\Phi-\frac{1}{c}\frac{\partial\chi}{\partial t}$, the
new $\gamma'$ and $U'$ can be written as
\begin{equation}
\gamma'=\varphi_{2}'-\varphi_{1}'=\varphi_{2}-\varphi_{1}-\frac{2e}{\hbar c}\chi_{2}+\frac{2e}{\hbar c}\chi_{1}=\gamma-\frac{2e}{\hbar c}\chi_{2}+\frac{2e}{\hbar c}\chi_{1},
\end{equation}
\begin{equation}
U'=\Phi_{2}'-\Phi_{1}'=\Phi_{2}-\Phi_{1}-\frac{1}{c}\dot{\chi}_{2}+\frac{1}{c}\dot{\chi}_{1}=U-\frac{1}{c}\dot{\chi}_{2}+\frac{1}{c}\dot{\chi}_{1}.
\end{equation}
After some calculations, we obtain
\begin{equation}
\begin{split}\dot{\gamma}'= & \dot{\gamma}-\frac{2e}{\hbar c}(\dot{\chi}_{2}-\dot{\chi}_{1})\\
= & \frac{2e}{\hbar}(U-\frac{1}{c}\dot{\chi}_{2}+\frac{1}{c}\dot{\chi}_{1})\\
= & \frac{2e}{\hbar}U'.
\end{split}
\end{equation}
Therefore, the second Josephson equation still holds. 

For the complete proof, we have to consider the vector potential $\overrightarrow{A}$
in which case $\gamma$ is defined as $\gamma=\varphi_{2}-\varphi_{1}+\frac{2e}{\hbar c}\int_{1}^{2}\overrightarrow{A}\cdot d\overrightarrow{s}$
and $U$ is defined as $U=-\int_{1}^{2}\overrightarrow{\bm{E}}\cdot d\overrightarrow{s}$
according to Eq. (\ref{eq:complete_proof_of_J2}). Now, we can prove
$\gamma$ and $U$ are both gauge-independent thus the second Josephson
equation is also gauge-independent. In fact, under the gauge transformation
$\overrightarrow{A}\to\overrightarrow{A}+\nabla\chi$ and we obtain
\begin{equation}
\begin{split}\gamma'= & \varphi_{2}'-\varphi_{1}'+\frac{2e}{\hbar c}\int_{1}^{2}\overrightarrow{A'}\cdot d\overrightarrow{s}\\
= & \varphi_{2}-\varphi_{1}-\frac{2e}{\hbar c}\chi_{2}+\frac{2e}{\hbar c}\chi_{1}+\frac{2e}{\hbar c}\int_{1}^{2}(\overrightarrow{A}+\nabla\chi)\cdot d\overrightarrow{s}\\
= & \gamma-\frac{2e}{\hbar c}\chi_{2}+\frac{2e}{\hbar c}\chi_{1}+\frac{2e}{\hbar c}\int_{1}^{2}\nabla\chi\cdot d\overrightarrow{s}\\
= & \gamma-\frac{2e}{\hbar c}\chi_{2}+\frac{2e}{\hbar c}\chi_{1}+\frac{2e}{\hbar c}\chi_{2}-\frac{2e}{\hbar c}\chi_{1}\\
= & \gamma.
\end{split}
\end{equation}
On the other hand, 
\begin{equation}
\begin{split}\overrightarrow{\bm{E}}'= & -(\nabla\Phi'+\frac{1}{c}\dot{\overrightarrow{A}}')\\
= & -\left(\nabla(\Phi-\frac{1}{c}\frac{\partial\chi}{\partial t})+\frac{1}{c}\dot{\overrightarrow{A}}+\frac{1}{c}\frac{\partial}{\partial t}\nabla\chi\right)\\
= & \overrightarrow{\bm{E}}+\frac{1}{c}\nabla\frac{\partial\chi}{\partial t}-\frac{1}{c}\frac{\partial}{\partial t}\nabla\chi\\
= & \overrightarrow{\bm{E}},
\end{split}
\end{equation}
hence $U=-\int_{1}^{2}\overrightarrow{\bm{E}}\cdot d\overrightarrow{s}$
is also gauge-independent. Since now, $\gamma$ and $U$ are both
gauge-independent, the second Josephson equation is automatically
gauge-independent. 
