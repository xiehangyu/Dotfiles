\documentclass[english]{ctexart}
\usepackage[T1]{fontenc}
\usepackage{geometry}
\geometry{verbose,tmargin=0cm,bmargin=2cm,lmargin=2cm,rmargin=2cm}
\setcounter{secnumdepth}{2}
\setcounter{tocdepth}{2}
\usepackage{graphicx}
\usepackage[authoryear]{natbib}

\makeatletter

%%%%%%%%%%%%%%%%%%%%%%%%%%%%%% LyX specific LaTeX commands.
%% Because html converters don't know tabularnewline
\providecommand{\tabularnewline}{\\}

%%%%%%%%%%%%%%%%%%%%%%%%%%%%%% User specified LaTeX commands.
\usepackage{braket}
\usepackage{tikz}
%\usepackage{braket}
%\usepackage{braket}
\usepackage{listings}
\usepackage{xcolor}
\usepackage{color}
\usepackage{diagbox}
\lstset{
numbers=left,
framexleftmargin=10mm,
frame=none,
keywordstyle=\bf\color{blue},
identifierstyle=\bf,
numberstyle=\color[RGB]{0,192,192},
commentstyle=\it\color[RGB]{0,96,96},
stringstyle=\rmfamily\slshape\color[RGB]{128,0,0}
}

%\usetheme{Darmstadt}
%\usetheme{Frankfurt}
% or ...

%\setbeamercovered{transparent}
\lstdefinelanguage
   [x64]{Assembler}     % add a "x64" dialect of Assembler
   [x86masm]{Assembler} % based on the "x86masm" dialect
   % with these extra keywords:
   {morekeywords={CDQE,CQO,CMPSQ,CMPXCHG16B,JRCXZ,LODSQ,MOVSXD, %
                  POPFQ,PUSHFQ,SCASQ,STOSQ,IRETQ,RDTSCP,SWAPGS, %
                  rax,rdx,rcx,rbx,rsi,rdi,rsp,rbp, %
                  r8,r8d,r8w,r8b,r9,r9d,r9w,r9b, %
                  r10,r10d,r10w,r10b,r11,r11d,r11w,r11b, %
                  r12,r12d,r12w,r12b,r13,r13d,r13w,r13b, %
                  r14,r14d,r14w,r14b,r15,r15d,r15w,r15b,retq,callq,cmpl}} % etc.

\lstset{language=[x64]Assembler}

\makeatother

\usepackage{babel}
\begin{document}
\title{计算方法第六次编程作业}
\author{戴梦佳\ \ PB20511879}
\maketitle

\section{题目}

通过快速傅里叶变换与快速傅里叶逆变换实现对给定函数的 Fourier 分析, 函数 $f$ 以及划分数 $n$ 如下: 

1. $f_{1}(t)=0.7\sin(2\pi\times2t)+\sin(2\pi\times5t),n=2^{4},2^{7}$.

2. $f_{2}(t)=0.7\sin(2\pi\times2t)+\sin(2\pi\times5t)+0.3\times\mathrm{random(t)}$,
其中 $\mathrm{random}(\mathrm{t})$ 为 $[0,1)$ 区间内的随机数, $\mathrm{n}=2^{7}$.
其中 $t\in[0,1)$, 将 $[0,1)$ 区间均匀划分为 $n$ 份, $f_{1,k}=f_{1}(k/n),k=0,1,\ldots,n-1,f_{2,k}$
为 $f_{1,k}$ 加上一个 随机扰动项, $f_{2,k}=f_{1,k}+0.3r_{k},r_{k}\in[0,1)$
为随机数。

\section{原理}

由于给定的$n$都是2的幂次,所以可以用逐次分半算法进行计算。复杂度为$O(n\log n)$。所有中间生成的数据结构都以.csv的格式保存在/src/data下的文件夹里。最后分析数据时通过python来作图。

\section{实验结果}

\subsection{频域分量的考察}

考察快速傅里叶变化后的频域时,每三个频率为一行,输出傅里叶变化后的实数部和虚数部。

对于函数$f_{1}$, $n=16$时,结果如下所示

\begin{tabular}{lllllllllllll}
Frequency & Real & Imaginary &  &  &  &  &  &  &  &  &  & \tabularnewline
0 & -1.53e-16 & 0 &  &  & 1 & 4.49e-17 & 1.06e-17 &  &  & 2 & 1.36e-16 & -0.35\tabularnewline
3 & 2.03e-17 & 1.67e-16 &  &  & 4 & -2.01e-19 & 1.11e-16 &  &  & 5 & -2.97e-16 & -0.5\tabularnewline
6 & -1.08e-16 & -8.33e-17 &  &  & 7 & 1.48e-16 & -1.97e-16 &  &  & 8 & 3.19e-16 & 0\tabularnewline
9 & 1.1e-16 & 2.39e-16 &  &  & 10 & -3.07e-17 & 1.11e-16 &  &  & 11 & -3.68e-16 & 0.5\tabularnewline
12 & -2.01e-19 & -1.11e-16 &  &  & 13 & 6.34e-17 & -1.67e-16 &  &  & 14 & 5.81e-17 & 0.35\tabularnewline
15 & 5.89e-17 & -5.33e-17 &  &  &  &  &  &  &  &  &  & \tabularnewline
\end{tabular}

对于函数$f_{1}$, $n=128$时,结果如下所示

\begin{tabular}{lllllllllllll}
Frequency & Real & Imaginary &  &  &  &  &  &  &  &  &  & \tabularnewline
0 & 9.95e-18 & 0 &  &  & 1 & 4.6e-18 & 2.34e-17 &  &  & 2 & -3.84e-18 & -0.35\tabularnewline
3 & 2.35e-17 & 6.37e-17 &  &  & 4 & 1.12e-17 & 4.68e-17 &  &  & 5 & -3.29e-16 & -0.5\tabularnewline
6 & 5.19e-18 & -7.23e-17 &  &  & 7 & 3.11e-17 & -6.45e-17 &  &  & 8 & 1.67e-17 & -7.12e-17\tabularnewline
9 & 1e-17 & -2.06e-17 &  &  & 10 & 3.35e-17 & -5.73e-17 &  &  & 11 & 6.63e-17 & 3.16e-17\tabularnewline
12 & 5.15e-17 & -5.47e-17 &  &  & 13 & 5.34e-18 & 3.71e-17 &  &  & 14 & 2.01e-17 & 3.38e-18\tabularnewline
15 & -3.48e-17 & -4.66e-18 &  &  & 16 & -1.45e-17 & -6.08e-18 &  &  & 17 & 8.59e-18 & -1.26e-17\tabularnewline
18 & 1.23e-17 & -4.28e-17 &  &  & 19 & 1.87e-17 & 6.85e-18 &  &  & 20 & 4.03e-17 & -3.68e-17\tabularnewline
21 & -3.92e-18 & 7.07e-17 &  &  & 22 & -1.31e-17 & 2.59e-17 &  &  & 23 & -1.01e-17 & 5.3e-18\tabularnewline
24 & -3.86e-17 & -4.21e-17 &  &  & 25 & 3.25e-17 & 1.58e-17 &  &  & 26 & -4.1e-17 & 1.15e-17\tabularnewline
27 & -1.76e-16 & -1.08e-16 &  &  & 28 & -9.7e-17 & -1.9e-17 &  &  & 29 & 1.76e-17 & -1.4e-16\tabularnewline
30 & -1.18e-17 & -1.03e-16 &  &  & 31 & -1.14e-18 & 3.66e-17 &  &  & 32 & -3.39e-17 & -1.86e-17\tabularnewline
33 & 1.51e-17 & 8.34e-18 &  &  & 34 & -2.3e-17 & -2.92e-17 &  &  & 35 & -5.97e-18 & 3.27e-18\tabularnewline
36 & -5.47e-17 & -1.77e-17 &  &  & 37 & 8.68e-18 & 1.34e-17 &  &  & 38 & -9.63e-17 & -2.02e-17\tabularnewline
39 & 1.82e-17 & -1.43e-16 &  &  & 40 & 9.99e-17 & -3.27e-17 &  &  & 41 & 2.06e-17 & 1.48e-17\tabularnewline
42 & 2.66e-17 & -3.43e-17 &  &  & 43 & 6.81e-18 & 4.03e-17 &  &  & 44 & -8.76e-18 & -1.29e-17\tabularnewline
45 & -2.98e-18 & -1.46e-17 &  &  & 46 & -1.88e-17 & 1.54e-17 &  &  & 47 & 2.97e-17 & -5.07e-17\tabularnewline
48 & 1.8e-17 & -4.35e-18 &  &  & 49 & -1.57e-17 & 2.57e-18 &  &  & 50 & 2.27e-17 & -2.83e-17\tabularnewline
51 & 1.9e-17 & 1.25e-17 &  &  & 52 & 5.22e-18 & 1.49e-17 &  &  & 53 & 1.86e-17 & -1.58e-17\tabularnewline
54 & -5.81e-17 & -3.89e-17 &  &  & 55 & 4.91e-17 & -5.55e-17 &  &  & 56 & 8.14e-17 & -3.4e-17\tabularnewline
57 & -1.72e-17 & -6.72e-18 &  &  & 58 & 7.02e-17 & -8.64e-18 &  &  & 59 & 2.68e-17 & -1.67e-16\tabularnewline
60 & 5.21e-17 & -1.72e-17 &  &  & 61 & -1.24e-17 & 2.77e-18 &  &  & 62 & 8.71e-17 & -1.11e-16\tabularnewline
63 & 4.01e-17 & 2.47e-17 &  &  & 64 & -1.02e-16 & 0 &  &  & 65 & 3.08e-17 & -2.86e-17\tabularnewline
66 & 1e-16 & -5.55e-17 &  &  & 67 & -1.64e-17 & 3.8e-18 &  &  & 68 & 5.21e-17 & 1.72e-17\tabularnewline
69 & 4.45e-18 & 0 &  &  & 70 & 6.03e-17 & 1.85e-18 &  &  & 71 & -7.74e-18 & 1.34e-17\tabularnewline
72 & 8.14e-17 & 3.4e-17 &  &  & 73 & 4.1e-17 & 5.52e-17 &  &  & 74 & -4.75e-17 & 3.96e-17\tabularnewline
75 & -1.29e-18 & 4.07e-17 &  &  & 76 & 5.22e-18 & -1.49e-17 &  &  & 77 & 3.41e-17 & -8.06e-18\tabularnewline
78 & 2.62e-17 & 5.03e-17 &  &  & 79 & -1.17e-17 & 8.84e-18 &  &  & 80 & 1.8e-17 & 4.35e-18\tabularnewline
81 & 2.7e-17 & 5.46e-17 &  &  & 82 & -4.45e-17 & 1.12e-17 &  &  & 83 & -1e-17 & 1.46e-17\tabularnewline
84 & -8.76e-18 & 1.29e-17 &  &  & 85 & 2.02e-17 & -2.05e-17 &  &  & 86 & 9.75e-18 & 2.4e-17\tabularnewline
87 & 1.68e-17 & -1.27e-17 &  &  & 88 & 9.99e-17 & 3.27e-17 &  &  & 89 & 9.13e-18 & 1.4e-16\tabularnewline
90 & -9.09e-17 & 2.82e-17 &  &  & 91 & 1.47e-16 & -1.42e-16 &  &  & 92 & -5.47e-17 & 1.77e-17\tabularnewline
93 & 3.61e-18 & -1.15e-17 &  &  & 94 & 6.8e-17 & -6.36e-17 &  &  & 95 & 1.07e-17 & -6.67e-18\tabularnewline
96 & -3.39e-17 & 1.86e-17 &  &  & 97 & -8.6e-18 & -3.47e-17 &  &  & 98 & 9.16e-17 & 1.4e-18\tabularnewline
99 & 9.2e-18 & 1.13e-16 &  &  & 100 & -9.7e-17 & 1.9e-17 &  &  & 101 & -5.53e-17 & -1.34e-17\tabularnewline
102 & -5.08e-17 & -2.99e-17 &  &  & 103 & 3.93e-17 & -1.09e-17 &  &  & 104 & -3.86e-17 & 4.21e-17\tabularnewline
105 & -1.53e-17 & -1.66e-17 &  &  & 106 & -3.66e-18 & -1.52e-17 &  &  & 107 & -2.01e-17 & -5.71e-17\tabularnewline
108 & 4.03e-17 & 3.68e-17 &  &  & 109 & 1.44e-17 & -1.5e-18 &  &  & 110 & -2.61e-17 & 6.97e-17\tabularnewline
111 & 1.44e-17 & 1.32e-17 &  &  & 112 & -1.45e-17 & 6.08e-18 &  &  & 113 & -1.68e-17 & -2.35e-18\tabularnewline
114 & -1.97e-17 & 3.21e-17 &  &  & 115 & -1.76e-17 & -5.08e-17 &  &  & 116 & 5.15e-17 & 5.47e-17\tabularnewline
117 & 3.84e-17 & -3.45e-17 &  &  & 118 & 3.46e-17 & 2.63e-17 &  &  & 119 & 1.16e-17 & 7.15e-17\tabularnewline
120 & 1.67e-17 & 7.12e-17 &  &  & 121 & 2.87e-17 & 5.69e-17 &  &  & 122 & 2.2e-17 & 1.47e-16\tabularnewline
123 & -4.17e-16 & 0.5 &  &  & 124 & 1.12e-17 & -4.68e-17 &  &  & 125 & 3.8e-18 & -3.09e-17\tabularnewline
126 & -8.64e-17 & 0.35 &  &  & 127 & 1.16e-17 & -7.46e-17 &  &  &  &  & \tabularnewline
\end{tabular}

对于函数$f_{2}$, $n=128$时,结果如下所示

\begin{tabular}{lllllllllllll}
Frequency & Real & Imaginary &  &  &  &  &  &  &  &  &  & \tabularnewline
0 & 0.139 & 0 &  &  & 1 & 0.0037 & 0.00434 &  &  & 2 & -0.00394 & -0.356\tabularnewline
3 & -0.0023 & 0.00231 &  &  & 4 & -0.00766 & 0.00058 &  &  & 5 & 0.00225 & -0.496\tabularnewline
6 & 0.00377 & 0.0069 &  &  & 7 & -0.00301 & 0.000887 &  &  & 8 & 0.0107 & -0.00149\tabularnewline
9 & -0.00663 & -0.00737 &  &  & 10 & -0.00555 & 0.00384 &  &  & 11 & 0.00322 & 0.0121\tabularnewline
12 & -0.00545 & -0.00344 &  &  & 13 & 0.00963 & -0.00119 &  &  & 14 & 0.00241 & 0.00518\tabularnewline
15 & -0.00197 & -0.0112 &  &  & 16 & -0.00215 & 0.00655 &  &  & 17 & -0.0022 & 0.00913\tabularnewline
18 & -0.00641 & -0.00767 &  &  & 19 & 0.00344 & -0.00604 &  &  & 20 & 0.0112 & -0.00553\tabularnewline
21 & 0.00359 & 0.00989 &  &  & 22 & -0.000897 & 0.0033 &  &  & 23 & 0.00851 & 0.00997\tabularnewline
24 & 0.000887 & -0.00468 &  &  & 25 & -0.00441 & 0.00731 &  &  & 26 & -0.00106 & 0.00841\tabularnewline
27 & 0.00387 & -0.00491 &  &  & 28 & 0.00134 & 0.00307 &  &  & 29 & -0.00703 & -0.00357\tabularnewline
30 & -0.00405 & -0.00537 &  &  & 31 & 0.00778 & -0.00263 &  &  & 32 & -0.00469 & 0.000897\tabularnewline
33 & -5.72e-05 & 0.000611 &  &  & 34 & -0.00356 & -0.00254 &  &  & 35 & 0.00254 & -0.00257\tabularnewline
36 & -0.00184 & -0.00291 &  &  & 37 & -0.00249 & 0.00629 &  &  & 38 & 0.0058 & -0.00123\tabularnewline
39 & -0.000974 & 0.00119 &  &  & 40 & 0.00692 & -0.00127 &  &  & 41 & 0.00225 & -0.0117\tabularnewline
42 & -0.000821 & -0.00256 &  &  & 43 & 0.0127 & 0.005 &  &  & 44 & 0.00108 & 0.00449\tabularnewline
45 & -0.00452 & 0.00033 &  &  & 46 & 0.00476 & 0.00787 &  &  & 47 & 0.00982 & 0.00326\tabularnewline
48 & 0.011 & 0.00394 &  &  & 49 & 0.0021 & 0.00259 &  &  & 50 & 0.00341 & -0.00455\tabularnewline
51 & 0.00567 & -0.00399 &  &  & 52 & 0.00482 & -0.00685 &  &  & 53 & -0.012 & 0.00118\tabularnewline
54 & -0.00474 & 0.00287 &  &  & 55 & 0.00251 & -0.000833 &  &  & 56 & -0.00309 & 0.00601\tabularnewline
57 & -0.00251 & 0.000417 &  &  & 58 & -0.00161 & -0.0115 &  &  & 59 & -0.00185 & -0.00241\tabularnewline
60 & 0.00608 & 0.00468 &  &  & 61 & -0.00531 & 0.00916 &  &  & 62 & 0.00653 & 0.00643\tabularnewline
63 & -0.00983 & 0.00319 &  &  & 64 & 0.00502 & 0 &  &  & 65 & -0.00983 & -0.00319\tabularnewline
66 & 0.00653 & -0.00643 &  &  & 67 & -0.00531 & -0.00916 &  &  & 68 & 0.00608 & -0.00468\tabularnewline
69 & -0.00185 & 0.00241 &  &  & 70 & -0.00161 & 0.0115 &  &  & 71 & -0.00251 & -0.000417\tabularnewline
72 & -0.00309 & -0.00601 &  &  & 73 & 0.00251 & 0.000833 &  &  & 74 & -0.00474 & -0.00287\tabularnewline
75 & -0.012 & -0.00118 &  &  & 76 & 0.00482 & 0.00685 &  &  & 77 & 0.00567 & 0.00399\tabularnewline
78 & 0.00341 & 0.00455 &  &  & 79 & 0.0021 & -0.00259 &  &  & 80 & 0.011 & -0.00394\tabularnewline
81 & 0.00982 & -0.00326 &  &  & 82 & 0.00476 & -0.00787 &  &  & 83 & -0.00452 & -0.00033\tabularnewline
84 & 0.00108 & -0.00449 &  &  & 85 & 0.0127 & -0.005 &  &  & 86 & -0.000821 & 0.00256\tabularnewline
87 & 0.00225 & 0.0117 &  &  & 88 & 0.00692 & 0.00127 &  &  & 89 & -0.000974 & -0.00119\tabularnewline
90 & 0.0058 & 0.00123 &  &  & 91 & -0.00249 & -0.00629 &  &  & 92 & -0.00184 & 0.00291\tabularnewline
93 & 0.00254 & 0.00257 &  &  & 94 & -0.00356 & 0.00254 &  &  & 95 & -5.72e-05 & -0.000611\tabularnewline
96 & -0.00469 & -0.000897 &  &  & 97 & 0.00778 & 0.00263 &  &  & 98 & -0.00405 & 0.00537\tabularnewline
99 & -0.00703 & 0.00357 &  &  & 100 & 0.00134 & -0.00307 &  &  & 101 & 0.00387 & 0.00491\tabularnewline
102 & -0.00106 & -0.00841 &  &  & 103 & -0.00441 & -0.00731 &  &  & 104 & 0.000887 & 0.00468\tabularnewline
105 & 0.00851 & -0.00997 &  &  & 106 & -0.000897 & -0.0033 &  &  & 107 & 0.00359 & -0.00989\tabularnewline
108 & 0.0112 & 0.00553 &  &  & 109 & 0.00344 & 0.00604 &  &  & 110 & -0.00641 & 0.00767\tabularnewline
111 & -0.0022 & -0.00913 &  &  & 112 & -0.00215 & -0.00655 &  &  & 113 & -0.00197 & 0.0112\tabularnewline
114 & 0.00241 & -0.00518 &  &  & 115 & 0.00963 & 0.00119 &  &  & 116 & -0.00545 & 0.00344\tabularnewline
117 & 0.00322 & -0.0121 &  &  & 118 & -0.00555 & -0.00384 &  &  & 119 & -0.00663 & 0.00737\tabularnewline
120 & 0.0107 & 0.00149 &  &  & 121 & -0.00301 & -0.000887 &  &  & 122 & 0.00377 & -0.0069\tabularnewline
123 & 0.00225 & 0.496 &  &  & 124 & -0.00766 & -0.00058 &  &  & 125 & -0.0023 & -0.00231\tabularnewline
126 & -0.00394 & 0.356 &  &  & 127 & 0.0037 & -0.00434 &  &  &  &  & \tabularnewline
\end{tabular}

关于这一结果的分析,见报告的最后一节分析与讨论内容。

\subsection{$|g_{i}|$的模长情况}

对于函数$f_{1}$, $n=16$时,$|g_{1}|$的模长折线图见下

\includegraphics[width=0.6\textwidth]{src/figs/f1_16FFT}

对于函数$f_{1},$$n=128$时,$|g_{1}|$的模长折线图见下

\includegraphics[width=0.6\textwidth]{src/figs/f1_128FFT}

对于函数$f_{2}$, $n=128$时,$|g_{2}|$的模长折线图见下

\includegraphics[width=0.6\textwidth]{src/figs/f2_128FFT}

\subsection{$f_{1}$的快速傅里叶逆变化}

对于$n=16$, $f_{1}$的采样点和快速傅里叶变化的逆变化见下所示(图中显示的都是实部)

\includegraphics[width=0.6\textwidth]{src/figs/f1_16IFFT}

对于$n=128$, 采样点和快速傅里叶变化的逆变化见下所示(图中显示的都是实部)

\includegraphics[width=0.6\textwidth]{src/figs/f1_128IFFT}

关于这些结果的讨论,见报告最后一节分析与讨论的内容。

\subsection{$f_{2}$的快速傅里叶逆变化}

对于$f_{2}$, 快速傅里叶变化的逆变化结果见下图所示。图中所标注的“Truncated”线,即代表快速傅里叶变换后取频率域前25\%的系数进行快速傅里叶逆变换所得最终结果图像(图中显示的都是实部)

\includegraphics[width=0.6\textwidth]{src/figs/f2_128IFFT}

\section{分析与讨论}
\begin{enumerate}
\item 采样数目$n$关系到原函数采样的逼真性。$n$越大,采样点越能反应原函数的细节信息,从而采样函数以及傅里叶逆变化后得到的最终结果都会更接近原函数。
\item 无论$n$取多大,快速傅里叶逆变化后的结果都精确等于采样函数(这里采样函数指的就是采样点处的原函数值)。
\item $n$的大小并不会明显影响傅里叶变化后频率分量$|g|$的峰值与形状分布,只会使得图像展宽。
\item 去掉高频系数后,显著影响了$f_{2}$的重建结果,使重建后的函数值下降了大约一半。这是可以预期的,因为这里舍系数的方式太粗糙了,没有考虑周期对称性。见下面的分析。
\item 由于函数$f$都是实函数,根据傅里叶变化公式$g_{l}=\frac{1}{n}\sum_{k}f_{k}e^{-i\frac{2\pi lk}{n}}$,我们得到
\begin{equation}
g_{l}^{*}=\frac{1}{n}\sum_{k}f_{k}e^{i\frac{2\pi lk}{n}}=\frac{1}{n}\sum_{k}f_{k}e^{-i\frac{2\pi(n-l)k}{n}}=g_{n-l}
\end{equation}
即在频域中,分量$g_{l}$和$g_{n-l}$互为共轭。这一点也在前面的数据和图表中得到了验证。
\item 从而,对于$f_{2}$重建时,只取频域前$25\%$的分量,必然造成严重误差。因为根据上面的分析,频域前$25\%$和频域后$25\%$的分量绝对值大小一样,对函数的重建贡献也一样。如果只取前$25\%$的数据,必然造成函数值下降一半左右。所以,对于频率滤波,我们应该取频域前$12.5\%$和频域后$12.\%$,
或者按照$g_{l}$的模长取模长最大的前$25\%$的数据,这样可以显著改善重建后的函数精度。
\item 注意到$\sin x=\frac{e^{ix}-e^{-ix}}{2i}$,所以对于函数$f_{1}$和$f_{2}$, 其应该在$2\mathrm{Hz}$,
$5\mathrm{Hz}$以及它们关于$\frac{n}{2}$的对称点上有峰值。这也在实验中被证实。另外,对于$f_{1}$,除了上述四个点外其余频域分量都为零,这是因为$f_{1}$是严格的正弦函数的叠加。而$f_{2}$由于有random
number,所以频域中其它分量也有不为零的值。
\item 由于$\{e^{i\omega jx},j=0,1,2,\cdots,n-1\}$是一组正交函数系,所以傅里叶逆变化可以重建原采样函数,这点也在实验中得到了验证。
\end{enumerate}

\end{document}
