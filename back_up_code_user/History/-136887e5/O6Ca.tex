\subsection{Experimental setup}

\subsubsection{Sample}
The sample we measured is a thin film layer contact consisting of niobium as superconductor and aluminium oxide as the separating layer. The thickness of the niobium layer is \SI{100}{nm}, and the thickness of the aluminium oxide is \SI{17.83}{nm}. The thin film is placed on a $\SI{6}{mm} \times \SI{10}{mm}$ silicon substrate which is $\SI{250}{\mu m}$ thick. The junction area is a square with an edge length of $\SI{19.5}{\mu m}$.

\subsubsection{Sample Holder}

\begin{figure}

\centering
\includegraphics[width=0.8\textwidth]{fig/sampleholder.png}

\caption{Sketch of the sample rod. This figure is taken from the manuscript \citep{manuscript}.}

\label{sampleholder}
\end{figure}

The sample with Josephson contact is located inside a coil, which can be used to generate magnetic field. The coil is fixed at the bottom of the dip stick. The dip stick can be inserted into a liquid helium to cool down to \SI{4.2}{K}, which is sufficient as the critical temperature for niobium is \SI{9.2}{K}. The junction is mounted on a board and connected to pads with aluminium bonds. The board is connected to the connection box of the NIM-frame with wiring inside the dip stick.

\subsubsection{Measurement Equipment}
For current and voltage measurement, we use a digital source measure unit (SMU) to measure the I-U characteristic of the Josephson contact. We perform a four-point measurement. The SMU is connected to a computer with a standard LAN cable. All the settings can be controlled with LabView program.

For the field dependence measurement, the current source drives a current in the coil, controlling the magnetic field inside the junction. The current is available at I-out, and the current source supplies a monitor voltage signal at Monitor Out-I. 


\subsection{Measurement}
\subsubsection{Cooling}

\begin{itemize}

\item Put on the safety glasses and gloves.
\item Remove the shutter valve from the He-can, and place the sample rod on the top of the can.
\item Lower the rod until the table tennis ball in the middle of the tube.
\item Wait until the ball is below the middle and lower the rod further down until the ball reaches the middle again.
\item Repeat until you reach the marker on the rod.

\end{itemize}

\subsubsection{Measuring the I-U curve}

\begin{itemize}

\item Set the start and stop point of the current in LabView to \SI{0}{A} and \SI{0.003}{A}. The number of points is set to 500.
\item Enable the up-down sweep and the four-wire sense.
\item Press Start Measurement, and save it as the zero-field dependence.

\end{itemize}


\subsubsection{Measuring the field dependent critical current}
Set the voltage from \SI{0}{V} and \SI{0.4}{V} with the step of \SI{0.01}{V}. The voltage of \SI{1}{V} produces a coil current of \SI{100}{mA}. At a coil current of \SI{200}{mA}, the coil produces a field of \SI{7.95}{mT}. Measure I-U characteristic with LabView program. Changing the stop point and the number of points to reduce the measurement time.

\subsubsection{Warming}
Put on the safety glasses and gloves. Pull the sample rod slowly up to the top and wait until it reached room temperature.
