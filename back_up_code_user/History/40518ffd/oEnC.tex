\documentclass[aps,onecolumn,superscriptaddress,notitlepage,longbibliography]{revtex4-1}
\usepackage{times}
\usepackage{graphicx}
\usepackage{feynmf}
\usepackage{tabularx}
\usepackage{amsmath}
\usepackage{amstext}
\usepackage{amssymb}
\usepackage{xfrac}
\usepackage[colorlinks,citecolor=blue]{hyperref}
\usepackage{graphicx}
\usepackage{amsmath}
\usepackage{amstext}
\usepackage{amssymb}
\usepackage{amsfonts}
\usepackage{longtable,booktabs}
\usepackage{hyperref}
\usepackage{url}
\usepackage{subfigure}
\usepackage{dsfont}
\usepackage{booktabs}
\usepackage{amsbsy}
\usepackage{dcolumn}
\usepackage{amsthm}
\usepackage{bm}
\usepackage{esint}
\usepackage{multirow}
\usepackage{hyperref}
\usepackage{cleveref}
\usepackage{mathrsfs}
\usepackage{amsfonts}
\usepackage{amsbsy}
\usepackage{dcolumn}
\usepackage{bm}
\usepackage{multirow}
\usepackage{color}
\usepackage{extarrows}
\usepackage{datetime}
\usepackage{comment}
\usepackage[super]{nth}
\hypersetup{
	colorlinks=magenta,
	linkcolor=blue,
	filecolor=magenta,
	urlcolor=magenta,
}
\def\Z{\mathbb{Z}}
\newcommand{\red}[1]{{\textcolor{red}{#1}}}
\newtheorem{theorem}{Theorem}
\newtheorem{statement}{Statement}
\newcommand{\mb}{\mathbb}
\newcommand{\bs}{\boldsymbol}
\newcommand{\wt}{\widetilde}
\newcommand{\mc}{\mathcal}
\newcommand{\bra}{\langle}
\newcommand{\ket}{\rangle}
\newcommand{\ep}{\epsilon}
\newcommand{\tf}{\textbf}

\begin{document}
	
	
	
\title{Supplemental Material for ``Yang-Lee Zeros, Semicircle Theorem, and Nonunitary Criticality in BCS Superconductivity''}

	\author{Hongchao Li}
\thanks{These two authors contributed equally to this work.}
\affiliation{Department of Physics, University of Tokyo, 7-3-1 Hongo, Tokyo 113-0033,
	Japan}
\email{lhc@cat.phys.s.u-tokyo.ac.jp}

\author{Xie-Hang Yu}
\thanks{These two authors contributed equally to this work.}
\affiliation{Max-Planck-Institut für Quantenoptik, Hans-Kopfermann-Straße 1, D-85748 Garching, Germany}
\affiliation{Munich Center for Quantum Science and Technology, Schellingstraße
4, 80799 München, Germany}
\email{xiehang.yu@mpq.mpg.de}

\author{Masaya Nakagawa}
\affiliation{Department of Physics, University of Tokyo, 7-3-1 Hongo, Tokyo 113-0033,
	Japan}
\email{nakagawa@cat.phys.s.u-tokyo.ac.jp}

\author{Masahito Ueda}
\affiliation{Department of Physics, University of Tokyo, 7-3-1 Hongo, Tokyo 113-0033,
	Japan}
\affiliation{RIKEN Center for Emergent Matter Science (CEMS), Wako, Saitama 351-0198,
	Japan}
\affiliation{Institute for Physics of Intelligence, University of Tokyo, 7-3-1
	Hongo, Tokyo 113-0033, Japan}
\email{ueda@cat.phys.s.u-tokyo.ac.jp}

	\begin{abstract}
		
	\end{abstract}
	\date{\today}
	\maketitle
	
	\tableofcontents

\section{Yang-Lee Zeros on the Phase Boundary}
We begin from considering the three-dimensional non-Hermitian BCS Hamiltonian
\begin{equation}
	H=\sum_{\boldsymbol{k}\sigma}\xi_{\boldsymbol{k}}c_{\boldsymbol{k}\sigma}^{\dagger}c_{\boldsymbol{k}\sigma}-\frac{U}{N}\sum_{\bm{k},\bm{k}'}{}^{'}c_{\bm{k}\uparrow}^{\dagger}c_{\bm{-k}\downarrow}^{\dagger}c_{\bm{-k}'\downarrow}c_{\bm{k}'\uparrow}, \label{eq:non-Hermitian}
	\end{equation}
where $U=U_R+iU_I$ and the prime in $\sum_{\bm{k}}^{'}$ indicates that the sum over $\bm{k}$ restricted to  $|\xi_{\boldsymbol{k}}|<\omega_D$ with $\omega_D$ being the cutoff energy. The mean-field Hamiltonian is given by 
\begin{equation}
H_{\mathrm{MF}}=\sum_{\boldsymbol{k}\sigma}\xi_{\boldsymbol{k}}c_{\boldsymbol{k}\sigma}^{\dagger}c_{\boldsymbol{k}\sigma}+\sum_{\bm{k}}{}^{'}[\bar{\Delta}_0c_{-\bm{k}\downarrow}c_{\bm{k}\uparrow}+\Delta_0 c_{\bm{k}\uparrow}^{\dagger}c_{-\bm{k}\downarrow}^{\dagger}]+\frac{N}{U}\bar{\Delta}_{0}\Delta_{0},
\end{equation}
 where $\Delta_{0}=-\frac{U}{N}\sum_{\boldsymbol{k}L}\langle c_{-\boldsymbol{k}\downarrow}c_{\boldsymbol{k}\uparrow}\rangle_{\mathrm{R}}$ and $\bar{\Delta}_0=-\frac{U}{N}\sum_{\boldsymbol{k}L}\langle c^{\dagger}_{\boldsymbol{k}\uparrow}c^{\dagger}_{-\boldsymbol{k}\downarrow}\rangle_{\mathrm{R}}$ represent the superconducting gap. The right and left ground states of the mean-filed Hamiltonian $H_{\mathrm{MF}}$ are given by \cite{Yamamoto2019}
\begin{align}
	|\text{BCS}\rangle_{R}&=\prod_{\bm{k}}(u_{\bm{k}}+v_{\bm{k}}c_{\boldsymbol{k}\uparrow}^{\dagger}c_{-\boldsymbol{k}\downarrow}^{\dagger})|0\rangle,\\
	|\text{BCS}\rangle_{L}&=\prod_{\bm{k}}(u^{*}_{\bm{k}}+\bar{v}^{*}_{\bm{k}}c_{\boldsymbol{k}\uparrow}^{\dagger}c_{-\boldsymbol{k}\downarrow}^{\dagger})|0\rangle,
\end{align}
where the parameters $u_{\bm{k}},v_{\bm{k}}$ and $\bar{v}_{\bm{k}}$ are complex coefficients and take the specific form of
\begin{equation}
	u_{\bm{k}}=\sqrt{\frac{E_{\bm{k}}+\xi_{\bm{k}}}{2E_{\bm{k}}}},\quad v_{\bm{k}}=-\sqrt{\frac{E_{\bm{k}}-\xi_{\bm{k}}}{2E_{\bm{k}}}}\sqrt{\frac{\Delta_0}{\bar{\Delta}_0}},\quad
	\bar{v}_{\bm{k}}=-\sqrt{\frac{E_{\bm{k}}-\xi_{\bm{k}}}{2E_{\bm{k}}}}\sqrt{\frac{\bar{\Delta}_0}{\Delta_0}}.
\end{equation}
%where $E_{\bm{k}}=\sqrt{\xi_{\bm{k}}^2+\Delta_0^2}$.
Here $E_{\bm{k}}=\sqrt{\xi_{\bm{k}}^2+\Delta_{\bm{k}}^2}$ is the dispersion relation of Bogoliubov quasiparticles where $\Delta_{\bm{k}}=\Delta_0\theta(\omega_D-|\xi_{\bm{k}}|)$ with $\theta(x)$ being the Heaviside step function. Note that $\bar{\Delta}_0\neq\Delta_0^*$. In the following we take a gauge  \cite{Yamamoto2019} in which $\Delta_0=\bar{\Delta}_0\in\mathbb{C}$.

The gap equation at absolute zero reads as \cite{Yamamoto2019}
\begin{equation}
  \frac{N}{U}=\sum_{\bm{k}}{}^{'}\frac{1}{2\sqrt{\xi_{\bm{k}}^2+\Delta_0^2}}.
\end{equation}
Provided that the density of states is constant and given by $\rho_0$, the above equation can be simplified as
\begin{equation}
  \frac{\sqrt{\omega_D^2 + \Delta_0^2} + \omega_D}{\Delta_0} =
  e^{\frac{1}{\rho_0 U}}\,.
\end{equation}
The solution to the gap equation is given by $\Delta_0 =
\frac{\omega_D}{\text{sinh} \left( \frac{1}{\rho_0 U} \right)}$. To be specific,
\begin{eqnarray}
  \Delta_0 & = & \frac{2 \omega_D}{\text{exp} \left[ \frac{1}{\rho_0 | U |^2}
  \left( U_R - i U_I \right) \right] - \text{exp} \left[ -
  \frac{1}{\rho_0 | U |^2} \left( U_R - iU_I \right) \right]}
  \nonumber\\
  & = & \frac{\omega_D}{\text{sinh} \left( \frac{U_R}{\rho_0 | U |^2} \right)
  \cos \left( \frac{U_I}{\rho_0 | U |^2} \right) - i \text{cosh}
  \left( \frac{U_R}{\rho_0 | U |^2} \right) \sin \left( \frac{U_I}{\rho_0 | U |^2} \right)}, 
  \label{Delta_0}
\end{eqnarray}
and its real part is given by
\begin{equation}
  \text{Re} [\Delta_0] = \omega_D \frac{\text{sinh} \left( \frac{U_R}{\rho_0 |
  U |^2} \right) \cos \left( \frac{U_I}{ \rho_0 | U |^2}
  \right)}{\left( \text{sinh} \left( \frac{U_R}{\rho_0 | U |^2} \right)
  \cos \left( \frac{U_I}{ \rho_0 | U |^2} \right) \right)^2 + \left(
  \text{cosh} \left( \frac{U_R}{\rho_0 | U |^2} \right) \sin \left(
  \frac{U_I}{ \rho_0 | U |^2} \right) \right)^2}\,.
    \label{RDelta_0}
\end{equation}
At the quantum phase transition point, the real part of the gap vanishes, which gives $\cos \left( \frac{U_I}{ \rho_0 | U |^2} \right) = 0$, or equivalently, $\frac{U_I}{ \rho_0 | U |^2} = \frac{\pi}{2}$. This determines the condition for the phase boundary \cite{Yamamoto2019}
\begin{equation}
  (\rho_0 \pi U_R)^2 + (\rho_0 \pi U_I- 1)^2 = 1
  \label{phase_transition}\,.
\end{equation}
This condition restricts the imaginary part of the gap $\Delta_0$ as
\begin{eqnarray}
  \text{Im} [\Delta_0] & = & \omega_D \frac{\text{cosh} \left(
  \frac{U_R}{\rho_0 | U |^2} \right) \sin \left( \frac{U_I}{ \rho_0
  | U |^2} \right)}{\left( \text{sinh} \left( \frac{U_R}{\rho_0 | U |^2}
  \right) \cos \left( \frac{U_I}{ \rho_0 | U |^2} \right) \right)^2
  + \left( \text{cosh} \left( \frac{U_R}{\rho_0 | U |^2} \right) \sin
  \left( \frac{U_I}{ \rho_0 | U |^2} \right) \right)^2} \nonumber\\
  & = & \frac{\omega_D}{\text{cosh} \left( \frac{U_R}{\rho_0 | U |^2}
  \right)}\,.
  \label{IDelta_0}
\end{eqnarray}

The phase transition is related to the existence of Yang-Lee zeros on the phase boundary. The partition function of Bogoliubov quasi-particles in a finite-size system at finite temperature $1/\beta$ is
\begin{equation}
  Z = \prod_{\bm{k},\sigma} (1 + e^{-\beta E_{\bm{k}}})\,.
\end{equation}
 Since we only consider the case with a large $\beta$, we directly substitute Eq. (\ref{Delta_0}) into the dispersion relation \footnote{To be precise, the gap equation at finite temperature is singular on the critical line since the Yang-Lee zeros make the expectation value ill-defined. However, we can define the value of the partition function on the critical line from the continuity of the partition function in a finite-size system.}. For the points not on the phase boundary, we have $\text{Re}
 [E_{\bm{k}}] > 0$, which indicates that the
 partition function cannot vanish. However, on the critical line (\ref{phase_transition}), the gap $\Delta_0$ becomes purely imaginary and therefore Yang-Lee zeros can emerge. The partition function on the phase boundary can be decomposed as
\begin{equation}
  Z = \prod_{\bm{k},|\xi_{\bm{k}}|<\text{Im}\Delta_0,\sigma} \left( 1 + e^{- i \beta \sqrt{\left( \text{Im}\Delta_0 \right)^2 - \xi_{\bm{k}}^2}} \right)\times \prod_{\bm{k},|\xi_{\bm{k}}|>\text{Im}\Delta_0,\simga}\left( 1 + e^{- \beta \sqrt{\xi_{\bm{k}}^2 +\Delta_{\bm{k}}^2 }} \right).
\end{equation}
The first product vanishes for the momentum $\bm{k}$ that satisfies the condition
\begin{equation}
  \beta \sqrt{\left( \text{Im} \Delta_0 \right)^2 - \xi_{\bm{k}}^2} = (2n + 1) \pi,
\end{equation}
where $n$ is an arbitrary integer. This condition is equivalent to
\begin{equation}
 |\xi_{\bm{k}}| = \sqrt{\left( \text{Im} \Delta_0
  \right)^2 - \left( \frac{2 n + 1}{\beta} \pi \right)^2}.
  \label{condition}
\end{equation}
 Further, we take the thermodynamic limit. Since $\text{Im}\Delta_0<\omega_D$, we can always find the momentum $\bm{k}$ in the energy shell satisfying the condition (\ref{condition}) for an arbitrarily large $\beta$. Hence, Yang-Lee zeros distribute on the phase boundary (\ref{phase_transition}). 

 \section{Renormalization Group Theory of Non-Hermitian BCS Superconductivity}\label{RG}

Here we consider the renormalization-group (RG) flow of the interaction strength to elucidate that the Yang-Lee singularity corresponds to the RG critical line. The one-loop beta function $\beta_1(U)$ is given at the order of $U^2$ by \cite{Shankar1994,Nagaosa}
\begin{equation}
  \frac{dV}{dt}=V^{2}=:\beta_1(U)\,,
  \label{one-loop}
\end{equation}
where $dt=-\frac{d\Xi}{\Xi}$ is the relative width of the high-energy
shell, $\Xi$ is the cutoff of the energy $\xi_{\bm{k}}$ and $V=\rho_0 U$ is the dimensionless interaction strength. Here $t$ is considered as the RG-flow parameter. We take the two-loop correction into account and consider the terms of the order of $U^3$. After the two-loop calculation, we will see that the RG equation reproduces the phase boundary shown in Fig. 1 in the main text.

Up to just one integral over the momenta, the higher-order contribution
to the beta function comes from the correction for the high-momentum propagator.
Actually, the self-energy for the high-momentum
propagator shown in Fig. \ref{feynman_diagram1} is given by

\begin{equation}
  \Sigma(\bm{k},\Omega)=\int\frac{d\omega}{2\pi}\frac{U}{i\omega-\xi_{\bm{k}}}e^{i\omega0^+}=U\theta(-\xi_{\bm{k}}),
\end{equation}
where $\Omega$ is the frequency for the external leg. Here we introduce a factor $e^{i\omega0^+}$ to ensure the convergence \cite{Shankar1994}. Therefore, the propagator is modified as

\begin{equation}
  G(\bm{k},\Omega)=\frac{1}{i\Omega-\xi_{\bm{k}}}+\frac{U\theta(-\xi_{\bm{k}})}{(i\Omega-\xi_{\bm{k}})^{2}}\,.
\end{equation}
By including the self-energy diagram in Fig. \ref{feynman_diagram1}, we can find the corrected contribution from the BCS diagram. After integrating out the energy shell $(-\Xi,-\Xi+d\Xi)$ of $\xi_{\bm{k}}$, we obtain the two-loop correction $\beta_2(U)$ to the beta function as
\begin{align}
 \beta_2(U)&= \frac{1}{2}\rho_0\Xi U^{2}\left(\int\frac{d\Omega}{2\pi}\frac{U}{(i\Omega-\Xi)^{2}}\frac{1}{-i\Omega-\Xi}+\int\frac{d\Omega}{2\pi}\frac{1}{i\Omega-\Xi}\frac{U}{(i\Omega+\Xi)^{2}}\right)\\
 & =-\frac{\rho_0^2U^3}{2}\,,
 \label{two-loop}
\end{align}
where we define $\rho_0=1/(2\Xi)$ since $\frac{1}{N}\sum_{\bm{k}}=1=\int\rho_0d\xi_{\bm{k}}$ is satisfied \cite{Yamamoto2019}. %Since we consider the electrons in the energy shell $(-\Xi,\Xi)$, the number $N$ also represents the number of electrons in this shell. 
Hence, the RG equation up to two-loop order is written as
\begin{equation}
\frac{dV}{dt}=V^{2}-\frac{1}{2}V^{3}\,.\label{eq:RG_Flow_Equ_Total}
\end{equation}
The RG flow diagram for Eq. (\ref{eq:RG_Flow_Equ_Total}) is shown in Fig. \ref{RG_Flow_Diagram}(c).
\begin{figure}
	\includegraphics[width=0.4\columnwidth]{Feynman diagram.pdf}
	\caption{Feynman diagrams for the self-energy correction and the renormalization of the interaction strength. The right diagram is the BCS diagram.}
	
	\label{feynman_diagram1}
\end{figure}

\begin{figure}
	\includegraphics[width=1\columnwidth]{phase_diagram_comparison.pdf}
	\caption{Canonical RG flow for coupling strength $V$ for general Hermitian and non-Hermitian Hamiltonians with the RG flow equation $\frac{dV}{dt}=aV^2+bV^3$. In the diagram (a) and (c), the parameters are set to be $a=1$ and $b=-1/2$. In the diagram (b) and (d), the parameters are set to be $a=-1$ and $b=1/2$. For $b>0$, there is neither a nontrivial stable fixed point nor a critical line.}
	
	\label{RG_Flow_Diagram}
\end{figure}
As can be seen from Eq. (\ref{eq:RG_Flow_Equ_Total}), the RG flow has a nontrivial fixed point at $V=2$ and a critical line depicted as the blue curve in Fig. \ref{RG_Flow_Diagram}. This critical line separates the whole space into two phases, with one flowing to the origin and the other flowing to the nontrivial fixed point. The analytical expression for the critical line can be derived as follows \cite{Nakagawa2018}. We rewrite the RG equation (\ref{eq:RG_Flow_Equ_Total}) as

\begin{align}
		\frac{dV_{R}}{dt} & =V_{R}^{2}-V_{I}^{2}-\frac{1}{2}V_{R}^{3}+\frac{3}{2}V_{R}V_{I}^{2},\\
		\frac{dV_{I}}{dt} & =2V_{R}V_{I}-\frac{3}{2}V_{I}V_{R}^{2}+\frac{1}{2}V_{I}^{3},
\end{align}
 where $V_{R}=\mathrm{Re}(V),V_{I}=\mathrm{Im}(V)$. On the critical line, the interaction parameter flows towards $V_R=\frac{2}{3},V_I=\infty$, which can be derived from the condition $\frac{dV_R}{dt}=0$ with $V_I\to\infty$.  The specific expression of the critical line can be obtained through integration of Eq. (\ref{eq:RG_Flow_Equ_Total}) as

\begin{equation}
t=-\frac{1}{V^{\mathrm{f}}}+\frac{1}{2}\ln V^{\mathrm{f}}-\frac{1}{2}\ln(2-V^{\mathrm{f}})+\frac{1}{V}-\frac{1}{2}\ln V+\frac{1}{2}\ln(2-V)\label{eq:Relation_between_t_and_U}\,,
\end{equation}
where the superscript f denotes the final value of the interaction parameter. Since $V_{I}^{\mathrm{f}}\to\infty$ and $V_{R}^{\mathrm{f}}\to\frac{2}{3}$ on the critical line,
the imaginary part of Eq. (\ref{eq:Relation_between_t_and_U}) reads as 
\begin{equation}
0=\frac{\pi}{2}-\frac{V_{I}}{V_{R}^{2}+V_{I}^{2}}-\frac{1}{2}\arctan\frac{V_{I}}{V_{R}}-\frac{1}{2}\arctan\frac{V_{I}}{2-V_{R}}\,.
\label{eq:Phase_boundary_by_RG}
\end{equation}
This is the equation defining the phase boundary. Around the origin, the expression of this critical line (\ref{eq:Phase_boundary_by_RG}) can be expanded as
\begin{equation}
0=-\frac{V_{I}}{V_{R}^{2}+V_{I}^{2}}+\frac{\pi}{2}\,,
\end{equation}
which is consistent with the phase boundary (\ref{phase_transition}) obtained from the mean-field theory. The RG result confirms the validity of the mean-field analysis.

By taking $V^{\mathrm{f}}$ to be pure imaginary, we obtain the energy scale $T_{\mathrm{recur}}$ that characterizes the reversion of the RG flow \cite{Nakagawa2018}:
\begin{equation}
	 t=\frac{i}{V_{I}^{\mathrm{f}}}+\frac{1}{2}\ln V_{I}^{\mathrm{f}}+i\frac{\pi}{4}-\frac{1}{4}\ln(4+(V_{I}^{\mathrm{f}})^{2})-\frac{i}{2}(2\pi-\arctan\frac{V_{I}^{\mathrm{f}}}{2})+\frac{1}{V}-\frac{1}{2}\ln V+\frac{1}{2}\ln(2-V)\,.
	\label{Delta t}
\end{equation}
If we assume $|V|\ll1$, the above equation (\ref{Delta t}) can be rewritten as
\begin{align}
		0 & =\frac{1}{V_{I}^{\mathrm{f}}}+\frac{\pi}{4}+\frac{1}{2}\arctan\frac{V_{I}^{\mathrm{f}}}{2}-\frac{V_{I}}{V_{R}^{2}+V_{I}^{2}}-\frac{1}{2}\arctan\frac{V_{I}}{V_{R}}-\frac{1}{2}\arctan(\frac{V_{I}}{2-V_{R}}),\\
		e^{-t} & =\sqrt{\frac{\sqrt{4+(V_{I}^{\mathrm{f}})^{2}}}{V_{I}^{\mathrm{f}}}}\exp(-\frac{V_{R}}{V_{R}^{2}+V_{I}^{2}})(\frac{|V|}{|2-V|})^{\frac{1}{2}}.
\end{align}
From the second equation, we have the reversion temperature $T_\mathrm{recur}$ at which the RG flow reaches a point $(0,V_I^f)$ on the imaginary axis:
\begin{equation}
  T_{\mathrm{recur}}\sim e^{-t}=\sqrt{\frac{\sqrt{4+(V_{I}^{\mathrm{f}})^{2}}}{V_{I}^{\mathrm{f}}}}\exp(-\frac{V_{R}}{V_{R}^{2}+V_{I}^{2}})(\frac{|V|}{|2-V|})^{\frac{1}{2}}\,.
\end{equation}
Near the phase boundary in which $V_I^f\gg1$, we can obtain the simplified expression for the reversion temperature as
\begin{equation}
  T_{\mathrm{recur}}\sim e^{-t}=\exp(-\frac{V_{R}}{V_{R}^{2}+V_{I}^{2}})(\frac{|V|}{|2-V|})^{\frac{1}{2}}\,.
\end{equation}


Finally, we consider a general canonical RG equation for a marginal interaction $V$ up to the order of $V^3$:
\begin{equation}
	\frac{dV}{dt}=aV^2+bV^3.
	\label{eq:RG_Flow_Equ_Marginal_canonical}
\end{equation}
In Fig. \ref{RG_Flow_Diagram}, we show both the Hermitian case and the non-Hermitian case. We find that there are only two types of RG flows for a marginal interaction. The first type with $b<0$ is shown in Fig. \ref{RG_Flow_Diagram}(a,c). In the Hermitian case, it has a nontrivial stable fixed point, whereas in the non-Hermitian case, it has a critical line. The second type with $b>0$ is shown in Fig. \ref{RG_Flow_Diagram}(b,d). In the Hermitian case, it has a non-trivial unstable fixed point, whereas in the non-Hermitian case, it shows no critical line. Hence, we can see  that only the first type of RG flows has a phase boundary and a phase transition.

We can also derive the critical line for the general canonical RG flow with arbitrary $a$ and $b<0$. Integrating Eq. (\ref{eq:RG_Flow_Equ_Marginal_canonical}) with respect to $t$, we obtain
\begin{equation}
	t=-\frac{1}{a V^{\mathrm{f}}}-\frac{b}{a^2}\ln{V^\mathrm{f}}+\frac{b}{a^2}\ln{(a+b V^\mathrm{f})}+\frac{1}{a V}+\frac{b}{a^2}\ln{V}-\frac{b}{a^2}\ln{(a+b V)}.
\label{eq:general_formula_final}
\end{equation}
On the critical line for $V_I\to\infty$ and $V_R\to-\frac{a}{3b}$, the imaginary part of Eq. (\ref{eq:general_formula_final}) reads as 
\begin{equation}
	0=-\frac{b\pi}{a|a|}-\frac{1}{a}\frac{V_I}{V_R^2+V_I^2}+\frac{b}{a^2}\arctan{\frac{V_I}{V_R}}+\frac{b}{a^2}\arctan{(-\frac{b V_I}{a+b V_R})}.
\label{eq:imaginary_part_boundary}
\end{equation}
Near the origin, this critical line can be expanded as 
\begin{equation}
0=-\frac{V_I}{V_R^2+V_I^2}-\frac{b\pi}{|a|}\,,%\:
%\begin{cases}
%	V_R>0, &\mathrm{if}\:a>0;\\
%	V_R<0, &\mathrm{if}\:a<0,
%\end{cases}
\end{equation}
where $V_R>0$ if $a>0$ and $V_R<0$ if $a<0$, which are both a semicircle.

\section{Correlation Functions on the Phase Boundary}\label{SectionOn}

In this section, we calculate the correlation functions on the phase boundary to elucidate the critical behavior at the Yang-Lee singularity. Here we firstly consider the momentum distribution of the particles
\begin{equation}
  {}_L\langle c_{\bm{k} \sigma}^{\dagger} c_{\bm{k} \sigma} \rangle_R = {}_L \left\langle
  \text{BCS} \right| c_{\bm{k} \sigma}^{\dagger} c_{\bm{k} \sigma} \left| \text{BCS}\right\rangle_R \,.
\end{equation}
Using the expression of the BCS states, we obtain
\begin{equation}
  {}_L\langle c_{\bm{k} \uparrow}^{\dagger} c_{\bm{k} \uparrow} \rangle_R = {}_L\langle c_{\bm{k}\downarrow}^{\dagger} c_{\bm{k} \downarrow} \rangle_R = v_{\bm{k}}^2 = \frac{1}{2} -
  \frac{\xi_{\bm{k}}}{2 E_{\bm{k}}} = \frac{1}{2} - \frac{\xi_{\bm{k}}}{2 \sqrt{\xi_{\bm{k}}^2 +
  \Delta_{\bm{k}}^2}}\,.
\end{equation}
Similarly, we have 
\begin{equation}
  {}_L\langle c_{\bm{k} \uparrow}^{\dagger} c_{\bm{k} \downarrow} \rangle_R = {}_L\langle c_{\bm{k}\downarrow}^{\dagger} c_{\bm{k} \uparrow} \rangle_R = 0\,.
\end{equation}
Then we perform the Fourier transformation to 
\begin{equation}
  C (\bm{x}-\bm{x}') := {}_L\langle c_{\sigma}^{\dagger} (\bm{x}) c_{\sigma} (\bm{x}') \rangle_R = \int \frac{d^3 \bm{k}}{(2 \pi)^3} \left(\frac{1}{2} - \frac{\xi_{\bm{k}}}{2 \sqrt{\xi_{\bm{k}}^2 + \Delta_{\bm{k}}^2}} \right)e^{i\bm{k} \cdot (\bm{x}-\bm{x}')}\,.
\label{eq:define_CF}
\end{equation}
We drop the first term on the right-hand side of Eq. (\ref{eq:define_CF}) since it is proportional to the delta function $\delta(\bm{x})$. Here, we replace the integral with $\int ' \frac{d^3 \bm{k}}{(2 \pi)^3}$ for the momentum with $|\xi_{\bm{k}}|<\omega_D$ since we are only concerned with the long-range behavior of the correlation function. 
In the following, we will replace $\bm{x}-\bm{x}'$
with $\bm{x}$ for convenience. Then the correlation
function is given by
\begin{equation}
  C (\bm{x}) \simeq -\int ' \frac{d^3 \bm{k}}{(2 \pi)^3} \frac{\xi_{\bm{k}}}{2 \sqrt{\xi_{\bm{k}}^2
  + \Delta_0^2}} e^{i\bm{k} \cdot \bm{x}}.
\end{equation}
 To extract the long-range behavior of the correlation function, we expand the energy spectrum around the Fermi surface as $\xi_{\bm{k}} \simeq v_F (k - k_F)$, where $v_F$ and $k_F$ are the Fermi velocity and the Fermi momentum, respectively. Then the integration can be simplified as
\begin{equation}
  C(\bm{x})\simeq-\int ' \frac{\rho_0}{2} d \xi_{\bm{k}} \sin \theta d \theta \frac{\xi_{\bm{k}}}{2
  \sqrt{\xi_{\bm{k}}^2 + \Delta_0^2}} e^{i \frac{\xi_{\bm{k}}}{v_F} x \cos \theta}
  e^{i k_F x \cos \theta}\,,
\end{equation}
where $x=|\bm{x}|$ and $\rho_0$ is the density of states at the Fermi surface. With the
integration over $\theta$, we have
\begin{align}
  C(\bm{x})&\simeq-\frac{1}{2} \int_{- \omega_D}^{\omega_D} \rho_0 d \xi_{\bm{k}} \frac{\xi_{\bm{k}}
  \sin \left( \left( \frac{\xi_{\bm{k}}}{v_F} + k_F \right) x \right)}{\left(
  \frac{\xi_{\bm{k}}}{v_F} + k_F \right) x \sqrt{\xi_{\bm{k}}^2 + \Delta_0^2}}\nonumber\\
  & =  -\rho_0 v_F \int_0^{\omega_D} d \xi_{\bm{k}} \frac{\xi_{\bm{k}}}{x \sqrt{\xi_{\bm{k}}^2 +
	\Delta_0^2}} \left[ (- \xi_{\bm{k}}) \frac{\sin (k_F x) \cos \left(
	\frac{\xi_{\bm{k}}}{v_F} x \right)}{(v_F k_F)^2 - \xi_{\bm{k}}^2} + v_F k_F
	\frac{\cos (k_F x) \sin \left( \frac{\xi_{\bm{k}}}{v_F} x \right)}{(v_F
	k_F)^2 - \xi_{\bm{k}}^2} \right] \nonumber\\
	& \simeq  -\rho_0 v_F \int_0^{\omega_D} d \xi_{\bm{k}} \frac{\xi_{\bm{k}}}{x \sqrt{\xi_{\bm{k}}^2
	+ \Delta_0^2}} v_F k_F \frac{\cos (k_F x) \sin \left(
	\frac{\xi_{\bm{k}}}{v_F} x \right)}{(v_F k_F)^2 - \xi_{\bm{k}}^2}\nonumber\\
	& \simeq -\rho_0 \frac{\cos
		(k_F x)}{k_Fx} \int_0^{\omega_D} d \xi_{\bm{k}} \frac{\xi_{\bm{k}}\sin \left( \frac{\xi_{\bm{k}}}{v_F}
		x \right)}{ \sqrt{\xi_{\bm{k}}^2 +(\text{Re}\Delta_0)^2- (\text{Im}\Delta_0)^2+2i\text{Re}\Delta_0\text{Im}\Delta_0}}\,.
		\label{correlation}
\end{align}
Since we expand the energy around the Fermi surface, we assume the condition $\omega_D\ll \mu=v_F k_F$ and neglect the first term in the bracket in the second equality of (\ref{correlation}). Here we also replace $(v_F k_F)^2 - \xi_{\bm{k}}^2$ with $(v_F k_F)^2$ due to this approximation. By defining
\begin{equation}
	k:=\frac{\omega_D}{\text{Im}\Delta_0}=\text{cosh}(\frac{U_R}{\rho_0|U|^2}),a:=\frac{\text{Re}\Delta_0}{\text{Im}\Delta_0},%=\frac{\pi U_R}{|U|^2}\text{tanh}(\frac{U_R}{\rho_0|U|^2})\delta U_R\,,
	\label{ka}
\end{equation} 
we rewrite the integral as
\begin{equation}
	C(\bm{x})\simeq-\rho_0 \frac{\text{Im}\Delta_0\cos(k_F x)}{k_Fx} \int_0^{k} dt \frac{t\sin ( \frac{\text{Im}\Delta_0x}{v_F}
		t )}{ \sqrt{t^2 +a^2- 1+2ia}}\,,
    \label{eq:Cx_near_boundary}
\end{equation}
where we change the integration variable from $\xi_{\bm{k}}$ to $t=\frac{\xi_{\bm{k}}}{\text{Im}\Delta_0} $. Here we separate the correlation function into the real and imaginary parts as 
\begin{equation}
	C(\bm{x})\simeq-\rho_0 \frac{\cos(k_F x)}{k_Fx}(F_1(\bm{x})+iF_2(\bm{x})),
\end{equation}
where
\begin{equation}
	F_1(\bm{x})=\text{Im}\Delta_0\text{Re}\left[\int_0^{k} dt \frac{t\sin ( \frac{\text{Im}\Delta_0x}{v_F}
	t )}{ \sqrt{t^2 +a^2- 1+2ia}}\right],F_2(\bm{x})=\text{Im}\Delta_0\text{Im}\left[\int_0^{k} dt \frac{t\sin ( \frac{\text{Im}\Delta_0x}{v_F}
	t )}{ \sqrt{t^2 +a^2- 1+2ia}}\right]\label{F_1F_2}.
\end{equation}
 Since this integral is dominated by the region where $t\simeq\pm\sqrt{1-a^2}$, changing the upper bound of the integral will not influence the long-range behavior of the correlation function. Hence, we set $k\to\infty$ here for convenience. We will illustrate this point in the following numerical simulation with a finite upper bound.
 When we consider the correlation function on the phase boundary, we have $a=0$. Hence, the function $F_1(x)$ in the real part of the correlation function takes the form of
\begin{equation}
	F_1(\bm{x})=\text{Im}\Delta_0\text{Re}\left[\int_0^{k} dt \frac{t\sin ( \frac{\text{Im}\Delta_0x}{v_F}
	t )}{ \sqrt{t^2 - 1}}\right]\simeq\text{Im}\Delta_0\int_1^{\infty} dt \frac{t\sin ( \frac{\text{Im}\Delta_0x}{v_F}
	t )}{ \sqrt{t^2 - 1}}.
	\label{definition_of_F1}
\end{equation}
To calculate this function, we introduce another function $G_1(x)$ as
\begin{equation}
	G_1 (x) := \text{Im}
		\Delta_0\int_1^{\infty} dt \frac{\cos ( \frac{\text{Im}\Delta_0x}{v_F}
		t )}{ \sqrt{t^2 - 1}}=-\text{Im}\Delta_0\frac{\pi}{2} N_0 \left( \frac{\text{Im} \Delta_0}{v_F} x \right),
\end{equation}
where $N_0$ is the 0-th order Bessel function of the second kind. The relationship between these two functions is 
\begin{equation}
	F_1 (x) \simeq - \frac{v_F}{\text{Im}\Delta_0} G_1' (x)\,.
\end{equation}
%The function $G_1(x)$ is given by
%\begin{equation}
%	G_1(x)=\text{Im}
%	\Delta_0\int_1^{\infty} dt \frac{\cos ( \frac{\text{Im}\Delta_0x}{v_F}
%	t )}{ \sqrt{t^2 - 1}} = -\text{Im}
%	\Delta_0
%	\frac{\pi}{2} N_0 \left( \frac{\text{Im} \Delta_0}{v_F} x \right)\,.
%\end{equation}
Therefore, the real part of the correlation function is
\begin{equation}
	\mathrm{Re}[C(\bm{x})]\simeq\frac{\pi}{2} \rho_0v_F \frac{\cos (k_F x)}{k_Fx} N_0' \left( \frac{\text{Im}
		\Delta_0}{v_F} x \right)\,.
\end{equation}
When we take the limit $x \rightarrow \infty$, we have
\begin{equation}
	\lim_{x \rightarrow \infty} N_0' (x) = \sqrt{\frac{2}{\pi}} \cos \left( x -
	\frac{\pi}{4} \right) \frac{1}{x^{1 / 2}} - \sqrt{\frac{2}{\pi}} \frac{1}{2}
	\sin \left( x - \frac{\pi}{4} \right) \frac{1}{x^{3 / 2}} \simeq 
	\sqrt{\frac{2}{\pi}} \cos \left( x - \frac{\pi}{4} \right) \frac{1}{x^{1 /
			2}}\,.
	\label{Realexact}
\end{equation}
Thus, the real part of the correlation function has the long-range behavior as
\begin{equation}
	\lim_{x \rightarrow \infty} \left[  \frac{\pi}{2} \rho_0 v_F \frac{\cos (k_F x)}{k_Fx}
	N_0' \left( \frac{\text{Im} \Delta_0}{v_F} x \right) \right] \sim \frac{1}{x^{3 / 2}}.
\end{equation}
Then we turn to consider the function $F_2(x)$ as the imaginary part of the correlation function. On the phase boundary, we can rewrite it as
\begin{equation}
	F_2(\bm{x})=\text{Im}\Delta_0\text{Im}\left[\int_0^{k} dt \frac{t\sin ( \frac{\text{Im}\Delta_0x}{v_F}
	t )}{ \sqrt{t^2 - 1}}\right]=-\text{Im}\Delta_0\int_0^{1} dt \frac{t\sin ( \frac{\text{Im}\Delta_0x}{v_F}
	t )}{ \sqrt{t^2 - 1}}
	\label{definition_of_F2}
\end{equation}
Similarly, we can also define another function $G_2(x)$ as
\begin{equation}
	G_2(x)=-\text{Im}\Delta_0\int_0^{1} dt \frac{\cos ( \frac{\text{Im}\Delta_0x}{v_F}
	t )}{ \sqrt{t^2 - 1}}=\frac{\pi}{2} J_0 \left( \frac{\text{Im}\Delta_0}{v_F} x \right),
\end{equation}
where $J_0$ is the 0-th order Bessel function of the first kind. The relationship between these two functions is also given by
\begin{equation}
	F_2 (x) = - v_F G_2' (x). %G_2 (x) = \frac{\pi}{2} J_0 \left( \frac{\text{Im}
		%\Delta_0}{v_F} x \right)\,.
\end{equation}
Then the imaginary part of the correlation function is equivalent to
\begin{equation}
	\mathrm{Im}[C(\bm{x})]\simeq\frac{\pi}{2} \rho_0 v_F\frac{\cos (k_F x)}{k_Fx} J_0' \left( \frac{\text{Im}
		\Delta_0}{v_F} x \right)\,.
\end{equation}
Thus, we have a similar long-range behavior for the imaginary part of the correlation function as
\begin{equation}
	\lim_{x \rightarrow \infty} \left[  \frac{\pi}{2} \rho_0 v_F \frac{\cos (k_F x)}{k_Fx}
	J_0' \left( \frac{\text{Im} \Delta_0}{v_F} x \right) \right] \sim \frac{1}{x^{3 / 2}}\,.
	\label{Imexact}
\end{equation}
To summarize, the correlation function takes the form of
\begin{align}
  \lim_{x\rightarrow\infty}{}_L \left\langle\text{BCS} \right| c_{\sigma}^{\dagger}(\bm{x}) c_{\sigma}(\bm{x}) \left| \text{BCS}
  \right\rangle_R&\simeq \lim_{x\rightarrow\infty}\frac{\pi}{2} \rho_0 v_F \frac{\cos (k_F x)}{k_Fx}( N_0' ( \frac{\text{Im}\Delta_0}{v_F} x )+iJ_0'( \frac{\text{Im}\Delta_0}{v_F} x ))\nonumber\\
  &=:(A(l)+iB(l))x^{-3/2}\,,
  \label{correlation_decay}
\end{align}
where
\begin{align}
    A(l)&=\sqrt{\frac{\pi}{2}}\rho_0 \frac{\sqrt{\text{Im}\Delta_0 v_F}}{k_F} \cos (k_F \frac{v_F}{\text{Im}\Delta_0}l)\cos(l-\frac{\pi}{4})\,,\nonumber\\
    B(l)&=\sqrt{\frac{\pi}{2}}\rho_0 \frac{\sqrt{\text{Im}\Delta_0 v_F}}{k_F} \cos (k_F \frac{v_F}{\text{Im}\Delta_0}l)\sin(l-\frac{\pi}{4})\,,
\end{align}
and $l= \frac{\text{Im}\Delta_0}{v_F} x$. The anomalous dimension is defined by $C(\bm{x})\propto x^{-D+2-\eta}$ \cite{Sachdev:2011uj}, where $D$ is the spatial dimension of the system. From Eq. (\ref{correlation_decay}), we can see that the correlation length diverges on the phase boundary and that the anomalous dimension is given by $\eta=1/2$. In addition, we present the numerical plot of the integrals $F_1$ and $F_2$ on the phase boundary in Fig. \ref{fig2} with detailed fitting parameters shown in Table. \ref{fitting_table_2}. In the numerical calculation we take a finite upper bound $k$ given in Eq. (\ref{ka}). These two functions are related to the correlation function as $C(x)=\rho_0 \frac{\cos (k_F x)}{k_Fx}(F_1(x)+iF_2(x))$. %In the following we also call $(F_1(x)+iF_2(x))$ as reduced correlation function.
The fitting results indicate that $F_1(l)$ is proportional to $\sin(l+\frac{\pi}{4})l^{-0.5}$ and $F_2(l)$ is proportional to $\sin(l+\frac{3\pi}{4})l^{-0.5}$. Those results are consistent with our analytical result in Eq. (\ref{correlation_decay}) and indicate that the correlation length diverges on the phase boundary.

\section{Correlation Function Near the Phase Transition}

To calculate the critical exponent of correlation length, we consider the correlation functions near the phase boundary.  The real part of the gap $\Delta_0$ is given by
\begin{equation}
  \text{Re} [\Delta_0] = \omega_D \frac{\text{sinh} \left( \frac{U_R}{\rho_0 |
  U |^2} \right) \cos \left( \frac{U_I}{ \rho_0 | U |^2}
  \right)}{\left( \text{sinh} \left( \frac{U_R}{\rho_0 | U |^2} \right)
  \cos \left( \frac{U_I}{ \rho_0 | U |^2} \right) \right)^2 + \left(
  \text{cosh} \left( \frac{U_R}{\rho_0 | U |^2} \right) \sin \left(
  \frac{U_I}{ \rho_0 | U |^2} \right) \right)^2}\,.\label{Real_gap}
\end{equation}
For convenience, we here consider the shift of the interaction strength by $\delta U_R\in\mathbb{R}$ from a point $U$ on the phase boundary. It can be replaced by an arbitrary amount $\delta U$ along any direction. Up to the first order of $\delta U_R$, we have $\cos \left(
\frac{U_I}{ \rho_0 | U |^2} \right) = \cos \left( \frac{\pi}{2}-\frac{\pi U_R\delta U_R}{|U|^2} \right)=\sin \left(\frac{\pi U_R\delta U_R}{|U|^2} \right) \simeq \frac{\pi U_R\delta U_R}{|U|^2}$. Then the real part is shown to be proportional to $\delta U_R$:
\begin{equation}
  \text{Re} \Delta_0 \simeq \frac{\pi\omega_D U_R}{| U |^2} \frac{\text{sinh}
  \left( \frac{U_R}{\rho_0 | U |^2} \right)}{\text{cosh}^2 \left(
  \frac{U_R}{\rho_0 | U |^2} \right)} \delta U_R \propto \delta U_R\,.
  \label{define_UR_deviation}
\end{equation}
The imaginary part of the gap remains the same as in Eq. (\ref{IDelta_0}) up to the same order of $\delta U_R$: $\text{Im} \Delta_0 =
\frac{\omega_D}{\text{cosh($\frac{U_R}{\rho_0 | U |^2}$)}}$ and   
\begin{equation}
 a=\frac{\pi U_R}{|U|^2}\text{tanh}(\frac{U_R}{\rho_0|U|^2})\delta U_R
 \label{detailed_expression_a}
\end{equation}
from Eq. (\ref{ka}). Then we turn to the correlation function near the phase boundary.

\begin{figure}
	\centering \includegraphics[width=0.6\columnwidth]{figs/fitting_k_2_509_on_the_boundary_total}
	\caption{(a) Real part $F_1(l)$ and (b) imaginary part $F_2(l)$ of the correlation function on the phase boundary which are defined in Eqs. (\ref{definition_of_F1}) and (\ref{definition_of_F2}). Here $(\rho_{0}U_{R},\rho_{0}U_{I})=(\frac{1}{\pi},\frac{1}{\pi})$ and $l=\frac{\text{Im}\Delta_0}{v_F}x$. The fitting parameters are shown in Table \ref{fitting_table_2}.}
	\label{fig2}
\end{figure}

\begin{table}
	\begin{tabular}{|c|c|}
		\hline 
		\multicolumn{2}{|c|}{$F_{1}(l)$}\tabularnewline
		\hline
		\hline 
		Fitting Function & $F_1(l)=\frac{a_{1}\sin(l+a_{2})}{l^{a_{3}}}$\tabularnewline
		\hline 
		$a_{1}$ & $1.246(1.129,1.363)$\tabularnewline
		\hline 
		$a_{2}$ & $0.786(0.783,0.789)$\tabularnewline
		\hline 
		$a_{3}$ & $0.499(0.485,0.513)$\tabularnewline
		\hline 
		$R^{2}$ & $0.9990$\tabularnewline
		\hline 
	\end{tabular}%
	\begin{tabular}{|c|c|}
		\hline 
		\multicolumn{2}{|c|}{$F_{2}(l)$}\tabularnewline
		\hline 
		\hline
		Fitting Function & $F_{2}(l)=\frac{a_{1}\sin(l+a_{2})}{l^{a_{3}}}$\tabularnewline
		\hline 
		$a_{1}$ & $-1.199(-1.190,-1.209)$\tabularnewline
		\hline 
		$a_{2}$ & $-0.7884(-0.7883,-0.7884)$\tabularnewline
		\hline 
		$a_{3}$ & $0.4924(0.4912,0.4936)$\tabularnewline
		\hline 
		$R^{2}$ & $0.9999$\tabularnewline
		\hline 
	\end{tabular}
	
	\caption{The left and right tables show fitting parameters for $F_1(l)$ and $F_2(l)$ on the boundary, respectively. Here $(\rho_{0}U_{R},\rho_{0}U_{I})=(\frac{1}{\pi},\frac{1}{\pi})$. The values in the parentheses show the range of error bars. The parameter $R^2$ represents the confidence of the fitting, which is defined as the ratio of the sum of squares of the regression (SSR) and the total sum of squares (SST).}
	
	\label{fitting_table_2}
\end{table}
\begin{figure}[t]
	\centering \includegraphics[width=0.6\columnwidth]{figs/fitting_a_4_526_1E-4_total}
	\caption{(a) Real part $F_1(l)$ and (b) imaginary part $F_2(l)$ of the correlation function near the boundary (see Eq. (\ref{F_1F_2})). These two figures are near the phase boundary with $(\rho_{0}U_{R},\rho_{0}U_{I})=(\frac{1}{\pi},\frac{1}{\pi})$ with $\rho_{0}\delta U_{R}=1\times10^{-4}\ll1$ and $l=\frac{\text{Im}\Delta_0x}{v_F}.$ The fitting parameters are shown in Table \ref{Fitting_table_1}.}
	\label{fig1}
\end{figure}
\begin{table}
	\begin{tabular}{|c|c|}
		\hline 
		\multicolumn{2}{|c|}{$F_{1}(l)$}\tabularnewline		
		\hline
		\hline 
		Fitting Function & $F_1(l)=a_{1}\sin(a_{2}l-a_{3})e^{-\frac{l}{\xi_{r}}}$\tabularnewline
		\hline 
		$a_{1}$ & $0.1287(0.1153,0.1420)$\tabularnewline
		\hline 
		$a_{2}$ & $1(0.9996,1)$\tabularnewline
		\hline 
		$a_{3}$ & $-0.7843(-0.6798,-0.8887)$\tabularnewline
		\hline 
		$\xi_{r}$ & $420.9(349.6,492.3)$\tabularnewline
		\hline 
		$R^{2}$ & $0.9962$\tabularnewline
		\hline 
	\end{tabular}%
	\begin{tabular}{|c|c|}
		\hline 
		\multicolumn{2}{|c|}{$F_{2}(l)$}\tabularnewline
		\hline
		\hline 
		Fitting Function & $F_2(l)=a_{1}\sin(a_{2}l-a_{3})e^{-\frac{l}{\xi_i}}$\tabularnewline
		\hline 
		$a_{1}$ & $-0.1292(-0.1285,-0.1299)$\tabularnewline
		\hline 
		$a_{2}$ & $1(1,1)$\tabularnewline
		\hline 
		$a_{3}$ & $0.7810(0.7756,0.7864)$\tabularnewline
		\hline 
		$\xi_{i}$ & $417.6(414,421.3)$\tabularnewline
		\hline 
		$R^{2}$ & $0.9937$\tabularnewline
		\hline 
	\end{tabular}
	
	\caption{The left and right tables show fitting parameters for $F_1(l)$ and $F_2(l)$ near the boundary, respectively. Here $(\rho_{0}U_{R},\rho_{0}U_{I})=(\frac{1}{\pi},\frac{1}{\pi})$ with $\rho_{0}\delta U_{R}=1\times10^{-4}\ll1$. The values in the parentheses are the corresponding error bars. The parameter $R^2$ represents the confidence of the fitting, which is defined as the ratio of the sum of squares of the regression (SSR) and the total sum of squares (SST).}
	\label{Fitting_table_1}
\end{table}
The long-range behavior of the correlation functions is governed by the properties of the integral in Eq. (\ref{eq:Cx_near_boundary}). In Fig. \ref{fig1}, we numerically plot $F_1$ and $F_2$ with $(\rho_{0}U_{R},\rho_{0}U_{I})=(\frac{1}{\pi},\frac{1}{\pi})$ and $\rho_{0}\delta U_{R}=1\times10^{-4}\ll1$. $F_1$ is fitted by an oscillating exponential decay  which is shown by the blue curve
in Fig. \ref{fig1}(a): $F_1(l)\propto\sin(l+\frac{\pi}{4})e^{-l/\xi_{r}}$,
where $\xi_{r}=420.9$. $F_2$ behaves similarly as
shown by the blue curve in Fig. \ref{fig1}(b): $F_2(l) \propto\sin(x+\frac{3\pi}{4})e^{-x/\xi_{i}}$
with $\xi_{i}=417.6$. Here $l=\frac{\text{Im}\Delta_0}{v_F}x$ is the same as in the previous section. We can see the behaviors of the real and the
imaginary parts of the correlation function are very close
to each other. The detailed fitting parameters are shown in Table \ref{Fitting_table_1}.



Furthermore, we numerically calculate the dependence of the correlation lengths on the deviation
$\rho_{0}\delta U_{R}$ from the phase boundary. The correlation lengths of the real part and the imaginary
part are shown separately in Fig. \ref{fig3}. We find that both of the correlation lengths are inversely proportional to the deviation from the phase boundary, i.e. $\xi^{-1}\propto a$. To derive the dependence analytically, we consider the integral $\int_0^{\infty}t\frac{\sin(tx)}{\sqrt{t^2+m^2}}$ with $m\in\mathbb{C}$ and take the limit of $k\to\infty$ in Eq. (\ref{F_1F_2}) since this limiting procedure will not change the long-range behavior. We have
\begin{equation}
   \int_0^{\infty}\frac{t\sin(tx)}{\sqrt{t^2+m^2}}=-\frac{d}{dx}\int_0^{\infty}\frac{\cos(tx)}{\sqrt{t^2+m^2}}=-K_0'(x)\,,
\end{equation}
where $K_{\nu}(x)$ is the $\nu$-th order modified Bessel function of the second kind. By substituting the expression for this special function, we have for $x\rightarrow\infty$
\begin{equation}
\int_{0}^{\infty}dt\frac{t\sin(xt)}{\sqrt{t^{2}+m^{2}}}\propto \frac{e^{-mx}}{\sqrt{mx}}= \frac{\text{exp}[-\text{Re}(m)x-i\text{Im}(m)x]}{\sqrt{mx}}\,.
\end{equation}
From the definition $C(\bm{x})\propto e^{-x/\xi}$ of the correlation length $\xi$ \cite{Sachdev:2011uj}, we obtain
\begin{equation}
    \xi^{-1}\sim\text{Re}(m)\,.
\end{equation}
From Eq. (\ref{eq:Cx_near_boundary}), the parameter $m$ is given by $m=a+i$. Hence, the relationship between $\xi$ and $a$ is given by
\begin{equation}
  \xi\propto a^{-1},\label{corre_length}
\end{equation}
which agrees with our numerical simulation results in Fig. \ref{fig3}. From the expression of $a$ in Eq. (\ref{detailed_expression_a}), we can see that the correlation length is inversely proportional to the deviation from the phase boundary $\xi^{-1}\propto\delta U_R$, which indicates the critical exponent $\nu=1$ from the definition $\xi\propto(\delta U)^{-\nu}$ \cite{Sachdev:2011uj}. However, on the real axis of the interaction strength $U$, this analysis fails because $\text{Im}\Delta_0=0$ on the whole real axis. From Eq. (\ref{correlation}),  the correlation function for $U_I=0$ is proportional to
\begin{figure}
	\centering \includegraphics[width=0.6\columnwidth]{figs/fitting_correlation_length_total}
	\caption{Dependence of the correlation lengths $\xi_r$ and $\xi_i$ on $\rho_{0}\delta U_{R}\ll1$. Here we use the function $f(x)=\frac{a}{x}$ to fit the data with
		$a=0.08815(0.08784,0.08846)$; $0.08985(0.08957,0.09014)$ and $R^{2}=0.9990$;
		$0.9992$, respectively.}
	\label{fig3}
\end{figure}
\begin{equation}
    C(\bm{x})\propto\int_0^{\infty}\frac{t\sin(xt)}{\sqrt{t^2+s^2}}\,,
\end{equation}
where we redefine $t=\frac{\xi_{\bm{k}}}{v_F}$ and $s=\text{Re}\Delta_0$. On the real axis, the real part of the gap is given by
\begin{equation}
    s=\text{Re}\Delta_0=\frac{\omega_D}{\sinh(\frac{1}{\rho_0U_R})}\,.
\end{equation}
For $U_R\to0$, we have $s=\omega_D\text{exp}(-\frac{1}{\rho_0U_R})\to0$ and thus the correlation length is given by
\begin{equation}
    \xi^{-1}\propto\text{exp}(-\frac{1}{\rho_0 U_R})\,.
\end{equation}
We can find that the correlation length cannot be represented by the polynomial form of $\delta U_R$, which indicates that the critical behavior on the real axis is indeed different from that on the upper half complex plane with $U_I\neq0$. As shown in Sec. \ref{RG}, this difference in the critical behavior can be understood from the RG flow.


\section{Thermodynamic Quantities on the Phase Boundary}

In this section, we calculate critical exponents associated with non-analyticity of thermodynamic quantities on the phase boundary. The condensation energy of the non-Hermitian BCS model is given by \cite{Yamamoto2019}
\begin{equation}
 \Delta E = - \frac{N}{U_R + i U_I} \left( \text{Im} \Delta_0 \right)^2
  - N \int_{- \omega_D}^{\omega_D} d \xi_{\bm{k}} \rho_0 \left( \sqrt{\xi_{\bm{k}}^2 +
  \Delta_0^2} - | \xi_{\bm{k}} | \right)\,.\label{free_energy}
\end{equation}
Note that here we have subtracted the energy of non-interacting fermions from the energy (\ref{free_energy}). The integration is separated into two parts. The second term of the integral is given by
\begin{equation}
  2 \rho_0 \int_0^{\omega_D} d \xi_{\bm{k}} \xi_{\bm{k}} = \rho_0 \omega_D^2\,.
\end{equation}
The first term on the phase boundary is given by
\[ - 2 \int_0^{\omega_D} d \xi_{\bm{k}} \rho_0 \sqrt{\xi_{\bm{k}}^2 + \Delta_0^2} = - 2
   \int_0^{\omega_D} d \xi_{\bm{k}} \rho_0 \sqrt{\xi_{\bm{k}}^2 - \left( \text{Im} \Delta_0
   \right)^2}. \]
We first focus on the real part of the energy. Since
\begin{eqnarray}
  \text{Re}\left[- 2 \int_0^{\omega_D} d \xi_{\bm{k}} \rho_0 \sqrt{\xi_{\bm{k}}^2 - \left( \text{Im}
  \Delta_0 \right)^2}\right] & = & - 2 \int_{\text{Im} \Delta_0}^{\omega_D}
  d \xi_{\bm{k}} \rho_0 \sqrt{\xi_{\bm{k}}^2 - \left( \text{Im} \Delta_0 \right)^2}
  \nonumber\\
  & = & (- 2 \rho_0) \left( \text{Im} \Delta_0 \right)^2 \left[ \frac{1}{4}
  \text{sinh} \left( \frac{2 U_R}{\rho_0 | U |^2} \right) - \frac{1}{2}
  \frac{U_R}{\rho_0 | U |^2} \right]\,,
\end{eqnarray}
the real part of the energy is
\begin{eqnarray}
  \text{Re} [\Delta E] & = & - N \frac{U_R}{| U |^2} \left( \text{Im} \Delta_0
  \right)^2 - 2 N \rho_0 \left( \text{Im} \Delta_0 \right)^2 \left[
  \frac{1}{4} \text{sinh} \left( \frac{2 U_R}{\rho_0 | U |^2} \right) -
  \frac{1}{2} \frac{U_R}{\rho_0 | U |^2} \right] + N \rho_0 \omega_D^2
  \nonumber\\
  & = & N \rho_0 \left[ \omega_D^2 - \frac{1}{2} \left( \text{Im} \Delta_0
  \right)^2 \text{sinh} \left( \frac{2 U_R}{\rho_0 | U |^2} \right) \right]
  \nonumber\\
  & = & N \rho_0 \omega_D^2 \left[ 1 - \frac{1}{2} \frac{\text{sinh} \left(
  \frac{2 U_R}{\rho_0 | U |^2} \right)}{\text{cosh}^2 \left( \frac{U_R}{\rho_0
  | U |^2} \right)} \right]\,.
  \label{free}
\end{eqnarray}

In a similar manner, we calculate the imaginary part of the condensation energy on the phase boundary as
\begin{eqnarray}
  \text{Im} [\Delta E] & = & - \frac{N}{| U |^2} \left( - U_I \right)
  \left( \text{Im} \Delta_0 \right)^2 - 2 N \rho_0 \int_0^{\text{Im} \Delta_0}
  d \xi_{\bm{k}} \sqrt{\xi_{\bm{k}}^2 - \text{Im} \Delta_0} \nonumber\\
  & = & \frac{2U_I N}{2 | U |^2} \left( \text{Im} \Delta_0 \right)^2 - 2 N
  \rho_0 \left( \text{Im} \Delta_0 \right)^2 \times \frac{\pi}{4} \nonumber\\
  & = & \frac{N}{2} \rho_0 \left( \text{Im} \Delta_0 \right)^2 \left[
  \frac{2U_I}{\rho_0 | U |^2} - \pi \right] \nonumber\\
  & = & 0 \,.
\end{eqnarray}
Thus, the imaginary part of the condensation energy vanishes on the phase boundary. 

We here relate the number of roots $\chi$ of the partition function with the nonanalyticity of the condensation energy. According to the definition of $\chi$ in the main text, we have
\begin{equation}
    \chi / \beta \simeq\frac{\omega_D}{\pi\text{cosh}(\frac{U_R}{\rho_0|U|^2})}
\end{equation}
in the zero-temperature limit. Using Eq. (\ref{free}), we find
\begin{equation}
    \Delta E=N\rho_0\omega_D^2(1-\sqrt{1-(\frac{\pi\chi}{\beta\omega_D})^2})\,.
\end{equation}
Since $\frac{U_R}{\rho_0|U|^2}\rightarrow\infty$ near $U=0$, the expressions for the condensation energy and the number of roots can be simplified as
\begin{equation}
    \Delta E=2N \rho_0 \omega_D^2 e^{-2\frac{U_R}{\rho_0|U|^2}},\chi/\beta=\frac{2\omega_D}{\pi}e^{-\frac{U_R}{\rho_0|U|^2}}\,.
\end{equation}
Thus, we have
\begin{equation}
    \Delta E=\frac{\pi^2N\rho_0}{2\beta^2}\chi^2\propto(\chi/\beta)^2\,
\end{equation}
near the origin, which means that condensation energy can be related to the number of roots $\chi$. Similarly, the condensation energy on the real axis is given by
\begin{equation}
	\Delta E=-2N \rho_0 \omega_D^2 e^{-\frac{2}{\rho_0 U_R}}.
\end{equation}
Hence, we have $\Delta E=-\frac{\pi^2N\rho_0}{2\beta^2}\chi^2\propto(\chi/\beta)^2$ near the origin. This tells us that the information of condensation energy on the real axis can be read from the number of roots on the complex plane.

Next, we show that the compressibility exhibits critical behavior near the phase boundary. The compressibility is defined by $\kappa=\frac{\partial^2F}{\partial\mu^2}$ where $\xi_{\bm{k}}=\epsilon_{\bm{k}}-\mu$ and $F=-\frac{1}{\beta}\text{log}Z$ is the free energy of the Bogoliubov quasiparticles. We have
\begin{equation}
  \kappa = - \sum_{\bm{k}} \frac{\Delta_{\bm{k}}^2}{(\xi_{\bm{k}}^2 +
  \Delta_{\bm{k}}^2)^{3 / 2}} = - N\int_{- \omega_D}^{\omega_D} \rho_0 d
  \xi_{\bm{k}} \frac{\Delta_0^2}{(\xi_{\bm{k}}^2 + \Delta_0^2)^{3
  / 2}}\,.
  \label{compressibility}
\end{equation}
Firstly, we consider the compressibility at points on the phase boundary. By substituting the gap in Eqs. (\ref{RDelta_0}) and (\ref{IDelta_0}) into the compressibility (\ref{compressibility}), we can rewrite it as
\begin{equation}
  \kappa= N\int_{- \omega_D}^{\omega_D} \rho_0 d \xi_{\bm{k}}
  \frac{(\text{Im} \Delta_0)^2}{(\xi_{\bm{k}}^2 - (\text{Im}
  \Delta_0)^2)^{3 / 2}}\,.
\end{equation}
This integral diverges since $\text{Im} \Delta_0 < \omega_D$ for all the points on the boundary. Hence, the compressibility exhibits singularity at each point on the boundary. This cannot occur in the Hermitian case since the integral in Eq. (\ref{compressibility}) is finite for a real gap $\Delta_0$. Near the phase boundary with an infinitesimal deviation $\delta U_R$, we obtain
\begin{equation}
	\kappa\simeq N\int_{- \omega_D}^{\omega_D} \rho_0 d \xi_{\bm{k}}
  \frac{(\text{Im} \Delta_0)^2}{(\xi_{\bm{k}}^2 - (\text{Im} \Delta_0)^2
  + 2 i \text{Re} \Delta_0 \text{Im} \Delta_0)^{3 / 2}} = N\int_{- A}^A \rho_0
  ds \frac{1}{(s^2 - 1 + 2 i \text{Re}
  \Delta_0 / \text{Im} \Delta_0)^{3 / 2}}\,,
\end{equation}
where $A := \cosh \left( \frac{U_R}{\rho_0 | U |^2} \right)$. Since the points near the value $s = 1$ dominantly contribute to the integral, we expand the integral around this point as
\begin{equation}
  \int_{0}^{c} \rho_0 d \delta s \frac{1}{(2 \delta
  s+ 2 i a)^{3 / 2}}=-[\frac{1}{\sqrt{2ia+c}}-\frac{1}{\sqrt{2ia}}],
\end{equation}
where $c$ is a positive constant. Under the limit $a\to0$, the integral is proportional to $a^{-1/2}$. Hence, we obtain the critical exponent $\zeta = 1 / 2$, which is defined as $\kappa \sim (\delta U)^{- \zeta}$ near the phase boundary.

We note that the critical exponents $\eta$ and $\zeta$ are not independent. Here we show the relation between $\eta$ and $\zeta$ for the energy spectrum that can be expanded as $(k-k_E)^{1/n}$ around a gapless point $k=k_E$. When $n$ is an integer, such dispersion relation appears near an $n$-th order exceptional point in non-Hermitian systems \cite{kato1995perturbation}. For those energy spectra, the long-range behavior of the correlation function takes the form as
\begin{equation}
  C(\bm{x})\propto x^{-2+1/n}\,.
\end{equation}
%Under the limitation $x\rightarrow\infty$, we have $J_{-1/2+1/n}(x)\sim\frac{1}{\sqrt{x}}$.
From the definition of the anomalous dimension, we find $\eta=1-\frac{1}{n}$. Similarly, we find that the critical behavior of the compressibility near the phase boundary is 
\begin{equation}
  \kappa\propto\int_{- A}^A \rho_0
  d \xi_{\bm{k}} \frac{1}{(\xi_{\bm{k}}^2 - 1 + 2 i \text{Re}
  \Delta_0 / \text{Im} \Delta_0)^{2-\frac{1}{n}}}\propto(\delta U)^{-1+\frac{1}{n}}\,,
\end{equation}
which indicates that $\zeta=1-1/n$. Thus, we have
\begin{equation}
  \eta=\zeta\,.
\end{equation}

Finally, we discuss the dynamical critical exponent $z$. From the definition of dynamical critical exponent $z$ in Ref. \cite{Sachdev:2011uj}, we here define it as
\begin{equation}
	\text{Re}\Delta_0\propto\xi^{-z}.
\end{equation}
By referencing (\ref{Real_gap}) and (\ref{corre_length}), we obtain
\begin{equation}
	\text{Re}\Delta_0\propto\xi^{-1}\propto\delta U.
\end{equation}
Hence, we have $z=1$.

\section{Pair Correlation Function}

The pair correlation function is defined by
\begin{equation}
	\rho_2 (\boldsymbol{r}_1 \sigma_1, \boldsymbol{r}_2 \sigma_2 ; \boldsymbol{r}_1' \sigma_1', \boldsymbol{r}_2' \sigma_2') := {}_L \langle
	c_{\sigma_1}^{\dagger} (\boldsymbol{r}_1) c_{\sigma_2}^{\dagger} (\boldsymbol{r}_2) c_{\sigma_2'} (\boldsymbol{r}_2') c_{\sigma_1'}(\boldsymbol{r}_1') \rangle_R\,.
\end{equation}
Here we set $\boldsymbol{r}_1 =\boldsymbol{r}_2 =\boldsymbol{R}$ and $\boldsymbol{r}_1'=\boldsymbol{r}_2' = 0$ to consider the correlation between two Cooper pairs. Without loss of generality, we assume $\sigma_1 = \sigma_1'=\uparrow, \sigma_2 = \sigma_2'= \downarrow$. With Wick's theorem, we can simplify the pair correlation function as
\begin{eqnarray}
	\rho_2 (\boldsymbol{R} \uparrow, \boldsymbol{R} \downarrow ; 0 \uparrow, 0
	\downarrow) & = & {}_L \langle c_{\uparrow}^{\dagger} (\boldsymbol{R})
	c_{\downarrow}^{\dagger} (\boldsymbol{R}) c_{\downarrow} (0) c_{\uparrow} (0)
	\rangle_R \nonumber\\
	& = & {}_L \langle c_{\uparrow}^{\dagger} (\boldsymbol{R})
	c_{\downarrow}^{\dagger} (\boldsymbol{R}) \rangle_R {}_L \langle
	c_{\downarrow} (0) c_{\uparrow} (0) \rangle_R + {}_L \langle
	c_{\uparrow}^{\dagger} (\boldsymbol{R}) c_{\uparrow} (0) \rangle_R {}_L
	\langle c_{\downarrow}^{\dagger} (\boldsymbol{R}) c_{\downarrow} (0) \rangle_R\,.
  \label{R0}
\end{eqnarray}
As shown in Eq. (\ref{correlation_decay}), the second term in Eq. (\ref{R0}) decays as $|\bm{R}|^{- 3 / 2}$ on the phase boundary. Thus, in the limit of $|\bm{R}|\to\infty$, the pair correlation function is given by
\begin{eqnarray}
	\lim_{|\bm{R}|\to\infty}\rho_2 (\boldsymbol{R} \uparrow, \boldsymbol{R} \downarrow ; 0 \uparrow, 0
	\downarrow) & = & {}_L \langle c_{\uparrow}^{\dagger} (0)
	c_{\downarrow}^{\dagger} (0) \rangle_R {}_L \langle c_{\downarrow} (0)
	c_{\uparrow} (0) \rangle_R \nonumber\\
	& = & \frac{1}{N^2} \sum_{\boldsymbol{k}_1, \boldsymbol{k}_2} {}_L \langle
	c_{\boldsymbol{k}_1 \uparrow}^{\dagger} c_{-\boldsymbol{k}_1
		\downarrow}^{\dagger} \rangle_R {}_L \langle c_{-\boldsymbol{k}_2 \downarrow}
	c_{\boldsymbol{k}_2 \uparrow}  \rangle_R \nonumber\\
	& = & \left( \frac{\Delta_0}{U} \right)^2 \,,
\end{eqnarray}
where we have used the translational invariance and the definition $\Delta_0 = - \frac{U}{N}
\sum_{\boldsymbol{k}} {}_L \langle c_{-\boldsymbol{k} \downarrow} c_{\boldsymbol{k}\uparrow}  \rangle_R$. We note that the long-distance limit of the pair correlation function does not vanish on the phase boundary:
\begin{equation}
	\lim_{|\bm{R}|\to\infty}\rho_2 (\boldsymbol{R} \uparrow, \boldsymbol{R} \downarrow ; 0 \uparrow, 0
	\downarrow) = - \frac{(\text{Im} \Delta_0)^2}{U^2} \neq 0\,.
\end{equation}
This non-vanishing behavior of the correlation function at the critical point is due to the non-Hermitian nature of the critical phenomenon. In fact, in nonunitary critical phenomena, the correlation function at the critical point can diverge as a function of the distance rather than decay \cite{Fisher:1978vn}.

\bibliography{MyCollection,export}
%\bibliography{MyCollection2}

\end{document}
