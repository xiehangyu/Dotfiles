\clearpage{}

\onecolumngrid
\renewcommand{\thefigure}{S\arabic{figure}}
\setcounter{figure}{0} 
\renewcommand{\thepage}{S\arabic{page}}
\setcounter{page}{1} 
\renewcommand{\theequation}{S.\arabic{equation}}
\setcounter{equation}{0} 
%\renewcommand{\thesection}{S.\Roman{section}} 
\setcounter{section}{0}

\begin{center}
\textbf{\textsc{\LARGE{}Supplementary Information}}{\LARGE\par}
\par\end{center}

\tableofcontents{}



In this Supplemental Mateiral, we provide detailed proof of Theorem 1 and 2 in the main text. We also provide the calculations of other results and conclusions in the main text and discuss their generalization.

\section{Proof of Canonical Typicality/Atypicality for the RFG ensemble}

In this section, we consider number conserving fermionic
Gaussian ensemble with $N$ modes occupied by $m$ fermions. The
notation follows the main text. In particular, %addition, 
$\langle\cdots\rangle$ is
used to denote the average value over the ensemble. The covariance
matrix of the subsystem $A$ for a particular random Gaussian state
is 
\begin{equation}
C_{A}=\Pi_{A}UC_{0}U^{\dagger}\Pi_{A}^{\dagger},
\end{equation}
where $\Pi_{A}$ is the projection operator on the the subsystem with size
$N_{A}\times N$, $U$ %can be 
is taken Haar randomly over $\mathbb{U}(N)$ and
$C_{0}$ satisfies $C_{0}^{2}=C_{0}$ and $\Tr C_{0}=m$. In the following,
the distance between two matrices is measured by Hilbert-Schmidt distance
$d_{\mathrm{HS}}$. We define a function $f:\mathbb{U}(N)\to\mathbb{R}$
as
\begin{equation}
f(U)=d_{\mathrm{HS}}(\Pi_{A}UC_{0}U^{\dagger}\Pi_{A}^{\dagger},\langle C_{A}\rangle).
\end{equation}
It is easy to check that $f$ is Lipschitz continuous with constant
$2$:

\begin{align*}
|f(U_{1})-f(U_{2})| & \leq d_{\mathrm{HS}}(U_{1}C_{0}U_{1}^{\dagger},U_{2}C_{0}U_{2}^{\dagger})\\
 & \leq d_{\mathrm{HS}}(U_{1}C_{0}U_{1}^{\dagger},U_{1}C_{0}U_{2}^{\dagger}) +d_{\mathrm{HS}}(U_{1}C_{0}U_{2}^{\dagger},U_{2}C_{0}U_{2}^{\dagger})\\
 & =\|C_{0}(U_{1}^{\dagger}-U_{2}^{\dagger})\|_{\mathrm{HS}}+\|(U_{1}-U_{2})C_{0}\|_{\mathrm{HS}}\\
 & \leq2d_{\mathrm{HS}}(U_{1,}U_{2}).
\end{align*}
The generalized Levy's lemma \citep{measure_concentration1,measure_concentration2,Meckes2019}
states that for any Lipschitz continuous function over some Riemann
manifolds with positive curvature, its values are concentrated around
the mean one. For the unitary group, %Formally, 
we have 
\begin{equation}
\mathbb{P}(|f(U)-\langle f(U)\rangle|\geq l\epsilon)\leq2e^{-\frac{N\epsilon^{2}}{12}},
\end{equation}
where $l$ is the Lipschitz constant. In what follows, we will bound $\langle f(U)\rangle=\langle d_{\mathrm{HS}}(\Pi_{A}UC_{0}U^{\dagger}\Pi_{A}^{\dagger},\langle C_{A}\rangle)\rangle$.
First, Since $\langle C_{A}\rangle=\Pi_{A}\int d_{\mathrm{H}}(U)UC_{0}U^{\dagger}\Pi_{A}^{\dagger}$
is invariant under any unitary on $\mathbb{U}(N_A)$, %group, 
according to Schur's lemma, $\langle C_{A}\rangle=\frac{m}{N}I_{A}$
and
\begin{align*}
\langle\|\Pi_{A}UC_{0}U^{\dagger}\Pi_{A}^{\dagger}-\langle C_{A}\rangle\|_{\mathrm{HS}}\rangle & \leq\sqrt{\langle\|\Pi_{A}UC_{0}U^{\dagger}\Pi_{A}^{\dagger}-\langle C_{A}\rangle\|_{\mathrm{HS}}^{2}\rangle}\\
 & =\sqrt{\langle\mathrm{Tr}(\Pi_{A}UC_{0}U^{\dagger}\Pi_{A}^{\dagger}-\langle C_{A}\rangle)^{2}\rangle}\\
 & =\sqrt{\langle\mathrm{Tr}(\Pi_{A}UC_{0}U^{\dagger}\Pi_{A}^{\dagger})^{2}\rangle-\frac{m^{2}}{N^{2}}N_{A}}.
\end{align*}
Next, we need to calculate $\langle\Tr(\Pi_{A}UC_{0}U^{\dagger}\Pi_{A}^{\dagger})^{2}\rangle$.
The idea is similar as in \citep{Popescu2006} and originally comes
from random quantum channel coding \citep{PhysRevA.55.1613}: we introduce
another reference space $R'$ which has the same dimension as the
original total system $R$. The following equation holds
\[
\langle\mathrm{Tr}(\Pi_{A}UC_{0}U^{\dagger}\Pi_{A}^{\dagger})^{2}\rangle=\int d_{H}(U)\mathrm{Tr}[(\Pi_{A}\otimes\Pi_{A})(UC_{0}U^{\dagger}\otimes UC_{0}U^{\dagger})\mathrm{SWAP}_{RR'}(\Pi_{A}^{\dagger}\otimes\Pi_{A}^{\dagger})],
\]
where $\mathrm{SWAP}_{RR'}$ is the SWAP operation between the original
system $R$ and the reference one $R'$. From Schur-Weyl duality \citep{Hayashi2017}, we obtain
\begin{equation}
\int d_{H}(U)(UC_{0}U^{\dagger}\otimes  UC_{0}U^{\dagger})\mathrm{SWAP}_{RR'}=\alpha I_{RR'}+\beta\mathrm{SWAP}_{RR'}.
\label{eq:unitary_2_design}
\end{equation}
Now, for simplicity, we can take $C_{0}=\begin{pmatrix}I_{m} & 0\\
0 & 0
\end{pmatrix}$. The following relations hold:
\begin{align*}
\text{Tr}\mathrm{SWAP}_{RR'} & =N,\\
\mathrm{Tr}[(U\otimes  U)(C_{0}\otimes  C_{0}) (U^{\dagger}\otimes  U^{\dagger})\mathrm{SWAP}_{RR'}] & =\mathrm{Tr}[(U\otimes  U)(C_{0}\otimes  C_{0})\mathrm{SWAP}_{RR'} (U^{\dagger}\otimes  U^{\dagger})] =\mathrm{Tr}C_{0}^{2}=m,\\
\mathrm{Tr}[(U\otimes  U) (C_{0}\otimes  C_{0}) (U^{\dagger}\otimes  U^{\dagger})] & =m^{2}.
\end{align*}
The trace of Eq. (\ref{eq:unitary_2_design}) gives $N^{2}\alpha+N\beta=m$.
Multiplying Eq. (\ref{eq:unitary_2_design}) by $\mathrm{SWAP}_{RR'}$
and tracing it, we have $N\alpha+N^{2}\beta=m^{2}$. Solving the equations
leads to $\begin{cases}
\alpha= & \frac{Nm-m^{2}}{N(N^{2}-1)}\\
\beta= & \frac{Nm^{2}-m}{N(N^{2}-1)}
\end{cases}.$ As a result, ~
\begin{align}
\langle\mathrm{Tr}(\Pi_{A}UC_{0}U^{\dagger}\Pi_{A}^{\dagger})^{2}\rangle & =\mathrm{Tr}[(\Pi_{A}\otimes \Pi_{A})(\alpha I_{RR'}+\beta\mathrm{SWAP}_{RR'})(\Pi_{A}^{\dagger}\otimes \Pi_{A}^{\dagger}\nonumber)] \\
 & =\alpha N_{A}^{2}+\beta N_{A}.
 \label{eq:value_of_average_of_tracesquare}
\end{align}
Assuming that in the thermodynamic limit $N\to\infty$, the density of charge
$\frac{m}{N}$ is fixed as $\mathcal{O}(1)$, we obtain
\begin{align*}
\langle f(U)\rangle^{2} & \leq\frac{mN_{A}^{2}}{N(N-1)}\sim \mathcal{O}\left(\frac{N_{A}^{2}}{N}\right)
\end{align*}
and the typicality
\begin{equation}
\mathbb{P}\left(d_{\mathrm{HS}}\left(C_{A},\frac{m}{N}I_{A}\right)\geq2\epsilon+\sqrt{\frac{mN_{A}^{2}}{N(N-1)}}\right)\leq2e^{-\frac{N\epsilon^{2}}{12}}.
\label{eq:meassure_concentration_on_covariance_matrix}
\end{equation}

For the other direction, %if 
we take $f(U)=d_{\mathrm{HS}}^{2}(C_{A},\langle C_{A}\rangle)$,
which %it 
is also Lipschitz continuous with constant calculated as
\begin{align}
|f(U_{1})-f(U_{2})| & \leq2(d_{\mathrm{HS}}(\Pi_{A}U_{1}C_{0}U_{1}^{\dagger}\Pi_{A}^{\dagger},\langle C_{A}\rangle)+d_{\mathrm{HS}}(\Pi_{A}U_{2}C_{0}U_{2}^{\dagger}\Pi_{A}^{\dagger},\langle C_{A}\rangle))d_{\mathrm{HS}}(U_{1,}U_{2})\nonumber \\
 & \leq4\sqrt{N_{A}}\left(1-\frac{m}{N}\right)d_{\mathrm{HS}}(U_{1},U_{2}).
\end{align}
Here we assume $m\leq\frac{N}{2}$ due to the particle-hole symmetry (otherwise, we may replace $1-\frac{m}{N}$ by $\frac{m}{N}$). According to Eq. (\ref{eq:value_of_average_of_tracesquare}), we obtain
\begin{equation}
\langle f(U)\rangle=\langle\mathrm{Tr}(\Pi_{A}UC_{0}U^{\dagger}\Pi_{A}^{\dagger})^{2}\rangle-\frac{m^{2}}{N^{2}}N_{A}\geq\frac{(N-m)mN_{A}^{2}}{N^{2}(N+1)}.
\end{equation}
If $\frac{N_{A}}{N}$ and $\frac{m}{N}$ are both fixed as $\mathcal{O}(1)$ in the thermodynamic limit, this formula scales linear with $N$. Applying generalized
Levy's lemma leads to 
\begin{equation}
\mathbb{P}\left(d_{\mathrm{HS}}^{2}\left(C_{A},\frac{m}{N}I_{A}\right)\leq\frac{(N-m)mN_{A}^{2}}{N^{2}(N+1)}-4\sqrt{N_{A}}\left(1-\frac{m}{N}\right)\epsilon\right)\leq2e^{-\frac{N}{12}\epsilon^{2}}. 
\label{eq:atypical_expression}
\end{equation}
For example, if we choose $\epsilon\sim\mathcal{O}(N^{\frac{1}{3}})$,
the above inequaility means that $C_{A}$ will deviate from its
ensemble average by an $\mathcal{O}(N)$ factor with almost unit probability. %almost $1$.
This is the atypicality discussed in the main text. 


\section{Some Applications of Measure Concentration Typicality}

\subsection{Measure concentration property for entropy}

In this subsection, we will use Eq. (\ref{eq:meassure_concentration_on_covariance_matrix})
and Eq. (\ref{eq:atypical_expression}) to derive the measure concentration
typicality/atypicality for subsystem entropy. For simplicity the half
filling condition is assumed. The eigenvalues of $C_{A}$ are denoted
as $\{\frac{1}{2}+\lambda_{i}\},i\in\{1,\cdots,N_{A}\}$ with $\sum_{i=1}^{N_{A}}\lambda_{i}^{2}=d_{\mathrm{HS}}(C_{A},\frac{I_{A}}{2})$
and $\lambda_{i}\in[-\frac{1}{2},\frac{1}{2}]$.

If the subsystem is microscopically small, we know the typicality of entropy
follows by noting that 
\begin{equation}
S_A=\sum_{i=1}^{N_{A}}H\left(\frac{1}{2}+\lambda_{i},\frac{1}{2}-\lambda_{i}\right)=\sum_{i=1}^{N_{A}}\left[1-\sum_{n=1}^{\infty}\frac{(2\lambda_{i})^{2n}}{2n(2n-1)\ln2}\right]\geq N_{A}-4d_{\mathrm{HS}}^{2}\left(C_{A},\frac{I_{A}}{2}\right),
\label{eq:Expansion_of_entropy_by_eigenvalues}
\end{equation}
where we replace $(2\lambda_i)^{2n}$ by $(2\lambda_i)^2$ in the last inequality since $(2\lambda_{i})^{2}\leq1$. Here $H(p_0,p_1)=-p_0\log_2 p_0 - p_1\log_2 p_1$ is the Shannon entropy. Combined with Eq. (\ref{eq:meassure_concentration_on_covariance_matrix})
we obtain
\begin{equation}
\begin{split}\mathbb{P}(N_{A}-S_A\geq x) & \leq\mathbb{P}\left(d_{\mathrm{HS}}\left(C_{A},\frac{I_{A}}{2}\right)\geq\frac{\sqrt{x}}{2}\right)\\
 & \leq\begin{cases}
2\exp\left[-\frac{N}{48}\left(\frac{\sqrt{x}}{2}-\sqrt{\frac{N_{A}^{2}}{2(N-1)}}\right)^{2}\right], & x>\frac{2N_{A}^{2}}{N-1};\\
1, & x\leq\frac{2N_{A}^{2}}{N-1}.
\end{cases}
\label{eq:typicality_for_subsystem_entropy}
\end{split}
\end{equation}
As long as $N\gg N_{A}^{2}$, we conclude the subsystem entropy will
be nearly maximal. 

%On 
For the other direction, if $N_{A}$ is macroscopically large, we can upper %lower 
bound the lhs of Eq. (\ref{eq:Expansion_of_entropy_by_eigenvalues}) by
\begin{equation}
S_A\leq N_{A}-\frac{2}{\ln2}d_{\mathrm{HS}}^{2}\left(C_{A},\frac{I_{A}}{2}\right).
\end{equation}
Following the atypicality of $d_{\mathrm{HS}}^{2}(C_{A},\frac{I_{A}}{2})$
in Eq.~(\ref{eq:atypical_expression}), the subsystem entropy density
will show an $\mathcal{O}(1)$ deviation from the maximal value%one
:
\begin{equation}
\begin{split}\mathbb{P}\left(S_A\geq N_{A}-\frac{N_{A}^{2}}{2\ln2(N+1)}+\frac{2}{\ln2}\epsilon\right) & \leq\mathbb{P}\left(d_{\mathrm{HS}}^{2}\left(C_{A},\frac{I_{A}}{2}\right)\leq\frac{N_{A}^{2}}{4(N+1)}-\epsilon\right)\\
 & \leq2e^{-\frac{N}{48N_{A}}\epsilon^{2}}.
\end{split}
\end{equation}
We may take $\epsilon\sim\mathcal{O}(N^{\alpha})$ for arbitrary $\alpha\in(0,1)$,
finding that the majority of subsystem entropy will be comparable
or smaller than $N_{A}-\frac{N_{A}^{2}}{2\ln2(N+1)}$. This clearly
illustrates the difference of the Page curve for the RFG ensemble from
the interacting one.



\subsection{Upper bound on the variance of entropy}

At the end of this section, we will discuss the variance of entropy
in microscopic region. Here the half filling condition is also assumed.
From Eq. (\ref{eq:typicality_for_subsystem_entropy}), we obtain
\[
\mathbb{P}((S_A-N_{A})^{2}\geq x)\leq\begin{cases}
1, & x\leq\frac{4N_{A}^{4}}{(N-1)^{2}};\\
2\exp\left[-\frac{N}{48}\left(\frac{x^{\frac{1}{4}}}{2}-\sqrt{\frac{N_{A}^{2}}{2(N-1)}}\right)^{2}\right], & x>\frac{4N_{A}^{4}}{(N-1)^{2}}.
\end{cases}
\]
Therefore
\begin{align*}
\langle(S_A-N_{A})^{2}\rangle & =\int\mathbb{P}((S_A-N_{A})^{2}\geq x)dx\\
 & \leq\frac{4N_{A}^{4}}{(N-1)^{2}}+2\int_{\frac{4N_{A}^{4}}{(N-1)^{2}}}^{\infty}dx\exp\left[-\frac{N}{48}\left(\frac{x^{\frac{1}{4}}}{2}-\sqrt{\frac{N_{A}^{2}}{2(N-1)}}\right)^{2}\right].
\end{align*}
For the last line, we can change the integral variable into $t=\sqrt{N}\left(\frac{x^{\frac{1}{4}}}{2}-\sqrt{\frac{N_A^2}{2(N-1)}}\right)$, obtaining
\begin{align*}
 & 2\int_{\frac{4N_{A}^{4}}{(N-1)^{2}}}^{\infty}dx\exp\left[-\frac{N}{48}\left(\frac{x^{\frac{1}{4}}}{2}-\sqrt{\frac{N_{A}^{2}}{2(N-1)}}\right)^{2}\right]
=  \frac{128}{\sqrt{N}}\int_{0}^{\infty}dt\left(\frac{t}{\sqrt{N}}+\sqrt{\frac{N_{A}^{2}}{2(N-1)}}\right){}^{3}e^{-\frac{t^{2}}{48}}\\
= & \frac{128}{N^{2}}\int_{0}^{\infty}dt\left(t+\sqrt{\frac{NN_{A}^{2}}{2(N-1)}}\right)^{3}e^{-\frac{t^{2}}{48}}
\sim  \mathcal{O}\left(\frac{1}{N^{2}}\right),
\end{align*}
%if keeping 
provided that $N_{A}$ is fixed as $\mathcal{O}(1)$. In conclusion, we obtain
\[
\mathrm{Var}(S_A)\leq\langle(S_A-N_{A})^{2}\rangle\sim\mathcal{O}(N^{-2}), %\frac{1}{N^{2}},
\]
which agrees with \citep{Bianchi2021,Bianchi2021a}.


\section{Detailed Calculation of the Dynamical %Emergent 
Page Curves}
As mentioned in the main text, for all the models in this section, we assume the initial state is a period-2 density wave with half filling. Following the same notation in the main text, we further define $X_{A}(t)=2C_{A}(t)-I_{A}$. Thus
\begin{equation}
S_{A}(t)=N_{A}-\sum_{n=1}^{\infty}\frac{\Tr X_{A}^{2n}(t)}{2n(2n-1)\ln2}.
\label{eq:Expanding_of_entropy_with_X}
\end{equation}

\subsection{Calculation for the minimal model}
We first consider the minimal model. %in this subsection.
Remember that the minimal model means only nearest neighbor hopping
is included. 
 After introducing the Fourier transformed 
 mode $a_{k}^{\dagger}=\frac{1}{\sqrt{N}}\sum_{j=1}^{N}e^{-ikj}a_{j}^{\dagger}$,
we can easily obtain the correlation function in momentum space: 
\begin{equation}
\Tr[\rho a_{k}^{\dagger}(t)a_{k'}(t)]=\frac{1}{2}\delta_{k,k'}+\frac{1}{2}\delta_{k,k'+\pi}e^{i\theta_{k}(t)},
\end{equation}
where $a_k(t)$ ($a^\dag_{k'}(t)$) is the annihilation (creaiton) operator in the Heisenberg picture, $\theta_{k}(t)=t(E_{k}-E_{k+\pi})$ and $\rho\textcolor{red}{=|\Psi_0\rangle\langle\Psi_0|}$ corresponds
to the initial density matrix. In the following, we may omit %suppress 
the
index $t$ if there is no ambiguity.

After inverse Fourier transformation back to position space, the covariance
matrix for subsystem $A$ reads  $[C_A]_{m_{1}m_{2}}=\frac{\delta_{m_{1},m_{2}}}{2}+\frac{1}{2N}\sum_{k}e^{i\theta_{k}}e^{ik(m_{1}-m_{2})}e^{i\pi m_{2}}$
and thus
\begin{equation}
[X_A]_{m_{1}m_{2}}=\frac{1}{N}\sum_{k}e^{i\theta_{k}}e^{ik(m_{1}-m_{2})}e^{i\pi m_{2}}.
\label{eq:Expression_for_X_in_NNH}
\end{equation}


\subsubsection{Second order in $X_{A}$}

With Eq. (\ref{eq:Expression_for_X_in_NNH}), we can calculate $\overline{\Tr X_{A}^{2}}$ as
\begin{align*}
\overline{\Tr X_{A}^{2}} & =\frac{1}{N^{2}}\sum_{k_{1},k_{2},m_{1},m_{2}}\overline{e^{i\theta_{k_{1}}}e^{ik_{1}(m_{1}-m_{2})}e^{i\pi m_{2}}e^{i\theta_{k_{2}}}e^{ik_{2}(m_{2}-m_{1})}e^{i\pi m_{1}}}\\
 & =\frac{1}{N^{2}}\sum_{k_{1},k_{2},m_{1},m_{2}}(\delta_{k_{1},k_{2}+\pi}e^{ik_{1}(m_{1}-m_{2})}e^{ik_{1}(m_{2}-m_{1})}e^{-i\pi(m_{2}-m_{1})}e^{i\pi(m_{1}+m_{2})}\\
 & +\delta_{k_{1}+k_{2},\pi}e^{i2k_{1}(m_{1}-m_{2})}e^{i\pi(m_{1}+m_{2})}e^{i\pi(m_{2}-m_{1})})\\
 & =\frac{N_{A}^{2}}{N}+\frac{1}{N}\sum_{k,m_{1},m_{2}}e^{i2k(m_{1}-m_{2})}=\frac{N_{A}^{2}}{N}+\frac{1}{N}\sum_{m_{1},m_{2}}(\delta_{m_{1}-m_{2},0}+\delta_{m_{1}-m_{2},\frac{N}{2}}+\delta_{m_{1}-m_{2},-\frac{N}{2}}).
\end{align*}
Noting that the first term in the middle step comes from $\theta_{k+\pi}=-\theta_{k}$,
which holds for general Hamiltonians according to the definition. However,
the second term is due to the reflection symmetry (which implies $E_k=E_{-k}$) of %model dependent and only holds for 
the minimal model and is thus not universal. 
%We call this as occasional symmetry. 
%Luckily, 
Fortunately, this model dependent term will vanish in the thermodynamic limit.

This calculation can be diagrammatically represented as shown in Fig.~\ref{Total_figure_for_FD_1}(a), where we draw an arrow from $m_1$ to $m_2$ ($m_2$ to $m_1$) because there is a factor $e^{ik_1(m_1-m_2)}$ ($e^{ik_2(m_2-m_1)}$). As will become clear below, such a diagrammatic representation provides a convenient and systematic way for dealing with higher order terms.

\subsubsection{Third order in $X_{A}$}
%Now we are calculating 
We move on to calculate $\overline{\Tr X_{A}^{3}}$ and will %to
see %why
it vanishes in thermodynamic limit. The expression is 
\begin{align*}
\overline{\Tr X_{A}^{3}} & =\frac{1}{N^{3}}\sum_{k_{1,2,3}m_{1,2,3}}\overline{e^{i\theta_{k_{1}}+i\theta_{k_{2}}+i\theta_{k_{3}}}}e^{ik_{1}(m_{1}-m_{2})+ik_{2}(m_{2}-m_{3})+ik_{3}(m_{3}-m_{1})}e^{i\pi(m_{1}+m_{2}+m_{3})}.
\end{align*}

The non-zero contribution of $\overline{e^{i\theta_{k_{1}}+i\theta_{k_{2}}+i\theta_{k_{3}}}}$
in the minimal model only comes from two cases:
\begin{enumerate}
\item $k_{2}=k_{1}+\pi,k_{3}=\pm\frac{\pi}{2}$ and cyclic permutations.
The contribution is proportional to 
\begin{align*}
 & \frac{1}{N^{3}}\sum_{k_{1},m_{1},m_{2},m_{3}}e^{ik_{1}(m_{1}-m_{3})}e^{i\frac{\pi}{2}(m_{1}+m_{3})} =\frac{N_{A}}{N^{2}}\sum_{m_{1},m_{3}}\delta_{m_{1},m_{3}}e^{i\pi m_{1}}\sim \mathcal{O}\left(\frac{1}{N}\right).
\end{align*}
\item $k_{1},k_{2}$ satisfy $|\cos k_{1}+\cos k_{2}|\leq1$ and $k_{3}=\arccos(-\cos k_{1}-\cos k_{2})$.
The contribution is proportional to
\[
\sim\frac{1}{N^{3}}\sum_{k_{1},k_{2}}\sum_{m_{1}}e^{im_{1}(k_{1}-k_{3}+\pi)}\sum_{m_{2}}e^{im_{2}(k_{2}-k_{1}+\pi)}\sum_{m_{3}}e^{im_{3}(k_{3}-k_{2}+\pi)}.
\]
With H\"older inequality \citep{hardy1988inequalities}: $\sum_{i}|a_{i}||b_{i}||c_{i}|\leq[(\sum_{i}|a_{i}|^{3})(\sum_{i}|b_{i}|^{3})(\sum_{i}|c_{i}|^{3})]^{\frac{1}{3}}$,
we can upper bound the above contribution as 
\[
\leq\frac{1}{N^{3}}\{[\sum_{k_{1},k_{2}}|\sum_{m_{1}}e^{im_{1}(k_{1}-k_{3}+\pi)}|^{3}][\sum_{k_{1},k_{2}}|\sum_{m_{2}}e^{im_{2}(k_{2}-k_{1}+\pi)}|^{3}][\sum_{k_{3},k_{2}}|\sum_{m_{3}}e^{im_{3}(k_{3}-k_{2}+\pi)}|^{3}]\}^{\frac{1}{3}}.
\]
Since here, no pair of two $k$ differ by $\pi$ (as it is the case
already considered in the first case), the sum of $m_{1},m_{2},m_{3}$
only contributes to $O(1)$, so the total contribution will be upper
bounded by $O(\frac{1}{N})$.
\end{enumerate}
In conclusion
\[
\overline{\Tr X_{A}^{3}}\sim \mathcal{O}\left(\frac{1}{N}\right)
\]
and thus vanishes in the thermodynamic limit. 

The above discussion can be generalized to higher orders: as long
as the degeneracy point of a Hamiltonian is not dense, we can safely
ignore the model dependent contribution %occasional symmetry 
and put $\overline{e^{i\theta_{k_{1}}}e^{i\theta_{k_{2}}}}=\delta_{k_{1},k_{2}+\pi}$,
which we call %this 
a contraction. In the following, %content, 
we will directly
use this contraction rule.\footnote{Another proof for general cases can be obtained with the techniques in Subsec.~\ref{subsec:Proof-of-main}.}


\begin{figure*}
\includegraphics[width=1\textwidth]{figs/first_order_and_second_order.png}
\caption{%The 
Feynman diagrams for calculating the entanglement entropy order by order. Here each
vertex represents a position index and each leg represents a momentum
index. Each leg is associated with $%\frac{1}{N}\sum_{k}
e^{i\theta_{k}}e^{ik(m-m')}e^{i\pi m'}$.
In these diagrams, the legs with same color need to be contracted.
Each color corresponds to one contraction.}
\label{Total_figure_for_FD_1}
\end{figure*}


\subsubsection{Forth Order in $X_{A}$}

Now we are moving to calculate $\overline{\Tr X_{A}^{4}}$:
\begin{equation}
\overline{\Tr X_{A}^{4}}=\frac{1}{N^{4}}\sum_{k_{1,2,3,4},m_{1,2,3,4}}\overline{\prod_{j}^{4}e^{i\theta_{k_{j}}}e^{ik_{j}(m_{j}-m_{j+1})}e^{im_{j}\pi}},
\label{eq:Expression_for_XA_4}
\end{equation}
where $m_{5}=m_{1}$. 

The first contracting class for $\overline{e^{i\theta_{k_{1}}}e^{i\theta_{k_{2}}}e^{i\theta_{k_{3}}}e^{i\theta_{k_{4}}}}$
is to contract the dynamical phase factors in pairs. %legs $2$ by $2$. 
Two patterns in this class are shown
in Fig.~\ref{Total_figure_for_FD_1}(b) and Fig.~\ref{Total_figure_for_FD_1}(c).
The legs with same colors mean that they are contracted together.
These patterns correspond to $\overline{e^{i\theta_{k_{1}}}e^{i\theta_{k_{4}}}}\;\overline{e^{i\theta_{k_{2}}}e^{i\theta_{k_{3}}}}=\delta_{k_{1},k_{4}+\pi}\delta_{k_{2},k_{3}+\pi}$
and $\overline{e^{i\theta_{k_{1}}}e^{i\theta_{k_{2}}}}\;\overline{e^{i\theta_{k_{4}}}e^{i\theta_{k_{3}}}}=\delta_{k_{1},k_{2}+\pi}\delta_{k_{4},k_{3}+\pi}$,
respectively. Substituting these delta functions to Eq. (\ref{eq:Expression_for_XA_4}),
we obtain 
\begin{equation}
2\times\frac{N_{A}^{2}}{N^{4}}\sum_{k_{1,3},m_{1,3}}e^{ik_{1}(m_{1}-m_{3})}e^{ik_{3}(m_{3}-m_{1})}=\frac{2N_{A}^{3}}{N^{2}},
\end{equation}
where the factor $2$ comes from the equal contribution of these two diagrams. 

Another pattern from this contracting class is shown in Fig.~\ref{Total_figure_for_FD_1}(d)
which gives $\overline{e^{i\theta_{k_{1}}}e^{i\theta_{k_{3}}}}\;\overline{e^{i\theta_{k_{2}}}e^{i\theta_{k_{4}}}}=\delta_{k_{1},k_{3}+\pi}\delta_{k_{2},k_{4}+\pi}$.
However, substituting this expression to Eq. (\ref{eq:Expression_for_XA_4})
leads to
\begin{equation}
\begin{aligned}\frac{1}{N^{4}}\sum_{k_{1,2},m_{1,2,3,4}}e^{i(k_{1}-k_{2})(m_{1}+m_{3}-m_{2}-m_{4})}e^{i\pi(m_{2}+m_{4})} & =\frac{1}{N^{2}}\sum_{m_{1,2,3,4}}\delta_{m_{1}+m_{3},m_{2}+m_{4}}e^{i\pi(m_{2}+m_{4})} \sim \mathcal{O}\left(\frac{1}{N}\right).
\end{aligned}
\end{equation}
This diagram thus %will 
vanishes in the thermodynamic limit. 

It should be emphasized that, in the above discussion, some terms %diagrams 
are calculated multiple times.
These forms the other contracting class: we contract the four legs
all together, as shown in Fig.~\ref{Total_figure_for_FD_1}(e). The
contribution from this diagram needs to be subtracted due to the multiple
calculation in Fig.~\ref{Total_figure_for_FD_1}(b) and (c):
\begin{equation}
-\frac{1}{N^{4}}\sum_{k_{1,2,3,4},m_{1,2,3,4}}\delta_{k_{2},k_{1}+\pi}\delta_{k_{3},k_{1}}\delta_{k_{4},k_{1}+\pi}\prod_{j}^{4}e^{ik_{j}(m_{j}-m_{j+1})}e^{im_{j}\pi}=-\frac{N_{A}^{4}}{N^{3}}.
\end{equation}
Combining all the contributions %calculation 
together, we obtain 
\begin{equation}
\overline{\Tr X_{A}^{4}}=\frac{2N_{A}^{3}}{N^{2}}-\frac{N_{A}^{4}}{N^{3}}.
\end{equation}

From the above discussion, we can see that the contraction rules here
are obviously different from %the 
Wick's theorem. 



\subsection{General Feynman Rules}

\begin{figure*}
\includegraphics[width=0.8\textwidth]{figs/third_order.png}
\caption{In this figure, more complicated Feynman diagrams are shown compared %complemented 
to Fig.~\ref{Total_figure_for_FD_1}. (a) shows a general Feynman diagram,
(b)-(e) are the Feynman diagrams for calculating $\overline{\Tr X_{A}^{6}}$.}
\label{Total_figure_for_FD_2}
\end{figure*}

%\begin{table}
%\begin{tabular}{|c|c|}
%\hline 
%$2j$ & $a_{2j}$\tabularnewline
%\hline 
%\hline 
%2 & 1\tabularnewline
%\hline 
%4 & -1\tabularnewline
%\hline 
%6 & 4\tabularnewline
%\hline 
%8 & -33\tabularnewline
%\hline 
%10 & 456\tabularnewline
%\hline 
%\end{tabular}

The method presented in the previous subsection allows us to calculate
Eq. (\ref{eq:Expanding_of_entropy_with_X}) to arbitrary orders. Here
we summarize our Feynman rules for contraction. A general Feynman diagram
for calculating $\overline{\Tr X_{A}^{2n}}$ is shown in Fig. \ref{Total_figure_for_FD_2}(a).
We will use the integer $i$ to label the legs associated with momentum
$k_{i}$. 
\begin{enumerate}
\item All the legs in Fig. \ref{Total_figure_for_FD_2}(a) must be contracted.  Each contraction leads to a delta function of momenta $k$s and has to include even number of legs, where half of the legs should be labeled
as even and the other half as odd. This last requirement arises from the phase factor $e^{i\pi m}$ in each leg and is to ensure the diagram %will 
not to vanish in the thermodynamic limit %, as in 
(we recall that Fig.~\ref{Total_figure_for_FD_1}(d) does not contribute as this requirement is not satisfied). For totally $2n$ legs with $l$ contractions, the $N,N_{A}$ dependence for this diagram is
$\frac{N_{A}^{2n-l+1}}{N^{2n-l}}$.
\item Each contraction with $2j$ legs should also be assigned a multiple
factor $a_{2j}$, accounting for the multiple calculations. $a_{2}$ and $a_{4}$ are obtained in previous discussion while higher $a_{2j}$ can be obtained iteratively, as shown below.
\item For each diagram, multiply the term $\frac{N_{A}^{2n-l+1}}{N^{2n-l}}$ and factors $a_{2j}$ obtained in Rule 1 and Rule 2 together. Some diagrams also need to multiply by subsystem correction factor $\beta$ (see details below). Sum over all possible diagrams leads to the desired result.
\end{enumerate}
The subsystem correction factor $\beta$ does not appear in $\overline{\Tr X_{A}^{4}}$ and $\overline{\Tr X_{A}^{2}}$, but will appear in calculating $\overline{\Tr X_{A}^{6}}$. For pedagogical purpose, we will now show how to calculate $\overline{\Tr X_{A}^{6}}$. Some diagrams are shown in Fig. \ref{Total_figure_for_FD_2}(b-e).
In diagram (b), there are three contractions, each contracts two legs.
The contribution for this diagram is $\frac{N_{A}^{4}}{N^{3}}a_{2}^{3}$.
In diagram (c), there are two contractions, one contracts four legs
together and the other contracts two legs. The contribution is $\frac{N_{A}^{5}}{N^{4}}a_{4}a_{2}$.
In diagram (d), all the six legs are contracted together, contributing
to $\frac{N_{A}^{6}}{N^{5}}a_{6}$. Attentions should be payed to %What should be attentioned is the 
diagram (e). After contracting the three pairs of legs, we obtain
\begin{equation}
\begin{split} & \sum_{k_{1,2,3}m_{1,2,3,4,5,6}}e^{ik_{1}(m_{1}-m_{2}+m_{4}-m_{5})}e^{ik_{2}(m_{2}-m_{3}+m_{5}-m_{6})}e^{ik_{3}(m_{3}-m_{4}+m_{6}-m_{1})}\\
= & N^{3}\sum_{m_{1,2,3,4,5,6}}\delta_{m_{1}+m_{4}=m_{2}+m_{5}=m_{3}+m_{6}\ \mathrm{mod\ }N}.
\end{split}
\end{equation}
Naively, one may conjecture %that 
the result of the last sum to %will 
be $N_{A}^{4}$.
However, this is only true when $f=\frac{N_{A}}{N}=1$. If $f\leq\frac{1}{2}$,
the sum will be much smaller. After carefully counting the pairs satisfying
the delta function, we obtain for $f\leq\frac{1}{2}$
\begin{equation}
\sum_{k_{1,2,3}m_{1,2,3,4,5,6}}e^{ik_{1}(m_{1}-m_{2}+m_{4}-m_{5})}e^{ik_{2}(m_{2}-m_{3}+m_{5}-m_{6})}e^{ik_{3}(m_{3}-m_{4}+m_{6}-m_{1})}=\frac{1}{2}N^{3}(N_{A}^{4}+N_{A}^{2})\to\beta_{1}N^{3}N_{A}^{4},
\end{equation}
where $\beta_{1}$ is defined as the subsystem correction factor in
thermal dynamical limit: $\beta_{1}=\begin{cases}
\frac{1}{2} & f\leq\frac{1}{2}\\
1 & f=1
\end{cases}$. Summing over all possible diagrams, the total contribution is 
\begin{equation}
\overline{\Tr X_{A}^{6}}=(5+\beta_{1})\frac{N_{A}^{4}}{N^{3}}-(6+3\beta_{2})\frac{N_{A}^{5}}{N^{4}}+\frac{N_{A}^{6}}{N^{5}}a_{6},
\end{equation}
where $\beta_{2}$ is another subsystem correction factor $\beta_{2}=\begin{cases}
\frac{2}{3} & f\leq\frac{1}{2}\\
1 & f=1
\end{cases}$. Here $a_{6}$ can be determined by considering the case when $N_{A}=N$
(i.e. $f=1$). In this case, $\overline{\Tr X_{A}^{6}}=N$, resulting
in $a_{6}=4$. Therefore, 
\begin{equation}
\overline{\Tr X_{A}^{6}}=\frac{11}{2}\frac{N_{A}^{4}}{N^{3}}-8\frac{N_{A}^{5}}{N^{4}}+4\frac{N_{A}^{6}}{N^{5}}\ \ \mathrm{if}\ \ f\leq\frac{1}{2}.
\end{equation}


\subsection{Proof of %main 
Theorem 2 \label{subsec:Proof-of-main}}

In this subsection, we go %are going 
beyond the minimal model and consider the general Hamiltonians satisfying the condition in %main 
Theorem 2 in the main text. We will find the Feynman rule as well as the subsystem entropy
is indeed the same as the previous subsection.

Since the Hamiltonian is period-2, we can use a modified Fourier
transformation to block diagonalize it:
\begin{equation}
A_{k}^{\dagger}=\sqrt{\frac{2}{N}}\sum_{j=1}^{\frac{N}{2}}e^{-ik(2j-1)}a_{2j-1}^{\dagger},\ \ B_k^\dag=\sqrt{\frac{2}{N}}\sum_{j=1}^{\frac{N}{2}}e^{-i2kj}a_{2j}^{\dagger},\;\;%k\in[0,\pi).
k\in\left\{\frac{2n\pi}{N}\right\}^{\frac{N}{2}-1}_{n=0}.
\label{eq:FT_2_periodic}
\end{equation}
Here $A_{k}^{\dagger},B_{k}^{\dagger}$ are related to the conserved (eigen) mode
$P_{k}^{\dagger},Q_{k}^{\dagger}$ via a $2\times2$ unitary transformation
$U^{k}$as $A_{k}^{\dagger}=U_{11}^{k}P_{k}^{\dagger}+U_{12}^{k}Q_{k}^{\dagger}$
and $B_{k}^{\dagger}=U_{21}^{k}P_{k}^{\dagger}+U_{22}^{k}Q_{k}^{\dagger}$.
Substituting into Eq. (\ref{eq:FT_2_periodic}) leads to 
\begin{align}
a_{2m}^{\dagger} & =\sqrt{\frac{2}{N}}\sum_{k=0}^{\pi}e^{ik2m}(U_{21}^{k}P_{k}^{\dagger}+U_{22}^{k}Q_{k}^{\dagger}),\;\;\;\;
a_{2m+1}^{\dagger}  =\sqrt{\frac{2}{N}}\sum_{k=0}^{\pi}e^{ik(2m+1)}(U_{11}^{k}P_{k}^{\dagger}+U_{12}^{k}Q_{k}^{\dagger}).
\label{eq:IFT}
\end{align}
If we define $Q_{k+\pi}=P_{k}$ %$P_{k}=Q_{k+\pi}$ 
and 
\begin{equation}
Z_{k}^{m}=\begin{cases}
\sqrt{2}U_{22}^{k}, & \mathrm{if\ }m\ \mathrm{is\ even\ and}\ k<\pi;\\
\sqrt{2}U_{21}^{k-\pi}, & \mathrm{if\ }m\ \mathrm{is\ even\ and}\ k\geq\pi;\\
\sqrt{2}U_{12}^{k}, & \mathrm{if\ }m\ \mathrm{is\ odd\ and}\ k<\pi;\\
-\sqrt{2}U_{11}^{k-\pi}, & \mathrm{if\ }m\ \mathrm{is\ odd\ and}\ k\geq\pi,
\end{cases}
\label{eq:Zkm}
\end{equation}
the above inverse Fourier transformation (\ref{eq:IFT}) can be rewritten as
\[
a_{m}^{\dagger}=\frac{1}{\sqrt{N}}\sum_{k=0}^{2\pi}Z_{k}^{m}Q_{k}^{\dagger}e^{ikm}.
\]
In the following, we will simplify %denote 
$\sum_{k=0}^{2\pi}$ as $\sum_{k}$
and $k$ should be understood as module $2\pi$. By assumption, all
the conserved quantity $\Tr(\rho Q_{k}^{\dagger}Q_{k})=\frac{1}{2}$
for $k\in[0,2\pi)$, thus
\begin{align*}
[C_A]_{ml} & =\frac{1}{2N}\sum_{k}Z_{k}^{m}Z_{k}^{l*}e^{ik(m-l)}+\frac{1}{2N}\sum_{k}e^{i\theta_{k}}Z_{k}^{m}Z_{k+\pi}^{l*}e^{ik(m-l)}e^{i\pi l}\\
 & =\frac{1}{2}\delta_{m,l}+\frac{1}{2N}\sum_{k}e^{i\theta_{k}}Z_{k}^{m}Z_{k+\pi}^{l*}e^{ik(m-l)}e^{i\pi l}
\end{align*}
where $\theta_{k+\pi}=-\theta_{k}$ by definition. In the last equality, we have
used the unitarity of $U^{k}$. Namely, if $m-l$ is odd:
\begin{align*}
\frac{1}{2N}\sum_{k}Z_{k}^{m}Z_{k}^{l*}e^{ik(m-l)} & =\frac{1}{4N}\sum_{k}[Z_{k}^{m}Z_{k}^{l*}e^{ik(m-l)}+Z_{k+\pi}^{m}Z_{k+\pi}^{l*}e^{i(k+\pi)(m-l)}]\\
 & =\frac{1}{4N}\sum_{k}(Z_{k}^{m}Z_{k}^{l*}-Z_{k+\pi}^{m}Z_{k+\pi}^{l*})e^{ik(m-l)}=0.
\end{align*}
A similar calculation can be carried out for the case in which %holds for 
$m-l$ is even. Therefore, we %still
have 
\[
[X_A]_{ml}=\frac{1}{N}\sum_{k}e^{i\theta_{k}}Z_{k}^{m}Z_{k+\pi}^{l*}e^{ik(m-l)}e^{i\pi l}.
\]
This expression is very similar to Eq. (\ref{eq:Expression_for_X_in_NNH})
except for the extra factors %terms 
$Z_{k}$. Nonetheless, we will show those extra
$Z_{k}$'s do %will 
not contribute %to 
in the thermodynamic limit. As a result,
the same Feynman rules and dynamical Page curve %entropy 
follows. 

When evaluating a Feynman diagram in the thermodynamic limit with %when 
$N_{A}$, $N$ both going to infinity,
we can first sum over the position indices. Introducing $s_{j}=k_{j}-k_{j-1}$, $k_{0}=k_{2n}$
and $m_{2n+1}=m_{1}$, we obtain
\begin{align*}
 & \sum_{m_{1,2\cdots2n}}\prod_{j=1}^{2n}e^{ik_{j}(m_{j}-m_{j+1})}e^{i\pi m_{j}}Z_{k_{j}}^{m_{j}}Z_{k_{j-1}+\pi}^{m_{j}*}\\
 & =\prod_{j=1}^{2n}e^{im_{j}s_{j}}e^{i\pi m_{j}}Z_{k_{j}}^{m_{j}}Z_{k_{j-1}+\pi}^{m_{j}*}\\
 & =\prod_{j=1}^{2n}\left[\sum_{m_{j}:\mathrm{even}}e^{im_{j}(s_{j}+\pi)}Z_{k_{j}}^{0}Z_{k_{j-1}+\pi}^{0*}+\sum_{m_{j}:\mathrm{odd}}e^{im_{j}(s_{j}+\pi)}Z_{k_{j}}^{1}Z_{k_{j-1}+\pi}^{1*}\right]\\
 & =\prod_{j=1}^{2n}\left[\frac{1-e^{iN_{A}(s_{j}+\pi)}}{1-e^{2i(s_{j}+\pi)}}\right]\prod_{j=1}^{2n}(Z_{k_{j}}^{0}Z_{k_{j-1}+\pi}^{0*}+e^{i(s_{j}+\pi)}Z_{k_{j}}^{1}Z_{k_{j-1}+\pi}^{1*})\\
 & =\frac{e^{i\frac{N_{A}}{2}\sum_{j=1}^{2n}(s_{j}+\pi)}}{e^{i\sum_{j=1}^{2n}(s_{j}+\pi)}}\prod_{j=1}^{2n}\left[\frac{\sin\frac{N_{A}(s_{j}+\pi)}{2}}{\sin(s_{j}+\pi)}\right]\prod_{j=1}^{2n}(Z_{k_{j}}^{0}Z_{k_{j-1}+\pi}^{0*}+e^{i(s_{j}+\pi)}Z_{k_{j}}^{1}Z_{k_{j-1}+\pi}^{1*})\\
 & =\prod_{j=1}^{2n}\left[\frac{\sin\frac{N_{A}(s_{j}+\pi)}{2}}{\sin(s_{j}+\pi)}\right]\prod_{j=1}^{2n}(Z_{k_{j}}^{0}Z_{k_{j-1}+\pi}^{0*}+e^{i(s_{j}+\pi)}Z_{k_{j}}^{1}Z_{k_{j-1}+\pi}^{1*})
\end{align*}
In the above calculation, we have assumed $N_{A}$ to be even for simplicity. We also emphasize that $s_{1}\cdots s_{2n}$ is not independent since $\sum_{i=1}^{2n}s_{i}=0$.
The factor $\frac{\sin\frac{N_{A}(s_{j}+\pi)}{2}}{\sin(s_{j}+\pi)}$
will be dominated by the contribution from $s_{j}=0$ or $s_{j}=\pi$
if $Z_{k}$ is smooth enough. This is due to the convergence of Fourier
series, see \citep{stein2003fourier}. In \citep{Kress1998}
the difference between the discrete sum over momenta and the integration is also upper bounded. However, if $s_{j}\simeq%\approx
0$, it will lead to 
\begin{equation}
Z_{k_{j}}^{0}Z_{k_{j-1}+\pi}^{0*}+e^{i(s_{j}+\pi)}Z_{k_{j}}^{1}Z_{k_{j-1}+\pi}^{1*}\simeq%=
Z_{k_{j}}^{0}Z_{k_{j}+\pi}^{0*}-Z_{k_{j}}^{1}Z_{k_{j}+\pi}^{1*}=0
\end{equation}
due to the unitarity of $U$. In the end, %conclusion, 
we obtain 
\begin{align}
\sum_{m_{1,2\cdots2n}}\prod_{j=1}^{2n}e^{ik_{j}(m_{j}-m_{j+1})}e^{i\pi m_{j}}Z_{k_{j}}^{m_{j}}Z_{k_{j-1}+\pi}^{m_{j}*} & \simeq(Z_{k}^{0}Z_{k}^{0*}+Z_{k}^{1}Z_{k}^{1*})^{2n}\prod_{j=1}^{2n}\left[\frac{\sin\frac{N_{A}(s_{j}+\pi)}{2}}{\sin(s_{j}+\pi)}\right]\,s_{j}\neq 0\nonumber \\
 & =2^{2n}\prod_{j=1}^{2n}\left[\frac{\sin\frac{N_{A}(s_{j}+\pi)}{2}}{\sin(s_{j}+\pi)}\right]\ s_{j}\neq 0
 \label{eq:Summing_Over_position}
\end{align}
 Now we can see in the final expression Eq. (\ref{eq:Summing_Over_position}) that the model-dependent factor %term 
$Z$ disappears. Therefore, those Hamiltonians
satisfying the conditions in %main 
Theorem 2 in the main text will have the same Feynman
rules and dynamical Page curve %entropy 
as the minimal model.


In Fig.~\ref{Taylor_completed_range3},
we plotted the dynamical Page curve for the Hamiltonian 
\begin{equation}
H=\sum_{i}a_{i}^{\dagger}a_{i+1}+0.3\sum_{i:\mathrm{even}}a_{i}^{\dagger}a_{i+3}-0.3\sum_{i:\mathrm{odd}}a_{i}^{\dagger}a_{i+3}+\mathrm{H.C.}.\label{eq:period2_range2_hamiltonian}
\end{equation}
This dynamical Page curve is nearly the same as the one of minimal
model in the main text.

\begin{figure}
\includegraphics[width=0.8\textwidth]{figs/period2range3Taylor_complete}

\caption{The dynamical Page curve for Hamiltonian (\ref{eq:period2_range2_hamiltonian})
(blue curve) and its comparison with the one for the RFG ensemble (red
curve) and the theoretical result (green curve). Here $N=200$. The
theoretical result is truncated up to order $\mathcal{O}(f^{4})$,
the same as in the main text. }

\label{Taylor_completed_range3}
\end{figure}

\subsection{Generalization to the %for 
atypical Page curves}

In this subsection, we %will 
further consider the case beyond the condition
in %main 
Theorem 2 in the main text, namely $\Tr(\rho Q_{k}^{\dagger}Q_{k})\neq\frac{1}{2}$
(we recall that $Q_{k+\pi}=P_{k}$). We denote $n_{k}=\Tr(\rho Q_{k}^{\dagger}Q_{k})$
and $\eta_{k}=\sqrt{n_{k}(1-n_{k})}$. Following the half filling condition,
we have 
\[
n_{k+\pi}=1-n_{k},\;\;\;\;\eta_{k+\pi}=\eta_{k}
\]
and still $a_{m}^{\dagger}=\frac{1}{\sqrt{N}}\sum_{k=0}^{2\pi}Z_{k}^{m}Q_{k}^{\dagger}e^{ikm}$
with $Z^m_{k}$ defined in Eq.~(\ref{eq:Zkm}). %previous subsection. 
The covariance matrix
can be calculated as
\begin{align*}
[C_A]_{ml} & =\frac{1}{N}\sum_{k_{1,2}}Z_{k_{1}}^{m}Z_{k_{2}}^{l*}e^{ik_{1}m}e^{-ik_{2}l}\Tr(\rho Q_{k_{1}}^{\dagger}Q_{k_{2}})\\
 & =\frac{1}{N}\sum_{k}Z_{k}^{m}Z_{k}^{l*}e^{ik(m-l)}n_{k}+\frac{1}{N}\sum_{k}e^{i\theta_{k}}Z_{k}^{m}Z_{k+\pi}^{l*}e^{ik(m-l)}e^{i\pi l}\eta_{k}.
\end{align*}
Since $n_{k}\neq\frac{1}{2}$, in general there is no simple expression
for $X_{A}$. 

Using the same techniques as in the previous subsection, we can still
establish the Feynman rules for this case. However, we have to distinguish
two kinds of legs, one like %represents 
$Z_{k_{j}}^{m_{j}}Z_{k_{j}}^{m_{j+1}*}e^{ik_{j}(m_{j}-m_{j+1})}n_{k_{j}}$
and the other like %represents 
$e^{i\theta_{k_{j}}}Z_{k_{j}}^{m_{j}}Z_{k_{j}+\pi}^{m_{j+1}*}e^{ik_{j}(m_{j}-m_{j+1})}e^{i\pi m_{j+1}}\eta_{k_{j}}$.
There is no dynamical phase $e^{i\theta_{k}}$ in the former, namely no delta functions associated with contraction. Also, there is no extra $e^{i\pi m_{j+1}}$ phase term in the former.
Due to the difference between these two kinds of legs, the rule is more complicated than the previous case. 

As an example, we can calculate the first three non-trivial terms to obtain:
\begin{align*}
\overline{\Tr C_{A}^{2}} & =\frac{N_{A}}{N}\sum_{k}n_{k}^{2}+\frac{N_{A}^{2}}{N^{2}}\sum_{k}\eta_{k}^{2},
\end{align*}
\[
\overline{\Tr C_{A}^{3}}=\frac{N_{A}}{N}\sum_{k}n_{k}^{3}+\frac{3N_{A}^{2}}{N^{2}}\sum_{k}n_{k}\eta_{k}^{2},
\]
\[
\overline{\Tr C_{A}^{4}}=\frac{N_{A}}{N}\sum_{k}n_{k}^{4}+4\frac{N_{A}^{2}}{N^{2}}\sum_{k}n_{k}^{2}\eta_{k}^{2}+2\frac{N_{A}^{2}}{N^{2}}\eta_{k}^{4}+\frac{2N_{A}^{3}}{N^{3}}\sum_{k}\eta_{k}^{4}-\frac{N_{A}^{4}}{N^{4}}\sum_{k}\eta_{k}^{4}.
\]
Therefore,  Up to $\overline{\Tr X_{A}^{4}}$, \footnote{Noting that in higher $n-$expansion of $\overline{\Tr X_{A}^{2n}}$, there
will also be contributions to the entropy density $\frac{\overline{S_A}}{N}$ at the order of $\mathcal{O}(f)$, like the term $\frac{N_{A}}{N}\frac{\sum_k n_{k}^{2n}}{N}$. Nonetheless, the convergence of expansion is guaranteed by $X_A^{2n}\leq X_A^{2n-2}$}
\[\overline{S_{A}}\simeq\frac{N_{A}(\ln2+\frac{3}{4})-\frac{N_{A}}{N}\sum_{k}(4n_{k}^{2}-\frac{8}{3}n_{k}^{3}+\frac{4}{3}n_{k}^{4})-\frac{N_{A}^{2}}{N^{2}}\sum_{k}(4\eta_{k}^{2}-8n_{k}\eta_{k}^{2}+\frac{16}{3}n_{k}^{2}\eta_{k}^{2}+\frac{8}{3}\eta_{k}^{4})-\frac{N_{A}^{3}}{N^{3}}\frac{8}{3}\sum_{k}\eta_{k}^{4}}{\ln2}.
\]

\section{Calculation of entanglement Entropy in the Quasi-Particle Picture}

In the quasi-particle picture, a nonequilibrium %pre-quenched 
initial state is a source for generating quasi-particles
with opposite momenta, which travel ballistically through the system.
Here the main assumption is %quasi particle picture 
 those quasi-particle pairs generated %generating
at different locations and times are incoherent. Therefore, the entanglement entropy
of subsystem $A$ is proportional to the number of pairs shared between
$A$ and its complement. Without loss of generality, we can assume
the subsystem $A$ is located in $[0,N_{A})$, $N_{A}\leq\frac{N}{2}$.
For a certain type of pairs with velocity $\pm v(k)$ ($v(k)>0$),
if the right-end of the pair is at position $0\leq x<N_{A}$, i.e.
within subsystem $A$, only when $x$ satisfies
\[
N_{A}-N\leq x-2v(k)t<0
\]
can this pair contribute to the entanglement entropy of $A$. Here the periodic
boundary condition is taken into account and $2v(k)t$ should be understood
as modulo $N$. The solutions of this inequality is a continuous range
$x\in[x_{\mathrm{min}},x_{\mathrm{max}})$, where $x_{\mathrm{min}}=\max\{0,2v(k)t+N_{A}-N\}$ and 
$x_{\mathrm{max}}=\min\{N_{A},2v(k)t\}$. Accordingly, we obtain
\[
\Delta x=x_{\mathrm{max}}-x_{\mathrm{min}}=\begin{cases}
2v(k)t, & 2v(k)t\leq N_{A};\\
N_{A}, & N_{A}<2v(k)t<N-N_{A};\\
N-2v(k)t, & 2v(k)t\geq N-N_{A}.
\end{cases}
\]
%The 
A similar argument holds if the left-end is in subsystem $A$. We
assume that after a sufficiently long %for long enough 
time, the quasi-particle pairs will distribute
uniformly among the system. Hence, %Thus, 
the contribution of %that kind of
quasi-particle pairs with momentum $k$ to the entanglement entropy upon the long-time average is given by %after time average is
\[
%S_{A}=
\frac{S_{A}(k)}{N^{2}}\int_{0}^{N}d(2v(k)t)\Delta x=S_{A}(k)\left[\frac{N_{A}}{N}-\left(\frac{N_{A}}{N}\right)^{2}\right],
\]
where the coefficient $S_A(k)$ is to be determined. Summing over all types of pairs, we obtain 
\[
S^{\rm qp}_{A}=\left(\frac{N_{A}}{N}-\frac{N_{A}^{2}}{N^{2}}\right)\sum_{k}S_{A}(k).
\]
If $N_{A}\to0$, the limit $S_{A}^\mathrm{qp}\to N_{A}\sum_{k}\frac{H(n_{k})}{N}$
should hold \citep{Alba2018}. Therefore, $S_{A}(k)=H(n_{k})$. If
$n_{k}=\frac{1}{2}$ for all $k$s, the entanglement entropy for subsystem
$A$ is
\[
S^{\rm qp}_{A}=N_{A}-\frac{N_{A}^{2}}{N},
\]
which deviates considerably from %is in stark contrast with 
the dynamical Page curve %as 
discussed in the main text.

