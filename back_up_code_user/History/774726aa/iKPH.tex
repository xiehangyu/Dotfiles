\documentclass{beamer}
\mode<presentation>
{
  \usetheme{ldv}
  %\setbeamercovered{transparent}
  \setbeamercovered{dynamic}
}

% Uncomment this if you're giving a presentation in german...
%\usepackage[ngerman]{babel}

% ...and rename this to "Folie"
\newcommand{\slidenomenclature}{Slide}


\usepackage[utf8]{inputenc}
\usepackage{amsmath,amssymb,amsfonts}
\usepackage{times}
\usepackage{graphicx}
\usepackage{bm}
\usepackage{fancyvrb}
\usepackage{array}
\usepackage{braket}
\usepackage{colortbl}
\usepackage{tabularx}

% Uncomment me when you need to insert code
\usepackage{color}
\usepackage{listings}
% End Code

% Uncomment me when you need video or sound
\usepackage{multimedia}
\usepackage{hyperref}
% End video

% Header
\newcommand{\zwischentitel}{Free-Fermion Page Curve}
\newcommand{\leitthema}{Xiehang Yu, Zongping Gong, Ignacio Cirac}
% End Header

% Titlepage
\title{Charge-conserved Free-fermion Page Curve :}
\author{Xiehang Yu}
\newcommand{\presdatum}{17th, Nov. 2022}
\institute
{
  Quantum Science and Technology (from 10.2021)\\
}
\subtitle{Canonical Typicality and Dynamical Emergence}
% End Titlepage


% Slides
\begin{document}

% 1. Slide: Titlepage
\begin{frame}
	Charge-conserved Free-fermion (CRFG) Ensemble:

 
        An abitrary random realization 

        
        where $T(\rho_{\phi},\Omega_{A})=||\rho_{\phi}-\Omega_{A}||_{1}/2$ is the trace distance and
\end{frame}

% 2. Slide: TOC
\begin{frame}
   \frametitle{Table of contents}
   \tableofcontents[subsectionstyle=hide]
\end{frame} 

% Further Slides
\section{Titles} 
\begin{frame}
   \frametitle{Title} 
A Lipschitz continuous function $f:\mathbb{U}(N)\to\mathbb{R}$ satisfying
\[
|f(x)-f(y)|\leq l \|x-y\|_{\mathrm{HS}}\ \mathrm{for}\ \forall x,y\in\mathbb{U}(N),
\]
will concentrate on its mean value as
\[
\mathbb{P}(|f(x)-\mathbb{E}(f)|\geq l \epsilon) \leq 2\exp\left(-\frac{N\epsilon^2}{12}\right)
\]
Here $\|x\|_\mathrm{HS}=\sqrt{\mathrm{Tr}(x^{\dagger}x)}$ is the Hilbert-Schmidt norm.
\begin{center}
To prove the typicality, we choose
\[f(U)=\|C_{A}(U)-\mathbb{E}(C_{A})\|_{\mathrm{HS}}=\|\Pi_{A}UC_{0}U^{\dagger}\Pi_{A}^{\dagger}-\mathbb{E}(C_{A})\|_{\mathrm{HS}}\]
\end{center}
where $\Pi_{A}$ is the projection operator onto the subsystem $A$. 

\end{frame}



\subsection{Subsection no.1.1  }
\begin{frame} 
   To prove the typicality, we choose
\[f(U)=\|C_{A}(U),\mathbb{E}(C_{A})\|_{\mathrm{HS}}=\|\Pi_{A}UC_{0}U^{\dagger}\Pi_{A}^{\dagger},\mathbb{E}(C_{A})\|_{\mathrm{HS}}\]
where $\Pi_{A}$ is the projection operator onto the subsystem $A$. 


We can verify $f(U)$ is Lipschitz continuous with $l=2$ 
\[
   \mathbb{E}(C_{A})=\frac{1}{2}I_{A},\ \ \ \ \mathbb{E}(f)\leq\sqrt{\frac{N_{A}^{2}}{2(N-1)}} 
   \]
The typicality then follows
\end{frame}

\begin{frame}
   On the other hand, if we choose \[f(U)=\|C_{A}-\mathbb{E}(C_{A})\|_{\mathrm{HS}}^{2}\]

$f(U)$ is also Lipschitz continuous with $l=2\sqrt{N_{A}}$ and

\[\mathbb{E}(f)\geq\frac{N_{A}^{2}}{4(N+1)}.\]

This can be used to prove the atypicality property.
\end{frame}
\begin{frame}
   Characterizing by Hilbert-Schmidt distance $d_{\mathrm{HS}}(x,y)=\|x-y\|_{\mathrm{HS}}$,

\[\mathbb{P}\left(d_{\mathrm{HS}}(C_{A},I_{A}/2)\geq\sqrt{\frac{N_{A}^{2}}{2(N-1)}}+2\epsilon\right)\leq2\exp\left(-\frac{N\epsilon^{2}}{12}\right)\]

and 

\[\mathbb{\mathbb{P}}\left(d_{\mathrm{HS}}^{2}(C_{A},I_{A}/2)\leq\frac{N_{A}^{2}}{4(N+1)}-2\epsilon\right)\leq2\exp\left(-\frac{\epsilon^{2}N}{12N_{A}}\right).\]

\end{frame}
\begin{frame}
   $f=\frac{N_A}{N}=\mathcal{O}(1)$
   $\epsilon=\mathcal{O}(N^{\alpha}),\ \alpha\in(0,1)$
\end{frame}
\begin{frame}
   \[
      S_A\geq N_A-4d^2_{\mathrm{HS}}\left(C_A,\frac{I_A}{2}\right)
   \]
   \[
      \mathbb{P}(N_{A}-S_{A}\geq\epsilon) \leq\begin{cases}
2\exp\left[-\frac{N}{48}\left(\frac{\sqrt{\epsilon}}{2}-\sqrt{\frac{N_{A}^{2}}{2(N-1)}}\right)^{2}\right], & \epsilon>\frac{2N_{A}^{2}}{N-1};\\
1, & \epsilon\leq\frac{2N_{A}^{2}}{N-1}.
\end{cases}
      \]
   \[
      S_A\leq N_A-\frac{2}{\ln{2}}d_{\mathrm{HS}}^2\left(C_A,\frac{I_A}{2}\right)
      \]
      \[
         \mathbb{P}\left(S_{A}\geq N_{A}-\frac{N_{A}^{2}}{2\ln2(N+1)}+\frac{2}{\ln2}\epsilon\right)  \leq2e^{-\frac{N}{48N_{A}}\epsilon^{2}}.
         \]
\end{frame}
\begin{frame}
   \begin{align*}
      H=\mathrm{Minimal\ Model}+H_1 
   \\
      J(\sum_{j:\mathrm{even}}a_{j}^{\dagger}a_{j+2m+1}-\sum_{j:\mathrm{odd}}a_{j}^{\dagger}a_{j+2m+1})+\mathrm{H.c.}
   \end{align*}
      hhhhhhh
\end{frame}
\begin{frame}
      \[H=(\sum_{j=1}^{N}a_{j}^{\dagger}a_{j+1}+0.3\sum_{j:\mathrm{even}}a_{j}^{\dagger}a_{j+2}-0.3\sum_{j:\mathrm{odd}}a_{j}^{\dagger}a_{j+2})+\mathrm{H.c}.\]
\end{frame} 
\begin{frame}
   $\mathbb{E}(C_{A})=\frac{1}{2}I_{A}$, $\mathbb{E}(a_k^{\dagger}a_k)=\frac{1}{2}$
   $\langle a_k^{\dagger}a_k\rangle=\frac{1}{2}$
   $\langle a_k^{\dagger}a_k\rangle\neq\frac{1}{2}$
   $J=0.3$, $m=1$
   \[
      S_A(C_A)=-\mathrm{Tr}C_A\log_2{C_A}-\mathrm{Tr}(I_A-C_A)\log_2{(I_A-C_A)}
      \]
      \[
         =N_A-\sum_{n=1}^\infty \frac{\mathrm{Tr}X_A^{2n}}{2n(2n-1)\ln{2}}
         \]
         $X_A=2C_A-I_A$
\end{frame}
\begin{frame}
   \[   C_A(t)=\Pi_AU_F\tilde{C}(t)U_F^{\dagger}\Pi_A^{\dagger}\]
   
   \begin{equation*}
   \tilde{C}(t)=\begin{pmatrix}\tilde{C}_{k_{1}}(t)\\
 & \tilde{C}_{k_{2}}(t)\\
 &  & \tilde{C}_{k_{3}}(t)\\
 &  &  & \cdots
\end{pmatrix}
   \end{equation*}
$\tilde{C}_{k}\sim e^{i\theta_k t}$
$\theta_k=E_k-E_{k+\pi}$
\end{frame}
\begin{frame}
   $\overline{e^{it(\theta_k+\theta_k')}}$
   $k'=k+\pi$ $k'=\pi-k$
   $e^{ik_1(m_1-m_2)}e^{i\pi m_2}$
\end{frame}
\begin{frame}
   \[\overline{\mathrm{Tr} X_{A}^{2}}= \frac{1}{N^{2}}\sum_{k_{1},k_{2},m_{1},m_{2}}\overline{e^{i\theta_{k_{1}}t}e^{ik_{1}(m_{1}-m_{2})}e^{i\pi m_{2}}e^{i\theta_{k_{2}}t}e^{ik_{2}(m_{2}-m_{1})}e^{i\pi m_{1}}}
= \frac{N_{A}^{2}}{N}.\]
$f=\frac{N_A}{N}$
\begin{equation}
\frac{\overline{S_{A}}}{N}=f-\frac{1}{\ln{2}}\left(\frac{1}{2}f^{2}+\frac{1}{6}f^{3}+\frac{1}{10}f^{4}\right)+\mathcal{O}(f^{5}).\label{eq:the_Page_curve_for_NNH}
\end{equation}
\begin{equation}
\frac{\mathbb{E}(S_{A})}{N}=f-\frac{1}{\ln{2}}\left(\frac{1}{2}f^{2}+\frac{1}{6}f^{3}+\frac{1}{12}f^{4}\right)+\mathcal{O}(f^{5}).\label{eq:the_Page_curve_result_for_CCRFG}
\end{equation}
\end{frame}
\begin{frame}
   $S_{A}=N_A-\sum_{n=1}^\infty \frac{\mathrm{Tr}X_A^{2n}}{2n(2n-1)\ln{2}}\geq N_{A}-\frac{\mathrm{Tr}(X_{A})^{2}}{\ln2}\sum_{n}\frac{1}{2n(2n-1)}=N_{A}-\frac{N_{A}^{2}}{N}$
   $X_{A}^{2n} \leq X_A^2,\ n>1$
\end{frame}
\begin{frame}
$N_A=N/2$
$\langle a_i^{\dagger}a_j^{\dagger}\rangle$
\end{frame}
\section{Lists} 
\subsection{Lists I}
\begin{frame}
   \frametitle{Unnumbered Lists}
   \begin{itemize}
      \item Introduction to  \LaTeX       \item Second bullet point 
      \item And a third one
      \item The last one
   \end{itemize} 
\end{frame}

\begin{frame}
   \[
\mathbb{P}((S_{A}-N_{A})^{2}\geq x)\leq\begin{cases}
1, & x\leq\frac{4N_{A}^{4}}{(N-1)^{2}};\\
2\exp\left[-\frac{N}{48}\left(\frac{x^{\frac{1}{4}}}{2}-\sqrt{\frac{N_{A}^{2}}{2(N-1)}}\right)^{2}\right], & x>\frac{4N_{A}^{4}}{(N-1)^{2}}.
\end{cases}
\]
\[
\mathrm{Var}(S_{A})\leq\langle(S_{A}-N_{A})^{2}\rangle=\mathcal{O}(N^{-2})\ \mathrm{if\ we\ keep}\ N_{A}\ \mathrm{fixed}.,
\]
\end{frame}
\begin{frame}
   \begin{align*}
&\langle(S_{A}-N_{A})^{2}\rangle =\int\mathbb{P}((S_{A}-N_{A})^{2}\geq x)dx\\
 & \leq\frac{4N_{A}^{4}}{(N-1)^{2}}+2\int_{\frac{4N_{A}^{4}}{(N-1)^{2}}}^{\infty}dx\exp\left[-\frac{N}{48}\left(\frac{x^{\frac{1}{4}}}{2}-\sqrt{\frac{N_{A}^{2}}{2(N-1)}}\right)^{2}\right].
\end{align*}
$e^{-\mathcal{O}(N)}$
\end{frame}
\begin{frame}
   $\mathrm{dim}^{\mathrm{eff}} \bm{E}=\frac{1}{\mathrm{Tr}_{E}(\mathrm{Tr}_{A}\epsilon_{S})^{2}}\geq\frac{\mathrm{dim} \bm{S}}{\mathrm{dim} \bm{A}}$
   $2^{N_A}, 2^N$
   \begin{equation*}
   \mathbb{P}\left(\left\{ \ket{\phi}|T(\rho_{\phi},\Omega_{A})\geq\epsilon+\frac{1}{2}\sqrt{\frac{\mathrm{dim} \bm{A}}{\mathrm{dim}^{\mathrm{eff}} \bm{E}}}\right\} \right)\leq4\exp{\left(-\frac{2\epsilon^2}{9\pi^{3}}\mathrm{dim} \bm{S}\right)}
   \end{equation*}
   $\mathcal{H}_S=\mathcal{H}_A\otimes\mathcal{H}_E$
   $\rho_{\phi}=\mathrm{Tr}_E\ket{\phi}\bra{\phi}$
   $\ket{\phi}\in\mathcal{H}_S$
   $\Omega_{A}=\mathrm{Tr}_E\epsilon_S$
   $\epsilon_S=\frac{\mathbb{I}_S}{d_S}$
\end{frame}
\begin{frame}
   $\langle a_{i}^{\dagger}a_{j}^{\dagger}\rangle=\langle a_{i}a_{j}\rangle=0$
    $\langle\cdots\rangle=\mathrm{Tr}(\rho\cdots)$
    $C_{i,j}=\langle a_i^\dag a_{j}\rangle$
    $\{C=UC_{0}U^{\dagger},U\in\mathbb{U}(N)\}$
    $S_{A}(C_{A})=-\mathrm{Tr} C_{A}\log_{2}C_{A}-\mathrm{Tr}(I_{A}-C_{A})\log_{2}(I_{A}-C_{A})$
    $\mathbb{E}(S_{A})=\int d_{\mathrm{H}}(U)S_{A}(C_{A}(U))$
$C_{0}=\begin{pmatrix}I_{\frac{N}{2}} & 0\\
0 & 0
\end{pmatrix}$
$C_{A}=\Pi_A C \Pi_A^{\dagger}$
\end{frame}
\begin{frame}
   \[
      \overline{S_A}=\lim_{T\to\infty}\int_0^T dt S_{A}(t)
   \]
   \[
      \ket{\psi}\in\mathbb{H}(2^N)
   \]
   \[
      
   \]
\end{frame}
\begin{frame}\frametitle{Lists with Pause}
   \begin{itemize}
      \item Introduction to  \LaTeX \pause 
      \item Second bullet point \pause 
      \item And a third one \pause 
      \item The last one
   \end{itemize} 
\end{frame}



\subsection{Lists II}
\begin{frame}
   \frametitle{Numbered Lists}
   \begin{enumerate}
      \item Introduction to  \LaTeX  
      \item Second bullet point 
      \item And a third one
      \item The last one
   \end{enumerate}
\end{frame}



\begin{frame}
   \frametitle{Numbered Lists with Pause}
   \begin{enumerate}
      \item Introduction to  \LaTeX \pause 
      \item Second bullet point \pause 
      \item And a third one \pause 
      \item The last one
   \end{enumerate}
\end{frame}



\section{Tables} 
\subsection{Tables}
\begin{frame}
   \frametitle{Tables}
   \begin{tabular}{|c|c|c|}
      \hline
      \textbf{Title} & \textbf{Title} & \textbf{Title} \\
      \hline
      First column & Second column &  \LaTeX  \\
      \hline
      Second row & \LaTeX & Last column \\
      \hline
   \end{tabular}
\end{frame}



\section{Blocks \& Math}
\subsection{Blocks}
\begin{frame}
   \frametitle{Blocks}

   \begin{block}{This is a simple block}
      It should contain some text.
   \end{block}

   \begin{exampleblock}{Example Block}
      This may be an example.
      $N_A$
   \end{exampleblock}


   \begin{alertblock}{Warning}
      The violent color indicates that this block may alert of something.
   \end{alertblock}
\end{frame}



\subsection{Math} 
\begin{frame}
   \frametitle{Math Expressions are a Breeze with \LaTeX \ldots}

   \begin{equation}
      p(\mathbf{x}_{k}|\mathbf{Z}_{k}) = \frac{p(\mathbf{z}_{k}|\mathbf{x     }_{k})p(\mathbf{x}_{k}|\mathbf{Z}_{k-1})}{\int \! p(\mathbf{z}_{k}|\mathbf{     x}_{k})p(\mathbf{x}_{k}|\mathbf{Z}_{k-1})\,d\mathbf{x}_{k}}
   \end{equation}

   \begin{equation}
      w_{k}^{i} \sim w_{k-1}^{i}\frac{p(\mathbf{z}_{k}|\mathbf{x}_{k}^{i})p(\mathbf{x}_{k}^{i}|\mathbf{x}_{k-1}^{i})}{q(\mathbf{x}_{k}^{i}|\mathbf{x}_{k-1}^{i},\mathbf{z}_{k})}
   \end{equation}

   \ldots and Bayes filtering is great!

\end{frame}



\section{Colors}
\begin{frame}
   \frametitle{An Overview of TUM's Colors}
   \tiny{Color blue sRGB 100\%: 0-101-189} \\
   \colorbox{tumcolor-blue}{}{}  \\
   \vspace{0.5cm}
   \tiny{Color green sRGB 100\%: 162-173-0} \\
   \colorbox{tumcolor-green}{}{} \\
   \vspace{0.5cm}
   \tiny{Color light grey sRGB 100\%: 218-215-203} \\
   \colorbox{tumcolor-lightgrey}{}{} \\
   \vspace{0.5cm}
   \tiny{Color orange sRGB 100\%: 227-114-34} \\
   \colorbox{tumcolor-orange}{}{} \\
   \vspace{0.5cm}
\end{frame}



\section{Multimedia}
\subsection{Split Screen}
\begin{frame}
   \frametitle{Splitting Screen}
   \begin{columns}

      \begin{column}{5cm}
         \begin{itemize}
            \item Here
            \item is some
            \item text
         \end{itemize}
      \end{column}

      \begin{column}{5cm}
         \begin{tabular}{|c|c|}
            \hline
            \textbf{On the} & \textbf{other side} \\
            \hline
            there may &  be a table \\
            \hline
            or even &  a picture as  \\
            \hline
            shown on the &  next frame  \\
            \hline
         \end{tabular}
      \end{column}

   \end{columns}
\end{frame}



\subsection{Pictures} 
\begin{frame}
   \frametitle{Pictures in Latex Beamer Class}
   \begin{figure}
      \includegraphics[scale=2.0]{images/video} 
      \caption{This is a picture!}
   \end{figure}
\end{frame}



\subsection{Pictures which Need More Space} 
\begin{frame}[plain]
   \frametitle{Plain, or a Way to Get More Space}
   \begin{figure}
      \includegraphics[scale=3.0]{images/video} 
      \caption{Picture again.}
   \end{figure}
\end{frame}



\subsection{Code Listings} 
% The [fragile] is important for listings!
\begin{frame}[fragile]
   \frametitle{Don't Ever Bore the Audience with Code Listings}
   \lstset{language=C, basicstyle=\small \ttfamily, showspaces=false, showtabs=false, tab= , keywordstyle=\bfseries, showstringspaces=false, framexleftmargin=5mm, frame=single, numbers=left, numberstyle=\tiny, stepnumber=1, numbersep=5pt, texcl=true}
   \begin{lstlisting}[caption={Especially when they are erroneous},frame=tlrb]
#include <stdio.h>

int main(void)
{
  printf("Hallo Welt\n");
  while(1);
  return 0;
}
   \end{lstlisting}
\end{frame}

% End Slides

\end{document}
