\subsection{Zero field measurement}

\subsubsection{Determination of $I_{c}$ and $I_{R}$}
The two parameters $I_{c}$ and $I_{R}$ can be determined directly from the current-voltage plot.


\begin{figure}[h]
\centering
\includegraphics[width=0.5\textwidth]{./fig/curve}
\caption{Determination of $I_{c}$ and $I_{R}$}
\label{UIcurvePicture}
\end{figure}


As shown in Fig. \ref{UIcurvePicture}, we plot the current-voltage-curve in the certain region to have a clear view of the switching of voltage,
thus from the plot $I_{c}=4.715mA$. For $I_{R}$ which is determined as a turning point of the curve, can be obtained approximately from the current-voltage-curve as $I_R=0.959mA$.
\subsubsection{Determination of $R_{n}$}
To determine $R_{n}$, we used Mathematica to get a fit of the current-voltage-curve. The fit is shown in Fig. \ref{determinationRNfigure}.

\begin{figure}[h]
\centering
\includegraphics[width=0.5\textwidth]{./fig/fit}
\caption{Determination of $R_{n}$}
\label{determinationRNfigure}
\end{figure}


From the fit, we can get the approximation value of $R_{n}$ as $R_{n}=0.253\Omega$.

\subsubsection{Calculation of McCumber parameter $\beta_{c}$}
The value of $\beta_{c}$ can be obtained from:
\begin{align*}
\frac{I_{R}}{I_{c}}=\frac{4}{\pi\sqrt{\beta_{c}}}
\end{align*}
Thus we can get $\beta_{c}$ by taking $I_{R}$ and $I_{c}$ in to the above:
\begin{align*}
\beta_c & =\frac{I_{c}^2}{I_{R}^2}\frac{16}{\pi^2}\\
    & =\frac{4.715^2}{0.959^2}\times\frac{16}{\pi^2}\\
    & =39.19
\end{align*}
The value of $\beta_c$ is larger than 1($\beta_{c}>1$), which means that the junction is in the underdamped regime.

\subsubsection{Calculation of the capacitance of the junction}
From the equation of McCumber-parameter, we can get the capacitance of the junction as:
\begin{align*}
    \beta_c=\frac{2\pi{C}{R_{n}^2}I_{c}}{\phi_{0}}
    C=\frac{\beta_c\phi_{0}}{2\pi{R_{n}^2}I_{c}}
\end{align*}
Thus, by taking the value of $\beta_{c}$, $\phi_{0}$, $R_{n}$ and $I_{c}$ in to the equation we can get the capacitance of the junction as:
\begin{align*}
    C=\frac{39.19\times2.0678\times10^{-15}}{2\pi\times0.253^2\times4.715\times10^{-3}A}\\
    C=4.271\times10^{-11}F
\end{align*}
In comparison, the capacitance of the capacitor with area of $L^2$, distance of $d$ and $AlO_{x}$ as dieletric is:
\begin{align*}
    C & =\frac{\epsilon\epsilon_{0}S}{d}\\
    & =\frac{9.34\times8.854\times10^{-12}\times{19.5\times10^{-6}}^2}{17.83\times10^{-9}}\\
    & =1.764\times10^{-12}F
\end{align*}

\subsection{Field dependent measurements}
\subsubsection{Plotting of $I_{c}$}

Since in our experiment, we can read out the voltage which controls
the magnitude of the flux inside the junction, our first task is to
determine the relation between the voltage and the flux $\Phi$:
\begin{align*}
\Phi & =t_{B}LB\\
 & =177.83\times19.5\times\frac{7.95}{200}\times0.1V\\
 & =13.78V
\end{align*}
where the unit of $V$ is volt and the unit of $\Phi$ is $10^{-15}\mathrm{Wb}$.
$I_{c}$ can be determined similar as in previous subsection. Therefore,
the behavior of $I_{c}$ with varying flux inside the junction can
be obtained in Fig. (\ref{Icwithvaryingflux})

\begin{figure}
\includegraphics[width=0.8\textwidth]{fig/Icwithvaryingflux}

\caption{The behavior of $I_{c}$ with varying flux inside the junction}

\label{Icwithvaryingflux}
\end{figure}
From the figure, we can clearly see that $I_{c}$ is oscillating with
$\Phi$ with a fixed period. Besides, there is a total decaying envelope
which suppress the amplitude of oscillation when increasing $\Phi$.

\subsubsection{Determine $\Phi_{0}$}

From the figure, we can extract that the periodicity of $I_{c}$ with
flux $\Phi$ is $\Delta\Phi=1.59\times10^{-15}\mathrm{Wb}$. According
to the formula in Sec. 2.2.6 in of the manual, this periodicity is
just the quantized flux unit $\Phi_{0}$. Therefore, we can conclude
from our experiment that the measured
\[
\Phi_{0}=1.59\times10^{-15}\mathrm{Wb}
\]


\subsubsection{Fitting of $I_{c}$}

The fitting result is shown in Fig. (\ref{Fitting_of_IC_varying_flux}).
The fitting parameter can be seen in Table. (\ref{fitting_parameter_table})

\begin{figure}
\includegraphics[width=0.8\textwidth]{fig/Fitting_varyingflux}

\caption{Fitting of $I_{c}$ with varying flux inside the junction}

\label{Fitting_of_IC_varying_flux}
\end{figure}

\begin{table}
\begin{center}
\begin{tabular}{|c|c|}
\hline 
\multicolumn{2}{|c|}{Fitting Function: $f(x)=a\times|\frac{\sin\frac{\pi x}{b}}{\frac{\pi x}{b}}|$}\tabularnewline
\hline 
\hline 
Fitting Parameters & Value\tabularnewline
\hline 
$a$ & $47.68(46.87,48.49)$\tabularnewline
\hline 
$b$ & $1.59(1.57,1.61)$\tabularnewline
\hline 
$R^{2}$ & $0.9952$\tabularnewline
\hline 
\end{tabular}
\end{center}

\caption{Fitting parameters of $I_{c}$ with varying flux inside the junction}

\label{fitting_parameter_table}
\end{table}


\subsubsection{Some discussions}

The literature of $\Phi_{0}=2.07\times10^{-15}\mathrm{Wb}$, larger
than our result. We propose the difference may come from the following
ways:
\begin{itemize}
\item In our experiment, instead one jump of volt, we noticed several jumps
of the volt (at most three jumps of the volt). This is a different
behavior from expected. This behavior may be due to some defects in
the junction. In other word, when the sample was growed, it may have
some defects which separate the sample into disjoint sections. Each
section become a superconductivity separately. Therefore, among different
sections, there will be normal currents besides the superfluid currents
and multi jumps have been observed. This phenomenon will lead to some
errors when we try to determine the critical currents $I_{c}$ of
the sample. Other non-uniform property of the sample will also cause
to this phenomenon.
\item We noticed that after the second minimum, the critical currents are
much higher than expected. This may be caused by a similar reason
as the previous point: In those cases, there will be multi jumps,
but most of them are too small to read out. Therefore, we just discover
one jump and omit others. However, it is possible that the true critical
currents should just be determined by the jumps we omit. As a result,
after the second minimum, the value of critical currents may be determined
incorrectly.
\item We also noticed that in our experiment, the voltage of the coil is
unstable, which leads to an unstable $\Phi$ during the measurement.
Indeed, when we measure the $U-I$ curve, the voltage of coil will
increase. This leads to an underestimate of $\Phi_{0}$, corresponding
to our result. This increase may be caused by some mutual inductance.
\item The sample may be heated in the measurement, which will cause some
errors to the measured $U-I$ curve and lead to the errors when determining
$I_{c}$.
\end{itemize}
