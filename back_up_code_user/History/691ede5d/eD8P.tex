\documentclass[english]{ctexart}
\usepackage[T1]{fontenc}
\usepackage{geometry}
\geometry{verbose,lmargin=1cm,rmargin=1cm}
\setcounter{secnumdepth}{2}
\setcounter{tocdepth}{2}
\usepackage{amsmath}
\usepackage{graphicx}
\usepackage[authoryear]{natbib}

\makeatletter

%%%%%%%%%%%%%%%%%%%%%%%%%%%%%% LyX specific LaTeX commands.
%% Because html converters don't know tabularnewline
\providecommand{\tabularnewline}{\\}

%%%%%%%%%%%%%%%%%%%%%%%%%%%%%% Textclass specific LaTeX commands.
\newenvironment{lyxcode}
	{\par\begin{list}{}{
		\setlength{\rightmargin}{\leftmargin}
		\setlength{\listparindent}{0pt}% needed for AMS classes
		\raggedright
		\setlength{\itemsep}{0pt}
		\setlength{\parsep}{0pt}
		\normalfont\ttfamily}%
	 \item[]}
	{\end{list}}

%%%%%%%%%%%%%%%%%%%%%%%%%%%%%% User specified LaTeX commands.
\usepackage{braket}
\usepackage{tikz}
%\usepackage{braket}
%\usepackage{braket}
\usepackage{listings}
\usepackage{xcolor}
\usepackage{color}
\usepackage{diagbox}
\lstset{
numbers=left,
framexleftmargin=10mm,
frame=none,
keywordstyle=\bf\color{blue},
identifierstyle=\bf,
numberstyle=\color[RGB]{0,192,192},
commentstyle=\it\color[RGB]{0,96,96},
stringstyle=\rmfamily\slshape\color[RGB]{128,0,0}
}

%\usetheme{Darmstadt}
%\usetheme{Frankfurt}
% or ...

%\setbeamercovered{transparent}
\lstdefinelanguage
   [x64]{Assembler}     % add a "x64" dialect of Assembler
   [x86masm]{Assembler} % based on the "x86masm" dialect
   % with these extra keywords:
   {morekeywords={CDQE,CQO,CMPSQ,CMPXCHG16B,JRCXZ,LODSQ,MOVSXD, %
                  POPFQ,PUSHFQ,SCASQ,STOSQ,IRETQ,RDTSCP,SWAPGS, %
                  rax,rdx,rcx,rbx,rsi,rdi,rsp,rbp, %
                  r8,r8d,r8w,r8b,r9,r9d,r9w,r9b, %
                  r10,r10d,r10w,r10b,r11,r11d,r11w,r11b, %
                  r12,r12d,r12w,r12b,r13,r13d,r13w,r13b, %
                  r14,r14d,r14w,r14b,r15,r15d,r15w,r15b,retq,callq,cmpl}} % etc.

\lstset{language=[x64]Assembler}

\makeatother

\usepackage{babel}
\begin{document}
\title{计算机体系结构上机实验第四次报告}
\author{戴梦佳\ \ PB20511879}
\maketitle

\section{循环展开和验证}

\subsection{循环展开过程}

在循环展开时,我们的展开步长取为$4$。这是考虑到,现代计算机大多支持在一个指令中同时计算4个浮点操作数的向量操作。

另一方面,我们使用乘法分配率重写了表达式
\begin{equation}
\begin{aligned}\alpha X[i]*X[i]+\beta X[i]+X[i]Y[i] & =X[i](\alpha X[i]+\beta+Y[i])\\
\alpha Y[i-1]+Y[i]+\alpha Y[i+1] & =Y[i]+\alpha(Y[i-1]+Y[i+1])
\end{aligned}
\end{equation}
这样的好处是可以减少浮点计算次数。在第一个算式中,我们将原先的四步乘法两部加法简化为两步乘法两步加法;而在第二个算式中,我们将两步乘法两步加法简化为两步加法一步乘法。

\subsection{验证}

我们通过比较原函数和循环展开后的函数分别计算的向量$Y_{1},Y_{2}$的欧几里得距离来验证循环展开的正确性。具体说,我们计算
\begin{lyxcode}
Y1<-原函数

Y2<-循环展开后的函数

sum=0

for~i=0:N-1:
\begin{lyxcode}
sum~+=~(Y1(i)-Y2(i)){*}(Y1(i)-Y2(i))
\end{lyxcode}
return~sum
\end{lyxcode}
计算得到的结果见下图

\includegraphics[width=0.6\textwidth]{report/figs/fig1}

其中的variance就是欧几里得距离。该距离为零,就说明循环展开是正确的。

\section{实验结果}
\begin{itemize}
\item 完成循环展开后,只使用优化选项$-O1$,不额外增加Simd元件,得到的仿真结果如下所示:
\end{itemize}
\begin{tabular}{lllllll}
 & daxpy & unrolled daxpy & daxsbxpxy & unrolled daxsbxpxy & stencil & unrolled stencil\tabularnewline
CPI & 1.781713 & 2.009455 & 2.094353 & 2.420291 & 1.962189 & 2.142371\tabularnewline
执行时长(CPU周期数) & 35640500 & 35175000 & 62839500 & 54469250 & 49053250 & 45530750\tabularnewline
指令条数 & 80014 & 70019 & 120017 & 90021 & 99997 & 85010\tabularnewline
\end{tabular}

从中我们可以看出,针对不同的函数,循环展开都能起到较好的优化效果,指令时常分别缩小了$1.3\%,13\%$和$7.2\%$。
\begin{itemize}
\item 进一步,我们额外增加三个SIMD浮点处理单元,得到的仿真结果如下所示:
\end{itemize}
\begin{tabular}{lllllll}
 & daxpy & unrolled daxpy & daxsbxpxy & unrolled daxsbxpxy & stencil & unrolled stencil\tabularnewline
CPI & 1.781713 & 2.009455 & 2.011032 & 2.420291 & 1.962169 & 2.152088\tabularnewline
执行时长(CPU周期数) & 35640500 & 35175000 & 60339500 & 54469250 & 49052750 & 45737250\tabularnewline
指令条数 & 80014 & 70019 & 120017 & 90021 & 99997 & 85010\tabularnewline
\end{tabular}

从中我们可以看出,增加三个浮点处理单元后,对于原先未循环展开的函数,性能略有提升(执行时长略微下降),但不明显。对于已经循环展开后的函数,则几乎没有变化。关于这方面的原因见之后的分析讨论。
\begin{itemize}
\item 加入优化选项$-O3$, 不增加浮点处理单元的仿真结果见下所示:
\end{itemize}
\begin{tabular}{lllllll}
 & daxpy & unrolled daxpy & daxsbxpxy & unrolled daxsbxpxy & stencil & unrolled stencil\tabularnewline
CPI & 1.822573 & 2.002564 & 2.061756 & 1.398738 & 2.239271 & 3.177234\tabularnewline
执行时长(CPU周期数) & 18235750 & 15035750 & 28361000 & 16622250 & 33588500 & 33787500\tabularnewline
指令条数 & 40022 & 30033 & 55023 & 47535 & 59999 & 42537\tabularnewline
\end{tabular}

我们发现,加入$-O3$的优化选项后,性能有了明显提升。同时,循环展开后,对于第一、第二个函数,性能可以得到进一步提升,指令时长分别减小了$17.5\%,41.3\%$。但对于第三个函数,循环展开反而会降低性能。
\begin{itemize}
\item 最后,我们加入$-O3$优化选项,并且,额外增加三个SIMD单元后,得到的仿真结果如下所示:
\end{itemize}
\begin{tabular}{lllllll}
 & daxpy & unrolled daxpy & daxsbxpxy & unrolled daxsbxpxy & stencil & unrolled stencil\tabularnewline
CPI & 1.822573 & 2.002564 & 2.061756 & 1.326454 & 2.208320 & 3.031972\tabularnewline
执行时长(CPU周期数) & 18235750 & 15035750 & 28361000 & 15763250 & 33124250 & 32242750\tabularnewline
指令条数 & 40022 & 30033 & 55023 & 47535 & 59999 & 42537\tabularnewline
\end{tabular}

同样的,相比于没有额外添加浮点单元的结果,执行时长略有下降,但并不十分明显。对于第一个函数没有任何变化,对于后两个函数的循环展开版本,时长分别减小了$5\%$和$3\%$。而对于未循环展开的版本,几乎没有变化。

\section{分析讨论}
\begin{enumerate}
\item 通过计算欧几里得距离,验证循环展开的结果是否正确。
\item 对于每一个函数,除了$-O3$优化下的第三个函数,循环展开都提升了性能(减小了执行时长)。执行时常的减小幅度不仅依赖于函数本身的性质,还依赖于优化选项。在我们的粒子中,第二个函数循环展开后性能提升最明显,这是因为,这个函数具有一定的复杂性,同时又是很规整的向量运算,从而循环展开可以很好地调度并利用浮点计算单元处理。但对于第三个函数,它每一次循环都还需要调用上一步循环中的Y{[}i-1{]},从而具有非常严重的数据相关冲突,即使循环展开,也很难调度。在$-O3$优化的选项下,循环展开的效果反而会下降

循环展开理论上可以减少名相关和控制相关。
\item 在本例子中,我们将循环展开到4次,除了首尾循环节外,中间的循环节都一样。
\begin{enumerate}
\item 如果没有展开足够多的循环的话,编译器无法找到足够的指令调度,同时在每一个循环节中,浮点计算单元Simd可能也无法完全发挥它的功用。
\item 相反,如果展开太多循环的话,指令代码量极度增大。这对数据cache和指令cache都是很沉重的负担,大大增大了cache miss的可能,反而会影响性能。
\end{enumerate}
\item 增加硬件的影响既依赖于函数本身的性质,又依赖于所选取的优化选项:
\begin{itemize}
\item 对于第一个函数,增加硬件始终不会产生影响,因为它特别简单,不会产生ALU堵塞,原本设置中自带的一个浮点计算单元已经足够用了。
\item 在$-O1$优化选项下,增加硬件对第二个函数的原函数可以减小运行时间,对第三个函数循环展开后的结果反而会延长时间。这是因为,在程序中,第二个函数的原函数在每一步循环中包括了四步乘法和两步加法,运算次数非常多,从而增加硬件可以避免ALU堵塞,提升性能。但对第三个函数,函数本身存在高度数据相关,难以实现指令间并行,
流水线效率低,也就不存在ALU堵塞了。所以,增加硬件反而可能会增加了线路判断的复杂性,造成运行速率的下降。
\item 在$-O3$优化选项下,对第二个函数的循环展开版本,第三个函数的原函数和循环展开版本,都可以一定程度上减少运行时间。这可能是因为,$-O3$优化选项进一步改变了代码的结构,发现了更多可以并行执行的指令,
增大了流水线的吞吐率,相应的ALU堵塞的可能增加。从而增加硬件可以避免ALU堵塞。
\end{itemize}
添加运算、访存硬件主要减少的是结构相关冲突。但如果,我们可以添加更多物理寄存器(或者保留站,ROB之类)的话,我们就有了更多寄存器用来完成寄存器重命名的工作。所以,添加这类硬件可以减少名相关。
\item 在本实验中,执行时长作为合适的指标比较性能表现较合适。这是因为,不同指令的效果和指令本身的执行时长(CPI)都不同。例如,SIMD指令一下可以处理多个浮点数,但它的CPI较普通指令较高。所以使用SIMD指令会增加仿真的CPI,降低指令条数,总体来看依然是降低总执行时长。另一方面,指令条数也会收到循环控制指令等的影响。所以,使用总的执行时长作为衡量标准,相对而言较客观,且涵盖了各种因素。它本身也是对计算机性能最直接的度量。
\item 至于手动循环展开优化有无意义,得分具体情况讨论。
\begin{enumerate}
\item 如果程序代码具有高度的向量性,循环节之间没有数据相关(例如本实验的第一、第二个函数),那么手动循环展开是有意义的。这也在上文的实验结果中可以反映出来。这是因为,手动循环展开后,可以更加方便硬件进行动态调度,避免堵塞以及使用SIMD指令。
\item 但如果程序较复杂,各个循环节之间存在强数据相关(例如本实验的第三个函数),那么手动循环展开的意义不大。对于这类情况,编译器开启最高优化($-O3$)后,已经能发现大部分可并行执行的指令了,再做循环展开甚至有可能降低效率。
\end{enumerate}
\item 在本次实验中,针对第三个函数强数据相关的情况,我们自己尝试使用软流水,试图尽可能隔开相关的指令。我们发现,使用软流水后,相较于原函数,性能确实有提升。但依然不如循环展开或者是编译器开启最高优化($-O3$)后的效果。这可能是因为,如果用C这种高级语言写软流水的话,我们不得不引入中间变量。数据在中间变量的传递过程中,会引入额外延迟。而循环展开和编译器优化,已经足够帮助硬件确定可以并行的指令,合理调度了。
\end{enumerate}

\end{document}
