\documentclass[aps,superscriptaddress,notitlepage,longbibliography]{revtex4-1}
\usepackage[latin9]{inputenc}
\setcounter{secnumdepth}{3}
\usepackage{bm}
\usepackage{amsmath}
\usepackage{amsthm}
\usepackage{amssymb}
\usepackage{esint}

\makeatletter
%%%%%%%%%%%%%%%%%%%%%%%%%%%%%% User specified LaTeX commands.
\usepackage{times}
\usepackage{comment}
\usepackage{graphicx}
\usepackage{feynmf}
\usepackage{tabularx}
\usepackage{amsmath}
\usepackage{amstext}
\usepackage{amssymb}
\usepackage{xfrac}
\usepackage[colorlinks,citecolor=blue]{hyperref}
\usepackage{graphicx}
\usepackage{amsmath}
\usepackage{amstext}
\usepackage{amssymb}
\usepackage{amsfonts}
\usepackage{longtable,booktabs}
\usepackage{hyperref}
\usepackage{url}
\usepackage{subfigure}
\usepackage{dsfont}
\usepackage{booktabs}
\usepackage{amsbsy}
\usepackage{dcolumn}
\usepackage{amsthm}
\usepackage{bm}
\usepackage{esint}
\usepackage{multirow}
\usepackage{hyperref}
\usepackage{cleveref}
\usepackage{mathrsfs}
\usepackage{amsfonts}
\usepackage{amsbsy}
\usepackage{dcolumn}
\usepackage{bm}
\usepackage{multirow}
\usepackage{color}
\usepackage{extarrows}
\usepackage{datetime}
\usepackage[super]{nth}
\hypersetup{
	colorlinks=magenta,
	linkcolor=blue,
	filecolor=magenta,
	urlcolor=magenta,
}
\def\Z{\mathbb{Z}}
\newcommand{\red}[1]{{\textcolor{red}{#1}}}
\newtheorem{theorem}{Theorem}\newtheorem{statement}{Statement}\newcommand{\mb}{\mathbb}
\newcommand{\bs}{\boldsymbol}
\newcommand{\wt}{\widetilde}
\newcommand{\mc}{\mathcal}
\newcommand{\bra}{\langle}
\newcommand{\ket}{\rangle}
\newcommand{\ep}{\epsilon}
\newcommand{\tf}{\textbf}

\makeatother

\begin{document}
\title{Supplemental Material for ``Superfluidity in Dissipative Bose-Einstein
Condensate''}
\author{Hongchao Li}
\thanks{These two authors contributed equally to this work.}
\affiliation{Department of Physics, University of Tokyo, 7-3-1 Hongo, Tokyo 113-0033,
Japan}
\email{lhc@cat.phys.s.u-tokyo.ac.jp}

\author{Xie-Hang Yu}
\thanks{These two authors contributed equally to this work.}
\affiliation{Max-Planck-Institut f�r Quantenoptik, Hans-Kopfermann-Stra�e 1, D-85748
Garching, Germany}
\affiliation{Munich Center for Quantum Science and Technology, Schellingstra�e
4, 80799 M�nchen, Germany}
\email{xiehang.yu@mpq.mpg.de}

\author{Masaya Nakagawa}
\affiliation{Department of Physics, University of Tokyo, 7-3-1 Hongo, Tokyo 113-0033,
Japan}
\email{nakagawa@cat.phys.s.u-tokyo.ac.jp}

\author{Masahito Ueda}
\affiliation{Department of Physics, University of Tokyo, 7-3-1 Hongo, Tokyo 113-0033,
Japan}
\affiliation{RIKEN Center for Emergent Matter Science (CEMS), Wako, Saitama 351-0198,
Japan}
\affiliation{Institute for Physics of Intelligence, University of Tokyo, 7-3-1
Hongo, Tokyo 113-0033, Japan}
\email{ueda@cat.phys.s.u-tokyo.ac.jp}

\date{\today}

\maketitle
 

\tableofcontents{}

\section{Derivation of Effective Action for the Quantum Phase}

We tend to derive the effective Keldysh action of the quantum phase
of bosonic fields. To begin with, we first review the superfluid theory
in the closed bosonic quantum systems \citep{Ueda2010}: 
\begin{equation}
S=\int dtd^{3}x[i\varphi_{+}^{\ast}(x)\partial_{t}\varphi_{+}(x)-H_{+}-i\varphi_{-}^{\ast}(x)\partial_{t}\varphi_{-}(x)+H_{-}],\label{eq:closed_action}
\end{equation}
where 
\begin{equation}
H_{\pm}=\frac{1}{2m}(\nabla\varphi_{\pm}^{\ast})\cdot(\nabla\phi_{\pm}^{\ast})-\mu|\varphi_{\pm}|^{2}+\frac{U}{2}|\varphi_{\pm}|^{4}.
\end{equation}
If we transform the fields into the retarded and advanced fields:
\begin{equation}
\varphi_{R}=\frac{1}{2}(\varphi_{+}+\varphi_{-}),\varphi_{A}=\varphi_{+}-\varphi_{-},
\end{equation}
the action can be rewritten as 
\begin{eqnarray}
S & = & \int dtd^{3}x\left[\frac{i}{2}\varphi_{R}^{\ast}(x)\partial_{t}\varphi_{A}(x)+\frac{i}{2}\varphi_{A}^{\ast}(x)\partial_{t}\varphi_{R}(x)-\frac{1}{2m}(\nabla\varphi_{R}^{\ast})\cdot(\nabla\varphi_{A})-\frac{1}{2m}(\nabla\varphi_{A}^{\ast})\cdot(\nabla\varphi_{R})\right]\nonumber \\
 &  & +\int dtd^{3}x\left[-\frac{U}{2}|\varphi_{+}|^{4}+\frac{U}{2}|\varphi_{-}|^{4}\right].
\end{eqnarray}
Here we consider the $U(1)$ symmetry breaking of the bosonic fields
$\phi_{+},\phi_{-}$ as \citep{PhysRevD.103.056020}
\begin{eqnarray}
\varphi_{+} & = & \varphi_{0}(1+\phi_{+})e^{i\theta_{+}},\label{eq:phi+}\\
\varphi_{-} & = & \varphi_{0}(1+\phi_{-})e^{i\theta_{-}}.\label{eq:phi-}
\end{eqnarray}
Here $\phi_{\alpha}$ and $\theta_{\alpha}$ represent gapped Higgs
modes and gapless Nambu-Goldstone mode on the contour $\alpha$ separately.
These expressions are supported by the mean-field solution of the
equation $\delta S/\delta\varphi_{A}=0$, which is given by 
\begin{equation}
(i\partial_{t}-\mu+U_{R}|\varphi_{R}|^{2})\varphi_{R}=0.
\end{equation}
The solution of this equation is 
\begin{eqnarray*}
\varphi_{R} & = & \sqrt{n},\\
\mu & = & \varphi_{R}^{2}U_{R},\\
\varphi_{A} & = & 0.
\end{eqnarray*}
By expanding the action \eqref{eq:closed_action} to the second order
term of $\phi_{\pm}$ and $\theta_{\pm}$, we have 
\begin{equation}
S=S_{+}-S_{-},\label{eq:total_action}
\end{equation}
where 
\begin{equation}
S_{\alpha}=\int dtd^{3}x\left[-\varphi_{0}^{2}\partial_{t}\theta_{\alpha}-\frac{1}{2m}\varphi_{0}^{2}(\nabla\theta_{\alpha})^{2}-2\varphi_{0}^{2}\phi_{\alpha}\partial_{t}\theta_{\alpha}-\frac{1}{2m}\varphi_{0}^{2}(\nabla\phi_{\alpha})^{2}-2U\varphi_{0}^{4}\phi_{\alpha}^{2}\right].
\end{equation}
By substituting $S_{\alpha}$ into Eq. \eqref{eq:total_action}, we
obtain 
\begin{eqnarray}
S & = & \int dtd^{3}x\left[-\varphi_{0}^{2}\partial_{t}\theta_{A}-\frac{1}{m}\varphi_{0}^{2}(\nabla\theta_{R})(\nabla\theta_{A})-2\varphi_{0}^{2}\phi_{R}\partial_{t}\theta_{A}-2\varphi_{0}^{2}\phi_{A}\partial_{t}\theta_{R}-\frac{1}{m}\varphi_{0}^{2}(\nabla\phi_{R})(\nabla\phi_{A})\right]\nonumber \\
 &  & +\int dtd^{3}x[-4U\varphi_{0}^{4}\phi_{R}\phi_{A}].\nonumber \\
 & = & \int dtd^{3}x\left[-\frac{1}{m}\varphi_{0}^{2}(\nabla\theta_{R})(\nabla\theta_{A})-2\varphi_{0}^{2}\left(\begin{array}{cc}
\phi_{R} & \phi_{A}\end{array}\right)\left(\begin{array}{c}
\partial_{t}\theta_{A}\\
\partial_{t}\theta_{R}
\end{array}\right)\right]\nonumber \\
 &  & +\int dtd^{3}x\left[-\left(\begin{array}{cc}
\phi_{R} & \phi_{A}\end{array}\right)\left(\begin{array}{cc}
0 & -\frac{\varphi_{0}^{2}}{2m}\nabla^{2}+2U\varphi_{0}^{4}\\
-\frac{\varphi_{0}^{2}}{2m}\nabla^{2}+2U\varphi_{0}^{4} & 0
\end{array}\right)\left(\begin{array}{c}
\phi_{R}\\
\phi_{A}
\end{array}\right)\right].
\end{eqnarray}
Then we integrate the amplitude fields $\phi_{R},\phi_{A}$ and obtain
\begin{eqnarray}
S & = & \int dtd^{3}x\left[-\frac{1}{m}\varphi_{0}^{2}(\nabla\theta_{R})(\nabla\theta_{A})\right]\nonumber \\
 &  & +\varphi_{0}^{4}\int dtd^{3}x\left[\left(\begin{array}{cc}
\partial_{t}\theta_{R} & \partial_{t}\theta_{A}\end{array}\right)\left(\begin{array}{cc}
0 & \left(-\frac{\varphi_{0}^{2}}{2m}\nabla^{2}+2U\varphi_{0}^{4}\right)^{-1}\\
\left(-\frac{\varphi_{0}^{2}}{2m}\nabla^{2}+2U\varphi_{0}^{4}\right)^{-1} & 0
\end{array}\right)\left(\begin{array}{c}
\partial_{t}\theta_{A}\\
\partial_{t}\theta_{R}
\end{array}\right)\right].\nonumber \\
 & = & \int dtd^{3}x\left[-\frac{1}{m}\varphi_{0}^{2}(\nabla\theta_{R})(\nabla\theta_{A})+\partial_{t}\theta_{R}\frac{\varphi_{0}^{4}}{-\frac{\varphi_{0}^{2}}{2m}\nabla^{2}+2U\varphi_{0}^{4}}\partial_{t}\theta_{A}\right].
\end{eqnarray}
Hence, we rotate the action back to the forward and backward contours
and obtain 
\begin{equation}
S=S_{+}^{\text{eff}}-S_{-}^{\text{eff}},
\end{equation}
where 
\begin{equation}
S_{\alpha}^{\text{eff}}=\int dtd^{3}x\left[-\frac{1}{2m}\varphi_{0}^{2}(\nabla\theta_{\alpha})(\nabla\theta_{\alpha})+\partial_{t}\theta_{\alpha}\frac{\varphi_{0}^{2}}{-\frac{1}{2m}\nabla^{2}+2U\varphi_{0}^{2}}\partial_{t}\theta_{\alpha}\right].\label{eq:action_2}
\end{equation}
By applying the Fourier transformation, we have the excitation spectrum:
\begin{equation}
\omega=\sqrt{\frac{k^{2}}{2m}\left(\frac{k^{2}}{2m}+2Un\right)},\varphi_{0}^{2}=n.
\end{equation}
By taking $\partial_{t}\theta_{+}=\partial_{t}\theta_{-}=0$, we can
see the first term in Eq. \eqref{eq:action_2} represents the phase
rigidity which takes the form after Wick rotation: $S=\int d\tau d^{3}x\left[\frac{1}{2m}\varphi_{0}^{2}(\nabla\theta_{+})^{2}-\frac{1}{2m}\varphi_{0}^{2}(\nabla\theta_{-})^{2}\right]$.

Then we turn to the Lindblad system, in which case the action becomes
\begin{equation}
S=\int dtd^{3}x[i\varphi_{+}^{\ast}(x)\partial_{t}\varphi_{+}(x)-H_{+}-i\varphi_{-}^{\ast}(x)\partial_{t}\varphi_{-}(x)+H_{-}-i\gamma\varphi_{-}^{\ast}(x)^{2}\varphi_{+}(x)^{2}],
\end{equation}
where 
\begin{equation}
H_{\pm}=\frac{1}{2m}(\nabla\varphi_{\pm}^{\ast})\cdot(\nabla\phi_{\pm}^{\ast})-\mu|\varphi_{\pm}|^{2}+\frac{U_{\pm}}{2}|\varphi_{\pm}|^{4}
\end{equation}
with $U_{\pm}=U_{R}\mp i\gamma$. We still apply the decompositions
in Eqs. \eqref{eq:phi+} and \eqref{eq:phi-}, which are supported
by the mean-field solution of the equation $\delta S/\delta\varphi_{A}=0$,
which is given by 
\begin{equation}
(i\partial_{t}-\mu+(U_{R}-i\gamma)|\varphi_{R}|^{2})\varphi_{R}=0.
\end{equation}
The solution of this equation is 
\begin{eqnarray*}
\varphi_{R}(t) & = & \frac{\varphi_{R}(0)}{\sqrt{1+2\gamma\varphi_{R}(0)^{2}t}}=\sqrt{\frac{n(0)}{1+2\gamma n(0)t}},\\
\mu(t) & = & \varphi_{R}^{2}(t)U_{R},\\
\varphi_{A} & = & 0.
\end{eqnarray*}
Hence, we just replace the approximation with the mean-field solution
as $\varphi_{0}=\varphi_{R}(t)$, which is a function of time $t$.

Still the bosonic fields can be decomposed into fields of $\theta_{\pm}$
and $\phi_{\pm}$ as Eqs. (\ref{eq:phi+}) and (\ref{eq:phi-}), which
results in
\begin{equation}
S=S_{+}-S_{-}+\int dtd^{3}x[(-4i\gamma)\varphi_{0}^{4}\phi_{+}\phi_{-}+2\gamma\varphi_{0}^{4}\theta_{A}+8\gamma\varphi_{0}^{4}\phi_{R}\theta_{A}],
\end{equation}
where 
\begin{equation}
S_{\alpha}=\int dtd^{3}x\left[-\varphi_{0}^{2}\partial_{t}\theta_{\alpha}-\frac{1}{2m}\varphi_{0}^{2}(\nabla\theta_{\alpha})^{2}-2\varphi_{0}^{2}\phi_{\alpha}\partial_{t}\theta_{\alpha}-\frac{1}{2m}\varphi_{0}^{2}(\nabla\phi_{\alpha})^{2}-2(U_{R}-i\alpha\gamma)\varphi_{0}^{4}\phi_{\alpha}^{2}\right].
\end{equation}
By substitution, we have the action

\begin{eqnarray}
S & = & \int dtd^{3}x\left[-\varphi_{0}^{2}\partial_{t}\theta_{A}-\frac{1}{m}\varphi_{0}^{2}(\nabla\theta_{R})(\nabla\theta_{A})-2\varphi_{0}^{2}\phi_{R}\partial_{t}\theta_{A}-2\varphi_{0}^{2}\phi_{A}\partial_{t}\theta_{R}-\frac{1}{m}\varphi_{0}^{2}(\nabla\phi_{R})(\nabla\phi_{A})\right]\nonumber \\
 &  & +\int dtd^{3}x\left[-4U_{R}\varphi_{0}^{4}\phi_{R}\phi_{A}+2i\gamma\varphi_{0}^{4}\left(2\phi_{R}^{2}+\frac{1}{2}\phi_{A}^{2}\right)+2\gamma\varphi_{0}^{4}\theta_{A}+8\gamma\varphi_{0}^{4}\phi_{R}\theta_{A}\right]\nonumber \\
 & = & \int dtd^{3}x\left[-\varphi_{0}^{2}\partial_{t}\theta_{A}-\frac{1}{m}\varphi_{0}^{2}(\nabla\theta_{R})(\nabla\theta_{A})-2\varphi_{0}^{2}\left(\begin{array}{cc}
\phi_{R} & \phi_{A}\end{array}\right)\left(\begin{array}{c}
\partial_{t}\theta_{A}-4\gamma\varphi_{0}^{2}\theta_{A}\\
\partial_{t}\theta_{R}
\end{array}\right)\right]\nonumber \\
 &  & +\int dtd^{3}x\left[-\left(\begin{array}{cc}
\phi_{R} & \phi_{A}\end{array}\right)\left(\begin{array}{cc}
-4i\gamma\varphi_{0}^{4} & -\frac{\varphi_{0}^{2}}{2m}\nabla^{2}+2U_{R}\varphi_{0}^{4}\\
-\frac{\varphi_{0}^{2}}{2m}\nabla^{2}+2U_{R}\varphi_{0}^{4} & -i\gamma\varphi_{0}^{4}
\end{array}\right)\left(\begin{array}{c}
\phi_{R}\\
\phi_{A}
\end{array}\right)+2\gamma\varphi_{0}^{4}\theta_{A}\right].
\end{eqnarray}
After integrating the amplitude fields $\phi_{R,A}$, we obtain the
effective action as 
\begin{eqnarray}
S & = & \int dtd^{3}x\left[-\varphi_{0}^{2}\partial_{t}\theta_{A}-\frac{1}{m}\varphi_{0}^{2}(\nabla\theta_{R})(\nabla\theta_{A})\right]\nonumber \\
 &  & +\int dtd^{3}x\left[\varphi_{0}^{4}\left(\begin{array}{cc}
\partial_{t}\theta_{A}-4\gamma\varphi_{0}^{2}\theta_{A} & \partial_{t}\theta_{R}\end{array}\right)A^{-1}\left(\begin{array}{c}
\partial_{t}\theta_{A}-4\gamma\varphi_{0}^{2}\theta_{A}\\
\partial_{t}\theta_{R}
\end{array}\right)+2\gamma\varphi_{0}^{4}\theta_{A}\right],
\end{eqnarray}
where $A=\left(\begin{array}{cc}
-4i\gamma & -\frac{\varphi_{0}^{2}}{2m}\nabla^{2}+2U_{R}\varphi_{0}^{4}\\
-\frac{\varphi_{0}^{2}}{2m}\nabla^{2}+2U_{R}\varphi_{0}^{4} & -i\gamma
\end{array}\right)$. When we take $\partial_{t}\theta_{+}=\partial_{t}\theta_{-}=0$,
we obtain 
\begin{eqnarray}
S & = & \int dtd^{3}x\left[-\frac{1}{m}\varphi_{0}^{2}(\nabla\theta_{R})(\nabla\theta_{A})+16\gamma^{2}\varphi_{0}^{4}\theta_{A}^{2}\frac{\varphi_{0}^{4}(-i\gamma\varphi_{0}^{4})}{-4\gamma^{2}\varphi_{0}^{8}-\left(-\frac{\varphi_{0}^{2}}{2m}\nabla^{2}+2U_{R}\varphi_{0}^{4}\right)^{2}}+2\gamma\varphi_{0}^{4}\theta_{A}\right]\nonumber \\
 & \simeq & \int dtd^{3}x\left[-\frac{1}{m}\varphi_{0}^{2}(\nabla\theta_{R})(\nabla\theta_{A})+4i\gamma^{2}\varphi_{0}^{4}\theta_{A}^{2}\frac{\gamma}{\gamma^{2}+U_{R}^{2}}+2\gamma\varphi_{0}^{4}\theta_{A}\right],
\end{eqnarray}
where we ignore the contribution from $\nabla^{2}$ since we consider
long-wave limit. We can see the second term on the R.H.S represents
the imaginary part of the contribution, which is an infinitesimal
quantity $O(\gamma\theta_{A}^{2})$. To concern the phase rigidity,
we only need to concern about the real part of the action after Wick
rotation, which is given by 
\begin{eqnarray}
S & = & \int d\tau d^{3}x\left[\frac{1}{m}\varphi_{0}^{2}(\nabla\theta_{R})(\nabla\theta_{A})-2\gamma\varphi_{0}^{4}\theta_{A}\right]\nonumber \\
 & = & \int d\tau d^{3}x\left[\frac{1}{2m}\varphi_{0}^{2}(\nabla\theta_{+})(\nabla\theta_{+})-\frac{1}{2m}\varphi_{0}^{2}(\nabla\theta_{-})(\nabla\theta_{-})-2\gamma\varphi_{0}^{4}(\theta_{+}-\theta_{-})\right]\nonumber \\
 & = & \int d\tau d^{3}x\varphi_{0}^{2}\left[\frac{1}{2m}(\nabla\theta_{+}+\bm{\psi}(x))(\nabla\theta_{+}+\bm{\psi}(x))-\frac{1}{2m}(\nabla\theta_{-}+\bm{\psi}(x))(\nabla\theta_{-}+\bm{\psi}(x))\right],\label{action-superfluid}
\end{eqnarray}
with $\nabla\cdot\bm{\psi}=2\gamma\varphi_{0}^{4}$. We can construct
the function as $\bm{\psi}=2\gamma\varphi_{0}^{4}\bm{x}/3$. Therefore,
we can still see the phase rigidity in the action. The vector field
$\bm{\psi}(x)$ represents the dissipative current from the environment.
Since the twist of quantum phase can be related to the velocity as
$\bm{v}_{s\alpha}=\nabla\theta_{\alpha}/m$ \citep{Coleman_2015},
the dissipative current can be determined by 
\begin{eqnarray}
\langle\bm{j}_{s}\rangle & = & \frac{\delta S}{m\delta\bm{v}_{s}}=\frac{1}{2}\left(\frac{\delta S}{m\delta\bm{v}_{s+}}-\frac{\delta S}{m\delta\bm{v}_{s-}}\right)|_{\bm{v}_{s+}=-\bm{v}_{s-}=\bm{v}_{s}}\nonumber \\
 & = & \frac{\varphi_{0}^{2}}{m}(m\bm{v}_{s}+\bm{\psi}),
\end{eqnarray}
which indicates that $\rho_{s}=\varphi_{0}^{2}=n(t)$ by definition
of the superfluid density $\rho_{s}=\partial\langle\bm{j}_{s}\rangle/\partial\bm{v}_{s}$.
Therefore, the superfluid density is just the total mass density.
All the bosons participate in the superfluid transport. Further, to
understand the current related to $\bm{\psi}$, here we define our
cuurent density operator as $\hat{\bm{j}}=\frac{1}{2}\sum_{\bm{k}}\bm{k}(c_{\bm{k}+}^{\dagger}c_{\bm{k}+}+c_{\bm{k}-}^{\dagger}c_{\bm{k}-})$
and we calculate the measured value of the current as 
\begin{equation}
\bm{j}=\langle\hat{\bm{j}}\rangle=\frac{1}{Z}\int D[c_{+}]D[c_{-}]\hat{\bm{j}}e^{iS}.
\end{equation}
This is the classical current we always introduce in the open quantum
systems. However, here we have a number density decay in the continuous
equation as 
\begin{equation}
\frac{dn}{dt}=-\nabla\cdot\bm{j}_{t},
\end{equation}
where $\bm{j}_{t}=\bm{j}_{c}+\bm{j}_{e}$ with the current $\bm{j}_{c}$
being the current flow between different space point and the current
$\bm{j}_{e}$ being the dissipative current due to the loss of particle
numbers. This dissipative current is given by 
\[
\nabla\cdot\bm{j}_{e}=-\frac{dn_{t}}{dt}=\frac{2\gamma n(0)^{2}}{(1+2\gamma n(0)t)^{2}}
\]
\[
\ 
\]
We assume the system is uniform and hence give the expression of $\bm{j}_{e}$
as 
\begin{equation}
\bm{j}_{e}=\frac{2\gamma n(0)^{2}}{3(1+2\gamma n(0)t)^{2}}\bm{x}=\frac{2}{3}\gamma n^{2}\bm{x},
\end{equation}
which is exactly the same as our field-theoretic calculation. Since
this is an isotropic term and has nothing to do with the external
perturbation, in the measurement, this term will not be influenced.
Hence, this current will not influence the phase stiffness and the
superfluid density since $D_{\alpha\beta}=\partial j_{\alpha}/\partial v_{\beta}$.
This can be also shown in the field-theoretic calculations.

\section{f-sum Rule}

This is also an important and relevant topic. We first consider such
a quantity 
\begin{equation}
\langle[\rho_{-\bm{k}},\mathcal{L}^{\dagger}(\rho_{\bm{k}})]\rangle=\text{Tr}[\rho[\rho_{-\bm{k}},\mathcal{L}^{\dagger}(\rho_{\bm{k}})]]
\end{equation}
where 
\begin{equation}
\rho_{\bm{k}}=\sum_{\bm{p}}a_{\bm{p}}^{\dagger}a_{\bm{p}+\bm{k}},\mathcal{L}^{\dagger}(O):=i[O,H]-\frac{\gamma}{2}\sum_{i}\{O,L_{i}^{\dagger}L_{i}\}+\gamma\sum_{i}L_{i}^{\dagger}OL_{i}.
\end{equation}
Here we consider the Heisenberg picture where the operators evolve
as 
\begin{equation}
\frac{dO}{dt}=\mathcal{L}^{\dagger}(O)=i[O,H]-\frac{\gamma}{2}\sum_{i}\{O,L_{i}^{\dagger}L_{i}\}+\gamma\sum_{i}L_{i}^{\dagger}OL_{i}.\label{eq:Ldynamics}
\end{equation}
From previous calculation \citep{Ueda2010}, we notice that 
\begin{equation}
i\left[{\rho_{\bm{k}}},H\right]=i\sum_{\bm{p}}(\varepsilon_{\bm{p}+\bm{k}}-\varepsilon_{\bm{p}})a_{\bm{p}}^{\dagger}a_{\bm{p}+\bm{k}}.
\end{equation}
Then we calculate the following terms in $\mathcal{L}(\rho_{\bm{k}})$.
We can see 
\begin{equation}
-\frac{\gamma}{2}\sum_{i}\{O,L_{i}^{\dagger}L_{i}\}+\gamma\sum_{i}L_{i}^{\dagger}OL_{i}=-\frac{\gamma}{2}\sum_{i}[O,L_{i}^{\dagger}]L_{i}+\frac{\gamma}{2}\sum_{i}L_{i}^{\dagger}[O,L_{i}].
\end{equation}
By substituting $L_{i}=a_{i}^{2}$ and apply the Fourier transformation,
we obtain 
\begin{eqnarray}
 &  & -\frac{\gamma}{2}\sum_{\bm{p},\bm{q},\bm{l}}[\rho_{\bm{k}},a_{\bm{p}+\bm{l}}^{\dagger}a_{\bm{q}-\bm{l}}^{\dagger}]a_{\bm{q}}a_{\bm{p}}+\frac{\gamma}{2}\sum_{\bm{p},\bm{q},\bm{l}}a_{\bm{p}+\bm{l}}^{\dagger}a_{\bm{q}-\bm{l}}^{\dagger}[\rho_{\bm{k}},a_{\bm{q}}a_{\bm{p}}]\nonumber \\
 & = & -\gamma\sum_{\bm{p},\bm{q},\bm{l}}\left[a_{\bm{l}}^{\dagger}a_{\bm{p}}^{\dagger}a_{\bm{l}+\bm{k}-\bm{q}}a_{\bm{p}+\bm{q}}+a_{\bm{l}}^{\dagger}a_{\bm{p}}^{\dagger}a_{\bm{p}-\bm{q}}{a_{\bm{l}+\bm{k}+\bm{q}}}\right]\nonumber \\
 & = & -2\gamma\left(\sum_{\bm{q}}\rho_{\bm{k}-\bm{q}}\rho_{\bm{q}}-\rho_{\bm{k}}\right).
\end{eqnarray}
By utilizing the fact that $[\rho_{\bm{k}},\rho_{\bm{p}}]=0$ for
arbitrary $\bm{p},\bm{k}$, we have 
\begin{eqnarray}
[\rho_{-\bm{k}},\mathcal{L}^{\dagger}(\rho_{\bm{k}})] & = & \left[\rho_{-\bm{k}},i\sum_{\bm{p}}(\varepsilon_{\bm{p}+\bm{k}}-\varepsilon_{\bm{p}})a_{\bm{p}}^{\dagger}a_{\bm{p}+\bm{k}}-2\gamma\left(\sum_{\bm{q}}\rho_{\bm{k}-\bm{q}}\rho_{\bm{q}}-\rho_{\bm{k}}\right)\right]\nonumber \\
 & = & -i\sum_{\bm{p}}(\varepsilon_{\bm{p}+\bm{k}}+\varepsilon_{\bm{p}-\bm{k}}-\varepsilon_{\bm{p}})a_{\bm{p}}^{\dagger}a_{\bm{p}}\nonumber \\
 & = & -2i\varepsilon_{\bm{k}}\hat{N}.
\end{eqnarray}
Therefore, 
\begin{equation}
\langle[\rho_{-\bm{k}},\mathcal{L}^{\dagger}(\rho_{\bm{k}})]\rangle=-2i\varepsilon_{\bm{k}}N.
\end{equation}
Below, we will show how to use the eigenspectrum of the Lindblad operator
to represent $\langle[\rho_{-\bm{k}},\mathcal{L}^{\dagger}(\rho_{\bm{k}})]\rangle$.
To begin with, we define the right eigenvectors $\hat{r}_{\alpha}$
and the left eigenvectors $\hat{l}_{\alpha}$ of $\mathcal{L}^{\dagger}$
as \citep{Scarlatella2019} 
\begin{eqnarray}
\mathcal{L}^{\dagger}(\hat{r}_{\alpha}) & = & \lambda_{\alpha}\hat{r}_{\alpha},\\
\mathcal{L}^{\dagger}(\hat{l}_{\alpha}) & = & \lambda_{\alpha}^{\ast}\hat{l}_{\alpha},
\end{eqnarray}
which satisfy 
\begin{equation}
\text{Tr}(\hat{l}_{\alpha}^{\dagger}\hat{r}_{\beta})=\delta_{\alpha\beta},\sum_{\alpha}\hat{r}_{\alpha}\hat{l}_{\alpha}^{\dagger}=\mathbb{I}.
\end{equation}
Then we can expand $\mathcal{L}^{\dagger}(\rho_{\bm{k}})$ as 
\begin{equation}
\mathcal{L}^{\dagger}(\rho_{\bm{k}})=\sum_{\alpha}\lambda_{\alpha}\hat{r}_{\alpha}\text{Tr}(\hat{l}_{\alpha}^{\dagger}\rho_{\bm{k}}).
\end{equation}
Therefore, we obtain 
\begin{eqnarray}
\langle[\rho_{-\bm{k}},\mathcal{L}^{\dagger}(\rho_{\bm{k}})]\rangle & = & \sum_{\alpha}\lambda_{\alpha}(\text{Tr}[\rho_{s}\rho_{-\bm{k}}\hat{r}_{\alpha}]-\text{Tr}[\rho_{-\bm{k}}\rho_{s}\hat{r}_{\alpha}])\text{Tr}(\hat{l}_{\alpha}^{\dagger}\rho_{\bm{k}})\nonumber \\
 & = & \sum_{\alpha}\lambda_{\alpha}\text{Tr}[[\rho_{s},\rho_{-\bm{k}}]\hat{r}_{\alpha}]\text{Tr}(\hat{l}_{\alpha}^{\dagger}\rho_{\bm{k}}),
\end{eqnarray}
where we use $\rho_{s}$ to represent the initial density matrix.
We further define a new retarded Green's function for open quantum
systems: 
\begin{equation}
\tilde{G}^{R}(\bm{k},t_{0},t)=-i\theta(t)\langle[\rho_{-\bm{k}}(t_{0}),\rho_{\bm{k}}(t_{0}+t)]\rangle
\end{equation}
in the Heisenburg picture. Hence, we can expand the Green's function
as 
\begin{eqnarray}
\tilde{G}^{R}(\bm{k},t_{0},t) & = & -i\theta(t)\sum_{\alpha}e^{\lambda_{\alpha}t}[\text{Tr}[\rho_{s}\rho_{-\bm{k}}\hat{r}_{\alpha}]\text{Tr}(\hat{l}_{\alpha}^{\dagger}\rho_{\bm{k}})-\text{Tr}[\rho_{-\bm{k}}\rho_{s}\hat{r}_{\alpha}]\text{Tr}(\hat{l}_{\alpha}^{\dagger}\rho_{\bm{k}})]\nonumber \\
 & = & -i\theta(t)\sum_{\alpha}e^{\lambda_{\alpha}t}\text{Tr}[[\rho_{s},\rho_{-\bm{k}}]\hat{r}_{\alpha}]\text{Tr}(\hat{l}_{\alpha}^{\dagger}\rho_{\bm{k}}).
\end{eqnarray}
Under the Fourier transformation of $t$, we obtain 
\begin{equation}
\tilde{G}^{R}(\bm{k},t_{0},\omega)=\int dte^{i\omega t}\tilde{G}^{R}(\bm{k},t_{0},t)=\sum_{\alpha}\frac{1}{\omega-i\lambda_{\alpha}+i0^{+}}\text{Tr}[[\rho_{s},\rho_{-\bm{k}}]\hat{r}_{\alpha}]\text{Tr}(\hat{l}_{\alpha}^{\dagger}\rho_{\bm{k}}).
\end{equation}
If we define a closed contour $C$ that contains all the poles $\lambda_{\alpha}-0^{+}$,
the relation becomes: 
\begin{equation}
\oint_{C}\frac{\omega d\omega}{2\pi}\tilde{G}^{R}(\bm{k},t_{0},\omega)=\text{Tr}[[\rho_{s},\rho_{-\bm{k}}]\hat{r}_{\alpha}]\text{Tr}(\hat{l}_{\alpha}^{\dagger}\rho_{\bm{k}})=-2i\varepsilon_{\bm{k}}N(t_{0}).\label{eq:f}
\end{equation}
Alternatively, the f-sum rule can also be used to derive the relation
between current-current correlation function and the number operator.
We begin from the dynamics of the operator $\rho_{\bm{k}}$ as 
\begin{equation}
\frac{\partial\rho_{\bm{r}}(t)}{\partial t}=\mathcal{L}^{\dagger}\rho_{\bm{r}}=i[H,\rho_{\bm{r}}]+\frac{\gamma}{2}\sum_{\bm{r}}[2L_{\bm{r}}^{\dagger}\rho_{\bm{r}}L_{\bm{r}}-L_{\bm{r}}^{\dagger}L_{\bm{r}}\rho_{\bm{r}}-\rho_{\bm{r}}L_{\bm{r}}^{\dagger}L_{\bm{r}}].\label{eq:continuous}
\end{equation}
After simplification, this dynamics can be reorganized as 
\begin{equation}
\frac{\partial\rho_{\bm{r}}(t)}{\partial t}=-\nabla\cdot(\bm{j}_{t}):=-\nabla\cdot(\bm{j}_{c}+\bm{j}_{e}),
\end{equation}
where $\bm{j}_{t}:=\bm{j}_{c}+\bm{j}_{e}$ is the total current with
$\bm{j}_{c}$ being the current flow in the closed quantum system
and $\bm{j}_{e}$ being the current induced by two-body loss. They
are separately given by 
\begin{eqnarray}
\bm{j}_{c} & = & \frac{i}{2}[\nabla a^{\dagger}(\bm{r})a(\bm{r})-a^{\dagger}(\bm{r})\nabla a(\bm{r})],\\
\bm{j}_{e} & = & \frac{2}{3}\gamma n^{2}\bm{r}.
\end{eqnarray}
By performing Fourier transformation to Eq. \eqref{eq:continuous},
we obtain 
\begin{equation}
-i\mathcal{L}^{\dagger}(\rho_{\bm{k}})=\omega\rho_{\bm{k}}(\omega)=-\bm{k}\cdot\bm{j}_{t}(\bm{k},\omega).
\end{equation}
The f-sum rule \eqref{eq:f} tells us 
\begin{equation}
m[\rho_{\bm{k}}(t),\mathcal{L}^{\dagger}(\rho_{-\bm{k}}(t))]=-ik^{2}N(t).
\end{equation}
Under the Fourier transformation, the f-sum rule becomes 
\begin{equation}
m\int dte^{i(\omega_{1}-\omega-\omega_{2})t}\frac{d\omega_{1}}{2\pi}\frac{d\omega_{2}}{2\pi}[\rho_{-\bm{k}}(\omega_{1}),\mathcal{L}^{\dagger}(\rho_{\bm{k}}(\omega_{2}))]=-ik^{2}N(\omega).\label{eq:f2}
\end{equation}
The L.H.S of Eq. \eqref{eq:f2} can be simplified as 
\begin{eqnarray}
 &  & m\int dt\frac{d\omega_{1}}{2\pi}\frac{d\omega_{2}}{2\pi}e^{i(\omega_{1}-\omega-\omega_{2})t}[\rho_{-\bm{k}}(\omega_{1}),\mathcal{L}^{\dagger}(\rho_{\bm{k}}(\omega_{2}))]\nonumber \\
 & = & \frac{m}{2\pi}\int d\omega_{1}d\omega_{2}\delta(\omega_{1}-\omega-\omega_{2})\left[\frac{-\bm{k}\cdot\bm{j}_{t}(\bm{k},\omega_{1})}{\omega_{1}},i\bm{k}\cdot\bm{j}_{t}(-\bm{k},\omega_{2})\right]\nonumber \\
 & = & -ik_{i}k_{j}m\int\frac{d\omega_{1}}{2\pi\omega_{1}}[j_{t}^{i}(\bm{k},\omega_{1}),j_{t}^{j}(-\bm{k},\omega-\omega_{1})]
\end{eqnarray}
and the f-sum rule can be rewritten as 
\[
N(\omega)=\frac{k_{i}k_{j}}{k^{2}}m\int\frac{d\omega_{1}}{2\pi\omega_{1}}[j_{t}^{i}(\bm{k},\omega_{1}),j_{t}^{j}(-\bm{k},\omega-\omega_{1})]
\]
Furthremore, we define the total current-current correlation function
as 
\begin{eqnarray}
\gamma_{t}^{i,j}(\bm{k},\omega,t_{0}) & = & m\int dte^{-i\omega t}[j_{t}^{i}(\bm{k},t+t_{0}),j_{t}^{j}(-\bm{k},t_{0})]\nonumber \\
 & = & m\int dte^{-i\omega(t+t_{0})}[j_{t}^{i}(\bm{k},t+t_{0}),j_{t}^{j}(-\bm{k},t_{0})]e^{i\omega t_{0}}\nonumber \\
 & = & m[j_{t}^{i}(\bm{k},\omega),j_{t}^{j}(-\bm{k},t_{0})]e^{i\omega t_{0}}.
\end{eqnarray}
By substituting the definition into the f-sum rule \eqref{eq:f2},
we have 
\begin{eqnarray}
N(t_{0}) & = & \frac{k_{i}k_{j}}{k^{2}}m\int\frac{d\omega_{1}}{2\pi\omega_{1}}\int\frac{d\omega}{2\pi}e^{i\omega t_{0}}[j_{t}^{i}(\bm{k},\omega_{1}),j_{t}^{j}(-\bm{k},\omega-\omega_{1})]\nonumber \\
 & = & \frac{k_{i}k_{j}}{k^{2}}m\int\frac{d\omega_{1}}{2\pi\omega_{1}}\int\frac{d\omega}{2\pi}e^{i\omega_{1}t_{0}}[j_{t}^{i}(\bm{k},\omega_{1}),j_{t}^{j}(-\bm{k},\omega-\omega_{1})]e^{i(\omega-\omega_{1})t_{0}}\nonumber \\
 & = & m\frac{k_{i}k_{j}}{k^{2}}\int\frac{d\omega_{1}}{2\pi\omega_{1}}e^{i\omega_{1}t_{0}}[j_{t}^{i}(\bm{k},\omega_{1}),j_{t}^{j}(-\bm{k},t_{0})]\nonumber \\
 & = & \frac{k_{i}k_{j}}{k^{2}}\int\frac{d\omega_{1}}{2\pi\omega_{1}}\gamma_{t}^{i,j}(\bm{k},\omega_{1},t_{0}).
\end{eqnarray}
Since the longitudial component of the correlation function is given
by $\gamma_{t}^{L}(\bm{k},\omega)=\frac{k_{i}k_{j}}{k^{2}}\gamma_{t}^{i,j}(\bm{k},\omega)$,
we can also represent the f-sum rule as 
\begin{equation}
\int\frac{d\omega}{2\pi\omega}\gamma_{t}^{L}(\bm{k},\omega,t)=N(t).\label{eq:total}
\end{equation}
Then we move to the normal fluid density, which is determined from
\begin{equation}
\langle J_{i}(t)\rangle=\rho_{n}^{i,j}u_{j},
\end{equation}
where we define the averaged current density as \citep{SM_footnote}
\begin{equation}
\langle J_{i}(t)\rangle=\frac{1}{V}\int d^{3}r\langle j_{i}(\bm{r},t)\rangle.
\end{equation}
Since the dissipative current is isotropic in the space, we assume
the dissipative current does not influence the averaged current density.
The only contribution originates from the closed current $\bm{j}_{c}$.
From the Lindbladian dynamics \eqref{eq:Ldynamics}, we derive the
linear response theory in open quantum system as 
\begin{eqnarray}
\frac{d}{dt}\langle J_{i}(t)\rangle & = & \frac{1}{V}\int d^{3}r\frac{d}{dt}\langle j_{i}(\bm{r},t)\rangle\nonumber \\
 & = & \frac{1}{V}\int d^{3}r\left\langle i[j_{i},H]-\frac{\gamma}{2}\sum_{\bm{r}}\{j_{i},L_{\bm{r}}^{\dagger}L_{\bm{r}}\}+\gamma\sum_{\bm{r}}L_{\bm{r}}^{\dagger}j_{i}L_{\bm{r}}\right\rangle \nonumber \\
 & = & -\frac{i}{V}m\int d^{3}r\int d^{3}r'\langle[j_{i}(\bm{r},t),j_{j}(\bm{r}',t)]\rangle u_{j}.
\end{eqnarray}
Here we assume the perturbation is only added on the Hamiltonian of
the closed quantum system. The averaged current density is thus given
by 
\begin{eqnarray}
\langle J_{i}(t)\rangle & = & -\frac{i}{V}m\int d^{3}r\int d^{3}r'\int^{t}dt'\langle[j_{i}(\bm{r},t),j_{j}(\bm{r}',t')]\rangle u_{j}\nonumber \\
 & = & -\frac{i}{V}m\int\frac{d\omega_{2}}{2\pi}\int d^{3}r\int d^{3}r'\int^{t}dt'e^{-i\omega_{2}t'}\langle[j_{i}(\bm{r},t),j_{j}(\bm{r}',-\omega_{2})]\rangle u_{j}\nonumber \\
 & = & \frac{m}{V}\int\frac{d\omega_{2}}{2\pi\omega_{2}}\int d^{3}r\int d^{3}r'e^{-i\omega_{2}t'}\langle[j_{i}(\bm{r},t),j_{j}(\bm{r}',-\omega_{2})]\rangle u_{j},
\end{eqnarray}
where we cancel the constant term with the initial condition. Then
the tensor for the normal fluid density is given by 
\begin{eqnarray}
\rho_{n}^{i,j}(t) & = & \frac{m}{V}\int\frac{d\omega_{2}}{2\pi\omega_{2}}\int d^{3}r\int d^{3}r'e^{-i\omega_{2}t'}\langle[j_{i}(\bm{r},t),j_{j}(\bm{r}',-\omega_{2})]\rangle\nonumber \\
 & = & \lim_{\bm{k}\rightarrow0}m\int\frac{d\omega_{2}}{2\pi\omega_{2}}e^{-i\omega_{2}t'}\langle[j_{i}(\bm{k},t),j_{j}(-\bm{k},-\omega_{2})]\rangle\nonumber \\
 & = & \lim_{\bm{k}\rightarrow0}\int\frac{d\omega}{2\pi\omega}\gamma^{i,j}(\bm{k},\omega,t).
\end{eqnarray}
Here we define the current-current correlation function for the closed
current operators. To connect this with the tensor $\gamma_{t}^{i,j}$,
we can see $[\bm{j}_{e}(\bm{r}),\bm{j}_{e}(\bm{r}')]=0$ and $[\bm{j}(\bm{r}),\bm{j}_{e}(\bm{r}')]$
is isotropic in the space. Hence, we can directly replace the current
operators with the total current operator as 
\begin{equation}
\rho_{n}^{i,j}(t)=\lim_{\bm{k}\rightarrow0}\int\frac{d\omega}{2\pi\omega}\gamma_{t}^{i,j}(\bm{k},\omega,t).
\end{equation}
With similar analysis with closed quantum system, we find that the
normal fluid density corresponds to the transverse component of the
total current-current correlation function. Together with Eq. \eqref{eq:total},
we can define the superfluid density with the difference between the
transverse part and longitudinal part of the correlation function
tensor.

\section{Derivation of Quantum Depletion Part}

We begin from constructing the mean-field Lindbladian action. The
Hamiltonian of the weakly interacting bosonic system reads as: 
\begin{eqnarray}
H & = & \sum_{\bm{k}}\varepsilon_{\bm{k}}a_{\bm{k}}^{\dagger}a_{\bm{k}}+\frac{U_{R}}{2}\sum_{\bm{r}}(a_{\bm{r}}^{\dagger})^{2}(a_{\bm{r}})^{2}\nonumber \\
 & = & \sum_{\bm{k}}\varepsilon_{\bm{k}}a_{\bm{k}}^{\dagger}a_{\bm{k}}+\frac{U_{R}}{2V}\sum_{\bm{k},\bm{q},\bm{p}}a_{\bm{k}}^{\dagger}a_{\bm{p}}^{\dagger}a_{\bm{p}-\bm{q}}a_{\bm{k}+\bm{q}}\label{Hamiltonian}
\end{eqnarray}
Here we assume a contact interaction for simplicity. By adding dissipation
to the system, we turn it into an open quantum system. This open quantum
system is described by the Lindblad equation: 
\begin{equation}
\frac{d\rho}{dt}=\mathcal{L}\rho=-i[H,\rho]-\frac{\gamma}{2}\sum_{\bm{r}}(\{L_{\bm{r}}^{\dagger}L_{\bm{r}},\rho\}-2L_{\bm{r}}^{\dagger}\rho L_{\bm{r}})\label{Lindbland}
\end{equation}
Here we take the Lindblad operator as $L_{\bm{r}}=a_{\bm{r}}^{2}$.
Then we consider generalization it to closed-time-contour path integral.
On Schwinger-Keldysh contour \citep{Sieberer_2016}, the action is
written as: 
\begin{equation}
S=\int_{-\infty}^{\infty}dt\left[\sum_{\bm{k}}(a_{\bm{k}+}^{\dagger}i\partial_{t}a_{\bm{k}+}-a_{\bm{k}-}^{\dagger}i\partial_{t}a_{\bm{k}-})-H_{+}+H_{-}+\frac{i\gamma}{2}\sum_{\bm{r}}(L_{\bm{r}+}^{\dagger}L_{\bm{r}+}+L_{\bm{r}-}^{\dagger}L_{\bm{r}-}-2L_{\bm{r}+}L_{\bm{r}-}^{\dagger})\right],
\end{equation}
where $H_{\alpha}=\sum_{\bm{k}}\varepsilon_{\bm{k}}a_{\bm{k}\alpha}^{\dagger}a_{\bm{k}\alpha}+\frac{U_{R}}{2V}\sum_{\bm{k},\bm{q},\bm{p}}a_{\bm{k}\alpha}^{\dagger}a_{\bm{p}\alpha}^{\dagger}a_{\bm{p}-\bm{q},\alpha}a_{\bm{k}+\bm{q},\alpha}(\alpha=+,-)$.
\ By applying Fourier transformation to the dissipation part, the
Schwinger-Keldysh action is given by: 
\begin{eqnarray}
S & = & \int_{-\infty}^{\infty}dt\left[\sum_{\bm{k}}(a_{\bm{k}+}^{\dagger}(i\partial_{t}-\varepsilon_{\bm{k}})a_{\bm{k}+}-a_{\bm{k}-}^{\dagger}(i\partial_{t}-\varepsilon_{\bm{k}})a_{\bm{k}-})-\frac{U}{2V}\sum_{\bm{k},\bm{q},\bm{p}}a_{\bm{k}+}^{\dagger}a_{\bm{p}+}^{\dagger}a_{\bm{p}-\bm{q},+}a_{\bm{k}+\bm{q},+}\right.\nonumber \\
 &  & +\frac{U^{\ast}}{2V}\sum_{\bm{k},\bm{q},\bm{p}}a_{\bm{k}-}^{\dagger}a_{\bm{p}-}^{\dagger}a_{\bm{p}-\bm{q},-}a_{\bm{k}+\bm{q},-}-i\frac{\gamma}{V}\sum_{\bm{k},\bm{q},\bm{p}}a_{\bm{k}-}^{\dagger}a_{\bm{p}-}^{\dagger}a_{\bm{p}-\bm{q},+}a_{\bm{k}+\bm{q},+},
\end{eqnarray}
where $U=U_{R}-i\gamma$. Under the mean-field approximation, we separate
the operators into the condensate parts and non-condensate parts.
Assumed that most of bosons in the system form the condensate, we
have: 
\begin{equation}
a_{0}^{\dagger}a_{0}\approx N,\sum_{\bm{k}}a_{\bm{k}}^{\dagger}a_{\bm{k}}\ll N.
\end{equation}
Here $N$ is the total number of electrons in the system. Since $N\gg1$,
we can equivalently assume that for condensate we have 
\begin{equation}
a_{0+}^{\dagger}=a_{0-}^{\dagger}=a_{0+}=a_{0-}\approx\sqrt{N}.
\end{equation}
By applying the approximations, we simplify the interaction term as:
\begin{equation}
\sum_{\bm{k},\bm{q},\bm{p}}a_{\bm{k}+}^{\dagger}a_{\bm{p}+}^{\dagger}a_{\bm{p}-\bm{q},+}a_{\bm{k}+\bm{q},+}\approx N^{2}+N\sum_{\bm{k}}a_{-\bm{k},+}a_{\bm{k}+}+N\sum_{\bm{k}}a_{\bm{k}+}^{\dagger}a_{-\bm{k},+}^{\dagger}+4N\sum_{\bm{k}}a_{\bm{k}+}^{\dagger}a_{\bm{k}+},\label{mean1}
\end{equation}

\begin{equation}
\sum_{\bm{k},\bm{q},\bm{p}}a_{\bm{k}-}^{\dagger}a_{\bm{p}-}^{\dagger}a_{\bm{p}-\bm{q},-}a_{\bm{k}+\bm{q},-}\approx N^{2}+N\sum_{\bm{k}}a_{-\bm{k},-}a_{\bm{k}-}+N\sum_{\bm{k}}a_{\bm{k}-}^{\dagger}a_{-\bm{k},-}^{\dagger}+4N\sum_{\bm{k}}a_{\bm{k}-}^{\dagger}a_{\bm{k}-},
\end{equation}
\begin{equation}
\sum_{\bm{k},\bm{q},\bm{p}}a_{\bm{k}-}^{\dagger}a_{\bm{p}-}^{\dagger}a_{\bm{p}-\bm{q},+}a_{\bm{k}+\bm{q},+}\approx N^{2}+N\sum_{\bm{k}}a_{-\bm{k},+}a_{\bm{k}+}+N\sum_{\bm{k}}a_{\bm{k}-}^{\dagger}a_{-\bm{k},-}^{\dagger}+4N\sum_{\bm{k}}a_{\bm{k}-}^{\dagger}a_{\bm{k}+}.\label{mean3}
\end{equation}
Then the action is simplified as: 
\begin{eqnarray}
S & = & \int_{-\infty}^{\infty}dt[\sum_{\bm{k}}(a_{\bm{k}+}^{\dagger}(i\partial_{t}-\varepsilon_{\bm{k}})a_{\bm{k}+}-a_{\bm{k}-}^{\dagger}(i\partial_{t}-\varepsilon_{\bm{k}})a_{\bm{k}-})-\frac{U_{R}}{2V}a_{0+}^{\dagger}a_{0+}^{\dagger}a_{0,+}a_{0,+}+\frac{U_{R}}{2V}a_{0-}^{\dagger}a_{0-}^{\dagger}a_{0,-}a_{0,-}\nonumber \\
 &  & -\frac{U^{\ast}n}{2}\sum_{\bm{k}}a_{-\bm{k},+}a_{\bm{k}+}-\frac{Un}{2}\sum_{\bm{k}}a_{\bm{k}+}^{\dagger}a_{-\bm{k},+}^{\dagger}-2Un\sum_{\bm{k}}a_{\bm{k}+}^{\dagger}a_{\bm{k}+}+\frac{U^{\ast}n}{2}\sum_{\bm{k}}a_{-\bm{k},-}a_{\bm{k}-}\nonumber \\
 &  & +\frac{Un}{2}\sum_{\bm{k}}a_{\bm{k}-}^{\dagger}a_{-\bm{k},-}^{\dagger}+2U^{\ast}n\sum_{\bm{k}}a_{\bm{k}-}^{\dagger}a_{\bm{k}-}-4i\gamma n\sum_{\bm{k}}a_{\bm{k}-}^{\dagger}a_{\bm{k}+}\large{\large{}]}\\
 & = & \int_{-\infty}^{\infty}dt\left[\sum_{\bm{k}}(a_{\bm{k}+}^{\dagger}i\partial_{t}a_{\bm{k}+}-a_{\bm{k}-}^{\dagger}i\partial_{t}a_{\bm{k}-})-H_{+}+H_{-}-4i\gamma n\sum_{\bm{k}}a_{\bm{k}-}^{\dagger}a_{\bm{k}+}\right],\label{action}
\end{eqnarray}
where: 
\begin{eqnarray}
H_{\alpha} & = & \frac{U_{R}}{2V}a_{0\alpha}^{\dagger}a_{0\alpha}^{\dagger}a_{0\alpha}a_{0\alpha}+\sum_{\bm{k}}\left[(\varepsilon_{\bm{k}}+2U_{R}n-2i\alpha\gamma n)a_{\bm{k}\alpha}^{\dagger}a_{\bm{k}\alpha}+\frac{U^{\ast}n}{2}a_{-\bm{k}\alpha}a_{\bm{k}\alpha}+\frac{Un}{2}a_{\bm{k}\alpha}^{\dagger}a_{-\bm{k}\alpha}^{\dagger}\right]\nonumber \\
 & = & \frac{U_{R}n}{2}N+\sum_{\bm{k}}\left[(\varepsilon_{\bm{k}}+U_{R}n-2i\alpha\gamma n)a_{\bm{k}\alpha}^{\dagger}a_{\bm{k}\alpha}+\frac{U^{\ast}n}{2}a_{-\bm{k}\alpha}a_{\bm{k}\alpha}+\frac{Un}{2}a_{\bm{k}\alpha}^{\dagger}a_{-\bm{k}\alpha}^{\dagger}\right].\label{H}
\end{eqnarray}
Then we move to calculate the quantum depletion part based on the
action \eqref{action}. Since the system is generally in dynamics
and difficult to solve, we only assue a case that $n\gamma\ll1$ which
represents quasi-steady state for the interacting bosonic systems.
Due to the complexity in Schwinger-Keldysh action, we here turn to
calculate the quantum depletion part perturbed by velocity introduced.
We consider the perturbation to the Schwinger-Keldysh theory as $\bm{k}\rightarrow\bm{k}-m\bm{v}$
for the forward contour and $\bm{k}\rightarrow\bm{k}+m\bm{v}$ for
the backward contour for the finite-momentum section with $\bm{k}\neq0$.
In this sense, the Schwinger-Keldysh action \eqref{action} is transformed
as 
\begin{equation}
S\rightarrow S-m\bm{v}\cdot(\bm{j}_{+}+\bm{j}_{-}),
\end{equation}
We define the phase stiffness with \citep{Coleman_2015}
\begin{equation}
Q_{ab}=\frac{1}{2V}\frac{\partial^{2}F}{\partial v_{a}\partial v_{b}}|_{v=0},\label{Qv}
\end{equation}
where we define the generating functional as 
\begin{equation}
F=-i\log Z[\bm{v}],
\end{equation}
Therefore, the averaged mass current can be substituted as 
\begin{equation}
\langle j_{a}\rangle=-\frac{1}{2mV}\frac{\partial F}{\partial v_{a}}
\end{equation}
since 
\begin{equation}
-\frac{1}{mV}\frac{\partial F}{\partial\bm{v}}=\frac{i}{mV}\frac{1}{Z}\frac{\partial Z}{\partial\bm{v}}=\frac{i}{V}\frac{1}{Z}\int D\phi e^{i[S(\phi)-\bm{v}\cdot(\bm{J}_{+}+\bm{J}_{-})]}(\bm{J}_{+}+\bm{J}_{-})|_{v=0}=\langle\hat{\bm{j}}_{+}+\hat{\bm{j}}_{-}\rangle=2\langle\bm{j}\rangle,
\end{equation}
where we use $\hat{\bm{j}}=(\hat{\bm{j}}_{+}+\hat{\bm{j}}_{-})/2$
standing for the classical current operator. By definition of the
density of the quantum depletion part 
\begin{equation}
j_{a}=n_{D}v_{a},
\end{equation}
where $\rho_{D}$ is the density of the quantum depletion part. We
can see the diagonal elements of the phase stiffness matrix $Q_{ab}$
is just the density of the quantum depletion part as 
\begin{equation}
Q_{ab}=mn_{D}\delta_{ab}.
\end{equation}
In addition, the definition of current here can be also reduced to
the closed quantum system where $\langle\hat{\bm{j}}_{+}\rangle=\langle\hat{\bm{j}}_{-}\rangle=\langle\hat{\bm{j}}\rangle/2$.
We will later show this point in our calculation for interacting Bose
gas without dissipation. Adding the perturbation $\bm{v}$ to the
system, the functional $Z$ is thus given by 
\begin{eqnarray}
Z & = & \int D[a_{\bm{k}+}(\tau),a_{-\bm{k}+}^{\dagger}(\tau),a_{\bm{k}-}(\tau),a_{-\bm{k},-}^{\dagger}(\tau)]e^{-S(a_{\bm{k}+}(\tau),a_{-\bm{k}+}^{\dagger}(\tau),a_{\bm{k}-}(\tau),a_{-\bm{k},-}^{\dagger}(\tau))}\nonumber \\
 & = & \int D[a_{\bm{k}+}(i\omega_{n}),a_{-\bm{k}+}^{\dagger}(i\omega_{n}),a_{\bm{k}-}(i\omega_{n}),a_{-\bm{k},-}^{\dagger}(i\omega_{n})]e^{\frac{1}{2}\Psi_{\bm{k}}^{\dagger}(i\omega_{n})G^{-1}(\bm{k},i\omega_{n})\Psi_{\bm{k}}(i\omega_{n})},
\end{eqnarray}
where we define 
\begin{equation}
\Psi_{\bm{k}}(i\omega_{n})=(a_{\bm{k}+}(i\omega_{n}),a_{-\bm{k}+}^{\dagger}(i\omega_{n}),a_{\bm{k}-}(i\omega_{n}),a_{-\bm{k},-}^{\dagger}(i\omega_{n}))^{T}.
\end{equation}
and the matrix $G$ as 
\begin{equation}
G(\bm{k},i\omega_{n})=\left(\begin{array}{cccc}
a_{1} & b & 0 & 0\\
b^{\ast} & a_{2} & 0 & c\\
c & 0 & -a_{2}^{\ast} & -b\\
0 & 0 & -b^{\ast} & -a_{1}^{\ast}
\end{array}\right)^{-1},\label{Green2}
\end{equation}
where $a_{1}=-(\varepsilon_{\bm{k}-m\bm{v}}+U_{R}n-2i\gamma n-\omega),b=-Un,a_{2}=-(\varepsilon_{\bm{k}+m\bm{v}}+U_{R}n-2i\gamma n+\omega),c=-4i\gamma n$.
In this case, we integrate the all bosonic degrees of freedom and
obtain the effective action: 
\begin{equation}
Z=e^{iS_{\text{eff}}}=e^{iF},
\end{equation}
which yields the form of the generating functional $F$ as 
\begin{equation}
F=-i\sum_{k}\text{Tr}\log[iG(\bm{k},\omega)]
\end{equation}
with $\sum_{k}=\sum_{\bm{k}}\sum_{\omega}$. Then we move to calculate
(\ref{Qv}). The first derivative gives the expression for the bosonic
current. 
\begin{equation}
-\langle J_{a}\rangle=-i\frac{1}{2mV}\frac{\partial F}{\partial v_{a}}=-i\frac{1}{2V}\sum_{k}\text{Tr}[(\sigma_{z}\otimes\sigma_{z})\nabla_{a}\varepsilon_{\bm{k}-m\bm{v}\sigma_{0}\otimes\sigma_{z}}G(\bm{k},\omega)].
\end{equation}
In this case the phase stiffness is given by: 
\begin{eqnarray}
Q_{ab} & = & \frac{-1}{2V}\frac{\partial^{2}F}{\partial v_{a}\partial v_{b}}|_{v=0}\nonumber \\
 & = & -i\frac{m^{2}}{2V}\sum_{k}\text{Tr}[(\sigma_{z}\otimes\sigma_{0})\nabla_{ab}^{2}\varepsilon_{\bm{k}}G(\bm{k},i\omega_{n})]+i\frac{m^{2}}{2V}\sum_{k}\nabla_{a}\varepsilon_{k}\nabla_{b}\varepsilon_{k}\text{Tr}[(\sigma_{z}\otimes\sigma_{z})G(\sigma_{z}\otimes\sigma_{z})G].
\end{eqnarray}
Here we apply the equality: $\nabla_{b}G=-G\nabla_{b}G^{-1}G$. Also
we use the part integral to the first part and obtain 
\begin{equation}
Q_{ab}=-i\frac{m^{2}}{2V}\sum_{k}\nabla_{a}\varepsilon_{k}\nabla_{b}\varepsilon_{k}\left\{ \text{Tr}[(\sigma_{z}\otimes\sigma_{z})G(\sigma_{z}\otimes\sigma_{z})G]-\text{Tr}[(\sigma_{z}\otimes\sigma_{0})G(\sigma_{z}\otimes\sigma_{0})G]\right\} .\label{Qab}
\end{equation}
Let's firstly consider a trace: $\text{Tr}[AGAG]$. We transform it
into: 
\begin{eqnarray}
\text{Tr}[AGAG] & = & \text{Tr}[A[G,A]G]+\text{Tr}[A^{2}G^{2}]\nonumber \\
 & = & \text{Tr}[[G,A][G,A]]+\text{Tr}[[G,A]AG]+\text{Tr}[A^{2}G^{2}]\nonumber \\
 & = & \text{Tr}[[G,A][G,A]]+2\text{Tr}[A^{2}G^{2}]-\text{Tr}[AGAG].
\end{eqnarray}
Hence, we reach a conclusion that: 
\begin{equation}
\text{Tr}[AGAG]=\text{Tr}[A^{2}G^{2}]+\frac{1}{2}\text{Tr}[[G,A]^{2}].
\end{equation}
In the expression (\ref{Qab}), we can replace $A$ with $\sigma_{z}\otimes\sigma_{z}$
and $\sigma_{z}\otimes\sigma_{0}$. In this case, both of the matrices
satisfy that $A^{2}=I$. Therefore, we simplify the formula into:
\begin{equation}
Q_{ab}=-i\frac{m^{2}}{4V}\sum_{k}\nabla_{a}\varepsilon_{k}\nabla_{b}\varepsilon_{k}\left\{ \text{Tr}[[G,\sigma_{z}\otimes\sigma_{z}]^{2}]-\text{Tr}[[G,\sigma_{z}\otimes\sigma_{0}]^{2}]\right\} .\label{Qab2}
\end{equation}
In this way, we firstly rewrite the Green's function as 
\begin{equation}
G=\left(\begin{array}{cc}
G_{11} & G_{12}\\
G_{21} & G_{22}
\end{array}\right).
\end{equation}
With the help of Green's function before (Eq. (\ref{Green2})), we
have: 
\begin{eqnarray}
G_{11} & = & \frac{1}{|G^{-1}|}\left(\begin{array}{cc}
a_{2}(a_{1}^{\ast}a_{2}^{\ast}-|b|^{2}) & -b(a_{1}^{\ast}a_{2}^{\ast}-|b|^{2})\\
b^{\ast}(|b|^{2}+c^{2}-a_{1}^{\ast}a_{2}^{\ast}) & a_{1}(a_{1}^{\ast}a_{2}^{\ast}-|b|^{2})
\end{array}\right),\\
G_{12} & = & \frac{1}{|G^{-1}|}\left(\begin{array}{cc}
|b|^{2}c & -a_{2}^{\ast}bc\\
-a_{1}b^{\ast}c & a_{1}a_{2}^{\ast}c
\end{array}\right),\\
G_{21} & = & \frac{1}{|G^{-1}|}\left(\begin{array}{cc}
a_{2}a_{1}^{\ast}c & -a_{1}^{\ast}bc\\
-a_{2}b^{\ast}c & |b|^{2}c
\end{array}\right),\\
G_{22} & = & \frac{1}{|G^{-1}|}\left(\begin{array}{cc}
-a_{1}^{\ast}(a_{1}a_{2}-|b|^{2}) & -b(|b|^{2}+c^{2}-a_{1}a_{2})\\
b^{\ast}(a_{1}a_{2}-|b|^{2}) & -a_{2}^{\ast}(a_{1}a_{2}-|b|^{2})
\end{array}\right)\,,
\end{eqnarray}
where we follow the notation above. Here we define the determinant
of the Green's function as $|G|$ with the form of: 
\begin{equation}
|G^{-1}|=|a_{1}|^{2}|a_{2}|^{2}-a_{1}a_{2}|b|^{2}+|b|^{2}c^{2}-a_{1}^{\ast}a_{2}^{\ast}|b|^{2}+|b|^{4}.\label{deter}
\end{equation}
Then let's focus on the commutators in the phase stiffness $\left(\ref{Qab2}\right)$
which has two components 
\begin{eqnarray}
\text{Tr}[G,\sigma_{z}\otimes\sigma_{z}]^{2} & = & \text{Tr}\left[\left(\begin{array}{cc}
G_{11} & G_{12}\\
G_{21} & G_{22}
\end{array}\right),\sigma_{z}\otimes\sigma_{z}\right]^{2}\nonumber \\
 & = & \text{Tr}[G_{11},\sigma_{z}]^{2}+\text{Tr}[G_{22},\sigma_{z}]^{2}-2\text{Tr}[\{G_{12},\sigma_{z}\}\{G_{21},\sigma_{z}\}],\\
\text{Tr}[[G,\sigma_{z}\otimes\sigma_{0}]^{2}] & = & \text{Tr}\left[\left(\begin{array}{cc}
G_{11} & G_{12}\\
G_{21} & G_{22}
\end{array}\right),\sigma_{z}\otimes\sigma_{0}\right]^{2}\nonumber \\
 & = & -4\text{Tr}[\{G_{12},G_{21}\}]\nonumber \\
 & = & -8\text{Tr}[G_{12}G_{21}].
\end{eqnarray}
Here we define the anti-commutator $\{A,B\}=AB+BA$. The first trace
has four components listed below 
\begin{eqnarray}
\text{Tr}[G_{11},\sigma_{z}]^{2} & = & \frac{1}{|G^{-1}|^{2}}8|b|^{2}(a_{1}^{\ast}a_{2}^{\ast}-|b|^{2})(|b|^{2}+c^{2}-a_{1}^{\ast}a_{2}^{\ast}),\label{G11}\\
\text{Tr}[G_{22},\sigma_{z}]^{2} & = & \frac{1}{|G^{-1}|^{2}}8|b|^{2}(a_{1}a_{2}-|b|^{2})(|b|^{2}+c^{2}-a_{1}a_{2}),\\
\text{Tr}[\{G_{12},\sigma_{z}\}\{G_{21},\sigma_{z}\}] & = & \frac{1}{|G^{-1}|^{2}}4|b|^{2}c^{2}(a_{1}^{\ast}a_{2}+a_{1}a_{2}^{\ast}),\\
\text{Tr}[G_{12}G_{21}] & = & \frac{1}{|G^{-1}|^{2}}|b|^{2}c^{2}[a_{2}a_{1}^{\ast}+|a_{2}|^{2}+|a_{1}|^{2}+a_{1}a_{2}^{\ast}].\label{G12}
\end{eqnarray}
Combining with (\ref{G11}-\ref{G12}), we obtain
\begin{equation}
Q_{ab}=i\frac{4m^{2}}{V}\sum_{k}\nabla_{a}\varepsilon_{k}\nabla_{b}\varepsilon_{k}\frac{-|b|^{2}}{2|G^{-1}|^{2}}\left[(a_{1}^{\ast}a_{2}^{\ast}-|b|^{2})(|b|^{2}+c^{2}-a_{1}^{\ast}a_{2}^{\ast})+\left(a_{1}a_{2}-|b|^{2}\right)(|b|^{2}+c^{2}-a_{1}a_{2})-c^{2}(|a_{2}|^{2}+|a_{1}|^{2})\right].
\end{equation}
Since we here only concern about the diagonal part, we rewrite the
$Q_{ab}$ into $Q(T)\delta_{ab}$. In this case, we obtain 
\begin{equation}
Q=\frac{4}{3}\int_{-\infty}^{\infty}d\omega\int_{-\infty}^{\infty}k^{4}dk\frac{-|b|^{2}}{4\pi^{2}|G^{-1}|^{2}}\left[(a_{1}^{\ast}a_{2}^{\ast}-|b|^{2})(|b|^{2}+c^{2}-a_{1}^{\ast}a_{2}^{\ast})+\left(a_{1}a_{2}-|b|^{2}\right)(|b|^{2}+c^{2}-a_{1}a_{2})-c^{2}(|a_{2}|^{2}+|a_{1}|^{2})\right]\,,\label{Qboson}
\end{equation}
Equation \eqref{Qboson} gives the complete expression of the phase
stiffness for arbitrary interaction strength $U_{R}$ and dissipation
strength $\gamma$. It is difficult to directly show the expression
of the phase stiffness in general. However, we can consider two extreme
cases: $U_{R}\gg\gamma$ and $U_{R}\ll\gamma$. To consider these
problems, we firstly deal with the determinant $|G|$. With the help
of Eq. (\ref{deter}), we have 
\begin{eqnarray}
|G^{-1}| & = & \varepsilon^{4}+2(4\gamma^{2}n^{2}-\omega^{2}-|U|^{2}n^{2})\varepsilon^{2}+(4\gamma^{2}n^{2}+\omega^{2})^{2}-2(4\gamma^{2}n^{2}-\omega^{2})|U|^{2}n^{2}+|U|^{4}n^{4}\nonumber \\
 & \equiv & \omega^{4}-2k_{1}\omega^{2}+k_{2},
\end{eqnarray}
where $k_{1}=(\varepsilon^{2}-4\gamma^{2}n^{2}-|U|^{2}n^{2}),k_{2}=(\varepsilon^{2}+4\gamma^{2}n^{2}-|U|^{2}n^{2})^{2}$.
The numerator of it can be shown by: 
\begin{eqnarray}
 &  & -[(\varepsilon+2i\gamma n)^{2}-\omega^{2}-|U|^{2}n^{2}]^{2}-[(\varepsilon-2i\gamma n)^{2}-\omega^{2}-|U|^{2}n^{2}]^{2}+32\gamma^{2}n^{2}(|U|^{2}n^{2}+8\gamma^{2}n^{2})\nonumber \\
 & = & -2[\omega^{4}-2k_{1}\omega^{2}+k_{1}^{2}-16\gamma^{2}n^{2}\varepsilon^{2}]+32\gamma^{2}n^{2}(|U|^{2}n^{2}+8\gamma^{2}n^{2})
\end{eqnarray}
with the notion $\varepsilon:=\varepsilon_{\bm{k}}+U_{R}n$ for simplicity. 

Let's first consider the case with $\gamma=0$ where there is no dissipation
and the system is closed. In such case, the quantities $a_{1},a_{2},b,c$
become 
\begin{equation}
a_{1}=\omega-\varepsilon_{\bm{k}}-U_{R}n,a_{2}=-\omega-\varepsilon_{\bm{k}}-U_{R}n,b=-U_{R}n,c=0.
\end{equation}
Then we substitute these quantities into superfluid density and obtain
\begin{equation}
Q=\frac{4i}{3}\int_{-\infty}^{\infty}d\omega\int_{-\infty}^{\infty}\frac{k^{4}}{2\pi^{2}}dk\frac{U_{R}^{2}n^{2}}{[-\omega^{2}+\varepsilon_{\bm{k}}^{2}+2\varepsilon_{\bm{k}}U_{R}n]^{2}},\label{Qzero}
\end{equation}
With the similar procedure, we can also calculate $Q'$ in the closed
system without using Schwinger-Keldysh Green's function and find its
relation with $Q$ as $Q'=Q$. The proof is as below.

The Green's function of the forward contour can be written as: 
\begin{equation}
G=-\left(\begin{array}{cc}
\varepsilon_{\bm{k}-m\bm{v}}+U_{R}n-i\omega_{n} & U_{R}n\\
U_{R}n & \varepsilon_{\bm{k}+m\bm{v}}+U_{R}n+i\omega_{n}
\end{array}\right)^{-1}=-(\varepsilon_{\bm{k}-m\bm{v}\sigma_{z}}+U_{R}n+U_{R}n\sigma_{x}-i\omega_{n}\sigma_{z})^{-1}.
\end{equation}
Then we still consider the second derivative to free energy. The first
derivative gives the expression for the bosonic current. 
\begin{equation}
-\langle J_{a}\rangle=\frac{1}{V}\frac{\partial F}{\partial v_{a}}=-\frac{m}{\beta V}\sum_{k}\text{Tr}[\sigma_{z}\nabla_{a}\varepsilon_{\bm{k}-m\bm{v}\sigma_{z}}G(\bm{k},i\omega_{n})].
\end{equation}
In this case the phase stiffness is also given by: 
\begin{eqnarray}
Q'_{ab} & = & -\frac{1}{V}\frac{\partial^{2}F}{\partial v_{a}\partial v_{b}}|_{v=0}\nonumber \\
 & = & -\frac{m^{2}}{\beta V}\sum_{k}\text{Tr}[\nabla_{ab}^{2}\varepsilon_{\bm{k}}G(\bm{k},i\omega_{n})]+\frac{m^{2}}{\beta V}\sum_{k}\nabla_{a}\varepsilon_{k}\nabla_{b}\varepsilon_{k}\text{Tr}[\sigma_{z}G\sigma_{z}G]\\
 & = & -\frac{m^{2}}{\beta V}\sum_{k}\nabla_{a}\varepsilon_{k}\nabla_{b}\varepsilon_{k}\text{Tr}[\sigma_{z}G\sigma_{z}G-G^{2}]\nonumber \\
 & = & -\frac{m^{2}}{2\beta V}\sum_{k}\nabla_{a}\varepsilon_{k}\nabla_{b}\varepsilon_{k}\text{Tr}[\sigma_{z},G]^{2}.
\end{eqnarray}
By explicitly writing down the form of the Green's function, we have
\begin{equation}
G=\frac{1}{(\omega_{n}^{2}+\varepsilon_{\bm{k}}^{2}+2\varepsilon_{\bm{k}}U_{R}n)}\left(\begin{array}{cc}
-\varepsilon_{\bm{k}}-U_{R}n-i\omega_{n} & U_{R}n\\
U_{R}n & -\varepsilon_{\bm{k}}-U_{R}n+i\omega_{n}
\end{array}\right)
\end{equation}
and the commutator takes the form of 
\begin{equation}
[\sigma_{z},G]=\frac{1}{(\omega_{n}^{2}+\varepsilon_{\bm{k}}^{2}+2\varepsilon_{\bm{k}}U_{R}n)}\left(\begin{array}{cc}
0 & -2U_{R}n\\
2U_{R}n & 0
\end{array}\right).
\end{equation}
Therefore, the behaviour of $Q'$ is given by 
\begin{equation}
Q'=\frac{4m^{2}}{\beta V}\sum_{k}\nabla_{a}\varepsilon_{k}\nabla_{b}\varepsilon_{k}\frac{U_{R}^{2}n^{2}}{(\omega_{n}^{2}+\varepsilon_{\bm{k}}^{2}+2\varepsilon_{\bm{k}}U_{R}n)^{2}}.
\end{equation}
By replacing the energy $\varepsilon_{\bm{k}}$ with $\varepsilon_{\bm{k}}=k^{2}/2m$,
we have 
\begin{equation}
Q'=\frac{4m^{2}}{\beta}\sum_{n}\int\frac{d^{3}k}{(2\pi)^{3}}\frac{k_{a}k_{b}}{m^{2}}\frac{U_{R}^{2}n^{2}}{(\omega_{n}^{2}+\varepsilon_{\bm{k}}^{2}+2\varepsilon_{\bm{k}}U_{R}n)^{2}}=\frac{4}{3\beta}\sum_{n}\int\frac{k^{4}dk}{2\pi^{2}}\frac{U_{R}^{2}n^{2}}{(\omega_{n}^{2}+\varepsilon_{\bm{k}}^{2}+2\varepsilon_{\bm{k}}U_{R}n)^{2}}.
\end{equation}
By comparing with (\ref{Qzero}), we reach the conclusion 
\begin{equation}
Q'=Q
\end{equation}
after Wick rotation. The phase stiffness thus becomes 
\begin{eqnarray}
Q' & = & \frac{4}{3}\int\frac{d\omega}{2\pi}\int\frac{k^{4}dk}{2\pi^{2}}\frac{U_{R}^{2}n^{2}}{(\omega^{2}+\varepsilon_{\bm{k}}^{2}+2\varepsilon_{\bm{k}}U_{R}n)^{2}}\nonumber \\
 & = & \frac{2}{3}\int_{0}^{\infty}\frac{k^{4}dk}{2\pi^{2}}\frac{U_{R}^{2}n^{2}}{[\varepsilon_{\bm{k}}^{2}+2\varepsilon_{\bm{k}}U_{R}n]^{3/2}}\nonumber \\
 & = & \frac{1}{3\pi^{2}}\sqrt{n^{3}U_{R}^{3}}m^{5/2}\nonumber \\
 & \propto & (U_{R}n)^{3/2}.\label{QQ}
\end{eqnarray}
This is consistent with the result calculated by linear response theory
\citep{Ueda2010}
\begin{equation}
n_{D}=\frac{1}{3\pi^{2}}\sqrt{(nU_{R})^{3}m^{3}}.
\end{equation}

If we assume $U_{R}\gg\gamma\neq0$, i.e., the dissipation is very
weak but non-vanishing, we derive the phase stiffness from the trace
in Eq. \eqref{Qab} 
\begin{eqnarray}
 &  & \text{Tr}[(\sigma_{z}\otimes\sigma_{z})G(\sigma_{z}\otimes\sigma_{z})G]-\text{Tr}[(\sigma_{z}\otimes\sigma_{0})G(\sigma_{z}\otimes\sigma_{0})G]\nonumber \\
 & = & \frac{8U_{R}^{2}n^{2}}{(\varepsilon_{\bm{k}}(\varepsilon_{\bm{k}}+2U_{R}n)-\omega^{2})^{2}}+\frac{8(-6U_{R}^{2}n^{2}(\varepsilon_{\bm{k}}(\varepsilon_{\bm{k}}+2U_{R}n)+7\omega^{2})+(\varepsilon_{\bm{k}}(\varepsilon_{\bm{k}}+2U_{R}n)-\omega^{2})^{2})}{(\varepsilon_{\bm{k}}(\varepsilon_{\bm{k}}+2U_{R}n)-\omega^{2})^{4}}\gamma^{2}n^{2}.\label{eq:divergent}
\end{eqnarray}
Here we only expand up to the second order of $\gamma n$. After integration
over $\omega$ under Wick rotation, we have 
\begin{eqnarray}
 &  & -i\int d\omega\text{Tr}[(\sigma_{z}\otimes\sigma_{z})G(\sigma_{z}\otimes\sigma_{z})G]-\text{Tr}[(\sigma_{z}\otimes\sigma_{0})G(\sigma_{z}\otimes\sigma_{0})G]\nonumber \\
 & = & \frac{4\pi U_{R}^{2}n^{2}}{(\varepsilon_{\bm{k}}(\varepsilon_{\bm{k}}+2U_{R}n))^{3/2}}+2\pi\frac{(3(U_{R}n)^{2}+4\varepsilon_{\bm{k}}U_{R}n+2\varepsilon_{\bm{k}}^{2})\sqrt{\varepsilon_{\bm{k}}(\varepsilon_{\bm{k}}+2U_{R}n)}}{(\varepsilon_{\bm{k}}(\varepsilon_{\bm{k}}+2U_{R}n))^{3}}\gamma^{2}n^{2},
\end{eqnarray}
which gives the form of the phase stiffness as 
\begin{eqnarray}
Q & = & \frac{m^{5/2}}{3\pi^{2}}\left[(U_{R}n)^{3/2}+\frac{(\gamma n)^{2}}{2\sqrt{2U_{R}n}}\int_{0}^{\infty}dxx^{\frac{3}{2}}\frac{(3+4x+2x^{2})\sqrt{x(2+x)}}{x^{3}(2+x)^{3}}\right]\nonumber \\
 & \simeq & \frac{m^{5/2}}{3\pi^{2}}(U_{R}n)^{3/2}\left[1+\frac{1}{2\sqrt{2}}\left(\frac{\gamma}{U_{R}}\right)^{2}\times\mathcal{O}(1)\right].\label{eq:integration}
\end{eqnarray}
Actually, the integration of the second term in Eq. (\ref{eq:divergent})
is divergent at $\omega=0$, since our theory is only consistent in
short time $\gamma t\ll1$ when $n$ can be regarded as a constant.
Therefore, to obtain the right result, one has to add a cutoff at
the small frequency domain of Eq. (\ref{eq:divergent}). Alternatively,
A simple regularization of Eq. (\ref{eq:integration}) by substituting
$3/2^{\frac{3}{2}}x(2+x)$ from the integral to remove the IR divergence
would give the integration result $\frac{4+3\ln2}{2\sqrt{2}}$.

Then we turn to the other limit $U_{R}=0$. In this way, we have 
\begin{equation}
a_{1}=-(\varepsilon_{\bm{k}}-2i\gamma n-\omega),b=i\gamma n,a_{2}=-(\varepsilon_{\bm{k}}-2i\gamma n+\omega),c=-4i\gamma n
\end{equation}
The numerator becomes 
\begin{equation}
-2[\omega^{4}-2k_{1}\omega^{2}+k_{1}^{2}-16\gamma^{2}n^{2}\varepsilon_{\bm{k}}^{2}]+288\gamma^{4}n^{4},
\end{equation}
where $k_{1}=(\varepsilon_{\bm{k}}^{2}-5\gamma^{2}n^{2}),k_{2}=(\varepsilon_{\bm{k}}^{2}+3\gamma^{2}n^{2})^{2}$.
Besides, the denominator becomes 
\begin{equation}
|G^{-1}|^{2}=(\omega^{4}-2k_{1}\omega^{2}+k_{2})^{2}
\end{equation}
Hence, the phase stiffness can be rewritten as 
\begin{equation}
Q=\frac{(2m)^{5/2}\gamma^{2}n^{2}}{3\pi^{2}}\int_{-\infty}^{\infty}d\omega\int_{0}^{\infty}\varepsilon_{\bm{k}}^{3/2}d\varepsilon_{\bm{k}}\frac{[\omega^{4}-2k_{1}\omega^{2}+k_{1}^{2}-16\gamma^{2}n^{2}\varepsilon_{\bm{k}}^{2}]-144\gamma^{4}n^{4}}{(\omega^{4}-2k_{1}\omega^{2}+k_{2})^{2}}\,,
\end{equation}
By substituting the integrated variables as $\omega\rightarrow\gamma n\omega,\varepsilon_{\bm{k}}\rightarrow\gamma n\varepsilon_{\bm{k}}$,
we have 
\begin{equation}
Q=\frac{(2m)^{5/2}}{3\pi^{2}}(\gamma n)^{3/2}\times A,\label{QU=00003D0}
\end{equation}
where $A$ takes the form of 
\begin{eqnarray*}
A & = & \int_{-\infty}^{\infty}dx\int_{0}^{\infty}y^{3/2}dy\frac{[x^{4}+2(y^{2}-5)x^{2}+(y^{2}-5)^{2}-16y^{2}]-144}{(x^{4}+2(y^{2}-5)x^{2}+(y^{2}+3)^{2})^{2}}\\
 & = & \frac{1}{2^{11/2}\sqrt{\pi}}\Gamma\left(\frac{1}{4}\right)^{2}>0.
\end{eqnarray*}
From the expression (\ref{QU=00003D0}), we can see when $U_{R}=0$,
the density of superfluid in quasi-steady state is propotional to
$(\gamma n)^{3/2}$, which indicates the quantum depletion part is
induced purely from dissipation. In addition, we can also see the
perturbation from the superfluid phase stiffness \eqref{Qboson}.
Further, if we assume $0\neq U_{R}\ll\gamma$ and expand the phase
stiffness \eqref{Qab} around $U_{R}=0$, we obtain 
\begin{eqnarray}
 &  & \text{Tr}[(\sigma_{z}\otimes\sigma_{z})G(\sigma_{z}\otimes\sigma_{z})G]-\text{Tr}[(\sigma_{z}\otimes\sigma_{0})G(\sigma_{z}\otimes\sigma_{0})G]\nonumber \\
 & = & -4\gamma^{2}n^{2}\left[\frac{1}{(\omega^{2}-4i\gamma n\omega-\varepsilon_{\bm{k}}^{2}-3\gamma^{2}n^{2})^{2}}+\frac{1}{(\omega^{2}+4i\gamma n\omega-\varepsilon_{\bm{k}}^{2}-3\gamma^{2}n^{2})^{2}}\right]\nonumber \\
 &  & +4\gamma nU_{R}n\left[\frac{4\varepsilon_{\bm{k}}}{(\omega^{2}-4i\gamma n\omega-\varepsilon_{\bm{k}}^{2}-3\gamma^{2}n^{2})^{3}}+\frac{4\varepsilon_{\bm{k}}}{(\omega^{2}+4i\gamma n\omega-\varepsilon_{\bm{k}}^{2}-3\gamma^{2}n^{2})^{2}}\right]\nonumber \\
 &  & +O(U_{R}^{2}n^{2}).
\end{eqnarray}
By substituting the expression into Eq. \eqref{Qab} and integrate
the variables $\omega$ and $\varepsilon_{\bm{k}}$, we have the phase
stiffness 
\begin{equation}
Q=\frac{m^{5/2}(\gamma n)^{3/2}}{24\pi^{5/2}}\Gamma\left(\frac{1}{4}\right)^{2}\left(1+6\frac{U_{R}}{\gamma}\frac{\Gamma(3/4)^{2}}{\Gamma(1/4)^{2}}\right)+O(U_{R}^{2}/\gamma^{2}).
\end{equation}


\section{Derivation of Spectral Function and Spectrum}

Here we show the Green's function of the dissipative superfluids.
The correlation-function matrix takes the form of 
\begin{eqnarray}
G & = & -i\left\langle \left(\begin{array}{c}
a_{k\bm{,+}}\\
a_{-k\bm{,+}}^{\dagger}\\
a_{k,-}\\
a_{-k,-}^{\dagger}
\end{array}\right)\left(\begin{array}{cccc}
a_{k,+}^{\dagger} & a_{-k,+} & a_{k,-}^{\dagger} & a_{-k,-}\end{array}\right)\right\rangle \nonumber \\
 & = & \left(\begin{array}{cccc}
-\frac{\varepsilon_{\bm{k}}+U_{R}n-2i\gamma n-\omega}{2} & -\frac{Un}{2}\\
-\frac{U^{\ast}n}{2} & -\frac{\varepsilon_{\bm{k}}+U_{R}n-2i\gamma n+\omega}{2} &  & -2i\gamma n\\
-2i\gamma n &  & \frac{\varepsilon_{\bm{k}}+U_{R}n+2i\gamma n-\omega}{2} & \frac{Un}{2}\\
 &  & \frac{U^{\ast}n}{2} & \frac{\varepsilon_{\bm{k}}+U_{R}n+2i\gamma n+\omega}{2}
\end{array}\right)^{-1}.
\end{eqnarray}
By directly deriving the inverse matrix, we have the following Green's
function \citep{Kamenev_2011}: 
\begin{eqnarray}
G^{T}(\bm{k},\omega) & \equiv & -i\langle a_{k,+}a_{k,+}^{\dagger}\rangle\nonumber \\
 & = & \frac{2(\omega+U_{R}n-2i\gamma n+\varepsilon_{\bm{k}})(\omega^{2}-\varepsilon_{\bm{k}}^{2}-2\varepsilon_{\bm{k}}(U_{R}n+2i\gamma n)+\gamma n(5\gamma n-4iU_{R}n))}{\omega^{4}+2(5\gamma^{2}n^{2}-\varepsilon_{\bm{k}}(\varepsilon_{\bm{k}}+2U_{R}n))\omega^{2}+[\varepsilon_{\bm{k}}(\varepsilon_{\bm{k}}+2U_{R}n)+3\gamma^{2}n^{2}]^{2}},\\
G^{<}(\bm{k},\omega) & \equiv & -i\langle a_{k,+}a_{k,-}^{\dagger}\rangle\nonumber \\
 & = & \frac{-8i\gamma n(U_{R}^{2}n^{2}+\gamma^{2}n^{2})}{\omega^{4}+2(5\gamma^{2}n^{2}-\varepsilon_{\bm{k}}(\varepsilon_{\bm{k}}+2U_{R}n))\omega^{2}+[\varepsilon_{\bm{k}}(\varepsilon_{\bm{k}}+2U_{R}n)+3\gamma^{2}n^{2}]^{2}},\\
G^{>}(\bm{k},\omega) & = & -i\langle a_{k,-}a_{k,+}^{\dagger}\rangle\nonumber \\
 & = & \frac{-8i\gamma n((\omega+\varepsilon_{\bm{k}}+U_{R}n)^{2}+4\gamma^{2}n^{2})}{\omega^{4}+2(5\gamma^{2}n^{2}-\varepsilon_{\bm{k}}(\varepsilon_{\bm{k}}+2U_{R}n))\omega^{2}+[\varepsilon_{\bm{k}}(\varepsilon_{\bm{k}}+2U_{R}n)+3\gamma^{2}n^{2}]^{2}},\\
G^{\tilde{T}}(\bm{k},\omega) & = & -i\langle a_{k,-}a_{k,-}^{\dagger}\rangle\nonumber \\
 & = & \frac{2(\omega+U_{R}n+2i\gamma n+\varepsilon_{\bm{k}})(-\omega^{2}+\varepsilon_{\bm{k}}^{2}+2\varepsilon_{\bm{k}}(U_{R}n-2i\gamma n)-\gamma n(5\gamma n+4iU_{R}n))}{\omega^{4}+2(5\gamma^{2}n^{2}-\varepsilon_{\bm{k}}(\varepsilon_{\bm{k}}+2U_{R}n))\omega^{2}+[\varepsilon_{\bm{k}}(\varepsilon_{\bm{k}}+2U_{R}n)+3\gamma^{2}n^{2}]^{2}}.
\end{eqnarray}
From these expression we can see the fundamental relation: $G^{T}=-(G^{\tilde{T}})^{\ast}$.
To observe the spectrum function, we need first investigate the retarded
Green's function, which is given by the relation 
\begin{eqnarray}
\left(\begin{array}{cc}
G^{K} & G^{R}\\
G^{A} & 0
\end{array}\right) & = & \left(\begin{array}{cc}
1 & 1\\
1 & -1
\end{array}\right)\left(\begin{array}{cc}
G^{T} & G^{<}\\
G^{>} & G^{\tilde{T}}
\end{array}\right)\left(\begin{array}{cc}
1 & 1\\
1 & -1
\end{array}\right).
\end{eqnarray}
Therefore, the retarded, advanced and Keldysh Green's functions are
given by 
\begin{eqnarray}
G^{K}(\bm{k},\omega) & = & \frac{-8i\gamma n((\omega+\varepsilon_{\bm{k}}+U_{R}n)^{2}+U_{R}^{2}n^{2}+5\gamma^{2}n^{2})}{\omega^{4}+2(5\gamma^{2}n^{2}-\varepsilon_{\bm{k}}(\varepsilon_{\bm{k}}+2U_{R}n))\omega^{2}+[\varepsilon_{\bm{k}}(\varepsilon_{\bm{k}}+2U_{R}n)+3\gamma^{2}n^{2}]^{2}},\\
G^{R}(\bm{k},\omega) & = & \frac{2(\omega+\varepsilon_{\bm{k}}+U_{R}n+2i\gamma n)}{\omega^{2}+4i\gamma n\omega-\varepsilon_{\bm{k}}^{2}-2U_{R}n\varepsilon_{\bm{k}}-3\gamma^{2}n^{2}},\\
G^{A}(\bm{k},\omega) & = & \frac{2(\omega+\varepsilon_{\bm{k}}+U_{R}n-2i\gamma n)}{\omega^{2}-4i\gamma n\omega-\varepsilon_{\bm{k}}^{2}-2U_{R}n\varepsilon_{\bm{k}}-3\gamma^{2}n^{2}}=(G^{R})^{\ast}.
\end{eqnarray}
Based on the retarded Green's function, we obtain the spectrum function
as \citep{Coleman_2015}
\begin{eqnarray}
A(\bm{k},\omega) & = & \frac{i}{2\pi}(G^{<}+G^{>})\nonumber \\
 & = & \frac{1}{\pi}\frac{4\gamma n((\omega+\varepsilon_{\bm{k}}+U_{R}n)^{2}+U_{R}^{2}n^{2}+5\gamma^{2}n^{2})}{\omega^{4}+2(5\gamma^{2}n^{2}-\varepsilon_{\bm{k}}(\varepsilon_{\bm{k}}+2U_{R}n))\omega^{2}+[\varepsilon_{\bm{k}}(\varepsilon_{\bm{k}}+2U_{R}n)+3\gamma^{2}n^{2}]^{2}}.
\end{eqnarray}
The figures of the spectral function are shown in the main text. We
note that when $U_{R}=0$, the spectrum function becomes 
\begin{eqnarray*}
A(\bm{k},\omega) & = & \frac{1}{\pi}\frac{4\gamma n((\omega+\varepsilon_{\bm{k}})^{2}+5\gamma^{2}n^{2})}{\omega^{4}+2(5\gamma^{2}n^{2}-\varepsilon_{\bm{k}}^{2})\omega^{2}+[\varepsilon_{\bm{k}}^{2}+3\gamma^{2}n^{2}]^{2}}\\
 & = & \frac{1}{\pi\gamma n}\frac{4((\omega'+\varepsilon')^{2}+5)}{{\omega'}^{4}+2\left({5-\varepsilon'}^{2}\right){\omega'}^{2}+\left({\varepsilon'}^{2}+3\right)^{2}}.
\end{eqnarray*}
where $\omega'=\omega/\gamma n,\varepsilon'=\varepsilon_{\bm{k}}/\gamma n$.

The spectrum of the Schwinger-Keldysh action is given by the poles
of the spectral function as 
\begin{equation}
\omega^{4}+2(5\gamma^{2}n^{2}-\varepsilon_{\bm{k}}(\varepsilon_{\bm{k}}+2U_{R}n))\omega^{2}+[\varepsilon_{\bm{k}}(\varepsilon_{\bm{k}}+2U_{R}n)+3\gamma^{2}n^{2}]^{2}=0.
\end{equation}
Nevertheless, the polynomial on the L.H.S can be factorized as 
\begin{equation}
(\omega^{2}+4i\gamma n\omega-\varepsilon_{\bm{k}}^{2}-2U_{R}n\varepsilon_{\bm{k}}-3\gamma^{2}n^{2})(\omega^{2}-4i\gamma n\omega-\varepsilon_{\bm{k}}^{2}-2U_{R}n\varepsilon_{\bm{k}}-3\gamma^{2}n^{2}).\label{eq:poles}
\end{equation}
The solutions to Eq. \eqref{eq:poles} can be expressed as 
\begin{eqnarray}
\omega_{1,2} & = & -2i\gamma n\pm\sqrt{\varepsilon_{\bm{k}}(\varepsilon_{\bm{k}}+2U_{R}n)-\gamma^{2}n^{2}},\\
\omega_{3,4} & = & 2i\gamma n\pm\sqrt{\varepsilon_{\bm{k}}(\varepsilon_{\bm{k}}+2U_{R}n)-\gamma^{2}n^{2}}.
\end{eqnarray}
For those momentum satisfying $\varepsilon_{\bm{k}}(\varepsilon_{\bm{k}}+2U_{R}n)<\gamma^{2}n^{2}$,
the real parts of all the spectrums are zero. For those momentum satisfying
$\varepsilon_{\bm{k}}(\varepsilon_{\bm{k}}+2U_{R}n)>\gamma^{2}n^{2}$,
the real parts of $\omega_{1(2)}$ are equal to $\omega_{3(4)}$.
The relations are given by $\omega_{1}=\omega_{3}^{\ast}$ and $\omega_{2}=\omega_{4}^{\ast}$.
Hence, we can reach the conclusion that 
\begin{equation}
\text{Re}[\omega_{1}]=\text{Re}[\omega_{3}]=-\text{Re}[\omega_{2}]=-\text{Re}[\omega_{4}].
\end{equation}
There is only one nontrivial real part in four spectrums. For the
evolution on the quantum states, the poles of the retarded Green's
function give the spectrum $\omega_{1,2}$. These two spectrums coincides
with the spectrums of the observables $A_{\bm{k}}$, indeed. We can
also see this point from another framework of the Schwinger-Keldysh
action. We first transform the action into another basis. By defining
the retarded or advanced operators $a_{\bm{k},R}=\frac{1}{2}(a_{\bm{k},+}+a_{\bm{k},-})$
and $a_{\bm{k},A}=a_{\bm{k},+}-a_{\bm{k},-}$ \citep{Kamenev_2011},
we have 
\begin{equation}
a_{\bm{k}+}^{\dagger}i\partial_{t}a_{\bm{k}+}-a_{\bm{k}-}^{\dagger}i\partial_{t}a_{\bm{k}-}=a_{\bm{k},A}^{\dagger}i\partial_{t}a_{\bm{k},R}-a_{\bm{k},R}^{\dagger}i\partial_{t}a_{\bm{k},A},
\end{equation}
\begin{eqnarray}
-H_{+}+H_{-}-4i\gamma n\sum_{\bm{k}}a_{\bm{k}-}^{\dagger}a_{\bm{k}+} & = & \sum_{\bm{k}}-(\varepsilon_{\bm{k}}+U_{R}n-2i\gamma n)a_{\bm{k},A}^{\dagger}a_{\bm{k},R}-(\varepsilon_{\bm{k}}+U_{R}n+2i\gamma n)a_{\bm{k},R}^{\dagger}a_{\bm{k},A}\nonumber \\
 &  & -\frac{U^{\ast}n}{2}(a_{-\bm{k},R}a_{\bm{k},A}+a_{-\bm{k},A}a_{\bm{k},R})-\frac{Un}{2}(a_{\bm{k},R}^{\dagger}a_{-\bm{k},A}^{\dagger}+a_{\bm{k},A}^{\dagger}a_{-\bm{k},R}^{\dagger}).
\end{eqnarray}
Hence, the action can be reorganized as 
\begin{equation}
S=\frac{1}{2}\sum_{\bm{k}}\int\frac{d\omega}{2\pi}\left(\begin{array}{cccc}
a_{\bm{k},R}^{\dagger} & a_{-\bm{k},R} & a_{\bm{k},A}^{\dagger} & a_{-\bm{k},A}\end{array}\right)\left(\begin{array}{cc}
O_{2\times2} & G\\
G^{\dagger} & O_{2\times2}
\end{array}\right)\left(\begin{array}{c}
a_{\bm{k},R}\\
a_{-\bm{k},R}^{\dagger}\\
a_{\bm{k},A}\\
a_{-\bm{k},A}^{\dagger}
\end{array}\right),\label{eq:action4}
\end{equation}
where 
\begin{equation}
G=\left(\begin{array}{cc}
\omega-(\varepsilon_{\bm{k}}+U_{R}n+2i\gamma n) & -Un\\
-U^{\ast}n & -\omega-(\varepsilon_{\bm{k}}+U_{R}n-2i\gamma n)
\end{array}\right).
\end{equation}
The poles of the Green's function are given by 
\begin{equation}
\det(G)=0\Rightarrow(\omega-(\varepsilon_{\bm{k}}+U_{R}n+2i\gamma n))(\omega+(\varepsilon_{\bm{k}}+U_{R}n-2i\gamma n))+|U|^{2}n^{2}=0.
\end{equation}
\begin{equation}
\det(G^{\dagger})=0\Rightarrow(\omega-(\varepsilon_{\bm{k}}+U_{R}n-2i\gamma n))(\omega+(\varepsilon_{\bm{k}}+U_{R}n+2i\gamma n))+|U|^{2}n^{2}=0.
\end{equation}
These two equations can be rewritten as 
\begin{eqnarray}
\omega^{2}-4i\gamma n\omega-\varepsilon_{\bm{k}}^{2}-2U_{R}n\varepsilon_{\bm{k}}-3\gamma^{2}n^{2} & = & 0,\\
\omega^{2}+4i\gamma n\omega-\varepsilon_{\bm{k}}^{2}-2U_{R}n\varepsilon_{\bm{k}}-3\gamma^{2}n^{2} & = & 0.
\end{eqnarray}
Hence, the spectrum $\omega_{1,2}$ represent the spectrums of the
retarded bosonic operators and the spectrum $\omega_{3,4}$ represent
the spectrums of the advanced bosnic operators. To see the fundamental
excitations in the system, we need to diagonalize the action \eqref{eq:action4}
with similar transformation. We first review the similar transformation
in the closed quantum systems. The mean-field Hamiltonian takes the
form of 
\begin{equation}
H=\frac{1}{2}\left(\begin{array}{cc}
a_{k}^{\dagger} & a_{-k}\end{array}\right)\left(\begin{array}{cc}
\varepsilon_{\bm{k}}+Un & Un\\
Un & \varepsilon_{\bm{k}}+Un
\end{array}\right)\left(\begin{array}{c}
a_{k}\\
a_{-k}^{\dagger}
\end{array}\right).
\end{equation}

To derive the Bogoliubov diagonalization matrix, we first transform
the matrix as 
\begin{equation}
\left(\begin{array}{cc}
\varepsilon_{\bm{k}}+Un & Un\\
Un & \varepsilon_{\bm{k}}+Un
\end{array}\right)\rightarrow\left(\begin{array}{cc}
\varepsilon_{\bm{k}}+Un & Un\\
Un & \varepsilon_{\bm{k}}+Un
\end{array}\right)\left(\begin{array}{cc}
1 & 0\\
0 & -1
\end{array}\right)=\left(\begin{array}{cc}
\varepsilon_{\bm{k}}+Un & -Un\\
Un & -(\varepsilon_{\bm{k}}+Un)
\end{array}\right)
\end{equation}
since the commutation relation differs for different indices in the
matrix. The eigenvectors of the matrix are 
\begin{eqnarray}
E_{1}=\sqrt{\varepsilon_{\bm{k}}(\varepsilon_{\bm{k}}+2Un)} & , & v_{1}=\left(1+x^{2}+x\sqrt{x^{2}+2},1\right)^{T},\\
E_{2}=-\sqrt{\varepsilon_{\bm{k}}(\varepsilon_{\bm{k}}+2Un)} & , & v_{2}=\left(1+x^{2}-x\sqrt{x^{2}+2},1\right)^{T},
\end{eqnarray}
where $x=\sqrt{\varepsilon_{\bm{k}}/Un}$. Hence, the similarity transformation
can be written as 
\begin{equation}
U=\left(\begin{array}{cc}
1+x^{2}+x\sqrt{x^{2}+2} & 1+x^{2}-x\sqrt{x^{2}+2}\\
1 & 1
\end{array}\right)\equiv\left(\begin{array}{cc}
\frac{1}{\alpha} & \alpha\\
1 & 1
\end{array}\right),
\end{equation}
where $\alpha=1+x^{2}-x\sqrt{x^{2}+2}$.

To recover the Bogoliubov transformation, we renormalize each eigenvector
as 
\begin{eqnarray}
v_{1} & \rightarrow & \beta_{1}v_{1}=\left(\frac{1}{\sqrt{1-\alpha^{2}}},\frac{\alpha}{\sqrt{1-\alpha^{2}}}\right)^{T},\\
v_{2} & \rightarrow & \beta_{2}v_{2}=\left(\frac{\alpha}{\sqrt{1-\alpha^{2}}},\frac{1}{\sqrt{1-\alpha^{2}}}\right)^{T},
\end{eqnarray}
with $\beta_{1}=\alpha/\sqrt{1-\alpha^{2}},\beta_{2}=1/\sqrt{1-\alpha^{2}}$.
In this case we obtain the Bogoliubov transformation matrix \citep{Ueda2010}
\begin{equation}
U=\left(\begin{array}{cc}
\frac{1}{\sqrt{1-\alpha^{2}}} & \frac{\alpha}{\sqrt{1-\alpha^{2}}}\\
\frac{\alpha}{\sqrt{1-\alpha^{2}}} & \frac{1}{\sqrt{1-\alpha^{2}}}
\end{array}\right).\label{Bogoliubov}
\end{equation}
Similarly, in the open quantum system, we need to diagonalize the
matrix 
\begin{equation}
J=\left(\begin{array}{cc}
\varepsilon_{\bm{k}}+U_{R}n+2i\gamma n & Un\\
U^{\ast}n & \varepsilon_{\bm{k}}+U_{R}n-2i\gamma n
\end{array}\right)
\end{equation}
from the matrix $G$. We follow the same procedure above and transform
$J$ as $J\sigma_{z}$ for diagonalization. The problem is equivalent
to diagonalize the matrix 
\begin{equation}
\left(\begin{array}{cc}
\varepsilon_{\bm{k}}+U_{R}n+2i\gamma n & -Un\\
U^{\ast}n & -(\varepsilon_{\bm{k}}+U_{R}n-2i\gamma n)
\end{array}\right).
\end{equation}
The eigenvectors are given by 
\begin{eqnarray}
v_{1} & = & \left(\frac{U_{R}n+\varepsilon_{\bm{k}}+\sqrt{\varepsilon_{\bm{k}}(\varepsilon_{\bm{k}}+2U_{R}n)-\gamma^{2}n^{2}}}{U^{\ast}n},1\right)^{T}\nonumber \\
 & \equiv & \left(\frac{1}{\bar{\alpha}},1\right)^{T},\\
v_{2} & = & \left(\frac{U_{R}n+\varepsilon_{\bm{k}}-\sqrt{\varepsilon_{\bm{k}}(\varepsilon_{\bm{k}}+2U_{R}n)-\gamma^{2}n^{2}}}{U^{\ast}n},1\right)^{T}\nonumber \\
 & \equiv & (\alpha,1)^{T},
\end{eqnarray}
with $\alpha=\frac{U_{R}n+\varepsilon_{\bm{k}}-\sqrt{\varepsilon_{\bm{k}}(\varepsilon_{\bm{k}}+2U_{R}n)-\gamma^{2}n^{2}}}{U^{\ast}n},\bar{\alpha}=\frac{U_{R}n+\varepsilon_{\bm{k}}-\sqrt{\varepsilon_{\bm{k}}(\varepsilon_{\bm{k}}+2U_{R}n)-\gamma^{2}n^{2}}}{Un}$.
We note that $\alpha^{\ast}=\bar{\alpha}$ only holds when $\varepsilon_{\bm{k}}(\varepsilon_{\bm{k}}+2U_{R}n)-\gamma^{2}n^{2}\geqslant0$
which corresponds to a nontrivial real part. Therefore, the diagonalization
matrix is given by 
\begin{equation}
U=\left(\begin{array}{cc}
\frac{1}{\bar{\alpha}} & \alpha\\
1 & 1
\end{array}\right).
\end{equation}
By gauge transformation 
\begin{eqnarray}
v_{1} & \rightarrow & \beta_{1}v_{1}=\left(\frac{1}{\sqrt{1-\alpha\bar{\alpha}}},\frac{\bar{\alpha}}{\sqrt{1-\alpha\bar{\alpha}}}\right)^{T},\\
v_{2} & \rightarrow & \beta_{2}v_{2}=\left(\frac{\alpha}{\sqrt{1-\alpha\bar{\alpha}}},\frac{1}{\sqrt{1-\alpha\bar{\alpha}}}\right)^{T},
\end{eqnarray}
we obtain the Bogoliubov matrix for the retarded(classical) operators
\begin{equation}
U=\left(\begin{array}{cc}
\frac{1}{\sqrt{1-\alpha\bar{\alpha}}} & \frac{\alpha}{\sqrt{1-\alpha\bar{\alpha}}}\\
\frac{\bar{\alpha}}{\sqrt{1-\alpha\bar{\alpha}}} & \frac{1}{\sqrt{1-\alpha\bar{\alpha}}}
\end{array}\right),\left(\begin{array}{c}
\beta_{\bm{k},R}\\
\bar{\beta}_{-\bm{k},R}
\end{array}\right)=\left(\begin{array}{cc}
\frac{1}{\sqrt{1-\alpha\bar{\alpha}}} & \frac{\alpha}{\sqrt{1-\alpha\bar{\alpha}}}\\
\frac{\bar{\alpha}}{\sqrt{1-\alpha\bar{\alpha}}} & \frac{1}{\sqrt{1-\alpha\bar{\alpha}}}
\end{array}\right)\left(\begin{array}{c}
a_{\bm{k},R}\\
a_{-\bm{k},R}^{\dagger}
\end{array}\right).
\end{equation}
and the Bogoliubov matrix for the advanced (quantum) operators 
\begin{equation}
U^{-1}=\left(\begin{array}{cc}
\frac{1}{\sqrt{1-\alpha\bar{\alpha}}} & -\frac{\alpha}{\sqrt{1-\alpha\bar{\alpha}}}\\
-\frac{\bar{\alpha}}{\sqrt{1-\alpha\bar{\alpha}}} & \frac{1}{\sqrt{1-\alpha\bar{\alpha}}}
\end{array}\right),\left(\begin{array}{c}
\beta_{\bm{k},A}\\
\bar{\beta}_{-\bm{k},A}
\end{array}\right)=\left(\begin{array}{cc}
\frac{1}{\sqrt{1-\alpha\bar{\alpha}}} & \frac{\alpha}{\sqrt{1-\alpha\bar{\alpha}}}\\
\frac{\bar{\alpha}}{\sqrt{1-\alpha\bar{\alpha}}} & \frac{1}{\sqrt{1-\alpha\bar{\alpha}}}
\end{array}\right)\left(\begin{array}{c}
a_{\bm{k},A}\\
a_{-\bm{k},A}^{\dagger}
\end{array}\right).
\end{equation}
Hence, the action becomes 
\begin{equation}
S=\frac{1}{2}\sum_{\bm{k}}\int\frac{d\omega}{2\pi}\left(\begin{array}{cccc}
\bar{\beta}_{\bm{k},R} & \beta_{-\bm{k},R} & \bar{\beta}_{\bm{k},A} & \beta_{-\bm{k},A}\end{array}\right)\left(\begin{array}{cc}
O_{2\times2} & H'\\
(H')^{\dagger} & O_{2\times2}
\end{array}\right)\left(\begin{array}{c}
\beta_{\bm{k},R}\\
\bar{\beta}_{-\bm{k},R}\\
\beta_{\bm{k},A}\\
\bar{\beta}_{-\bm{k},A}
\end{array}\right),
\end{equation}
where 
\begin{equation}
H'=\left(\begin{array}{cc}
i\partial_{t}\sigma_{z}-\left(-2i\gamma-\sqrt{\varepsilon_{\bm{k}}(\varepsilon_{\bm{k}}+2U_{R}n)-\gamma^{2}n^{2}}\right) & 0\\
0 & i\partial_{t}\sigma_{z}-\left(-2i\gamma+\sqrt{\varepsilon_{\bm{k}}(\varepsilon_{\bm{k}}+2U_{R}n)-\gamma^{2}n^{2}}\right)
\end{array}\right).
\end{equation}
In the expressions above, we can obtain $\bar{\beta}=\beta^{\dagger}$
when the momentum satisfies $\varepsilon_{\bm{k}}(\varepsilon_{\bm{k}}+2U_{R}n)>\gamma^{2}n^{2}$.
 

\bibliographystyle{apsrev4-2}
\bibliography{MyNewCollection}

\end{document}
