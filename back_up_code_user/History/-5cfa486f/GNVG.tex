%% LyX 2.3.6.1 created this file.  For more info, see http://www.lyx.org/.
%% Do not edit unless you really know what you are doing.
\documentclass[14pt,english]{article}
\usepackage[T1]{fontenc}
\usepackage{geometry}
\geometry{verbose,lmargin=0.5cm,rmargin=0.5cm}
\usepackage{amstext}
\usepackage{amssymb}
\usepackage[authoryear]{natbib}

\makeatletter

%%%%%%%%%%%%%%%%%%%%%%%%%%%%%% LyX specific LaTeX commands.
%% Because html converters don't know tabularnewline
\providecommand{\tabularnewline}{\\}

%%%%%%%%%%%%%%%%%%%%%%%%%%%%%% User specified LaTeX commands.
\usepackage{braket}
\usepackage{tikz}
%\usepackage{braket}
%\usepackage{braket}
\usepackage{listings}
\usepackage{xcolor}
\usepackage{color}
\usepackage{hyperref}
\usepackage{diagbox}
\usepackage{extsizes}
\makeatother

\usepackage{babel}
\begin{document}
\begin{tabular}{|c|c|c|c|}
\hline 
 & Dual Unitary & $\overline{\mathfrak{L}}_{2}$ & $\overline{\mathfrak{L}}_{k_{\text{L/R}}}(k_{\text{L/R}}\geq2)$ on
left (right) side\tabularnewline
\hline 
\hline 
Non Vanishing at & Light Rays & Light Rays$+$Time Axes & Light Rays$+v_{\mathrm{L/R}}=\frac{k_{\text{L/R}}-2}{k_{\text{L/R}}}$\tabularnewline
\hline 
\end{tabular}
\end{document}
