\documentclass[aps,onecolumn,nofootinbib,superscriptaddress,notitlepage,longbibliography]{revtex4-1}
\usepackage{times}
\usepackage{tikz-feynman}
\usepackage{graphicx}
\usepackage{feynmf}
\usepackage{tabularx}
\usepackage{amsmath}
\usepackage{amstext}
\usepackage{amssymb}
\usepackage{xfrac}
\usepackage[colorlinks,citecolor=blue]{hyperref}
\usepackage{graphicx}
\usepackage{amsmath}
\usepackage{amstext}
\usepackage{amssymb}
\usepackage{amsfonts}
\usepackage{longtable,booktabs}
\usepackage{hyperref}
\usepackage{url}
\usepackage{subfigure}
\usepackage{dsfont}
\usepackage{booktabs}
\usepackage{amsbsy}
\usepackage{dcolumn}
\usepackage{amsthm}
\usepackage{bm}
\usepackage{esint}
\usepackage{multirow}
\usepackage{hyperref}
\usepackage{cleveref}
\usepackage{mathrsfs}
\usepackage{amsfonts}
\usepackage{amsbsy}
\usepackage{dcolumn}
\usepackage{bm}
\usepackage{multirow}
\usepackage{color}
\usepackage{extarrows}
\usepackage{datetime}
\usepackage{comment}
\usepackage[super]{nth}
\hypersetup{
	colorlinks=magenta,
	linkcolor=blue,
	filecolor=magenta,
	urlcolor=magenta,
}
\def\Z{\mathbb{Z}}
\newcommand{\red}[1]{{\textcolor{red}{#1}}}
\newtheorem{theorem}{Theorem}
\newtheorem{statement}{Statement}
\newcommand{\mb}{\mathbb}
\newcommand{\bs}{\boldsymbol}
\newcommand{\wt}{\widetilde}
\newcommand{\mc}{\mathcal}
\newcommand{\bra}{\langle}
\newcommand{\ket}{\rangle}
\newcommand{\ep}{\epsilon}
\newcommand{\tf}{\textbf}

\begin{document}
	
	
	
\title{Supplementary Material for "Yang-Lee Singularity in BCS Superconductivity"}

	\author{Hongchao Li}
\thanks{These two authors contributed equally to this work.}
\affiliation{Department of Physics, University of Tokyo, 7-3-1 Hongo, Tokyo 113-0033,
	Japan}
\email{lhc@cat.phys.s.u-tokyo.ac.jp}

\author{Xie-Hang Yu}
\thanks{These two authors contributed equally to this work.}
\affiliation{Max-Planck-Institut für Quantenoptik, Hans-Kopfermann-Straße 1, D-85748 Garching, Germany}
\affiliation{Munich Center for Quantum Science and Technology, Schellingstraße
4, 80799 München, Germany}
\email{xiehang.yu@mpq.mpg.de}

\author{Masaya Nakagawa}
\affiliation{Department of Physics, University of Tokyo, 7-3-1 Hongo, Tokyo 113-0033,
	Japan}
\email{nakagawa@cat.phys.s.u-tokyo.ac.jp}

\author{Masahito Ueda}
\affiliation{Department of Physics, University of Tokyo, 7-3-1 Hongo, Tokyo 113-0033,
	Japan}
\affiliation{RIKEN Center for Emergent Matter Science (CEMS), Wako, Saitama 351-0198,
	Japan}
\affiliation{Institute for Physics of Intelligence, University of Tokyo, 7-3-1
	Hongo, Tokyo 113-0033, Japan}
\email{ueda@cat.phys.s.u-tokyo.ac.jp}

	\begin{abstract}
		
	\end{abstract}
	\date{\today}
	\maketitle
	
	\tableofcontents

\section{Yang-Lee Zeros on the Boundary}

The gap equation with constant density of states $\rho_{0}$ and $\beta\rightarrow\infty$
reads: 
\begin{equation}
\frac{\sqrt{\omega_{D}^{2}+\Delta_{0}^{2}}+\omega_{D}}{\Delta_{0}}=e^{\frac{1}{\rho_{0}U}}\,.
\end{equation}
The solution to the gap equation is given by $\Delta_{0}=\frac{\omega_{D}}{\text{sinh}\left(\frac{1}{\rho_{0}U}\right)}$.
To be specific: 
\begin{eqnarray}
\Delta_{0} & = & \frac{2\omega_{D}}{\text{exp}\left[\frac{1}{\rho_{0}|U|^{2}}\left(U_{R}-iU_{I}\right)\right]-\text{exp}\left[-\frac{1}{\rho_{0}|U|^{2}}\left(U_{R}-iU_{I}\right)\right]}\,.\nonumber \\
 & = & \frac{\omega_{D}}{\text{sinh}\left(\frac{U_{R}}{\rho_{0}|U|^{2}}\right)\cos\left(\frac{U_{I}}{\rho_{0}|U|^{2}}\right)-i\text{cosh}\left(\frac{U_{R}}{\rho_{0}|U|^{2}}\right)\sin\left(\frac{U_{I}}{\rho_{0}|U|^{2}}\right)}
\end{eqnarray}
with the real part of it given by 
\begin{equation}
\text{Re}[\Delta_{0}]=\omega_{D}\frac{\text{sinh}\left(\frac{U_{R}}{\rho_{0}|U|^{2}}\right)\cos\left(\frac{U_{I}}{\rho_{0}|U|^{2}}\right)}{\left(\text{sinh}\left(\frac{U_{R}}{\rho_{0}|U|^{2}}\right)\cos\left(\frac{U_{I}}{\rho_{0}|U|^{2}}\right)\right)^{2}+\left(\text{cosh}\left(\frac{U_{R}}{\rho_{0}|U|^{2}}\right)\sin\left(\frac{U_{I}}{\rho_{0}|U|^{2}}\right)\right)^{2}}\,.\label{RDelta_0}
\end{equation}
When phase transition occurs, the real part of the gap vanishes, which
gives the result $\cos\left(\frac{U_{I}}{\rho_{0}|U|^{2}}\right)=0$.
Or equivalently, $\frac{U_{I}}{\rho_{0}|U|^{2}}=\frac{\pi}{2}$. This
determines the expression of phase boundary: 
\begin{equation}
(\rho_{0}\pi U_{R})^{2}+(\rho_{0}\pi U_{I}-1)^{2}=1\,,\label{phase_transition}
\end{equation}
which restricts the imaginary part of the gap $\Delta_{0}$ as: 
\begin{eqnarray}
\text{Im}[\Delta_{0}] & = & \omega_{D}\frac{\text{cosh}\left(\frac{U_{R}}{\rho_{0}|U|^{2}}\right)\sin\left(\frac{U_{I}}{\rho_{0}|U|^{2}}\right)}{\left(\text{sinh}\left(\frac{U_{R}}{\rho_{0}|U|^{2}}\right)\cos\left(\frac{U_{I}}{\rho_{0}|U|^{2}}\right)\right)^{2}+\left(\text{cosh}\left(\frac{U_{R}}{\rho_{0}|U|^{2}}\right)\sin\left(\frac{U_{I}}{\rho_{0}|U|^{2}}\right)\right)^{2}}\nonumber \\
 & = & \frac{\omega_{D}}{\text{cosh}\left(\frac{U_{R}}{\rho_{0}|U|^{2}}\right)}\,.\label{IDelta_0}
\end{eqnarray}

On the boundary, we can also connect the phase transition with the
existence of Yang-Lee zeros. Here we see for each momentum of quasi-particles,
the number of quasi-particle can only be 0 or 1. Hence, the partition
function for a quasi particle with momentum $\bm{k}$ is given by:
\begin{equation}
Z_{\bm{k}}\equiv{}_{L}\langle u_{\bm{k}}|e^{-\beta H}|u_{\bm{k}}\rangle_{R}=1+e^{-\beta E_{\bm{k}}}\,.
\end{equation}
Hence, the partition function of all Bogoliubov quasi-particles is
\begin{equation}
Z=\prod_{\bm{k}}Z_{\bm{k}}=\prod_{\bm{k}}(1+e^{-\beta E_{\bm{k}}})\,.
\end{equation}
For the points not on the phase transition critical line, we have
$\text{Re}[E_{\bm{k}}]>0$, which indicates that there shouldn't be
any zeros of partition function. However, on the critical line, the
system becomes a gapless one and Yang-Lee zeros can take place. To
prove this, let's see the partition function on the boundary: 
\begin{equation}
Z=\prod_{\bm{k}}\left(1+e^{-i\beta\sqrt{\left(\text{Im}\Delta_{0}\right)^{2}-\xi_{\bm{k}}^{2}}}\right)\,.
\end{equation}
If we require $Z=0$, we only need to find a $\bm{k}$ to satisfy
$e^{-i\beta\sqrt{\left(\text{Im}\Delta_{0}\right)^{2}-\xi_{\bm{k}}^{2}}}=-1$,
which is eqivalent to 
\begin{equation}
\beta\sqrt{\left(\text{Im}\Delta_{0}\right)^{2}-\xi_{\bm{k}}^{2}}=(2n+1)\pi\Rightarrow|\xi_{\bm{k}}|=\sqrt{\left(\text{Im}\Delta_{0}\right)^{2}-\left(\frac{2n+1}{\beta}\pi\right)^{2}}\cong\text{Im}\Delta_{0}\,.
\end{equation}
As we know, $\xi_{\bm{k}}\in[-\omega_{D},\omega_{D}]$. From the formula
(\ref{IDelta_0}) we can see $\text{Im}\Delta_{0}<\omega_{D}$ makes
sense for all the points on the phase boundary. Therefore, $Z=0$
is satisfied on the boundary (\ref{phase_transition}) indicating
that Yang-Lee zeros distribute on the phase boundary

\section{Correlation Functions on The Phase Transition Boundary}

In this section, we study the model with correlation function. Here
we firstly consider the equal-time correlation functions in momentum
space. The correlaton function is given by: 
\begin{equation}
\langle c_{\bm{k}\sigma}c_{\bm{k}\sigma}^{\dagger}\rangle={}_{L}\left\langle \text{BCS}\right|c_{\bm{k}\sigma}c_{\bm{k}\sigma}^{\dagger}\left|\text{BCS}\right\rangle _{R}={}_{L}\left\langle \text{BCS}\right|(1-c_{\bm{k}\sigma}^{\dagger}c_{\bm{k}\sigma})\left|\text{BCS}\right\rangle _{R}\,.
\end{equation}
In this way, the correlator takes the form of: 
\begin{equation}
\langle c_{\bm{k}\uparrow}c_{\bm{k}\uparrow}^{\dagger}\rangle=\langle c_{\bm{k}\downarrow}c_{\bm{k}\downarrow}^{\dagger}\rangle=1-v_{k}^{2}=\frac{1}{2}+\frac{\xi_{k}}{2E_{k}}=\frac{1}{2}+\frac{\xi_{k}}{2\sqrt{\xi_{k}^{2}+\Delta_{0}^{2}}}\,.
\end{equation}
Similarly, the correlators between electrons with different spins
vanish: 
\begin{equation}
\langle c_{\bm{k}\uparrow}c_{\bm{k}\downarrow}^{\dagger}\rangle=\langle c_{\bm{k}\downarrow}c_{\bm{k}\uparrow}^{\dagger}\rangle=0\,.
\end{equation}
Then we perform the Fourier transformation to the real space with
the result: 
\begin{equation}
\langle c_{\sigma}(\bm{x})c_{\sigma}^{\dagger}(\bm{x}')\rangle:=C(|\bm{x}-\bm{x}'|)=\int\frac{d^{3}k}{(2\pi)^{3}}\left(\frac{1}{2}+\frac{\xi_{k}}{2\sqrt{\xi_{k}^{2}+\Delta_{0}^{2}}}\right)e^{i\bm{k}\cdot(\bm{x}-\bm{x}')}\,.
\end{equation}
In the following text I will directly replace $\bm{x}-\bm{x}'$ with
$\bm{x}$ for convenience. Then on the boundary, the correlation function
is given by: 
\begin{equation}
C(\bm{x})=\int\frac{d^{3}k}{(2\pi)^{3}}\frac{\xi_{k}}{2\sqrt{\xi_{k}^{2}+\Delta_{0}^{2}}}e^{i\bm{k}\cdot\bm{x}}\,.
\end{equation}
Here we drop the first term since it gives a result propotional to
delta function. We only concern about long-range behaviour of the
correlation function. This will be quite difficult to calculate for
the ambiguity of kinetic energy $\xi_{k}$ in the formula. Here we
perform a simple approximation. Since we consider the kinetic energy
around the Fermi surface, we can expand the energy spectrum as a linear
one around the Fermi surface with Fermi velocity $v_{F}$. To be specific,
we have $\xi_{k}\cong v_{F}(k-k_{F})$. Then the integration can be
simplified as: 
\begin{equation}
\int\frac{\rho_{0}}{2}d\xi_{k}\sin\theta d\theta\frac{\xi_{k}}{2\sqrt{\xi_{k}^{2}+\Delta_{0}^{2}}}e^{i\frac{\xi_{k}}{v_{F}}x\cos\theta}e^{ik_{F}x\cos\theta}\,.
\end{equation}
Here $\rho_{0}$ is still the density of state for simplicity. With
the integration of variable $\theta$, we have: 
\begin{equation}
\frac{1}{2}\int_{-\omega_{D}}^{\omega_{D}}\rho_{0}d\xi_{k}\frac{\xi_{k}\sin\left(\left(\frac{\xi_{k}}{v_{F}}+k_{F}\right)x\right)}{\left(\frac{\xi_{k}}{v_{F}}+k_{F}\right)x\sqrt{\xi_{k}^{2}+\Delta_{0}^{2}}}\,.
\end{equation}
But on the boundary, $\Delta_{0}$ is an imaginary number. So we seperate
the integration region into two parts:$\left[-\text{Im}\Delta_{0},\text{Im}\Delta_{0}\right]$
and $\left[-\omega_{D},-\text{Im}\Delta_{0}\right]\cup\left[\text{Im}\Delta_{0},\omega_{D}\right]$.
In the first part, the correlation function is a completely imaginary
number. This will be the same for real-part properties. We here discuss
the second region for reference. 
\begin{eqnarray}
 &  & \frac{1}{2}\left(\int_{\text{Im}\Delta_{0}}^{\omega_{D}}+\int_{-\omega_{D}}^{-\text{Im}\Delta_{0}}\right)\rho_{0}d\xi_{k}\frac{\xi_{k}\sin\left(\left(\frac{\xi_{k}}{v_{F}}+k_{F}\right)x\right)}{\left(\frac{\xi_{k}}{v_{F}}+k_{F}\right)x\sqrt{\xi_{k}^{2}-\left(\text{Im}\Delta_{0}\right)^{2}}}\nonumber \\
 & = & \frac{\rho_{0}}{2}\int_{\text{Im}\Delta_{0}}^{\omega_{D}}d\xi_{k}\left[\frac{\xi_{k}\sin\left(\left(\frac{\xi_{k}}{v_{F}}+k_{F}\right)x\right)}{\left(\frac{\xi_{k}}{v_{F}}+k_{F}\right)x\sqrt{\xi_{k}^{2}-\left(\text{Im}\Delta_{0}\right)^{2}}}-\frac{\xi_{k}\sin\left(\left(-\frac{\xi_{k}}{v_{F}}+k_{F}\right)x\right)}{\left(-\frac{\xi_{k}}{v_{F}}+k_{F}\right)x\sqrt{\xi_{k}^{2}-\left(\text{Im}\Delta_{0}\right)^{2}}}\right]\nonumber \\
 & = & \frac{\rho_{0}}{2}\int_{\text{Im}\Delta_{0}}^{\omega_{D}}d\xi_{k}\frac{\xi_{k}}{x\sqrt{\xi_{k}^{2}-\left(\text{Im}\Delta_{0}\right)^{2}}}\frac{\left(-\frac{\xi_{k}}{v_{F}}+k_{F}\right)\sin\left(\left(\frac{\xi_{k}}{v_{F}}+k_{F}\right)x\right)-\left(\frac{\xi_{k}}{v_{F}}+k_{F}\right)\sin\left(\left(-\frac{\xi_{k}}{v_{F}}+k_{F}\right)x\right)}{k_{F}^{2}-\left(\frac{\xi_{k}}{v_{F}}\right)^{2}}\nonumber \\
 & = & \rho_{0}v_{F}\int_{\text{Im}\Delta_{0}}^{\omega_{D}}d\xi_{k}\frac{\xi_{k}}{x\sqrt{\xi_{k}^{2}-\left(\text{Im}\Delta_{0}\right)^{2}}}\left[(-\xi_{k})\frac{\sin(k_{F}x)\cos\left(\frac{\xi_{k}}{v_{F}}x\right)}{(v_{F}k_{F})^{2}-\xi_{k}^{2}}+v_{F}k_{F}\frac{\cos(k_{F}x)\sin\left(\frac{\xi_{k}}{v_{F}}x\right)}{(v_{F}k_{F})^{2}-\xi_{k}^{2}}\right]\,,
\end{eqnarray}
where we assume that $v_{F}k_{F}\gg\omega_{D}$ so we ignore the first
term in the bracket. However, the second term is still difficult to
calculate. We make an additional assumption that $\omega_{D}\rightarrow\infty$.
Since $\text{Im}\Delta_{0}\sim\omega_{D}$, all $\xi_{k}$ in the
factor $\frac{\xi_{k}}{(v_{F}k_{F})^{2}-\xi_{k}^{2}}$ approximately
takes the value: $\omega_{D}$ and the integration is approximately
given by: 
\begin{equation}
\rho_{0}\frac{\omega_{D}v_{F}k_{F}}{(v_{F}k_{F})^{2}-\omega_{D}^{2}}\cos(k_{F}x)\int_{\text{Im}\Delta_{0}}^{\omega_{D}}d\xi_{k}\frac{\sin\left(\xi_{k}\frac{x}{v_{F}}\right)}{(x/v_{F})\sqrt{\xi_{k}^{2}-\left(\text{Im}\Delta_{0}\right)^{2}}}\,.\label{Reintegration}
\end{equation}
Now we focus on the integration: $\lim_{\omega_{D}\rightarrow\infty}\int_{\text{Im}\Delta_{0}}^{\omega_{D}}d\xi_{k}\frac{\sin\left(\xi_{k}\frac{x}{v_{F}}\right)}{(x/v_{F})\sqrt{\xi_{k}^{2}-\left(\text{Im}\Delta_{0}\right)^{2}}}$,
which gives an exact result: 
\begin{equation}
\int_{\text{Im}\Delta_{0}}^{\infty}d\xi_{k}\frac{\sin\left(\xi_{k}\frac{x}{v_{F}}\right)}{(x/v_{F})\sqrt{\xi_{k}^{2}-\left(\text{Im}\Delta_{0}\right)^{2}}}=\frac{\pi}{2}\frac{J_{0}\left(\text{Im}\Delta_{0}\frac{x}{v_{F}}\right)}{(x/v_{F})}\,.\label{Reapprox}
\end{equation}
Here $J_{0}(x)$ is the zero-th order Bessel function. Therefore,
on the boundary, the correlation function decreases as $\frac{J_{0}(ax)}{x}$.
When the variable $x$ is taken to infinity, the integral (\ref{Reapprox})
can be rewritten into: 
\begin{equation}
\frac{J_{0}(ax)}{x}\sim\sqrt{\frac{2}{\pi ax}}\frac{\cos\left(ax-\frac{\pi}{4}\right)}{x}\sim\frac{1}{x^{3/2}}\,.
\end{equation}
As a conclusion, we find that on the boundary, the anomalous dimension
$\eta=\frac{1}{2}$ which is induced by exceptional points. Here anomalous
dimension is defined by $C(\bm{x})\propto x^{-D+2-\eta}$. From the
expression, we can see the correlatioin length diverges on the phase
boundary.

Besides, we can still have the imaginary part with similar calculation.
\begin{equation}
\int_{0}^{\text{Im}\Delta_{0}}d\xi_{k}\frac{\sin\left(\xi_{k}\frac{x}{v_{F}}\right)}{(x/v_{F})\sqrt{\left(\text{Im}\Delta_{0}\right)^{2}-\xi_{k}^{2}}}=\frac{\pi}{2}\frac{H_{0}\left(\text{Im}\Delta_{0}\frac{x}{v_{F}}\right)}{(x/v_{F})}\,.
\end{equation}
Here $H_{0}(x)$ is the zero-th order Struve function. When $x\rightarrow\infty$,
we have: 
\begin{equation}
\int_{0}^{\text{Im}\Delta_{0}}d\xi_{k}\frac{\sin\left(\xi_{k}\frac{x}{v_{F}}\right)}{(x/v_{F})\sqrt{\left(\text{Im}\Delta_{0}\right)^{2}-\xi_{k}^{2}}}\sim\frac{\sin(\frac{\text{Im}\Delta_{0}x}{v_{F}})}{x^{3/2}}\,.\label{Imapprox}
\end{equation}
We can reconsider the formula (\ref{Reintegration}) again without
further approximation. 
\begin{equation}
\rho_{0}\frac{v_{F}k_{F}}{(v_{F}k_{F})^{2}-\omega_{D}^{2}}\frac{\cos(k_{F}x)}{x/v_{F}}\int_{\text{Im}\Delta_{0}}^{\omega_{D}}d\xi_{k}\frac{\xi_{k}}{\sqrt{\xi_{k}^{2}-(\text{Im}\Delta_{0})^{2}}}\sin\left(\frac{\xi_{k}}{v_{F}}x\right)\,,
\end{equation}
where we define: 
\begin{equation}
F_{1}(x)\equiv\int_{\text{Im}\Delta_{0}}^{\omega_{D}}d\xi_{k}\frac{\xi_{k}}{\sqrt{\xi_{k}^{2}-(\text{Im}\Delta_{0})^{2}}}\sin\left(\frac{\xi_{k}}{v_{F}}x\right),G_{1}(x)\equiv\int_{\text{Im}\Delta_{0}}^{\omega_{D}}d\xi_{k}\frac{1}{\sqrt{\xi_{k}^{2}-(\text{Im}\Delta_{0})^{2}}}\cos\left(\frac{\xi_{k}}{v_{F}}x\right)\,.
\end{equation}
The relationship between these two functions can be shown as: 
\begin{equation}
F_{1}(x)=-v_{F}G_{1}'(x)\,.
\end{equation}
In addition, under the limit of $\omega_{D}\rightarrow\infty$, we
have: 
\begin{equation}
\int_{\text{Im}\Delta_{0}}^{\omega_{D}}d\xi_{k}\frac{1}{\sqrt{\xi_{k}^{2}-(\text{Im}\Delta_{0})^{2}}}\cos\left(\frac{\xi_{k}}{v_{F}}x\right)=-\frac{\pi}{2}N_{0}\left(\frac{\text{Im}\Delta_{0}}{v_{F}}x\right)\,.
\end{equation}
Therefore, the original integral is given by: 
\begin{equation}
\frac{\pi}{2}\rho_{0}\frac{\text{Im}\Delta_{0}v_{F}k_{F}}{(v_{F}k_{F})^{2}-\omega_{D}^{2}}\frac{\cos(k_{F}x)}{x/v_{F}}N_{0}'\left(\frac{\text{Im}\Delta_{0}}{v_{F}}x\right)\,.
\end{equation}
When we take the limit $x\rightarrow\infty$ again, we have: 
\begin{equation}
\lim_{x\rightarrow\infty}N_{0}'(x)=\sqrt{\frac{2}{\pi}}\cos\left(x-\frac{\pi}{4}\right)\frac{1}{x^{1/2}}-\sqrt{\frac{2}{\pi}}\frac{1}{2}\sin\left(x-\frac{\pi}{4}\right)\frac{1}{x^{3/2}}\cong\sqrt{\frac{2}{\pi}}\cos\left(x-\frac{\pi}{4}\right)\frac{1}{x^{1/2}}\,.\label{Realexact}
\end{equation}
We can see its long-range behaviour is exactly the same with the expression
(\ref{Reapprox}) on which we made approximations. Similar procedure
can be applied to the imaginary part. The exact form of the expression
(\ref{Imapprox}) is: 
\begin{equation}
\rho_{0}\frac{v_{F}k_{F}}{(v_{F}k_{F})^{2}-\omega_{D}^{2}}\frac{\cos(k_{F}x)}{x/v_{F}}\int_{0}^{\text{Im}\Delta_{0}}d\xi_{k}\frac{\xi_{k}}{\sqrt{(\text{Im}\Delta_{0})^{2}-\xi_{k}^{2}}}\sin\left(\frac{\xi_{k}}{v_{F}}x\right)\,.
\end{equation}
We can also define: 
\begin{equation}
F_{2}(x)\equiv\int_{0}^{\text{Im}\Delta_{0}}d\xi_{k}\frac{\xi_{k}}{\sqrt{(\text{Im}\Delta_{0})^{2}-\xi_{k}^{2}}}\sin\left(\frac{\xi_{k}}{v_{F}}x\right),G_{2}(x)\equiv\int_{0}^{\text{Im}\Delta_{0}}d\xi_{k}\frac{1}{\sqrt{(\text{Im}\Delta_{0})^{2}-\xi_{k}^{2}}}\cos\left(\frac{\xi_{k}}{v_{F}}x\right)\,.
\end{equation}
The relationship between these two functions is also given by: 
\begin{equation}
F_{2}(x)=-v_{F}G_{2}'(x),G_{2}(x)=\frac{\pi}{2}J_{0}\left(\frac{\text{Im}\Delta_{0}}{v_{F}}x\right)\,.
\end{equation}
Then the imaginary part is also equivalent to: 
\begin{equation}
-\frac{\pi}{2}\rho_{0}\frac{\text{Im}\Delta_{0}v_{F}k_{F}}{(v_{F}k_{F})^{2}-\omega_{D}^{2}}\frac{\cos(k_{F}x)}{x/v_{F}}J_{0}'\left(\frac{\text{Im}\Delta_{0}}{v_{F}}x\right)\,.
\end{equation}
So we have the similar long-range behaviour for imaginary part: 
\begin{equation}
\lim_{x\rightarrow\infty}\left[-\frac{\pi}{2}\rho_{0}\frac{\text{Im}\Delta_{0}v_{F}k_{F}}{(v_{F}k_{F})^{2}-\omega_{D}^{2}}\frac{\cos(k_{F}x)}{x/v_{F}}J_{0}'\left(\frac{\text{Im}\Delta_{0}}{v_{F}}x\right)\right]\sim\frac{\sin\left(\frac{\text{Im}\Delta_{0}}{v_{F}}x-\frac{\pi}{4}\right)}{x^{3/2}}\,,\label{Imexact}
\end{equation}
which is exactly the same with the \ref{Imapprox}). To be summarized,
the correlation function takes the form of: 
\begin{align}
\lim_{\bm{x}\rightarrow\infty}{}_{L}\left\langle \text{BCS}\right|C_{\sigma}(\bm{x})C_{\sigma}^{\dagger}(\bm{x})\left|\text{BCS}\right\rangle _{R} & =-\lim_{\bm{x}\rightarrow\infty}\frac{\pi}{2}\rho_{0}\frac{\text{Im}\Delta_{0}v_{F}k_{F}}{(v_{F}k_{F})^{2}-\omega_{D}^{2}}\frac{\cos(k_{F}x)}{x/v_{F}}(N_{0}'(\frac{\text{Im}\Delta_{0}}{v_{F}}x)+iJ_{0}'(\frac{\text{Im}\Delta_{0}}{v_{F}}x))\nonumber \\
 & :=(A(l)+iB(l))x^{-3/2}\,,
\end{align}
where: 
\begin{align}
A(l) & =-\sqrt{\frac{\pi}{2}}\rho_{0}\frac{\text{Im}\Delta_{0}v_{F}k_{F}}{(v_{F}k_{F})^{2}-\omega_{D}^{2}}\frac{\cos(k_{F}x)}{x/v_{F}}\cos(\frac{\text{Im}\Delta_{0}}{v_{F}}x-\frac{\pi}{4})\,,\nonumber \\
B(l) & =-\sqrt{\frac{\pi}{2}}\rho_{0}\frac{\text{Im}\Delta_{0}v_{F}k_{F}}{(v_{F}k_{F})^{2}-\omega_{D}^{2}}\frac{\cos(k_{F}x)}{x/v_{F}}\sin(\frac{\text{Im}\Delta_{0}}{v_{F}}x-\frac{\pi}{4})\,.
\end{align}
They are obtained by taking the limitations (\ref{Realexact}) and
(\ref{Imexact}). In the following section, for simplicity, we directly
take this approximation to calculate the integral since derivative
of $x$ will also not influence the exponential decay of the correlation
function. In the following section we can see the numerical results
of correlation functions coincides with the oscillation of (\ref{Realexact})
and (\ref{Imexact}).

Then we move to the region near the phase boundary.

\section{Correlation Function Near The Phase Transition}

In the superfluid phase, we consider the points deviating a little
from the phase boundary: $U_{R}\rightarrow U_{R}+\delta U_{R}$, where
$\delta U_{R}\rightarrow0$. The real part of gap $\Delta_{0}$ is
given by: 
\begin{equation}
\text{Re}[\Delta_{0}]=\omega_{D}\frac{\text{sinh}\left(\frac{U_{R}}{\rho_{0}|U|^{2}}\right)\cos\left(\frac{U_{I}}{\rho_{0}|U|^{2}}\right)}{\left(\text{sinh}\left(\frac{U_{R}}{\rho_{0}|U|^{2}}\right)\cos\left(\frac{U_{I}}{\rho_{0}|U|^{2}}\right)\right)^{2}+\left(\text{cosh}\left(\frac{U_{R}}{\rho_{0}|U|^{2}}\right)\sin\left(\frac{U_{I}}{\rho_{0}|U|^{2}}\right)\right)^{2}}\,.
\end{equation}
For convenience we here choose the $\delta U=\delta U_{R}$ along
the real axis. It can be replaced with arbitrary $\delta U$ along
any direction. At these points, we have: $\cos\left(\frac{U_{I}}{\rho_{0}|U|^{2}}\right)=\cos\left(\frac{\pi}{2}-\frac{\pi U_{R}\delta U_{R}}{|U|^{2}}\right)=\sin\left(\frac{\pi U_{R}\delta U_{R}}{|U|^{2}}\right)\cong\frac{\pi U_{R}\delta U_{R}}{|U|^{2}}$.
Then the real part can be considered as: 
\begin{equation}
\text{Re}\Delta_{0}\cong\frac{\pi\omega_{D}U_{R}}{|U|^{2}}\frac{\text{sinh}\left(\frac{U_{R}}{\rho_{0}|U|^{2}}\right)}{\text{cosh}^{2}\left(\frac{U_{R}}{\rho_{0}|U|^{2}}\right)}\delta U_{R}\propto\delta U_{R}\,.
\end{equation}
The imaginary part remains the same as (\ref{IDelta_0}): $\text{Im}\Delta_{0}=\frac{\omega_{D}}{\text{cosh(\ensuremath{\frac{U_{R}}{\rho_{0}|U|^{2}}})}}$.
Then we turn to the correlation function near the boundary: 
\begin{eqnarray}
C(\bm{x}) & = & \int\frac{d^{3}k}{(2\pi)^{3}}\frac{\xi_{k}}{2\sqrt{\xi_{k}^{2}+\Delta_{0}^{2}}}e^{i\bm{k}\cdot\bm{x}}\nonumber \\
 & = & \int\frac{d^{3}k}{(2\pi)^{3}}\frac{\xi_{k}}{2\sqrt{\xi_{k}^{2}+\left(\text{Re}\Delta_{0}\right)^{2}-\left(\text{Im}\Delta_{0}\right)^{2}+2i\text{Re}\Delta_{0}\text{Im}\Delta_{0}}}e^{i\bm{k}\cdot\bm{x}}\,.
\end{eqnarray}
With the similar analysis, we still have: 
\begin{align}
C(\bm{x}) & =\int\frac{d^{3}k}{(2\pi)^{3}}\frac{\xi_{k}}{2\sqrt{\xi_{k}^{2}+\Delta_{0}^{2}}}e^{i\bm{k}\cdot\bm{x}}\nonumber \\
 & =\rho_{0}v_{F}\int_{0}^{\omega_{D}}d\xi_{k}\frac{\xi_{k}}{x\sqrt{\xi_{k}^{2}+\Delta_{0}^{2}}}\left[(-\xi_{k})\frac{\sin(k_{F}x)\cos\left(\frac{\xi_{k}}{v_{F}}x\right)}{(v_{F}k_{F})^{2}-\xi_{k}^{2}}+v_{F}k_{F}\frac{\cos(k_{F}x)\sin\left(\frac{\xi_{k}}{v_{F}}x\right)}{(v_{F}k_{F})^{2}-\xi_{k}^{2}}\right]\nonumber \\
 & \cong\rho_{0}v_{F}\int_{0}^{\omega_{D}}d\xi_{k}\frac{\xi_{k}}{x\sqrt{\xi_{k}^{2}+\Delta_{0}^{2}}}v_{F}k_{F}\frac{\cos(k_{F}x)\sin\left(\frac{\xi_{k}}{v_{F}}x\right)}{(v_{F}k_{F})^{2}-\xi_{k}^{2}}\nonumber \\
 & =\rho_{0}\frac{\omega_{D}v_{F}k_{F}}{(v_{F}k_{F})^{2}-\omega_{D}^{2}}\frac{\cos(k_{F}x)}{x/v_{F}}\int_{0}^{\omega_{D}}d\xi_{k}\frac{\sin\left(\frac{\xi_{k}}{v_{F}}x\right)}{\sqrt{\xi_{k}^{2}+(\text{Re}\Delta_{0})^{2}-(\text{Im}\Delta_{0})^{2}+2i\text{Re}\Delta_{0}\text{Im}\Delta_{0}}}\,.\label{correlation}
\end{align}
By defining: 
\begin{equation}
k=\frac{\omega_{D}}{\text{Im}\Delta_{0}^{2}}=\frac{1}{\text{cosh}(\frac{U_{R}}{\rho_{0}|U|^{2}})},\frac{\text{Re}\Delta_{0}}{\text{Im}\Delta_{0}}=\frac{\pi U_{R}}{|U|^{2}}\text{tanh}(\frac{U_{R}}{\rho_{0}|U|^{2}})\delta U_{R}\equiv a\,,
\end{equation}
we rewrite the integral as: 
\begin{equation}
C(\bm{x})=\rho_{0}\frac{\omega_{D}v_{F}k_{F}}{(v_{F}k_{F})^{2}-\omega_{D}^{2}}\frac{\cos(k_{F}x)}{x/v_{F}}\int_{0}^{k}dt\frac{\sin(\frac{\text{Im}\Delta_{0}x}{v_{F}}t)}{\sqrt{t^{2}+a^{2}-1+2ia}}\,,\label{integral_expresion}
\end{equation}
where we substitude the variable with $\xi_{k}=\text{Im}\Delta_{0}t$.

Our next focus is to investigate the properties of the integral. Here
we take the coordinate $(\rho_{0}U_{R},\rho_{0}U_{I})=(\frac{1}{\pi},\frac{1}{\pi})$
as an example, here the correlation function behaves like an oscillating
exponential function when we turn to the points near the boundary.
We here give the plot of $\rho_{0}\delta U_{R}=3\times10^{-6}<<1$
in the FIG. \ref{fig1}. The fittings of these points tell us the
correlation function decay exponentially with the distance $x$. The
real part of correlation function is given by the blue curve in (a):
$\text{Re}C(x)\propto\sin(x+\frac{\pi}{4})e^{-x/\xi_{r}}$, where
$\xi_{r}=453.6$. The imaginary part gives a similar resul as the
blue curve in (b): $\text{Im}C(x)\propto\sin(x+\frac{3\pi}{4})e^{-x/\xi_{i}}$
with $\xi_{i}=451.6$. We can see the behaviours of the real and the
imaginary parts of the correlation function are very close to each
other. The only difference is in the coefficients.

Likewise, when we consider the problem on the boundary, the decay
will be much slower than the results above. We present the correlation
function on the boundary in the FIG. \ref{fig2}. The fitting results
of these points indicate that the real part of the correlation function
is propotional to $\sin(x+\frac{\pi}{4})x^{-0.5}$ and the imaginary
part of it is porpotional to $\sin(x+\frac{3\pi}{4})x^{-0.5}$. Those
fittings tell us the conclusion in the last section is correct and
the correlation length will diverge on the boundary.

Further, we sum the correlation lengths with different deviations
$\rho_{0}\delta2U_{I}$ from the boundary. The real part and imaginary
part are given seperately in the FIG. \ref{fig3}. We can see the
two diagrams are similar. Note that we will give the exact solution
to this integral and behaviour of correlation function in the next
section.

\begin{figure}[t]
\centering \includegraphics[width=0.8\columnwidth,bb = 0 0 200 100, draft, type=eps]{Re2467-3}\phantom{AAA}\\
 \includegraphics[width=0.8\columnwidth,bb = 0 0 200 100, draft, type=eps]{Im2467-3}\phantom{AAA}\\
 \caption{Real part(upper one) and imaginary part(down one) of the correlation
function near the boundary with $\rho_{0}\delta U_{R}=3\times10^{-6}<<1$.}
\label{fig1} 
\end{figure}

\begin{figure}
\centering \includegraphics[width=0.8\columnwidth,bb = 0 0 200 100, draft, type=eps]{Re2467+0}\phantom{AAA}\\
 \includegraphics[width=0.8\columnwidth,bb = 0 0 200 100, draft, type=eps]{Im_2467+0}\phantom{AAA}\\
 \caption{Real part(upper one) and imaginary part(down one) of the correlation
function on the boundary.}
\label{fig2} 
\end{figure}

\begin{figure}
\centering \includegraphics[width=0.8\columnwidth,bb = 0 0 200 100, draft, type=eps]{fitting_correlation_length_real.png}\\
 \includegraphics[width=0.8\columnwidth,bb = 0 0 200 100, draft, type=eps]{fitting_correlation_length_IM.png}\\
 \caption{Relationship between correlation length of real part(upper one) and
imaginary part(down one) of correlation function and $\rho_{0}\delta U_{R}<<1$.}
\label{fig3} 
\end{figure}

Above we give an analysis on the numerical simulation result of correlation
function near the boundary. Here we show the analytic derivation of
correlation and prove that $\xi^{-1}\propto a$. This can be obtained
from generalizing the integration result in $\int_{0}^{\infty}\frac{\sin(tx)}{\sqrt{t^{2}+m^{2}}}$
from $m\in\mathbb{R}$ to $m\in\mathbb{C}$. It can be estimated that
the modification of upper boundary will not change its exponential
decay when $x\to\infty$. According to previous research: 
\begin{equation}
\int_{0}^{\infty}\frac{\sin(tx)}{\sqrt{t^{2}+m^{2}}}\propto\frac{1}{2}\pi[I_{0}(mx)-L_{0}(mx)]\,,
\end{equation}
where $I_{\nu}$ is the $\nu-$order first-class corrected Bessel
function and $L_{\nu}$ is the $\nu-$order first-class corrected
Struve function. As $x\to\infty$, by substituding the expression
for these two special functions, we have 
\begin{equation}
\int_{0}^{\infty}dt\frac{\sin(xt)}{\sqrt{t^{2}+m^{2}}}\propto\frac{e^{-mx}}{\sqrt{mx}}=\frac{\text{exp}[-\text{Re}(m)x-i\text{Im}(m)x]}{\sqrt{mx}}\,.
\end{equation}
From the definition of the orrelation length: $C(\bm{x})\propto e^{-x/\xi}$,
we obtain the form of it as 
\begin{equation}
\xi^{-1}\sim\text{Re}(m)\,.
\end{equation}
In the (\ref{integral_expresion}), the parameter $m=a+i$. Hence,
the relationship between $\xi$ and $a$ is given by: 
\begin{equation}
\xi\propto a^{-1}\,,
\end{equation}
which indicates the critical exponent $\nu=1$.

Then we move to prove that $a$ is an infinitesimal number when we
deviate a little from the boundary. From the real part and imaginary
part of gap, we find the expression of $a$: 
\begin{equation}
a=\frac{\text{Re}\Delta_{0}}{\text{Im}\Delta_{0}}=\frac{\tanh(\frac{U_{R}}{\rho_{0}|U|^{2}})}{\tan(\frac{U_{I}}{\rho_{0}|U|^{2}})}\,.
\end{equation}
When we consider the points near the boundary, we have: 
\begin{equation}
\frac{U_{I}}{\rho_{0}|U|^{2}}\to\frac{U_{I}}{\rho_{0}|U|^{2}}-\frac{2U_{I}U_{R}\delta U_{R}}{\rho_{0}|U|^{4}}=\frac{\pi}{2}--\frac{2U_{I}U_{R}\delta U_{R}}{\rho_{0}|U|^{4}}\,,
\end{equation}
where $\delta U_{R}\to0$. In this case, the parameter $a$ can be
shown as: 
\begin{equation}
a=\frac{\tanh(\frac{U_{R}}{\rho_{0}|U|^{2}})}{\tan(\frac{\pi}{2}-\frac{2U_{I}U_{R}\delta U_{R}}{\rho_{0}|U|^{4}})}\approx\tanh(\frac{U_{R}}{\rho_{0}|U|^{2}})\frac{2U_{I}U_{R}\delta U_{R}}{\rho_{0}|U|^{4}}\propto\delta U_{R}\,.
\end{equation}
Hence the correlation length will be $\xi^{-1}\propto\delta U_{R}$.
When $\delta U_{R}$ goes to 0, the correlation length will diverge
on the boundary, which coincides with our previous conclusion. However,
on the real axis this analysis fail because $\text{Im}\Delta_{0}=0$
on the whole real axis. Therefore, we can only access the correlation
length from the definition of correlation function again. From (\ref{correlation}),
we know that on the real axis where $U_{I}=0$, the correlation function
is equivalent to: 
\begin{equation}
C(\bm{x})\propto\int_{0}^{\infty}\frac{\sin(xt)}{\sqrt{t^{2}+s^{2}}}\,,
\end{equation}
where we redefine $t=\frac{\xi_{\bm{k}}}{v_{F}}$ and $s=\text{Re}\Delta_{0}$.
On the real axis, the real part of the gap is given by: 
\begin{equation}
s=\text{Re}\Delta_{0}=\frac{1}{\sinh(\frac{1}{\rho_{0}U_{R}})}\,.
\end{equation}
When $U_{R}\to0$, we have: $s=\text{exp}(-\frac{1}{\rho_{0}U_{R}})\to0$
with the correlation length here shown by: 
\begin{equation}
\xi^{-1}\propto\text{exp}(-\frac{1}{\rho_{0}U_{R}})\,.
\end{equation}
We can see here correlation function can not be represented by the
polynomial form of $\delta U_{R}$, which indicates that real axis
is really different from the whole upper plane with $U_{I}\neq0$.
This can be also figured out from the RG diagram.

\section{Thermodynamic quantities on The Boundary}

In this part, we introduce the free energy and compressibility on
the boundary. The free energy of the dissipative Hubbard model is
given by: 
\begin{equation}
E=-\frac{N}{U_{R}+iU_{I}}\left(\text{Im}\Delta_{0}\right)^{2}-N\int_{-\omega_{D}}^{\omega_{D}}d\xi_{k}\rho_{0}\left(\sqrt{\xi_{k}^{2}+\Delta_{0}^{2}}-|\xi_{k}|\right)\,.
\end{equation}
Note that here $E$ is free energy between the superfluid phase and
the free-electronic phase. The integration is seperated into two parts.
The second term is easy to figure out. 
\begin{equation}
2\rho_{0}\int_{0}^{\omega_{D}}d\xi_{k}\xi_{k}=\rho_{0}\omega_{D}^{2}\,.
\end{equation}
The first term is given on the boundary. 
\[
-2\int_{0}^{\omega_{D}}d\xi_{k}\rho_{0}\sqrt{\xi_{k}^{2}+\Delta_{0}^{2}}=-2\int_{0}^{\omega_{D}}d\xi_{k}\rho_{0}\sqrt{\xi_{k}^{2}-\left(\text{Im}\Delta_{0}\right)^{2}}
\]
\,. Since we care about the real part of the energy, we have: 
\begin{eqnarray}
\text{Re}\left[-2\int_{0}^{\omega_{D}}d\xi_{k}\rho_{0}\sqrt{\xi_{k}^{2}-\left(\text{Im}\Delta_{0}\right)^{2}}\right] & = & -2\int_{\text{Im}\Delta_{0}}^{\omega_{D}}d\xi_{k}\rho_{0}\sqrt{\xi_{k}^{2}-\left(\text{Im}\Delta_{0}\right)^{2}}\nonumber \\
 & = & (-2\rho_{0})\left(\text{Im}\Delta_{0}\right)^{2}\left[\frac{1}{4}\text{sinh}\left(\frac{2U_{R}}{\rho_{0}|U|^{2}}\right)-\frac{1}{2}\frac{U_{R}}{\rho_{0}|U|^{2}}\right]\,.
\end{eqnarray}
Then the real part of free energy is: 
\begin{eqnarray}
\text{Re}[E] & = & -N\frac{U_{R}}{|U|^{2}}\left(\text{Im}\Delta_{0}\right)^{2}-2N\rho_{0}\left(\text{Im}\Delta_{0}\right)^{2}\left[\frac{1}{4}\text{sinh}\left(\frac{2U_{R}}{\rho_{0}|U|^{2}}\right)-\frac{1}{2}\frac{U_{R}}{\rho_{0}|U|^{2}}\right]+N\rho_{0}\omega_{D}^{2}\nonumber \\
 & = & N\rho_{0}\left[\omega_{D}^{2}-\frac{1}{2}\left(\text{Im}\Delta_{0}\right)^{2}\text{sinh}\left(\frac{2U_{R}}{\rho_{0}|U|^{2}}\right)\right]\nonumber \\
 & = & N\rho_{0}\omega_{D}^{2}\left[1-\frac{1}{2}\frac{\text{sinh}\left(\frac{2U_{R}}{\rho_{0}|U|^{2}}\right)}{\text{cosh}^{2}\left(\frac{U_{R}}{\rho_{0}|U|^{2}}\right)}\right]\,.\label{free}
\end{eqnarray}
Hence, we have the form of real part of free energy on the boundary.
Note that this is not the difference of free energy between the normal
phase and the superfluid phase since the mean-field approximation
breaks down in the normal phase. We cannot figure out the analytic
form of free energy and partition function in the normal phase. In
the following we prove that the normal phase behaves differently from
the free-electronic phase. In the following part and main text, we
will use $\Delta F$ to express $\text{Re}[E]$.

In a similar way, we calculate the imaginary part of the free energy.
\begin{eqnarray}
\text{Im}[E] & = & -\frac{N}{|U|^{2}}\left(-U_{I}\right)\left(\text{Im}\Delta_{0}\right)^{2}-2N\rho_{0}\int_{0}^{\text{Im}\Delta_{0}}d\xi_{k}\sqrt{\xi_{k}^{2}-\text{Im}\Delta_{0}}\nonumber \\
 & = & \frac{2U_{I}N}{2|U|^{2}}\left(\text{Im}\Delta_{0}\right)^{2}-2N\rho_{0}\left(\text{Im}\Delta_{0}\right)^{2}\times\frac{\pi}{4}\nonumber \\
 & = & \frac{N}{2}\rho_{0}\left(\text{Im}\Delta_{0}\right)^{2}\left[\frac{2U_{I}}{\rho_{0}|U|^{2}}-\pi\right]\nonumber \\
 & = & 0\,.
\end{eqnarray}
We can see on the boundary the condensate does not have imaginary
part. Furthermore, we consider how to express the order of Yang-Lee
zeros with this free energy. According to the definition of Yang-Lee
zeros in the main context, we have: 
\begin{equation}
\kappa=\frac{\beta\omega_{D}}{2\pi\text{cosh}(\frac{U_{R}}{\rho_{0}|U|^{2}})}\,.
\end{equation}
With (\ref{free}), we find that: 
\begin{equation}
\Delta F=4N\rho_{0}\omega_{D}^{2}(1-\sqrt{1-4(\frac{2\pi\kappa}{\beta\omega_{D}})^{2}})\,.
\end{equation}

Near the origin, we can see $\frac{U_{R}}{\rho_{0}|U|^{2}}\rightarrow\infty$.
The expressions for free energy and density of Yang-Lee zeros can
be simplified as: 
\begin{equation}
\Delta F=N\rho_{0}\omega_{D}^{2}e^{-2\frac{U_{R}}{\rho_{0}|U|^{2}}},\kappa=\frac{\beta\omega_{D}}{2\pi}e^{-\frac{U_{R}}{\rho_{0}|U|^{2}}}\,.
\end{equation}
Their relation is also simplified as: 
\begin{equation}
\Delta F=\frac{4\pi^{2}N\rho_{0}}{\beta^{2}}\kappa^{2}\propto\kappa^{2}\,,
\end{equation}
which means we can read the free energy change on the axis from the
order of Yang-Lee zeros on the upper complex plane. There is a power
relationship between them, which can be used to define the critical
exponent $\phi$ in the main context.

Then we move to consider the compressibility. Here we focus on the
compressibility defined by $\chi=\frac{\partial^{2}E}{\partial\mu^{2}}$.
This will be given by: 
\begin{equation}
\chi=-\sum_{\bm{k}}\frac{\Delta_{0}^{2}}{(\xi_{\bm{k}}^{2}+\Delta_{0}^{2})^{3/2}}=-\int_{-\omega_{D}}^{\omega_{D}}\rho_{0}d\xi_{\bm{k}}\frac{\Delta_{0}^{2}}{(\xi_{\bm{k}}^{2}+\Delta_{0}^{2})^{3/2}}\,.\label{compressibility}
\end{equation}
Firstly, we focus on the points near the boundary. By substituding
the gap (\ref{RDelta_0},\ref{IDelta_0}) into the compressibility
(\ref{compressibility}), we can rewrite the $\chi$ as: 
\begin{equation}
\chi=\int_{-\omega_{D}}^{\omega_{D}}\rho_{0}d\xi_{\bm{k}}\frac{(\text{Im}\Delta_{0})^{2}}{(\xi_{\bm{k}}^{2}-(\text{Im}\Delta_{0})^{2})^{3/2}}\rightarrow\infty\,.
\end{equation}
It diverges since $\text{Im}\Delta_{0}<\omega_{D}$ for all the points
on the boundary. Hence, we find that compressibility has singularity
on the boundary for each point. This cannot be discovered in Hermitian
physics since for Eq. (\ref{compressibility}), a real gap $\Delta_{0}$
can always lead to a finite result instead of divergence. Near the
boundary with a deviation $\delta U_{1}\rightarrow0$, we find the
gap has another form: 
\begin{equation}
\text{Re}\Delta_{0}\approx\frac{\pi U_{1}\omega_{D}}{|U|^{2}}\frac{\sinh\left(\frac{U_{1}}{\rho_{0}|U|^{2}}\right)}{\cosh^{2}\left(\frac{U_{1}}{\rho_{0}|U|^{2}}\right)}\delta U_{1}\rightarrow0,\text{Im}\Delta_{0}\approx\frac{\omega_{D}}{\cosh\left(\frac{U_{1}}{\rho_{0}|U|^{2}}\right)}\,.
\end{equation}
Therefore, we can rewrite the compressibility as: 
\begin{equation}
\chi\approx\int_{-\omega_{D}}^{\omega_{D}}\rho_{0}d\xi_{\bm{k}}\frac{(\text{Im}\Delta_{0})^{2}}{(\xi_{\bm{k}}^{2}-(\text{Im}\Delta_{0})^{2}+2i\text{Re}\Delta_{0}\text{Im}\Delta_{0})^{3/2}}=\int_{-A}^{A}\rho_{0}d\xi_{\bm{k}}\frac{1}{(\xi_{\bm{k}}^{2}-1+2i\text{Re}\Delta_{0}/\text{Im}\Delta_{0})^{3/2}}\,,
\end{equation}
where $A\equiv\cosh\left(\frac{U_{1}}{\rho_{0}|U|^{2}}\right)$. Here
we know the integral must take the form of $f\left(A,\frac{\text{Re}\Delta_{0}}{\text{Im}\Delta_{0}}\right)$.
Obviously, we cannot calculate the analytic expression for this integral.
However, here we prove when $\text{Re}\Delta_{0}\rightarrow0$ the
function should be propotional to $(\text{Re}\Delta_{0}/\text{Im}\Delta_{0})^{-1/2}$
qualitatively. Since only the points near the value $\xi_{k}=1$ contributes
to the integral, we expand the integral around the point with the
width $2a$ where $a=\text{Re}\Delta_{0}/\text{Im}\Delta_{0}$: 
\begin{equation}
\int_{-a}^{a}\rho_{0}d\xi_{\bm{k}}\frac{1}{(2\delta\xi_{\bm{k}}+2ia)^{3/2}}\sim a^{-1/2}\hat{A}\yen,.
\end{equation}
Hence, we here have another critical exponent defined as $\chi\sim(\delta U)^{-\zeta},\zeta=1/2$
for near the boundary $\text{Re}\Delta_{0}\propto\delta U$.

Indeed, there is a relation between the critical exponents $\eta$
and $\zeta$ for any kinds of exceptional points. For the energy spectrums
which can be expanded as $(k-k_{E})^{1/n}$ around the exceptional
point $k=k_{E}$, the correlation function takes the form as: 
\begin{equation}
C(\bm{x})\propto\frac{J_{-1/2+1/n}(x)}{x^{3/2-1/n}}\,.
\end{equation}
Under the limitation $x\rightarrow\infty$, we have $J_{-1/2+1/n}(x)\sim\frac{1}{\sqrt{x}}$.
From the definition of the anomalous dimension, we find that $\eta=1-\frac{1}{n}$.
Similarly, we find the compressibility near the boundary satisfied:
\begin{equation}
\chi\propto\int_{-A}^{A}\rho_{0}d\xi_{\bm{k}}\frac{1}{(\xi_{\bm{k}}^{2}-1+2i\text{Re}\Delta_{0}/\text{Im}\Delta_{0})^{1+1/n}}\propto(\delta U)^{1/n}\,,
\end{equation}
which indicates that $\zeta=1/n$. Therefore, we have: 
\begin{equation}
\eta+\zeta=1\,.
\end{equation}
When $n$ is an integer here, the gapless points correspond to the
$n$-th order exceptional points in the non-Hermitian systems.

\section{Pair Correlation Function}

In the discussion above we only focus on the correlation function
for single electron. In this section we turn to the pair correlation
function frequently used in superconducting system.

The pair correlation function is defined here below: 
\begin{equation}
\rho_{2}(\boldsymbol{r}_{1}\sigma_{1},\boldsymbol{r}_{2}\sigma_{2};\boldsymbol{r}_{1}'\sigma_{1}',\boldsymbol{r}_{2}'\sigma_{2}')\equiv{}_{L}\langle C_{\sigma_{1}}^{\dagger}(\boldsymbol{r}_{1})C_{\sigma_{2}}^{\dagger}(\boldsymbol{r}_{2})C_{\sigma_{2}'}(\boldsymbol{r}_{2}')C_{\sigma_{1}'}(\boldsymbol{r}_{1}')\rangle_{R}\,.
\end{equation}
Here what we concern is the relative position of the two Cooper pairs
instead of relative position of two electrons in the same Cooper pair.
Hence, we set $\boldsymbol{r}_{1}=\boldsymbol{r}_{2}=\boldsymbol{r}$
and $\boldsymbol{r}_{1}'=\boldsymbol{r}_{2}'=0$ here. Without lossing
generality, we assume $\sigma_{1}=\uparrow=\sigma_{1}',\sigma_{2}=\downarrow=\sigma_{2}'$.
With Wick theorem, we can simplify the pair correlation function as:
\begin{eqnarray}
\rho_{2}(\boldsymbol{R}\uparrow,\boldsymbol{R}\downarrow;0\uparrow,0\downarrow) & = & _{L}\langle C_{\uparrow}^{\dagger}(\boldsymbol{R})C_{\downarrow}^{\dagger}(\boldsymbol{R})C_{\downarrow}(0)C_{\uparrow}(0)\rangle_{R}\nonumber \\
 & = & _{L}\langle C_{\uparrow}^{\dagger}(\boldsymbol{R})C_{\downarrow}^{\dagger}(\boldsymbol{R})\rangle_{R}{}_{L}\langle C_{\downarrow}(0)C_{\uparrow}(0)\rangle_{R}+{}_{L}\langle C_{\uparrow}^{\dagger}(\boldsymbol{R})C_{\uparrow}(0)\rangle_{R}{}_{L}\langle C_{\downarrow}^{\dagger}(\boldsymbol{R})C_{\downarrow}(0)\rangle_{R}\,.
\end{eqnarray}
The second term can be figured out in the upper calculation. If we
take the limit $\boldsymbol{R}\rightarrow\infty$, it will decay to
0 as $R^{-3/2}$ on the transition boundary. What is important here
is the first term. Due to the translational invariance here, we can
simplify it as: 
\begin{eqnarray}
\rho_{2}(\boldsymbol{R}\uparrow,\boldsymbol{R}\downarrow;0\uparrow,0\downarrow) & = & _{L}\langle C_{\uparrow}^{\dagger}(0)C_{\downarrow}^{\dagger}(0)\rangle_{R}{}_{L}\langle C_{\downarrow}(0)C_{\uparrow}(0)\rangle_{R}\nonumber \\
 & = & \frac{1}{N^{2}}\sum_{\boldsymbol{k}_{1},\boldsymbol{k}_{2}}{}_{L}\langle C_{\boldsymbol{k}_{1}\uparrow}^{\dagger}C_{-\boldsymbol{k}_{1}\downarrow}^{\dagger}\rangle_{R}{}_{L}\langle C_{-\boldsymbol{k}_{2}\downarrow}C_{\boldsymbol{k}_{2}\uparrow}\rangle_{R}\nonumber \\
 & = & \left(\frac{\Delta_{0}}{U}\right)^{2}\,.
\end{eqnarray}
Here we use the definition of $\Delta_{0}$: $\Delta_{0}=-\frac{U}{N}\sum_{\boldsymbol{k}}{}_{L}\langle C_{-\boldsymbol{k}\downarrow}C_{\boldsymbol{k}\uparrow}\rangle_{R}$.
Even on the phase boundary, this will still not vanish. In contrast,
the result is given by: 
\begin{equation}
\rho_{2}(\boldsymbol{R}\uparrow,\boldsymbol{R}\downarrow;0\uparrow,0\downarrow)=-\frac{(\text{Im}\Delta_{0})^{2}}{U^{2}}\neq0\,.
\end{equation}
This also indicates that the phase transition in non-Hermitian system
is completely different from the Hermtian one between normal metal
and superconducting phase, where the gap $\Delta_{0}$ vanishes and
pair correlation function decays to 0 at infinity.

\section{Calculation for Renormalization Group(RG) Theory}

Here we consider the RG flow of the interaction strength. In standard
Wilsonian RG theory, the one-loop beta function is well know which
is at the order of $O(U^{2})$: 
\begin{equation}
\frac{dU}{dt}=\frac{1}{4\pi}U^{2}:=\beta_{1}(U)\,.
\end{equation}
However, this is not enough to obtain the phase boundary in the main
text. Here we take the two-loop correction into account and consider
the terms at the order of $O(U^{3})$.

Up to just one integral over the momentum, the higher order contribution
to RG flow comes from the correction for the high-momentum propagator.
Actually, we have the following self-interaction for high-momentum
propagator, as shown in Fig. \ref{feynman_diagram1}:

\begin{equation}
\Sigma(\bm{k},\Omega)=\int\frac{d\omega}{2\pi}\frac{U}{i\omega-v_{F}k}=U\,,
\end{equation}

where $k$ and $\Omega$ are the momentum and frequency for the external
leg respectively. Therefore, the propagator is modified as

\begin{equation}
G(\bm{k},\Omega)=\frac{1}{i\Omega-v_{F}k}+\frac{U}{(i\Omega-v_{F}k)^{2}}\,.
\end{equation}

By considering the self-energy diagrams in Fig. \ref{feynman_diagram1},
there is two-loop beta function in the BCS diagram. By applying the
Wilsonian RG and integrating out the energy shell $(\Lambda-d\Lambda,\Lambda)$,
we have: 
\begin{align}
\beta_{2}(U) & =-\{\rho_{0}d\Lambda U^{2}(\int\frac{d\Omega}{2\pi}\frac{U}{(i\Omega-\Lambda)^{2}}\frac{1}{-i\Omega-\Lambda}+\int\frac{d\Omega}{2\pi}\frac{1}{i\Omega-\Lambda}\frac{U}{(i\Omega+\Lambda)^{2}}\}\times\frac{1}{2}\\
 & =\frac{\rho_{0}^{2}U^{3}}{2}dt\,,
\end{align}
where we define $\rho_{0}=1/\Lambda$ with $\Lambda$ the upper bound
of fermionic energy and $dt=-\frac{d\Lambda}{\Lambda^{2}}$ representing
the RG flow parameter. Hence, we combine both the one-loop and two-loop
contributions together and the RG-flow equation is written as: 
\begin{equation}
\frac{dV}{dt}=V^{2}-\frac{1}{2}V^{3}\,.\label{eq:RG_Flow_Equ_Total}
\end{equation}
Here $V=\rho_{0}U$ is a dimensionless coupling strength. The RG flow
diagram for Eq. (\ref{eq:RG_Flow_Equ_Total}) is shown in the Fig.
\ref{RG_Flow_figure}. 
\begin{figure}
\begin{fmffile}{diagram} \begin{fmfgraph*}(120,80) \fmfleft{i}
\fmfright{o} \fmf{plain}{i,v,v,o} \fmfdot{v} \end{fmfgraph*}
\end{fmffile}\feynmandiagram[{[}layered layout, horizontal=a to
b][{[}edges] { {i1, i2} -- a{[}dot{]} -- {[}half left{]} b{[}dot{]}
-- {[}half left{]} a, b -- {f1, f2}, };

\caption{The Feynman diagram for self energy correction and interaction strength
renormalization. The right diagram is known as BCS diagram.}

\label{feynman_diagram1} 
\end{figure}

\begin{figure}
\includegraphics[width=0.5\columnwidth,bb = 0 0 200 100, draft, type=eps]{phase_diagram}

\caption{RG flow diagram for Eq. (\ref{eq:RG_Flow_Equ_Total})}

\label{RG_Flow_figure} 
\end{figure}

Obviously, there is a non-trivial fixed point as $V=2$ and a critical
line depicted as a blue curve. This line separates the whole space
into two phases, with one flowing to the origin and the other flowing
to the nontrivial fixed point. The equation of this line can be derived
as follow:

Noting that

\begin{equation}
\begin{cases}
\frac{dV_{R}}{dt} & =V_{R}^{2}-V_{I}^{2}-\frac{1}{2}V_{R}^{3}+\frac{3}{2}V_{R}V_{I}^{2}\\
\frac{dV_{I}}{dt} & =2V_{R}V_{I}-\frac{3}{2}V_{I}V_{R}^{2}+\frac{1}{2}V_{I}^{3}
\end{cases}
\end{equation}
, where $V_{R}=\mathrm{Re}(V),V_{I}=\mathrm{Im}(V)$. On this line,
when we consider the fixed point at $V_{I}\rightarrow\infty$, we
see $\frac{dV_{R}}{dt}=0$. Hence, the only solution is $V_{R}=\frac{2}{3},V_{I}=\infty$.
The specific expression of the critical line can be obtained by integrating
the Eq. (\ref{eq:RG_Flow_Equ_Total}):

\begin{equation}
\Delta t=-\frac{1}{V^{\mathrm{f}}}+\frac{1}{2}\ln V^{\mathrm{f}}-\frac{1}{2}\ln(2-V^{\mathrm{f}})+\frac{1}{V}-\frac{1}{2}\ln V+\frac{1}{2}\ln(2-V)\,,\label{eq:Relation_between_t_and_U}
\end{equation}
with the superscribe f denoting the final state. At the critical line,
$V_{I}^{\mathrm{f}}\to\infty,V_{R}^{\mathrm{f}}\to\frac{2}{3}$, the
imaginary part of the above equation reads as 
\begin{equation}
0=\frac{\pi}{2}-\frac{V_{I}}{V_{R}^{2}+V_{I}^{2}}-\frac{1}{2}\arctan\frac{V_{I}}{V_{R}}-\frac{1}{2}\arctan\frac{V_{I}}{2-V_{R}}\,.\label{eq:Phase_boundary_by_RG}
\end{equation}
This is the equation define the phase boundary. Around the origin,
the expression of this critical line (\ref{eq:Phase_boundary_by_RG})
can be expanded as: 
\begin{equation}
0=-\frac{V_{I}}{V_{R}^{2}+V_{I}^{2}}+\frac{\pi}{2}\,,
\end{equation}
which is consistent with the phase boundary (\ref{phase_transition}).

In the Eq. (\ref{eq:Relation_between_t_and_U}), if we take $V^{\mathrm{f}}$
to be purely imaginary, we will get the recursion temperature: 
\begin{equation}
\Delta t=\frac{i}{V_{I}^{\mathrm{f}}}+\frac{1}{2}\ln V_{I}^{\mathrm{f}}+i\frac{\pi}{4}-\frac{1}{4}\ln(4+(V_{I}^{\mathrm{f}})^{2})-\frac{i}{2}(2\pi-\arctan\frac{V_{I}^{\mathrm{f}}}{2})+\frac{1}{V}-\frac{1}{2}\ln V+\frac{1}{2}\ln(2-V)\,.\label{Delta t}
\end{equation}
If we assume $|V|<<1$, the above equation (\ref{Delta t}) can be
rewritten as 
\begin{equation}
\begin{cases}
0 & =\frac{1}{V_{I}^{\mathrm{f}}}+\frac{\pi}{4}+\frac{1}{2}\arctan\frac{V_{I}^{\mathrm{f}}}{2}-\frac{V_{I}}{V_{R}^{2}+V_{I}^{2}}-\frac{1}{2}\arctan\frac{V_{I}}{V_{R}}-\frac{1}{2}\arctan(\frac{V_{I}}{2-V_{R}})\\
e^{-\Delta t} & =\sqrt{\frac{\sqrt{4+(V_{I}^{\mathrm{f}})^{2}}}{V_{I}^{\mathrm{f}}}}\exp(-\frac{V_{R}}{V_{R}^{2}+V_{I}^{2}})(\frac{|V|}{|2-V|})^{\frac{1}{2}}
\end{cases}
\end{equation}
. From the second equation, we have the recursion temperature with
the ending point $(0,V_{I}^{f})$ in RG flow. 
\begin{equation}
T_{\mathrm{recur}}\sim e^{-\Delta t}=\sqrt{\frac{\sqrt{4+(V_{I}^{\mathrm{f}})^{2}}}{V_{I}^{\mathrm{f}}}}\exp(-\frac{V_{R}}{V_{R}^{2}+V_{I}^{2}})(\frac{|V|}{|2-V|})^{\frac{1}{2}}\,.
\end{equation}
If we further assume that $V_{I}^{\mathrm{f}}\to\infty$, which take
place near the phase boundary, we can obtain the simplified recursion
temperature. 
\begin{equation}
T_{\mathrm{recur}}\sim e^{-\Delta t}=\exp(-\frac{V_{R}}{V_{R}^{2}+V_{I}^{2}})(\frac{|V|}{|2-V|})^{\frac{1}{2}}\,.
\end{equation}

\end{document}
