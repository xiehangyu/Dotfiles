\documentclass[twocolumn,english,prl,aps,superscriptaddress,amsmath,amssymb,floatfix,nofootinbib]{revtex4-2}
\usepackage[T1]{fontenc}
\usepackage{verbatim}
%\usepackage[letterpaper]{geometry}
%\geometry{verbose}
\setcounter{secnumdepth}{2}
\setcounter{tocdepth}{2}
\usepackage{amsmath}
\usepackage{amssymb}
\usepackage{graphicx}

\makeatletter
%%%%%%%%%%%%%%%%%%%%%%%%%%%%%% User specified LaTeX commands.
%\documentclass [prl,aps,letterpaper,preprint,amsmath,amssymb,floatfix] {revtex4}
%\documentclass [prl,aps,letterpaper,preprint,superscriptaddress,amsmath,amssymb,floatfix] {revtex4}
\usepackage{times}
\usepackage{textcomp}
\usepackage{epstopdf}
\usepackage{braket}
\usepackage{tikz}
\usepackage[colorlinks,linkcolor=blue,citecolor=blue]{hyperref}
\usepackage{tikz-network}
\usepackage{amsfonts}
\newtheorem{theorem}{Theorem}



%%%%%%%%%%%%%%%%%%%%%%%%%%%%%% LyX specific LaTeX commands.
\pdfpageheight\paperheight
\pdfpagewidth\paperwidth

%% Because html converters don't know tabularnewline
\providecommand{\tabularnewline}{\\}

%%%%%%%%%%%%%%%%%%%%%%%%%%%%%% Textclass specific LaTeX commands.

\@ifundefined{textcolor}{}{%
 \definecolor{BLACK}{gray}{0}
 \definecolor{WHITE}{gray}{1}
 \definecolor{RED}{rgb}{1,0,0}
 \definecolor{GREEN}{rgb}{0,1,0}
 \definecolor{BLUE}{rgb}{0,0,1}
 \definecolor{CYAN}{cmyk}{1,0,0,0}
 \definecolor{MAGENTA}{cmyk}{0,1,0,0}
 \definecolor{YELLOW}{cmyk}{0,0,1,0}
}

\usepackage{xcolor}\usepackage{soul}
\setcounter{MaxMatrixCols}{10}
%TCIDATA{OutputFilter=Latex.dll}
%TCIDATA{Version=5.50.0.2953}
%TCIDATA{<META NAME="SaveForMode" CONTENT="1">}
%TCIDATA{BibliographyScheme=BibTeX}
%TCIDATA{LastRevised=Wednesday, July 06, 2016 07:06:51}
%TCIDATA{<META NAME="GraphicsSave" CONTENT="32">}
%TCIDATA{Language=American English}

\newcommand{\dg}{$^\circ$ }
\newcommand{\dgc}{$^\circ\mathrm{C}$}
\def \Tr {\mathrm{Tr}}
\definecolor{blue}{rgb}{0,0,1}
\definecolor{red}{rgb}{1,0,0}
\definecolor{green}{rgb}{0,1,0}
\newcommand{\red}[1]{\textcolor{red}{ #1}}
\newcommand{\blue}[1]{\textcolor{blue}{ #1}}
\newcommand{\green}[1]{\textcolor{green}{ #1}}

%\@ifundefined{showcaptionsetup}{}{%
% \PassOptionsToPackage{caption=false}{subfig}}
%\usepackage{subfig}
%\makeatother

\usepackage{babel}
\begin{document}
\title{Free-fermion %Gaussian 
Page Curve: Canonical Typicality and Dynamical Emergence}
\author{Xie-Hang Yu}
\affiliation{Max-Planck-Institut f\"ur Quantenoptik, Hans-Kopfermann-Stra{\ss}e 1, D-85748 Garching, Germany}
\affiliation{Munich Center for Quantum Science and Technology, Schellingstra{\ss}e 4, 80799 M\"unchen, Germany}
\author{Zongping Gong}
\affiliation{Max-Planck-Institut f\"ur Quantenoptik, Hans-Kopfermann-Stra{\ss}e 1, D-85748 Garching, Germany}
\affiliation{Munich Center for Quantum Science and Technology, Schellingstra{\ss}e 4, 80799 M\"unchen, Germany}
\author{J. Ignacio Cirac}
\affiliation{Max-Planck-Institut f\"ur Quantenoptik, Hans-Kopfermann-Stra{\ss}e 1, D-85748 Garching, Germany}
\affiliation{Munich Center for Quantum Science and Technology, Schellingstra{\ss}e 4, 80799 M\"unchen, Germany}
\begin{abstract}
% ZG: F->f, no need to use Capital $T$
The celebrated Page curve proposed almost three decades ago describes the averaged bipartite entanglement for general interacting random quantum states. Surprisingly, the noninteracting (free-fermion) counterpart has only been established very recently. %and exhibits remarkable difference from the former. 
Here we provide further analytical insights into the latter, focusing on both the kinematic and dynamical aspects. First, we unveil the underlying canonical typicality \textcolor{blue}{and atypicality} for random free-fermion states%, which
. \textcolor{blue}{The former} appears for a small subsystem and is exponentially weaker than the well-known result in the interacting case. The latter explains why the free-fermion Page curve differs remarkably from the interacting one when the subsystem is macroscopically large, i.e., comparable with the entire system. Second, we find that the free-fermion Page curve emerges with unexpectedly high accuracy in some simple \textcolor{red}{tight binding} %local free-fermion 
models in long-time quench dynamics. This contributes a rare analytical result concerning quantum thermalization on a macroscopic scale, where conventional paradigms such as the generalized Gibbs ensemble and quasi-particle picture are not applicable. %This highlights the relevance to real physical systems
\begin{comment}
The random fermionic Gaussian(RFG-) ensemble is %a 
\textcolor{red}{widely used} %wide-used platform
to illustrate many physical concepts, such as thermalization and entanglement
entropy. Our work first studies the ergodicity of these ensemble and
prove the canonical typicality. We found that the thermalization rate
of the free Fermonic systems is only polynomial in environment size,
rather than exponential. This slower convergence rate can predict
very different local behavior of entanglement entropy for RFG-ensemble.
Later, we turn to some global behavior such as Page curve of the RFG-ensemble.
Our work suggests that the Page curve can dynamically emerge with
a class of time independent, translationally symmetric Hamiltonians.
We analytically give a full description of this class of Hamiltonians.
We also pointed out the original method of calculating free-fermionic
dynamical entropy in quasi-particle picture can only give a lower
bound, because this picture ignores correlations among multiple pairs
generated simultaneously. 
\end{comment}
\end{abstract}
\maketitle


\emph{Introduction.--}As
a central concept in quantum information science \citep{nielsen00},
entanglement has been recognized to play vital roles in describing
and understanding quantum many-body systems in and out of equilibrium
\citep{Luigi2008,relation_entropy_Phase,Eisert2015,Abanin2019}. %,relation_entropy_thermal,relation_entropy_thermal2,relation_entropy_thermal3,relation_entropy_Phase}. 
For example, entanglement area laws for ground states
of gapped local Hamiltonians enable their efficient descriptions based
on tensor networks \citep{RevModPhys.93.045003}, \textcolor{blue}{%and
\textcolor{red}{while} their violations \textcolor{red}{may} signature %entanglement also characterizes 
quantum phase transitions \citep{Vidal2003,Calabrese2004}.} %PhysRevB.94.195121,Shapourian2021,PhysRevA.85.030302,PhysRevA.85.062331}.}
The emergence of thermal ensemble from unitary evolution, a process
known as quantum thermalization \cite{Srednicki1994}, is ultimately attributed to the entanglement
generated between a subsystem and the complement \citep{Nandkishore2015}. %Popescu2006,Srednicki1994,QuantumChaosETH}.
%Entanglement is important: QI, CM, StatMech, HE}

\textcolor{red}{[ZG: Have deleted or rearranged some citations. Please also remove all the references that do not appear in the main text or SM.]} Almost thirty years ago, Page considered the fundamental
problem of bipartite entanglement in a fully random many-body system
and found a maximal entanglement entropy up to finite-size corrections \cite{Page1993}.
This seminal work was originally motivated by the black-hole information
problem \cite{Page1993blackhole}. %,Page2013}. 
Remarkably, it
has been attracted increasing and much broader interest in the past
decade, %even outside the high-energy physics community. 
due not only %especially 
to the new theoretical insights from quantum thermalization \citep{PhysRevLett.115.267206,PhysRevLett.119.220603,Nakagawa2018,PageCurve_Thermal,PageCurve_Thermal2,PhysRevLett.125.021601} and %PageCurve_Thermal3},
quantum chaos \citep{Sekino_2008,%Chaos_scramb2,%Chaos_scramb3,Chaos_scramb4,
Chaos_Scrambl,Nahum2018},
%and mathematical physics \citep{%Akemann2015,Collins2006,Matsumoto2013,Edelman2005,PhysRevE.105.014109,PhysRevLett.115.267206,Wigner1955,PhysRevLett.119.220603,Matsumoto2013},
but also to the practical relevance in light of the rapid experimental development in quantum simulations %Recently, the long-range control of quantum simulators provides further experimental access to this physical quantity 
\citep{entropy_measure1,entropy_measure2,entropy_measure3,long_range_accessible,long_range_accessible2,Semeghini2021,%Dai2016,
Yang2020,
Liu2022}. %Experimental Realization}\textcolor{red}{%in generic interacting many-body systems. 
In particular, the saturation of maximal entropy has been found to
be a consequence of canonical typicality \citep{Popescu2006,Goldstein2006,Reimann2007},
which means most random states behave locally like the canonical ensemble. %Also, 
This \textcolor{red}{typicality} %\textcolor{blue}{kinematic} 
behavior
has been argued to emerge in generic interacting many-body
systems satisfying the eigenstate thermalization hypothesis \citep{Srednicki1994,QuantumChaosETH,Abanin2019,
Rigol2008,Nandkishore2015,Moessner2017} %QuantumChaos,Collura2014,Ishii2019,Sotiriadis_2014},  
\textcolor{red}{%implying the ultimate connection to quantum thermalization. This relation 
and} can even be rigorously established \textcolor{red}{or ruled out} in specific situations \cite{%PhysRevLett.124.200604,
ETH_CT_proof3,Hamazaki2018}. %,PhysRevLett.116.030401}. %\textcolor{blue}{The relation between canonical typicality and ETH has been proved both numerically and analytically in many quantum models \citep{ETH_CT_proof1,ETH_CT_proof2,ETH_CT_proof3,ETH_CT_proof4,ETH_CT_proof5}.}%many-body system, say \$N\$ qubits. For a subsystem with \$N\_A<N/2\$ qubits, the averaged entanglement entropy was found to saturate the maximal value up to finite-size corrections. 

%Page curve: recent Gaussian, why important? }

In this Letter, we provide analogous insights into
the noninteracting counterpart of Page's problem.  %\textcolor{blue}{due to the importance of Gaussian resources in quantum information \citep{matchgates,Circuit_complexity_free_fermion,fermionicTomograph1}, quantum computations \citep{Matos2022} and quantum chaos \citep{Ishii2019,Analytical_example_fermionic_gaussian,Cassidy2011}.}%Why Gaussian
%Precisely speaking, we focus on the random fermionic Gaussian (RFG) ensemble, 
\textcolor{red}{That is, we focus on free fermions or (fermionic) Gaussian states}, which are of their own interest in quantum many-body physics, quantum information and computation \citep{PhysRevLett.116.030401,PhysRevLett.121.200501,PhysRevA.65.032325,PhysRevLett.120.190501,matchgates,Circuit_complexity_free_fermion,fermionicTomograph1,Matos2022}. Somehow surprisingly, in the seemingly simpler noninteracting
case, the subsystem-size dependence of averaged entanglement entropy,
which is described by the Page curve pictorially, was not solved until
very recently \citep{Bianchi2021,Bianchi2021a}. It turns out to be similar to the interacting case for a small subsystem, but differ significantly otherwise. \textcolor{blue}{See Fig.~\ref{Setup_of_CCRFG_ensemble}(b) for an illustration.} We establish the corresponding %a qualitatively weaker 
canonical typicality and \textcolor{blue}{atypicality} that explicitly explains the similarity and difference. In addition, we show that the free-fermion Page curve can be relevant to extremely simple tight-binding models via long-time quench dynamics. %although it still obeys the volume law. Here comes a natural question: ? 
 %Our work: two questions, why different? Physical relevance?}


\begin{comment}\textbf{\emph{Introduction-}} Ergodicity is the foundation of %classical
statistial physics. It becomes a great conceptual puzzle when we consider
quantum mechanics, where the universe is a pure state and evolves %d
unitarily. How to re-establish the foundation in quantum statistical
mechanics has been studied for decades. Up to now, there are two definitions
of thermalization in quantum statistics. One is dynamical and the
other is kinematic. The dynamical thermalization focuses on eigenstate
thermalization hypothesis (ETH) \citep{Srednicki1994,QuantumChaosETH,Abanin2019}.
This hypothesis argues that the eigenstates of \textcolor{red}{typical} %most 
Hamiltonians will
locally reduce to the canonical ensemble. %and the off diagonal correlation can be smeared out in long time average. 
On the other hand, the kinematic
thermalization studied the behavior of each pure state in universe
rather than time average. If the environmental size is large enough,
each pure state in the ensemble will behave locally like the average
of the ensemble\textcolor{red}{, %. This 
a property known} %refers to 
as canonical typicality \citep{Popescu2006,Goldstein2006,Reimann2007}.
%In our work, 
\textcolor{red}{Here,} we are interested in the fermionic gaussian state because
it plays important roles in quantum information in terms of matchgates
\citep{matchgates}, quantum computations \citep{Matos2022} and quantum
chaos \citep{Ishii2019,Analytical_example_fermionic_gaussian,Cassidy2011}.
We use kinematical thermalization to characterize its statistical
behavior.

%When talking about thermalization, 
\textcolor{red}{Entanglement entropy is} an important macroscopic quantity
%is entropy, 
which has been shown deeply connected to thermalization
\citep{relation_entropy_thermal,relation_entropy_thermal2,relation_entropy_thermal3}.
Some experiments also measure the entropy \citep{entropy_measure1,entropy_measure2,entropy_measure3}
to indicate thermalization. %The entanglement entropy also relates to phase transition in condensed matter physics \citep{relation_entropy_Phase}.
Nowadays, with the development of experimental technics, we can access
large scale systems and non local operations \citep{long_range_accessible,long_range_accessible2,Semeghini2021,Dai2016,Yang2020}.
The study of non local entanglement entropy becomes possible, which
is described by Page curve\citep{Page1993}.

Originally, Page studies the average of entanglement for general pure
state ensembles in order to address the black hole information paradox
\citep{Page1993blackhole,Page2013}. Later, the Page curve for other
ensembles have also been studied and shown to related to random matrix
theory \citep{Bianchi2021}. Besides the original purpose, the Page
curve can be used to describe quantum chaos and scrambling \citep{Chaos_scramb2,Chaos_scramb3,Chaos_scramb4,Chaos_Scrambl,QuantumChaos}.
It also appears in the study of thermalization \citep{PageCurve_Thermal,PageCurve_Thermal2,PageCurve_Thermal3}. 

The Page curve for RFG-ensemble was first obtained in \citep{Bianchi2021a}
and later generalized to charge conserved symmetric case in \citep{Bianchi2021}.
It still has volume law but with a different factor and variance from
\textcolor{red}{the interacting} %interaction 
one. Besides, this random quantum resource has many applications
in quantum information process \citep{Circuit_complexity_free_fermion,fermionicTomograph1}
and may also have advantage in quantum metrology when generalizing
\citep{Bosonic_metrology} to fermionic system \citep{fermionic_Gauss_metrology}.
Therefore, for those practical reasons we need to consider how the
Page curve can appear in real physical system rather than the manifold
of random matrices\citep{Kaneko2020,Nakata2017,Ho2022,Lucas2022}.

In this letter, we first derive the canonical typicality of RFG-ensemble. We will use our typicality result to illustrate the difference of Page
curve between free and \textcolor{red}{interacting} %interaction 
fermionic system. Then, we propose
a simple class of time-independent Hamiltonian with translational
symmetry, which can dynamically emerge the Page curve for RFG-ensemble
to a very high accuracy. Some of those previous works \citep{Kaneko2020,Nakata2017,Ho2022,Lucas2022}
consider non-integrable systems or floquet systems which are hard
to keep track of analytically and only have numerical results. Others
require local physical systems plus infinite degrees of ancilla. Our
proposal is much simpler than those. It has no reference to ancilla
and is a pure global property.    

%\textcolor{red}{(ZG: Try to explain why Winter's proof cannot apply to Gaussian states)}
\end{comment}
\emph{Canonical typicality and atypicality.--}%In our work, 
We start by generalizing %generalize
the main result in \citep{Popescu2006} to \textcolor{red}{the random fermionic Gaussian (RFG)} ensemble. While \citep{Popescu2006} already considers possible restrictions, we stress that Gaussianity is inadequate since Gaussian states do not constitute a Hilbert subspace. 
%which will show great difference from the \textcolor{red}{interacting} %interaction one. 
For simplicity, we %will only 
consider number-conserving systems with totally $N$ modes occupied by $N/2$ fermions, i.e., the half-filling case. Compared to the fully random case without number conservation, this appears to be more physically comprehensible and experimentally relevant, while displaying exactly the same Page curve %half filling case, because it has been shown this case dominates the contribution of Page curve for RFG-ensemble. The volume-law coefficient of half-filling case also converges to the latter one
\citep{Bianchi2021a,Bianchi2021}. %Of course, there is no difficulty to consider other filling numbers \cite{SM}.} 
%The discussion on 
More general ensembles are discussed %can be found 
in Supplemental Material \cite{SM}. 

\begin{figure}
\includegraphics[width=1\columnwidth]{figs/new_set_up_fig}\caption{(a) The entire %total 
\textcolor{red}{free-fermion} system has $N$ sites with half filling. The %physical 
subsystem of interest %we are interested in 
has $N_{A}$ ($N_{A}<N$) sites. %shows the states are in the 
The RFG-ensemble is generated by Haar-random Gaussian unitaries \textcolor{red}{with number conservation}. %induced by Haar measure over $\mathbb{\textcolor{red}{U}}(N)$.
\textcolor{blue}{(b) %shows the state is evolved with some hopping terms.}
The Page curve of RFG-ensemble and interacting ensemble in the thermal dynamical limit $N\to\infty$. From the figure, it is obvious that these two Page curves agree with each other in microscopical region but show a $\mathcal{O}(1)$ deviation in macroscopical region. The interacting Page curve in the thermal dynamical limit is always saturated. (c) The table summarizes the typicality/atypicality property for RFG-ensemble and interacting ensemble}}
\label{Setup_of_CCRFG_ensemble}
\end{figure}

A pictorial illustration of our setup is shown %can be seen 
in Fig.~\ref{Setup_of_CCRFG_ensemble}(a). 
Due to Wick's theorem \citep{Hackl2021}, a fermionic Gaussian %RFG-
state $\rho$ is %can be 
fully captured %described 
by its covariance matrix $C_{j,j'}=\mathrm{Tr}(\rho a_{j}^{\dagger}a_{j'})$.
Here $a_{j}$ is the annihilation operator for mode $j$, which may label, e.g., a lattice site. %at the site $l$. 
%The vanishing of $\Tr(\rho a^{\dagger}a^{\dagger})$ is because we are considering half filling case. The more general form of covariance matrix includes a unitary transformation into Majorana basis. 
As the covariance matrix for any RFG-pure state can be related to each
other by a unitary transformation, %on the space of creation operators,
the uniform distribution over this ensemble can be generated at the
level of the covariance matrix $\{C=UC_{0}U^{\dagger}\}$. Here %where
$U$ is taken Haar-randomly over the unitary group $\mathbb{U}(N)$ %$\mathbb{SU}(N)$ 
\citep{Bianchi2021,Bianchi2021a} and
$C_{0}$ is an arbitrary reference %is a specific 
covariance matrix in the ensemble satisfying $C_0^2=C_0$ and $\Tr C_0=N/2$. %, which can be taken as $\begin{pmatrix}I_{\textcolor{red}{N/2}} & 0\\ 0 & 0 \end{pmatrix}$. 
%for simplicity. Using 

An important property of Gaussian states is that their subsystems will also be Gaussian. %and thus fully characterized by the covariance matrix. 
We denote $C_{A}$ as %denotes 
the $N_A\times N_A$ covariance matrix restricted
to %the physical 
subsystem $A$ with $N_A$ modes. The entanglement entropy $S_A=-\Tr(\rho_A\log_2\rho_A)$ of the reduced state $\rho_A=\Tr_{\bar A}\rho$ ($\bar A$: complement of $A$) then reads:
\begin{equation}
\begin{split}
    S_{A}=&-\Tr(C_A\log_2 C_A)\\
    &-\Tr((I_A-C_A)\log_2 (I_A-C_A)), %\mathrm{Tr}\begin{bmatrix}C_{A}\\ & I_{A}-C_{A} \end{bmatrix}\log_{2}\begin{bmatrix}C_{A}\\& I_{A}-C_{A}\end{bmatrix}
    \end{split}
    \label{eq:SA}
\end{equation}%, the distance between two covariance matrix is calculated with Hilbert-Schmidt distance $d_{\mathrm{H-S}}(C_{1},C_{2})=\sqrt{(C_{2}-C_{1})^{\dagger}(C_{2}-C_{1})}.
where $I_A$ is the identity matrix with dimension $N_A$.

Our first result is the measure concentration property of the covariance
matrix for RFG-ensemble:

\textbf{Theorem 1}: For arbitrary $\epsilon>0$ and subsystem $A$, the probability that the reduced covariance matrix of a state in the RFG ensemble deviates from the ensemble average \textcolor{red}{satisfies} %for each arbitrary state in RFG-ensemble with half filling, its covariance matrix restricted in subsystem $A$ is almost identity with high probability $\mathbb{P}$ as long as the environment system size is large enough
\begin{equation}
\mathbb{P}(d_{\rm HS}(C_{A},%\frac{1}{2}
I_{A}/2)\geq\eta+2\epsilon)\leq2e^{-\frac{\epsilon^{2}}{\eta'}},
\label{eq:Canonicality_result_in_CCRFG}
\end{equation}
\begin{equation}
\textcolor{red}{\mathbb{P}(d_{\rm HS}^2(C_{A},%\frac{1}{2}
I_{A}/2)\leq\eta_{\rm a}-2\epsilon)\le 2e^{-\frac{\epsilon^2}{\eta'_{\rm a}}}},    
\label{eq:atypicality}
\end{equation}
with $\eta=\sqrt{\frac{N_{A}^{2}}{2(N-1)}}$, %and 
$\eta'=\frac{12}{N}$, \textcolor{red}{$\eta_{\rm a}=\frac{N_A^2}{4(N+1)}$, $\eta'_{\rm a}=\frac{12N_A}{N}$} and $d_{\rm HS}(C,C')=\sqrt{\Tr(C-C')^2}$ being the Hilbert-Schmidt distance. 

%One of the skills we used in the proof is
The proof largely relies on the generalized Levy's lemma
%in 
for Riemann manifolds with positive curvature \citep{measure_concentration1,measure_concentration2,Meckes2019},
which allows us to turn the upper bound on %of
the distance average $\langle d_{\mathrm{HS}}(C_{A},I_{A}/2)\rangle\leq\sqrt{\frac{N_A^2}{2(N-1)}}$ \textcolor{red}{or the lower bound $\langle d^2_{\mathrm{HS}}(C_{A},I_{A}/2)\rangle\geq\frac{N_A^2}{4(N+1)}$}
into a probability inequality \citep{SM}. %\textcolor{red}{See} %The details of the proof can be seen in 
%Supplemental Material \textcolor{red}{for detail}. 
%Although we are using the Hilbert-Schmidt distance here, one can easily obtain the result for other distance measures, since all measures of distance in finite space is equivalent. 
From Eq. (\ref{eq:Canonicality_result_in_CCRFG}) we can easily see for
infinite environments $N\to\infty$, the local microscopic system will have maximal entropy
$S_{A}\to N_{A}$.

%It is also worthy noting 
We emphasize that in Eq. (\ref{eq:Canonicality_result_in_CCRFG}),
$\eta$ and $\eta'$ only scales polynomially with the system size. %sites.
%It is a big contradictory to 
This contrasts starkly with the exponential scaling canonical typicality %result 
for random interacting %interaction 
ensemble \citep{Popescu2006}. %in which they scales exponentially.
Intuitively, this is because in the interacting %interaction 
case, the %dimension of the 
Hilbert-space dimension scales exponentially with the (sub)system size, %sites, while 
which, however, simply equals to the size %degrees of} freedom 
of the covariance matrix \textcolor{blue}{in the free fermion case}. %only scales polynomially with the system sites. 
Physically, the Gaussian \textcolor{red}{constraint} %constriction 
makes the ensemble
%can 
only explore a very limited sub-manifold in the \textcolor{red}{entire} %total 
Hilbert space.
%\textcolor{red}{(ZG: Try to explain in further detail: for a macroscopically large subsystem, $\eta/N_A$ can be exponentially small in the interacting case, but is $\mathcal{O}(1)$)}
This polynomial scaling means that, for a fixed subsystem size $N_A$, the reduced state still exhibits canonical typicality, while the atypicality is only polynomially suppressed by the environment size. \textcolor{red}{Accordingly, the averaged entanglement entropy should achieve the maximal value but with a polynomial finite-size correction.} \textcolor{red}{In fact, such an exponentially weaker canonical typicality (\ref{eq:Canonicality_result_in_CCRFG}) can also be exploited to explain the qulitatively larger variance of the entanglement entropy for the RFG ensemble, which is $\mathcal{O}(N^{-2})$ %$\frac{1}{N^{2}}$ 
in comparison to  
%for RFG-ensemble while 
$e^{-\mathcal{O}(N)}$ in the interacting case} %for interacting %interaction one
\citep{Bianchi2021a,Bianchi2021,SM}.  

\textcolor{red}{On the other hand}, %Moreover, 
if the subsystem is macroscopically large, implying that $f=N_A/N$ is $\mathcal{O}(1)$, the concentration inequality (\ref{eq:Canonicality_result_in_CCRFG}) becomes meaningless since $\eta$ is $\mathcal{O}(\textcolor{red}{\sqrt{N}})$, \textcolor{red}{the same order as the Hilbert-Schmidt norm of $C_A$. Instead, we may take $\epsilon=\mathcal{O}(N^\alpha)$ with $\alpha\in(0,1)$ in Eq.~(\ref{eq:atypicality}), finding that the majority of the reduced covariance matrix differs significantly from the ensemble average. In other words, the RFG ensemble exhibits canonical atypicality in this case. In particular, this result implies an $\mathcal{O}(1)$ deviation in the entanglement entropy density from the maximal value}. \textcolor{red}{We recall that, in stark} contrast, the canonical typicality for interacting states is exponentially stronger and persists even on \textcolor{red}{any} macroscopic scale \textcolor{red}{with $f<1/2$}. %\textcolor{blue}{except when $N_{A}\geq\frac{N}{2}$.


%\textcolor{blue}{More specifically, for RFG ensemble with macroscopical subsystem,
%the atypicality is lower bounded by 
%\begin{equation*}
%\begin{split}
%\mathbb{P}\{d_{\mathrm{HS}}^{2}(C_{A},I_{A}/2) & \leq\frac{N_{A}^{2}}{4(N+1)}-2\sqrt{N_{A}}/N^{\frac{1}{3}}\}\\
% & \leq2e^{-\frac{N^{\frac{1}{3}}}{4}}
%\end{split}
%\end{equation*}
%This means that when $f\sim \mathcal{O}(1)$, the Gaussian state is anti-concentrated
%by an $\mathcal{O}(N)$ factor with probability very close to $1$.} 
%although the free fermions can still be kinematically thermalized, its thermalization rate is much slower than the \textcolor{red}{interacting} %interaction one. 

The above discussions \textcolor{blue}{can be made more straight forward in \citep{SM}, where we consider the measure concentration property of the subsystem
entropy $S_{A}$, obtaining}
\begin{equation}
\begin{split} & \mathbb{P}(S_{A}\leq N_{A}-\epsilon)\\
 & \leq\begin{cases}
2e^{-\frac{(\sqrt{\epsilon}-\xi)^{2}}{\xi'}}, & \epsilon>\xi^{2};\\
1, & \epsilon\leq\xi^{2}
\end{cases}
\end{split}
\label{eq:main_text_typicality_for_entropy}
\end{equation}

\textcolor{blue}{in microscopic region and}
\begin{equation}
\begin{split}\mathbb{P}(S_{A}\geq N_{A}-\xi_{a}+\frac{2}{\ln2}\epsilon) & \leq2e^{-\frac{\epsilon^{2}}{\xi'_{a}}}\end{split}
\end{equation}
\textcolor{blue}{in macroscopic region. Here $\xi=\sqrt{\frac{2N_{A}^{2}}{N-1}}$, $\xi'=\frac{192}{N}$, $\xi_{a}=\frac{N_{A}^{2}}{2\ln2(N+1)}$ and $\xi'_{a}=\frac{48N_{A}}{N}$. It is worthy emphasizing that Eq. (\ref{eq:main_text_typicality_for_entropy}) also becomes meaningless in macroscopic region since $\xi^2$ will be comparable with $N_A$.}
%slower convergence rate can 
\textcolor{red}{Thus we fully} explain the \textcolor{red}{similarity and difference, which occur on the microscopic and macroscopic scales, respectively, between} % obvious deviation of 
the Page curve for Gaussian states \textcolor{red}{and} %with 
the one for interacting %interaction 
states, see Fig.~\ref{Setup_of_CCRFG_ensemble}(b) and (c).
%This rate will also lead to very different scaling behavior of mutual information. In \citep{Shapourian2021}, the mutual information will almost vanish if $N_{A}<\frac{N}{2}$. However, for free fermions, the mutual information never vanishes even if $N_{A}<\frac{N}{2}$, see Supplemental Material.

\emph{Dynamically emergent Page curve.--}\textcolor{red}{%On top of the above kinematic results, 
We recall that a particularly intriguing point of the (interacting) Page curve is its emergence in physical many-body systems with local interactions, which are typically chaotic but yet far from fully random. Indeed, a popular phenomenological theory for describing generic entanglement dynamics on the macroscopic level, the so-called entanglement membrane theory \cite{Nahum2018}, explicitly assumes that the entanglement profile of the thermalized system follows the Page curve. It is thus natural to ask whether the free-fermion Page curve could be relevant to thermalization in physical systems without interactions. Note that this question is complementary to the (a)typicality results which are kinematic, i.e., irrelevant to dynamics, as in the interacting case.} %Here, 


\begin{figure*}
\includegraphics[width=0.9\textwidth]{figs/dynamical_page_curve_total}
\caption{\textcolor{blue}{(a) and (b) show the tight binding Hamiltonians with periodicity 2 in our dynamical model. (a) only includes the odd-range hopping, while (b) includes even-range hopping.} (c) Dynamical Page curve for the minimal model %nearest neighbor hoping Hamiltonian
(blue) %line) 
and its comparison with the Page curve for RFG-ensemble
(red) %line) 
as well as our theoretic result Eq. (\ref{eq:the_Page_curve_for_NNH})
(green). %line). 
Here $N=200$. %We can see 
These three lines are very close to each other, with a relative difference $\sim\mathcal{O}(10^{-3})$ which agrees with our analysis. \textcolor{blue}{This figure can also represent the general dynamical Page curve for Hamiltonians in (a).} %argument. 
(d) Dynamical Page curve for Hamiltonian
$H=(\sum_{j=1}^{N}a_{j}^{\dagger}a_{j+1}+0.3\sum_{j:\mathrm{even}}a_{j}^{\dagger}a_{j+2}-0.3\sum_{j:\mathrm{odd}}a_{j}^{\dagger}a_{j+2})+{\rm H.c.}$. Here also $N=200$. The dynamical Page curve is obviously different
from the Page curve for RFG-ensemble. The considerable %big
deviation between the
theoretical result and the dynamical Page curve near $f=\frac{1}{2}$, where higher-order terms become least negligible, is because we only calculate up to the third term in Eq. (\ref{eq:Taylor_expansion_for_entropy}) \cite{SM}. \textcolor{blue}{This figure also illustrates the property of dynamical Page curve for general Hamiltonians in (b).} \textcolor{red}{[ZG: make the subfigure labels ``(a)" and ``(b)" smaller and lower.]}
%, see Supplemental Material. 
%The range where $f=\frac{1}{2}$ will receive large corrections from higher order terms. 
}
\label{dynamical_page_curve_total}
\end{figure*}


\textcolor{red}{We try to address the above question by %will 
analytically investigating the long-time averaged entanglement entropy in the quench dynamics governed by} %that the free-fermion Page curve \textcolor{red}{does emerge, although not perfectly,} in the long-time dynamics of
some simple \textcolor{red}{local quadratic} Hamiltonians \textcolor{red}{with number conservation. Hereafter, we use the term ``dynamical Page curve" to refer to this long-time averaged entanglement profile. %can dynamically emerge the Page curve to a very high accuracy with long time average. 
%For simplicity, the periodic boundary condition is assumed.
To simplify the analytic calculations, we assume the Hamiltonians to be translation-invariant (under the periodic boundary condition) and specify} our initial state \textcolor{red}{to be} %is 
a period-2 density wave \textcolor{red}{with half filling. See Fig.~\ref{dynamical_page_curve_total}(a-b) for a schematic illustration. It is worth mentioning that the dynamical Page curve is ensured to be concave by translation invariance, as a result of the strong subadditivity of quantum entropy \cite{Wolf2008}.}
%The Hamiltonian is a \textcolor{red}{noninteracting} %non-interaction
%spinless one which conserves the total charge number $\frac{N}{2}$, see Fig.~\ref{Setup_of_CCRFG_ensemble}(b). First, let us consider 

\textcolor{red}{We primarily focus on the minimal} %a rather simple 
model\textcolor{red}{, i.e., a one-dimensional lattice with nearest-neighbor hopping:}
\begin{equation}
H_{0}=\sum_{j}a_{j}^{\dagger}a_{j+1}+{\rm H.c.},
\label{eq:NNH_Hamiltonian}
\end{equation}
\textcolor{red}{whose %Assuming the periodic boundary condition, one immediately obtains the 
band dispersion reads $E_k=2\cos k$. We believe that the exact results for the large (spatiotemporal) scale dynamical behaviors of this fundamental model are interesting on their own.  Moreover, our method and results actually apply to much broader situations, as will soon become clear below.} %Later we will use $\overline{O}$ to denote the long time average of the quantity $O$.


Surprisingly, \textcolor{red}{despite the additional translation-invariant and energy-conserving constraints compared to the RFG ensemble,} this minimal model (\ref{eq:NNH_Hamiltonian}) turns out to give rise to %simple nearest neighbor hopping Hamiltonian can dynamically emerge 
\textcolor{red}{a dynamical Page curve extremely close to that for the %the Page curve of 
RFG ensemble %in the long time average 
%to a very high accuracy, 
(see blue and red curves in Fig.~\ref{dynamical_page_curve_total}(c))}. 
%We %will 
%use the term dynamical Page curve to refer to this long-time average behavior. 
Our calculation strategy is by perturbatively expanding
the entropy expression (\ref{eq:SA}) %, which can be calculated as $S_{A}=-\mathrm{Tr}\begin{pmatrix}C_{A}\\ & I_{A}-C_{A}
%\end{pmatrix}\log_{2}\begin{pmatrix}C_{A}\\ & I_{A}-C_{A}\end{pmatrix}$. 
%Expanding it 
around $C_{A}=\frac{I_{A}}{2}$, obtaining %we will get 
\begin{equation}
S_{A}(t)=N_{A}-%\frac{1}{\ln2}
\sum_{n=1}^{\infty}\frac{\mathrm{Tr}(2C_{A}(t)-I_{A}){}^{2n}}{2n(2n-1)\ln2}.
\label{eq:Taylor_expansion_for_entropy}
\end{equation}
Thanks to the translational invariance, $C_{A}(t)$ can be related to the block-diagonal momentum-space covariance matrix $\tilde C(t)=\bigoplus_k\tilde C_k(t)$ via %in Fourier space 
$C_{A}(t)=\Pi_{A}U_{\rm F}\tilde{C}(t)U^{\dagger}_{\rm F}\Pi_{A}$. Here
$U_{\rm F}$ and $\Pi_{A}$ are the Fourier transformation matrix and projector %projection operator 
to subsystem $A$, respectively. %$\tilde{C}$ is the covariance matrix in momentum space with block-diagonal structure. 
The off-diagonal
elements of $\tilde{C}_k(t)$ involve %will have 
a time-dependent %dynamical 
phase
$e^{i\theta_{k}(t)}$ with %which relates to eigenenergies 
$\theta_{k}(t)=t(E_{k}-E_{k+\pi})$.
When calculating $\overline{\mathrm{Tr}(2C_{A}(t)-I_{A})^{2n}}$, where $\overline{f(t)}=\lim_{T\to\infinty}T^{-1}\int^T_0 dt f(t)$ denotes the long-time average,
we will encounter %meet the 
terms like $\overline{e^{i\theta_{k}(t)}e^{i\theta_{k'}(t)}}$,
which equals to $\delta_{k,k'+\pi}$ in the thermodynamic limit.
This contraction allows us to establish a set of Feynman rules for %with which we can 
systematically calculating Eq.~(\ref{eq:Taylor_expansion_for_entropy})
order by order \cite{SM}. %Details can be found in Supplemental Material. 

%Denoting $f=\frac{N_{A}}{N}$. 
\textcolor{red}{Since the bipartite entanglement entropy is identical for either of the subsystems, the Page curve is reflection-symmetric with respect to $f=\frac{1}{2}$ and thus%Due to the symmetry of bipartition entropy
} it suffices to %we can only 
focus on $f=N_A/N\leq\frac{1}{2}$. In the thermodynamic limit, the dynamical Page curve turns out to be \cite{SM} 
%is:
\begin{equation}
\frac{\overline{S_{A}}}{N}=f-\frac{1}{\ln{2}}\left(\frac{1}{2}f^{2}+\frac{1}{6}f^{3}+\frac{1}{10}f^{4}\right)+\mathcal{O}(f^{5}).\label{eq:the_Page_curve_for_NNH}
\end{equation}
 On the other hand, the %volume-law coefficient in 
 Page curve for RFG ensemble
is \citep{Bianchi2021}
\begin{equation}
\frac{\langle S_{A}\rangle}{N}=f-\frac{1}{\ln{2}}\left(\frac{1}{2}f^{2}+\frac{1}{6}f^{3}+\frac{1}{12}f^{4}\right)+\mathcal{O}(f^{5}).\label{eq:the_Page_curve_result_for_CCRFG}
\end{equation}
%the difference between 
The above two equations differ %are 
only by $\frac{1}{60\ln{2}}f^{4}+\mathcal{O}(f^5)$, %\leq\frac{1}{480}$, 
which is as small as about $\mathcal{O}(10^{-3})$ even for $f$ near $1/2$.
%indicating the dynamical process of nearest neighbor hopping Hamiltonian can emerge the Page curve for RFG-ensemble to a very high accuracy!


Even if we add a perturbation $H_{1}$ to Eq.~(\ref{eq:NNH_Hamiltonian}),
as long as $H_{1}$ is periodic-2 %fulfills the symmetry of the initial state (namely it is symmetric when translating 2 sites) 
and %it 
only includes %the 
odd-range hopping, as represented by Fig.~\ref{dynamical_page_curve_total}(a),  %term with odd range, 
the dynamical Page curve can be analytically proved to be the same as Eq. (\ref{eq:the_Page_curve_for_NNH}) in the thermodynamic %thermal dynamical 
limit, as the same Feynman rules apply \cite{SM}. %see the discussion below Theorem 3 and the Supplemental Material. 
One example %of $H_{1}$ 
is $H_{1}=J(\sum_{j:\mathrm{even}}a_{j}^{\dagger}a_{j+2m+1}-\sum_{j:\mathrm{odd}}a_{j}^{\dagger}a_{j+2m+1})+{\rm H.c.}$
for arbitrary $J$ and integer $m$. Thus, we have defined another ensemble
of fermionic Gaussian states by dynamical evolution, which covers a
wide class of Hamiltonians and this ensemble has remarkably similar
Page curve as the RFG ensemble. 

However, if $H_{1}$ includes even-range %the 
hopping, as represented by Fig.~\ref{dynamical_page_curve_total}(b) %term with even range, 
the dynamical Page curve will be very different, %from the one for RFG-ensemble, 
as shown in Fig.~\ref{dynamical_page_curve_total}(d).
This can be easily explained with the canonical typicality property
proved above: for this class of Hamiltonians, their conserved \textcolor{red}{(eigen)} mode
occupation number $n_{k}$ deviates from the average value of RFG ensemble,
which is %will be 
$\frac{1}{2}$. %This deviation is a strong evidence that this ensemble for the dynamical process corresponding to this class of Hamiltonians must be very different from RFG-ensemble. 
Thus, %its Page curve 
the dynamical ensemble is naturally %reasonable to be 
``atypical" \textcolor{blue}{even for microscopical scale because the local conserved observable is constructed from mode occupation numbers \citep{Ishii2019}}. \textcolor{red}{This result implies the reduced state on a small subsystem deviates considerably from being maximally mixed so that the tangent slope of the dynamical Page curve at $f=0$ is well below that for the RFG ensemble.
%\textcolor{red}{Accordingly, the Feynman rules for perturbatively calculating the Page curve become more complicated.} 
In contrast,} one can show that all the conserved mode occupation number for the class of Hamiltonians mentioned in the last paragraph are $\frac{1}{2}$. 

%Now we can introduce our 
All the observations above constitute our second main result: %third 
%theorem: 

\textbf{Theorem 2}: The RFG-ensemble-like dynamical Page curve (\ref{eq:the_Page_curve_for_NNH}) emerges for a period-2 \textcolor{red}{short-range free-fermion} Hamiltonian if and only if the conserved mode occupation numbers are $1/2$. %In order for a Non-interaction charge-conserved fermionic system with periodic-2 symmetry in space to dynamically emerge the Page curve for RFG-ensemble, a necessary condition is that all the conserved mode occupation number must be $\frac{1}{2}$. 

%The necessary part of the theorem can be argued similarly as above. For sufficient part, in Supplemental Material, we will show that those Hamiltonians satisfying the conditions above will have the same Feynman rule as for the nearest neighbor hopping Hamiltonian, thus the same dynamical Page curve.

%Our techniques in calculating Eq. (\ref{eq:Taylor_expansion_for_entropy}) can also be generalized to the class of Hamiltonians in Fig. (\ref{The_figure_of_dynamical_page_curve_for_range2}) with a different and much more complicated Feynman rules.


\emph{Discussions.--}%Sometimes 
It is well-known that the generalized Gibbs ensemble (GGE) %is also used to 
characterizes %argue
the local thermalization of integrable systems including free fermions \citep{PhysRevLett.98.050405,doi:10.1126/science.1257026,Ishii2019, Cassidy2011, Lucas2022, Essler2016} \textcolor{red}{[ZG: add citations.]}. %non-interaction states. 
However, in principle,
GGE only predicts the expectation values of observables, \textcolor{red}{which do not include} the entropy. Note that the former (latter) is linear (nonlinear) in $\rho_A$ and thus commmutes (does not commute) with time average. 
\textcolor{red}{Moreover}, %In addition, 
\textcolor{blue}{we also study the macroscopic scale, which can not be captured by GGE} \textcolor{red}{as well as its recently proposed refined version \cite{Lucas2022} following the interacting counterpart \cite{Ho2022}. In this sense, our study goes well beyond the conventional paradigm of quantum thermalization in integrable systems, pointing out especially the highly nontrivial behaviors on the macroscopic level, where typicality may completely break down.} %compared to GGE.
%The canonical (a)typicality also upper bounds the thermalization rate for finite systems. This can be used to predict the $\mathcal{O}(\frac{1}{N})$ finite size effect in Page curve as well as the correct variance scaling \cite{SM}. %The details are shown in Supplemental Material. 

Finally, let us mention the relation between our %calculation 
strategy
and the quasi-particle picture, which is widely %can also be 
used to calculate %dynamical 
entropy growth \citep{PhysRevLett.127.060404,Jurcevic2014,Castro2016_,Essler2016,Calabrese2005,Fagotti2008,Bertini2018,BertiniB2018_2}. %Alba2018,}.
However, this picture fails to %can not 
\textcolor{red}{reproduce} the %give the exact result for 
dynamical Page curve. Under the %When considering 
periodical boundary condition, the quasi-particle
picture predicts %the subsystem entropy is 
$\overline{S_{A}}=N-\frac{N_{A}^{2}}{N}$
for the Hamiltonian satisfying the conditions in Theorem 2 \cite{SM}. \textcolor{red}{This result is obtained by counting the steady number of entangled pairs shared by $A$ and $\bar A$.} %and the discussions there. 
On the other hand, noting that $(2C_{A}-I_{A})^{2n}\leq(2C_{A}-I_{A})^{2}$,
if we replace all the higher-order \textcolor{red}{terms} of $(2C_{A}-I_{A})$ in Eq. (\ref{eq:Taylor_expansion_for_entropy})
with $(2C_{A}-I_{A})^{2}$, we will get a lower entropy bound, %of the entropy,
which coincides with the prediction %is just the result predicted 
by the quasi-particle picture: $S_{A}\geq N_{A}-\frac{\mathrm{Tr}(2C_{A}-I_{A})^{2}}{\ln2}\sum_{n}\frac{1}{2n(2n-1)}=N_{A}-\frac{N_{A}^{2}}{N}$.
It is reasonable to argue that the quasi-particle picture ignores
possible higher-order correlations beyond %among multiple 
quasi-particle pairs. %which are generated simultaneously. 
%Therefore, the quasi-particle picture can only give a lower bound of the dynamical Page curve.


\emph{Conclusion and outlook.--}%In this letter, 
We have derived the canonical
(a)typicality for the RFG ensemble %This canonical typicality makes the concept of ergodicity and thermalization clearer in free-fermions system.
%What's more, we also 
and pointed out \textcolor{blue}{the quatitative }%\textcolor{red}{the qualitative} %that there is obvious 
scaling difference in atypicality suppression %of the thermalization rate 
from %the interaction 
interacting systems. This %slower thermalization rate which is due to the property of Gaussian ensemble can 
explains the very different behaviors of the Page curves. %, even for some local quantities such as variance, as discovered in \citep{Bianchi2021a}.
We have also explored the relevance to long-time quench dynamics of %also focus on the dynamical Page curve of 
free-fermion systems. %, which is a global property and can not be fully characterized by the canonical typicality. 
To our surprise, some simple time-independent
Hamiltonians are enough to 
%\textcolor{red}{already accommodate a dynamical Page curve that highly resemble the one for the RFG ensemble.} 
make %dynamically emerge 
the \textcolor{red}{free-fermion} Page curve %of RFG ensemble 
emerge to a very high accuracy. 
We %study this problem further and 
analytically prove a necessary and sufficient condition about this behavior. The \textcolor{red}{breakdown} %correctness 
of the quasi-particle picture was also discussed.

Strictly speaking, we define a new %\textcolor{red}{RFG} 
ensemble arising from %covering 
a wide class of free-fermion Hamiltonians, whose dynamical Page curve \textcolor{red}{resembles a lot but yet differs from} %to yet different from 
the fully random one. %for RFG-ensemble. 
%However, there is a tiny difference from any page curve we know so far, such as the fully random Page curve and the Page curve of averaging all eigenstates \citep{Bianchi2021,Bianchi2021a,Vidmar2017}.
The properties of this new ensemlbe and its corresponding Page curve %It is a new kind of Page curve whose connection to others 
merit further study. Another interesting question is if how the %there will be more kinds of 
dynamical Page curves will be enriched upon imposing %regarding 
additional symmetries (such as the \textcolor{red}{Altland-Zirnbauer symmetries \cite{Altland1997}}). %to the Hamiltonians.
Our work proposes a methodology to study this question. Besides, whether
or not the fully random Page curve can emerge exactly for a time-independent free Hamiltonian %can exactly dynamically emerge  
also remains open.

\bibliographystyle{utphys}

\bibliographystyle{apsrev4-2}
\addcontentsline{toc}{section}{\refname}%\nocite{*}
\bibliography{MyCollection}
\clearpage{}

\onecolumngrid
\renewcommand{\thefigure}{S\arabic{figure}}
\setcounter{figure}{0} 
\renewcommand{\thepage}{S\arabic{page}}
\setcounter{page}{1} 
\renewcommand{\theequation}{S.\arabic{equation}}
\setcounter{equation}{0} 
%\renewcommand{\thesection}{S.\Roman{section}} 
\setcounter{section}{0}

\begin{center}
\textbf{\textsc{\LARGE{}Supplementary Information}}{\LARGE\par}
\par\end{center}

\tableofcontents{}

\begin{center}
\textcolor{red}{[ZG: add an abstract summarizing the contents of the Supplemental Materials. Such a summary will also be required when submitting the manuscript to PRL.]}
\end{center}
In this Supplemental Mateiral, we provide detailed proof of Theorem 1 and 2 in the main text. We also provide the calculations of other results and conclusions in the main text and discuss their generalization.

\section{Proof of Canonical Typicality/Atypicality for RFG-ensemble}

In this section, we consider number conserving fermionic
Gaussian ensemble with $N$ modes occupied by $m$ fermions. The
notation follows the main text. In \textcolor{red}{particular, %addition, 
$\langle\cdots\rangle$} is
used to denote the average value over the ensemble. The covariance
matrix of the subsystem $A$ for a particular random Gaussian state
is 
\begin{equation}
C_{A}=\Pi_{A}UC_{0}U^{\dagger}\Pi_{A}^{\dagger},
\end{equation}
\textcolor{red}{where} $\Pi_{A}$ is the projection operator on the the subsystem with size
$N_{A}\times N$, $U$ %can be 
\textcolor{red}{is} taken Haar randomly over $\mathbb{\textcolor{red}{U}}(N)$ \textcolor{red}{and}
$C_{0}$ satisfies $C_{0}^{2}=C_{0}$ and $\Tr C_{0}=m$. In the following,
the distance between two matrices is measured by Hilbert-Schmidt distance
$d_{\mathrm{HS}}$. We define a function $f:\mathbb{\textcolor{red}{U}}(N)\to\mathbb{R}$
as
\begin{equation}
f(U)=d_{\mathrm{HS}}(\Pi_{A}UC_{0}U^{\dagger}\Pi_{A}^{\dagger},\langle C_{A}\rangle).
\end{equation}
It is easy to check that $f$ is Lipschitz continuous with constant
$2$:

\begin{align*}
|f(U_{1})-f(U_{2})| & \leq d_{\mathrm{HS}}(U_{1}C_{0}U_{1}^{\dagger},U_{2}C_{0}U_{2}^{\dagger})\\
 & \leq d_{\mathrm{HS}}(U_{1}C_{0}U_{1}^{\dagger},U_{1}C_{0}U_{2}^{\dagger}) +d_{\mathrm{HS}}(U_{1}C_{0}U_{2}^{\dagger},U_{2}C_{0}U_{2}^{\dagger})\\
 & =\|C_{0}(U_{1}^{\dagger}-U_{2}^{\dagger})\|_{\mathrm{HS}}+\|(U_{1}-U_{2})C_{0}\|_{\mathrm{HS}}\\
 & \leq2d_{\mathrm{HS}}(U_{1,}U_{2}).
\end{align*}
The generalized Levy's lemma \citep{measure_concentration1,measure_concentration2,Meckes2019}
states that for any Lipschitz continuous function over some Riemann
manifolds with positive curvature, its values are concentrated around
the mean one. \textcolor{red}{For the unitary group,} %Formally, 
we have 
\begin{equation}
\mathbb{P}(|f(U)-\langle f(U)\rangle|\geq l\epsilon)\leq2e^{-\frac{N\epsilon^{2}}{12}},
\end{equation}
where $l$ is the Lipschitz constant. In what follows, we will bound $\langle f(U)\rangle=\langle d_{\mathrm{HS}}(\Pi_{A}UC_{0}U^{\dagger}\Pi_{A}^{\dagger},\langle C_{A}\rangle)\rangle$.
First, Since $\langle C_{A}\rangle=\Pi_{A}\int d_{\mathrm{H}}(U)UC_{0}U^{\dagger}\Pi_{A}^{\dagger}$
is invariant under any unitary \textcolor{red}{on $\mathbb{U}(N_A)$}, %group, 
according to Schur's lemma, $\langle C_{A}\rangle=\frac{m}{N}I_{A}$
and
\begin{align*}
\langle\|\Pi_{A}UC_{0}U^{\dagger}\Pi_{A}^{\dagger}-\langle C_{A}\rangle\|_{\mathrm{HS}}\rangle & \leq\sqrt{\langle\|\Pi_{A}UC_{0}U^{\dagger}\Pi_{A}^{\dagger}-\langle C_{A}\rangle\|_{\mathrm{HS}}^{2}\rangle}\\
 & =\sqrt{\langle\mathrm{Tr}(\Pi_{A}UC_{0}U^{\dagger}\Pi_{A}^{\dagger}-\langle C_{A}\rangle)^{2}\rangle}\\
 & =\sqrt{\langle\mathrm{Tr}(\Pi_{A}UC_{0}U^{\dagger}\Pi_{A}^{\dagger})^{2}\rangle-\frac{m^{2}}{N^{2}}N_{A}}.
\end{align*}
Next, we need to calculate $\langle\textcolor{red}{\Tr}(\Pi_{A}UC_{0}U^{\dagger}\Pi_{A}^{\dagger})^{2}\rangle$.
The idea is similar as in \citep{Popescu2006} and originally comes
from random quantum channel coding \citep{PhysRevA.55.1613}: we introduce
another reference space $R'$ which has the same dimension as the
original total system $R$. The following equation holds:
\[
\langle\mathrm{Tr}(\Pi_{A}UC_{0}U^{\dagger}\Pi_{A}^{\dagger})^{2}\rangle=\int d_{H}(U)\mathrm{Tr}[\textcolor{red}{(\Pi_{A}\otimes\Pi_{A})}(UC_{0}U^{\dagger}\textcolor{red}{\otimes}\Pi_{A}UC_{0}U^{\dagger})\mathrm{SWAP}_{RR'}\textcolor{red}{(\Pi_{A}^{\dagger}\otimes\Pi_{A}^{\dagger})}],
\]
where $\mathrm{SWAP}_{RR'}$ is the SWAP operation between the original
system $R$ and the reference one $R'$. From Schur-Weyl duality \citep{Hayashi2017}, we obtain
\begin{equation}
\int d_{H}(U)(UC_{0}U^{\dagger}\textcolor{red}{\otimes} UC_{0}U^{\dagger})\mathrm{SWAP}_{RR'}=\alpha I_{RR'}+\beta\mathrm{SWAP}_{RR'}.
\label{eq:unitary_2_design}
\end{equation}
Now, for simplicity, we can take $C_{0}=\begin{pmatrix}I_{m} & 0\\
0 & 0
\end{pmatrix}$. The following relations hold:
\begin{align*}
\text{Tr}\mathrm{SWAP}_{RR'} & =N,\\
\mathrm{Tr}[(U\textcolor{red}{\otimes} U)(C_{0}\textcolor{red}{\otimes} C_{0}) (U^{\dagger}\textcolor{red}{\otimes} U^{\dagger})\mathrm{SWAP}_{RR'}] & =\mathrm{Tr}[(U\textcolor{red}{\otimes} U)(C_{0}\textcolor{red}{\otimes} C_{0})\mathrm{SWAP}_{RR'} (U^{\dagger}\textcolor{red}{\otimes} U^{\dagger})] =\mathrm{Tr}C_{0}^{2}=m,\\
\mathrm{Tr}[(U\textcolor{red}{\otimes} U) (C_{0}\textcolor{red}{\otimes} C_{0}) (U^{\dagger}\textcolor{red}{\otimes} U^{\dagger})] & =m^{2}.
\end{align*}
The trace of Eq. (\ref{eq:unitary_2_design}) gives $N^{2}\alpha+N\beta=m$.
Multiplying Eq. (\ref{eq:unitary_2_design}) by $\mathrm{SWAP}_{RR'}$
and tracing it, we have $N\alpha+N^{2}\beta=m^{2}$. Solving the equations
leads to $\begin{cases}
\alpha= & \frac{Nm-m^{2}}{N(N^{2}-1)}\\
\beta= & \frac{Nm^{2}-m}{N(N^{2}-1)}
\end{cases}.$ As a result, ~
\begin{align}
\langle\mathrm{Tr}(\Pi_{A}UC_{0}U^{\dagger}\Pi_{A}^{\dagger})^{2}\rangle & =\mathrm{Tr}[(\Pi_{A}\textcolor{red}{\otimes}\Pi_{A})(\alpha I_{RR'}+\beta\mathrm{SWAP}_{RR'})(\Pi_{A}^{\dagger}\textcolor{red}{\otimes}\Pi_{A}^{\dagger}\nonumber)] \\
 & =\alpha N_{A}^{2}+\beta N_{A}.
 \label{eq:value_of_average_of_tracesquare}
\end{align}
Assuming that in the \textcolor{red}{thermodynamic} limit \textcolor{red}{$N\to\infty$}, the density of charge
$\frac{m}{N}$ is fixed \textcolor{red}{as $\mathcal{O}(1)$}, we obtain
\begin{align*}
\langle f(U)\rangle^{2} & \leq\frac{mN_{A}^{2}}{N(N-1)}\sim \mathcal{O}\left(\frac{N_{A}^{2}}{N}\right)
\end{align*}
and the typicality
\begin{equation}
\mathbb{P}\left(d_{\mathrm{HS}}\left(C_{A},\frac{m}{N}I_{A}\right)\geq2\epsilon+\sqrt{\frac{mN_{A}^{2}}{N(N-1)}}\right)\leq2e^{-\frac{N\epsilon^{2}}{12}}.
\label{eq:meassure_concentration_on_covariance_matrix}
\end{equation}

\textcolor{red}{For} the other direction, %if 
we take $f(U)=d_{\mathrm{HS}}^{2}(C_{A},\langle C_{A}\rangle)$,
\textcolor{red}{which} %it 
is also Lipschitz continuous with constant calculated as:
\begin{align}
|f(U_{1})-f(U_{2})| & \leq2(d_{\mathrm{HS}}(\Pi_{A}U_{1}C_{0}U_{1}^{\dagger}\Pi_{A}^{\dagger},\langle C_{A}\rangle)+d_{\mathrm{HS}}(\Pi_{A}U_{2}C_{0}U_{2}^{\dagger}\Pi_{A}^{\dagger},\langle C_{A}\rangle))d_{\mathrm{HS}}(U_{1,}U_{2})\nonumber \\
 & \leq4\sqrt{N_{A}}\left(1-\frac{m}{N}\right)d_{\mathrm{HS}}(U_{1},U_{2}).
\end{align}
Here we assume $m\leq\frac{N}{2}$ due to the particle-hole symmetry (\textcolor{red}{otherwise, we may replace $1-\frac{m}{N}$ by $\frac{m}{N}$}). According to Eq. (\ref{eq:value_of_average_of_tracesquare}), we obtain
\begin{equation}
\langle f(U)\rangle=\langle\mathrm{Tr}(\Pi_{A}UC_{0}U^{\dagger}\Pi_{A}^{\dagger})^{2}\rangle-\frac{m^{2}}{N^{2}}N_{A}\geq\frac{(N-m)mN_{A}^{2}}{N^{2}(N+1)}.
\end{equation}
If $\frac{N_{A}}{N}$ and $\frac{m}{N}$ are both fixed \textcolor{red}{as $\mathcal{O}(1)$} in the \textcolor{red}{thermodynamic} limit, this formula scales linear with $N$. Applying generalized
Levy's lemma leads to 
\begin{equation}
\mathbb{P}\left(d_{\mathrm{HS}}^{2}\left(C_{A},\frac{m}{N}I_{A}\right)\leq\frac{(N-m)mN_{A}^{2}}{N^{2}(N+1)}-4\sqrt{N_{A}}\left(1-\frac{m}{N}\right)\epsilon\right)\leq2e^{-\frac{N}{12}\epsilon^{2}}. 
\label{eq:atypical_expression}
\end{equation}
For example, if we choose $\epsilon\sim\textcolor{red}{\mathcal{O}(N^{\frac{1}{3}})}$,
the above inequaility means that $C_{A}$ will deviate from its
ensemble average by an $\textcolor{red}{\mathcal{O}}(N)$ factor with \textcolor{red}{almost unit} probability. %almost $1$.
This is the atypicality discussed in the main text. 


\section{Some Applications of Measure Concentration Typicality}

\subsection{Measure concentration property for entropy}

In this subsection, we will use Eq. (\ref{eq:meassure_concentration_on_covariance_matrix})
and Eq. (\ref{eq:atypical_expression}) to derive the measure concentration
typicality/atypicality for subsystem entropy. For simplicity the half
filling condition is assumed. The eigenvalues of $C_{A}$ are denoted
as $\{\frac{1}{2}+\lambda_{i}\},i\in\{1,\cdots,N_{A}\}$ with $\sum_{i=1}^{N_{A}}\lambda_{i}^{2}=d_{\mathrm{HS}}(C_{A},\frac{I_{A}}{2})$
and $\lambda_{i}\in[-\frac{1}{2},\frac{1}{2}]$.

If the subsystem is microscopically small, \textcolor{red}{we know} the typicality of entropy
follows by noting that 
\begin{equation}
S_A=\sum_{i=1}^{N_{A}}H\left(\frac{1}{2}+\lambda_{i},\frac{1}{2}-\lambda_{i}\right)=\sum_{i=1}^{N_{A}}\left[1-\sum_{n=1}^{\infty}\frac{(2\lambda_{i})^{2n}}{2n(2n-1)\ln2}\right]\geq N_{A}-4d_{\mathrm{HS}}^{2}\left(C_{A},\frac{I_{A}}{2}\right),
\label{eq:Expansion_of_entropy_by_eigenvalues}
\end{equation}
where we replace $(2\lambda_i)^{2n}$ by $(2\lambda_i)^2$ in the last inequality since $(2\lambda_{i})^{2}\leq1$. \textcolor{red}{Here $H(p_0,p_1)=-p_0\log_2 p_0 - p_1\log_2 p_1$ is the Shannon entropy.} Combined with Eq. (\ref{eq:meassure_concentration_on_covariance_matrix})
we obtain
\begin{equation}
\begin{split}\mathbb{P}(N_{A}-S_A\geq x) & \leq\mathbb{P}\left(d_{\mathrm{HS}}\left(C_{A},\frac{I_{A}}{2}\right)\geq\frac{\sqrt{x}}{2}\right)\\
 & \leq\begin{cases}
2\exp\left[-\frac{N}{48}\left(\frac{\sqrt{x}}{2}-\sqrt{\frac{N_{A}^{2}}{2(N-1)}}\right)^{2}\right], & x>\frac{2N_{A}^{2}}{N-1};\\
1, & x\leq\frac{2N_{A}^{2}}{N-1}.
\end{cases}
\label{eq:typicality_for_subsystem_entropy}
\end{split}
\end{equation}
As long as $N\gg N_{A}^{2}$, we conclude the subsystem entropy will
be nearly maximal. 

%On 
\textcolor{red}{For} the other direction, if $N_{A}$ is macroscopically large, we can \textcolor{red}{upper} %lower 
bound \textcolor{red}{the lhs of} Eq. (\ref{eq:Expansion_of_entropy_by_eigenvalues}) by
\begin{equation}
S_A\leq N_{A}-\frac{2}{\ln2}d_{\mathrm{HS}}^{2}\left(C_{A},\frac{I_{A}}{2}\right).
\end{equation}
Following the atypicality of $d_{\mathrm{HS}}^{2}(C_{A},\frac{I_{A}}{2})$
in Eq.~(\ref{eq:atypical_expression}), the subsystem entropy density
will show an $\mathcal{O}(1)$ deviation from the maximal \textcolor{red}{value}%one
:
\begin{equation}
\begin{split}\mathbb{P}\left(S_A\geq N_{A}-\frac{N_{A}^{2}}{2\ln2(N+1)}+\frac{2}{\ln2}\epsilon\right) & \leq\mathbb{P}\left(d_{\mathrm{HS}}^{2}\left(C_{A},\frac{I_{A}}{2}\right)\leq\frac{N_{A}^{2}}{4(N+1)}-\epsilon\right)\\
 & \leq2e^{-\frac{N}{48N_{A}}\epsilon^{2}}.
\end{split}
\end{equation}
We may take $\epsilon\sim\mathcal{O}(N^{\alpha})$ for arbitrary $\alpha\in(0,1)$,
finding that the majority of subsystem entropy will be comparable
or smaller than $N_{A}-\frac{N_{A}^{2}}{2\ln2(N+1)}$. This clearly
illustrates the difference of the Page curve for the RFG ensemble from
the interacting one.

%\textcolor{red}{[ZG: Add a paragraph showing that the entropy density deviates from the maximal value by an $\mathcal{O}(1)$ value using the atypicality property. Hint: The deviation from the maximal value may be written as a KL divergence (essentially classical since one of them is identity, allowing us to choose the basis arbitrarily), which satisfies the Pinsker's inequality.]}


\subsection{Upper bound on the variance of entropy}

At the end of this section, we will discuss the variance of entropy
in microscopic region. Here the half filling condition is also assumed.
From Eq. (\ref{eq:typicality_for_subsystem_entropy}), we obtain
\[
\mathbb{P}((S_A-N_{A})^{2}\geq x)\leq\begin{cases}
1, & x\leq\frac{4N_{A}^{4}}{(N-1)^{2}};\\
2\exp\left[-\frac{N}{48}\left(\frac{x^{\frac{1}{4}}}{2}-\sqrt{\frac{N_{A}^{2}}{2(N-1)}}\right)^{2}\right], & x>\frac{4N_{A}^{4}}{(N-1)^{2}}.
\end{cases}
\]
Therefore
\begin{align*}
\langle(S_A-N_{A})^{2}\rangle & =\int\mathbb{P}((S_A-N_{A})^{2}\geq x)dx\\
 & \leq\frac{4N_{A}^{4}}{(N-1)^{2}}+2\int_{\frac{4N_{A}^{4}}{(N-1)^{2}}}^{\infty}dx\exp\left[-\frac{N}{48}\left(\frac{x^{\frac{1}{4}}}{2}-\sqrt{\frac{N_{A}^{2}}{2(N-1)}}\right)^{2}\right].
\end{align*}
For the last line, we can change the integral variable \textcolor{red}{into $t=\sqrt{N}\left(\frac{x^{\frac{1}{4}}}{2}-\sqrt{\frac{N_A^2}{2(N-1)}}\right)$, obtaining}
\begin{align*}
 & 2\int_{\frac{4N_{A}^{4}}{(N-1)^{2}}}^{\infty}dx\exp\left[-\frac{N}{48}\left(\frac{x^{\frac{1}{4}}}{2}-\sqrt{\frac{N_{A}^{2}}{2(N-1)}}\right)^{2}\right]
=  \frac{128}{\sqrt{N}}\int_{0}^{\infty}dt\left(\frac{t}{\sqrt{N}}+\sqrt{\frac{N_{A}^{2}}{2(N-1)}}\right){}^{3}e^{-\frac{t^{2}}{48}}\\
= & \frac{128}{N^{2}}\int_{0}^{\infty}dt\left(t+\sqrt{\frac{NN_{A}^{2}}{2(N-1)}}\right)^{3}e^{-\frac{t^{2}}{48}}
\sim  \mathcal{O}\left(\frac{1}{N^{2}}\right),
\end{align*}
%if keeping 
\textcolor{red}{provided that $N_{A}$ is fixed as $\mathcal{O}(1)$}. In conclusion, we obtain
\[
\mathrm{Var}(S_A)\leq\langle(S_A-N_{A})^{2}\rangle\sim\textcolor{red}{\mathcal{O}(N^{-2})}, %\frac{1}{N^{2}},
\]
which agrees with \citep{Bianchi2021,Bianchi2021a}.


\section{Detailed Calculation of the Dynamical %Emergent 
Page Curves}
\textcolor{red}{As mentioned in the main text, for all the models} in this section, we assume the initial state is \textcolor{red}{a period-2} density wave with half filling. Following the same notation in the main text, we further define $X_{A}(t)=2C_{A}(t)-I_{A}$. Thus
\begin{equation}
S_{A}(t)=N_{A}-\sum_{n=1}^{\infty}\frac{\Tr X_{A}^{2n}(t)}{2n(2n-1)\ln2}.
\label{eq:Expanding_of_entropy_with_X}
\end{equation}

\subsection{Calculation for the minimal model}

\textcolor{red}{We} first consider the minimal model. %in this subsection.
Remember that the minimal model means only nearest neighbor hopping
is included. 
 After introducing the Fourier \textcolor{red}{transformed} 
 mode $a_{k}^{\dagger}=\frac{1}{\sqrt{N}}\sum_{j=1}^{N}e^{-ikj}a_{j}^{\dagger}$,
we can easily obtain the correlation function in momentum space: 
\begin{equation}
\Tr[\rho a_{k}^{\dagger}(t)a_{k'}(t)]=\frac{1}{2}\delta_{k,k'}+\frac{1}{2}\delta_{k,k'+\pi}e^{i\theta_{k}(t)},
\end{equation}
where \textcolor{red}{$a_k(t)$ ($a^\dag_{k'}(t)$) is the annihilation (creaiton) operator in the Heisenberg picture,} $\theta_{k}(t)=t(E_{k}-E_{k+\pi})$ and $\rho$ corresponds
to the initial density matrix. In the following, we may \textcolor{red}{omit} %suppress 
the
index $t$ if there is no ambiguity.

After inverse Fourier transformation back to position space, the covariance
matrix for subsystem $A$ reads  $\textcolor{red}{[C_A]}_{m_{1}m_{2}}=\frac{\delta_{m_{1},m_{2}}}{2}+\frac{1}{2N}\sum_{k}e^{i\theta_{k}}e^{ik(m_{1}-m_{2})}e^{i\pi m_{2}}$
and \textcolor{red}{thus}
\begin{equation}
\textcolor{red}{[X_A]}_{m_{1}m_{2}}=\frac{1}{N}\sum_{k}e^{i\theta_{k}}e^{ik(m_{1}-m_{2})}e^{i\pi m_{2}}.
\label{eq:Expression_for_X_in_NNH}
\end{equation}


\subsubsection{Second order in $X_{A}$}

With Eq. (\ref{eq:Expression_for_X_in_NNH}), we can calculate $\overline{\Tr X_{A}^{2}}$ as
\begin{align*}
\overline{\Tr X_{A}^{2}} & =\frac{1}{N^{2}}\sum_{k_{1},k_{2},m_{1},m_{2}}\overline{e^{i\theta_{k_{1}}}e^{ik_{1}(m_{1}-m_{2})}e^{i\pi m_{2}}e^{i\theta_{k_{2}}}e^{ik_{2}(m_{2}-m_{1})}e^{i\pi m_{1}}}\\
 & =\frac{1}{N^{2}}\sum_{k_{1},k_{2},m_{1},m_{2}}(\delta_{k_{1},k_{2}+\pi}e^{ik_{1}(m_{1}-m_{2})}e^{ik_{1}(m_{2}-m_{1})}e^{-i\pi(m_{2}-m_{1})}e^{i\pi(m_{1}+m_{2})}\\
 & +\delta_{k_{1}+k_{2},\pi}e^{i2k_{1}(m_{1}-m_{2})}e^{i\pi(m_{1}+m_{2})}e^{i\pi(m_{2}-m_{1})})\\
 & =\frac{N_{A}^{2}}{N}+\frac{1}{N}\sum_{k,m_{1},m_{2}}e^{i2k(m_{1}-m_{2})}=\frac{N_{A}^{2}}{N}+\frac{1}{N}\sum_{m_{1},m_{2}}(\delta_{m_{1}-m_{2},0}+\delta_{m_{1}-m_{2},\frac{N}{2}}+\delta_{m_{1}-m_{2},-\frac{N}{2}}).
\end{align*}
Noting that the first term \textcolor{red}{in the middle step} comes from $\theta_{k+\pi}=-\theta_{k}$,
which holds for general Hamiltonians according to the definition. However,
the second term is \textcolor{red}{due to the reflection symmetry (which implies $E_k=E_{-k}$) of} %model dependent and only holds for 
the minimal model \textcolor{red}{and is thus not universal}. 
%We call this as occasional symmetry. 
%Luckily, 
\textcolor{red}{Fortunately}, this model dependent term will vanish in the \textcolor{red}{thermodynamic} limit.

This calculation can be diagrammatically represented \textcolor{red}{as shown} in Fig.~\ref{Total_figure_for_FD_1}(a), \textcolor{red}{where we draw an arrow from $m_1$ to $m_2$ ($m_2$ to $m_1$) because there is a factor $e^{ik_1(m_1-m_2)}$ ($e^{ik_2(m_2-m_1)}$). As will become clear below, such a diagrammatic representation provides a convenient and systematic way for dealing with higher order terms.}

\subsubsection{Third order in $X_{A}$}
%Now we are calculating 
\textcolor{red}{We move on to calculate} $\overline{\Tr X_{A}^{3}}$ and \textcolor{red}{will} %to
see %why
it vanishes in \textcolor{red}{thermodynamic} limit. The expression is 
\begin{align*}
\overline{\Tr X_{A}^{3}} & =\frac{1}{N^{3}}\sum_{k_{1,2,3}m_{1,2,3}}\overline{e^{i\theta_{k_{1}}+i\theta_{k_{2}}+i\theta_{k_{3}}}}e^{ik_{1}(m_{1}-m_{2})+ik_{2}(m_{2}-m_{3})+ik_{3}(m_{3}-m_{1})}e^{i\pi(m_{1}+m_{2}+m_{3})}.
\end{align*}

The non-zero contribution of $\overline{e^{i\theta_{k_{1}}+i\theta_{k_{2}}+i\theta_{k_{3}}}}$
in the minimal model only comes from two cases:
\begin{enumerate}
\item $k_{2}=k_{1}+\pi,k_{3}=\pm\frac{\pi}{2}$ and cyclic permutations.
The contribution is proportional to 
\begin{align*}
 & \frac{1}{N^{3}}\sum_{k_{1},m_{1},m_{2},m_{3}}e^{ik_{1}(m_{1}-m_{3})}e^{i\frac{\pi}{2}(m_{1}+m_{3})} =\frac{N_{A}}{N^{2}}\sum_{m_{1},m_{3}}\delta_{m_{1},m_{3}}e^{i\pi m_{1}}\sim \mathcal{O}\left(\frac{1}{N}\right).
\end{align*}
\item $k_{1},k_{2}$ satisfy $|\cos k_{1}+\cos k_{2}|\leq1$ and $k_{3}=\arccos(-\cos k_{1}-\cos k_{2})$.
The contribution is proportional to
\[
\sim\frac{1}{N^{3}}\sum_{k_{1},k_{2}}\sum_{m_{1}}e^{im_{1}(k_{1}-k_{3}+\pi)}\sum_{m_{2}}e^{im_{2}(k_{2}-k_{1}+\pi)}\sum_{m_{3}}e^{im_{3}(k_{3}-k_{2}+\pi)}.
\]
With H\"older inequality \citep{hardy1952inequalities}: $\sum_{i}|a_{i}||b_{i}||c_{i}|\leq[(\sum_{i}|a_{i}|^{3})(\sum_{i}|b_{i}|^{3})(\sum_{i}|c_{i}|^{3})]^{\frac{1}{3}}$,
we can upper bound the above contribution as 
\[
\leq\frac{1}{N^{3}}\{[\sum_{k_{1},k_{2}}|\sum_{m_{1}}e^{im_{1}(k_{1}-k_{3}+\pi)}|^{3}][\sum_{k_{1},k_{2}}|\sum_{m_{2}}e^{im_{2}(k_{2}-k_{1}+\pi)}|^{3}][\sum_{k_{3},k_{2}}|\sum_{m_{3}}e^{im_{3}(k_{3}-k_{2}+\pi)}|^{3}]\}^{\frac{1}{3}}.
\]
Since here, no pair of two $k$ differ by $\pi$ (as it is the case
already considered in the first case), the sum of $m_{1},m_{2},m_{3}$
only contributes to $O(1)$, so the total contribution will be upper
bounded by $O(\frac{1}{N})$.
\end{enumerate}
In conclusion
\[
\overline{\Tr X_{A}^{3}}\sim \mathcal{O}\left(\frac{1}{N}\right)
\]
and \textcolor{red}{thus} vanishes in the \textcolor{red}{thermodynamic} limit. 

The above discussion can be generalized to higher orders: as long
as the degeneracy point of a Hamiltonian is not dense, we can safely
ignore the model dependent \textcolor{red}{contribution} %occasional symmetry 
and put $\overline{e^{i\theta_{k_{1}}}e^{i\theta_{k_{2}}}}=\delta_{k_{1},k_{2}+\pi}$,
\textcolor{red}{which} we call %this 
a contraction. In the following, %content, 
we will directly
use this contraction rule.\footnote{Another proof for general cases can be obtained with the techniques in Subsec.~\ref{subsec:Proof-of-main}.}


\begin{figure*}
\includegraphics[width=1\textwidth]{figs/first_order_and_second_order.png}
\caption{%The 
Feynman diagrams for calculating \textcolor{red}{the entanglement} entropy order by order. Here each
vertex represents a position index and each leg represents a momentum
index. Each leg is associated with $%\frac{1}{N}\sum_{k}
e^{i\theta_{k}}e^{ik(m-m')}e^{i\pi m'}$.
In these diagrams, the legs with same color need to be contracted.
Each color corresponds to one contraction.}
\label{Total_figure_for_FD_1}
\end{figure*}


\subsubsection{Forth Order in $X_{A}$}

Now we are moving to calculate $\overline{\Tr X_{A}^{4}}$:
\begin{equation}
\overline{\Tr X_{A}^{4}}=\frac{1}{N^{4}}\sum_{k_{1,2,3,4},m_{1,2,3,4}}\overline{\prod_{j}^{4}e^{i\theta_{k_{j}}}e^{ik_{j}(m_{j}-m_{j+1})}e^{im_{j}\pi}},
\label{eq:Expression_for_XA_4}
\end{equation}
where $m_{5}=m_{1}$. 

The first contracting class for $\overline{e^{i\theta_{k_{1}}}e^{i\theta_{k_{2}}}e^{i\theta_{k_{3}}}e^{i\theta_{k_{4}}}}$
is to contract \textcolor{red}{the dynamical phase factors in pairs.} %legs $2$ by $2$. 
Two patterns in this class are shown
in Fig.~\ref{Total_figure_for_FD_1}(b) and Fig.~\ref{Total_figure_for_FD_1}(c).
The legs with same colors mean that they are contracted together.
These patterns correspond to $\overline{e^{i\theta_{k_{1}}}e^{i\theta_{k_{4}}}}\;\overline{e^{i\theta_{k_{2}}}e^{i\theta_{k_{3}}}}=\delta_{k_{1},k_{4}+\pi}\delta_{k_{2},k_{3}+\pi}$
and $\overline{e^{i\theta_{k_{1}}}e^{i\theta_{k_{2}}}}\;\overline{e^{i\theta_{k_{4}}}e^{i\theta_{k_{3}}}}=\delta_{k_{1},k_{2}+\pi}\delta_{k_{4},k_{3}+\pi}$,
respectively. Substituting these delta functions to Eq. (\ref{eq:Expression_for_XA_4}),
we obtain 
\begin{equation}
2\times\frac{N_{A}^{2}}{N^{4}}\sum_{k_{1,3},m_{1,3}}e^{ik_{1}(m_{1}-m_{3})}e^{ik_{3}(m_{3}-m_{1})}=\frac{2N_{A}^{3}}{N^{2}},
\end{equation}
where the factor $2$ comes from the equal contribution of \textcolor{red}{these} two diagrams. 

Another pattern from this contracting class is shown in Fig.~\ref{Total_figure_for_FD_1}(d)
which gives $\overline{e^{i\theta_{k_{1}}}e^{i\theta_{k_{3}}}}\;\overline{e^{i\theta_{k_{2}}}e^{i\theta_{k_{4}}}}=\delta_{k_{1},k_{3}+\pi}\delta_{k_{2},k_{4}+\pi}$.
However, substituting this expression to Eq. (\ref{eq:Expression_for_XA_4})
leads to
\begin{equation}
\begin{aligned}\frac{1}{N^{4}}\sum_{k_{1,2},m_{1,2,3,4}}e^{i(k_{1}-k_{2})(m_{1}+m_{3}-m_{2}-m_{4})}e^{i\pi(m_{2}+m_{4})} & =\frac{1}{N^{2}}\sum_{m_{1,2,3,4}}\delta_{m_{1}+m_{3},m_{2}+m_{4}}e^{i\pi(m_{2}+m_{4})} \sim \mathcal{O}\left(\frac{1}{N}\right).
\end{aligned}
\end{equation}
This diagram \textcolor{red}{thus} %will 
vanishes in the \textcolor{red}{thermodynamic} limit. 

\textcolor{red}{It should be emphasized that,} in the above discussion, some \textcolor{red}{terms} %diagrams 
are calculated multiple times.
These forms the other contracting class: we contract the four legs
all together, as shown in Fig.~\ref{Total_figure_for_FD_1}(e). The
contribution from this diagram needs to be subtracted due to the multiple
calculation in Fig.~\ref{Total_figure_for_FD_1}(b) and (c):
\begin{equation}
-\frac{1}{N^{4}}\sum_{k_{1,2,3,4},m_{1,2,3,4}}\delta_{k_{2},k_{1}+\pi}\delta_{k_{3},k_{1}}\delta_{k_{4},k_{1}+\pi}\prod_{j}^{4}e^{ik_{j}(m_{j}-m_{j+1})}e^{im_{j}\pi}=-\frac{N_{A}^{4}}{N^{3}}.
\end{equation}
Combining all the \textcolor{red}{contributions} %calculation 
together, we obtain 
\begin{equation}
\overline{\Tr X_{A}^{4}}=\frac{2N_{A}^{3}}{N^{2}}-\frac{N_{A}^{4}}{N^{3}}.
\end{equation}

From the above discussion, we can see that the contraction rules here
are obviously different from %the 
Wick's theorem. 



\subsection{General Feynman Rules}

\begin{figure*}
\includegraphics[width=0.8\textwidth]{figs/third_order.png}
\caption{In this figure, more Feynman diagrams are shown \textcolor{red}{compared} %complemented 
to Fig.~\ref{Total_figure_for_FD_1}. (a) shows a general Feynman diagram,
(b)-(e) are the Feynman diagrams for calculating $\overline{\Tr X_{A}^{6}}$.}
\label{Total_figure_for_FD_2}
\end{figure*}

%\begin{table}
%\begin{tabular}{|c|c|}
%\hline 
%$2j$ & $a_{2j}$\tabularnewline
%\hline 
%\hline 
%2 & 1\tabularnewline
%\hline 
%4 & -1\tabularnewline
%\hline 
%6 & 4\tabularnewline
%\hline 
%8 & -33\tabularnewline
%\hline 
%10 & 456\tabularnewline
%\hline 
%\end{tabular}

The method presented in the previous subsection allows us to calculate
Eq. (\ref{eq:Expanding_of_entropy_with_X}) to arbitrary orders. Here
we summarize our Feynman rules \textcolor{red}{for contraction}. A general Feynman diagram
for calculating $\overline{\Tr X_{A}^{2n}}$ is shown in Fig. \ref{Total_figure_for_FD_2}(a).
We will use the integer $i$ to label the legs associated with momentum
$k_{i}$. 
\begin{enumerate}
\item All the legs in Fig. \ref{Total_figure_for_FD_2}(a) must be contracted.  Each contraction leads to a delta function of momenta $k$s and has to include even number of legs, where half of the legs should be labeled
as even and the other half as odd. This last requirement \textcolor{red}{arises from the phase factor $e^{i\pi m}$ in each leg and} is to ensure the diagram %will 
not to vanish in the \textcolor{red}{thermodynamic} limit %, as in 
\textcolor{red}{(we recall that Fig.~\ref{Total_figure_for_FD_1}(d) does not contribute as this requirement is not satisfied)}. For totally $2n$ legs with $l$ contractions, the $N,N_{A}$ dependence for this diagram is
$\frac{N_{A}^{2n-l+1}}{N^{2n-l}}$.
\item Each contraction with $2j$ legs should also be assigned a multiple
factor $a_{2j}$, accounting for the multiple calculations. $a_{2}$ and $a_{4}$ are obtained in previous discussion while higher $a_{2j}$ can be obtained iteratively, as shown below.
\item For each diagram, multiply the term $\frac{N_{A}^{2n-l+1}}{N^{2n-l}}$ and factors $a_{2j}$ obtained in Rule 1 and Rule 2 together. Some diagrams also need to multiply by subsystem correction factor $\beta$ (see details below). Sum over all possible diagrams leads to the desired result.
\end{enumerate}
\textcolor{blue}{The subsystem correction factor $\beta$ does not appear in $\overline{\Tr X_{A}^{4}}$ and $\overline{\Tr X_{A}^{2}}$, but will appear in calculating $\overline{\Tr X_{A}^{6}}$. For pedagogical purpose, we will now show how to calculate $\overline{\Tr X_{A}^{6}}$. Some diagrams are shown in Fig. \ref{Total_figure_for_FD_2}(b-e).
In diagram (b), there are three contractions, each contracts two legs.
The contribution for this diagram is $\frac{N_{A}^{4}}{N^{3}}a_{2}^{3}$.
In diagram (c), there are two contractions, one contracts four legs
together and the other contracts two legs. The contribution is $\frac{N_{A}^{5}}{N^{4}}a_{4}a_{2}$.
In diagram (d), all the six legs are contracted together, contributing
to $\frac{N_{A}^{6}}{N^{5}}a_{6}$. What should be attentioned is
the diagram (e). After contracting the three pairs of legs, we obtain
\begin{equation}
\begin{split} & \sum_{k_{1,2,3}m_{1,2,3,4,5,6}}e^{ik_{1}(m_{1}-m_{2}+m_{4}-m_{5})}e^{ik_{2}(m_{2}-m_{3}+m_{5}-m_{6})}e^{ik_{3}(m_{3}-m_{4}+m_{6}-m_{1})}\\
= & N^{3}\sum_{m_{1,2,3,4,5,6}}\delta_{m_{1}+m_{4}=m_{2}+m_{5}=m_{3}+m_{6}\ \mathrm{mod\ }N}.
\end{split}
\end{equation}
Naively, one may conjecture that the result of last sum will be $N_{A}^{4}$.
However, this is only true when $f=\frac{N_{A}}{N}=1$. If $f\leq\frac{1}{2}$,
the sum will be much smaller. After carefully counting the pairs satisfying
the delta function, we obtain for $f\leq\frac{1}{2}$
\begin{equation}
\sum_{k_{1,2,3}m_{1,2,3,4,5,6}}e^{ik_{1}(m_{1}-m_{2}+m_{4}-m_{5})}e^{ik_{2}(m_{2}-m_{3}+m_{5}-m_{6})}e^{ik_{3}(m_{3}-m_{4}+m_{6}-m_{1})}=\frac{1}{2}N^{3}(N_{A}^{4}+N_{A}^{2})\to\beta_{1}N^{3}N_{A}^{4},
\end{equation}
where $\beta_{1}$ is defined as the subsystem correction factor in
thermal dynamical limit: $\beta_{1}=\begin{cases}
\frac{1}{2} & f\leq\frac{1}{2}\\
1 & f=1
\end{cases}$. Summing over all possible diagrams, the total contribution is 
\begin{equation}
\overline{\Tr X_{A}^{6}}=(5+\beta_{1})\frac{N_{A}^{4}}{N^{3}}-(6+3\beta_{2})\frac{N_{A}^{5}}{N^{4}}+\frac{N_{A}^{6}}{N^{5}}a_{6},
\end{equation}
where $\beta_{2}$ is another subsystem correction factor $\beta_{2}=\begin{cases}
\frac{2}{3} & f\leq\frac{1}{2}\\
1 & f=1
\end{cases}$. Here $a_{6}$ can be determined by considering the case when $N_{A}=N$
(i.e. $f=1$). In this case, $\overline{\Tr X_{A}^{6}}=N$, resulting
to $a_{6}=4$. Therefore, 
\begin{equation}
\overline{\Tr X_{A}^{6}}=\frac{11}{2}\frac{N_{A}^{4}}{N^{3}}-8\frac{N_{A}^{5}}{N^{4}}+4\frac{N_{A}^{6}}{N^{5}}\ \ \mathrm{if}\ \ f\leq\frac{1}{2}.
\end{equation}}


\subsection{Proof of %main 
Theorem 2 \label{subsec:Proof-of-main}}

In this subsection, we \textcolor{red}{go} %are going 
beyond the minimal model and consider the general Hamiltonians satisfying the condition in %main 
Theorem 2 \textcolor{red}{in the main text}. We will find the Feynman rule as well as the subsystem entropy
is indeed the same as the previous subsection.

Since the Hamiltonian is period-2, we can use a modified Fourier
transformation to block diagonalize it:
\begin{equation}
A_{k}^{\dagger}=\sqrt{\frac{2}{N}}\sum_{j=1}^{\frac{N}{2}}e^{-ik(2j-1)}a_{2j-1}^{\dagger},\ \ B_k^\dag=\sqrt{\frac{2}{N}}\sum_{j=1}^{\frac{N}{2}}e^{-i2kj}a_{2j}^{\dagger},\;\;%k\in[0,\pi).
\textcolor{red}{k\in\left\{\frac{2n\pi}{N}\right\}^{\frac{N}{2}-1}_{n=0}}.
\label{eq:FT_2_periodic}
\end{equation}
\textcolor{red}{Here} $A_{k}^{\dagger},B_{k}^{\dagger}$ are related to the conserved \textcolor{red}{(eigen)} mode
$P_{k}^{\dagger},Q_{k}^{\dagger}$ via a $2\times2$ unitary transformation
$U^{k}$as $A_{k}^{\dagger}=U_{11}^{k}P_{k}^{\dagger}+U_{12}^{k}Q_{k}^{\dagger}$
and $B_{k}^{\dagger}=U_{21}^{k}P_{k}^{\dagger}+U_{22}^{k}Q_{k}^{\dagger}$.
Substituting into Eq. (\ref{eq:FT_2_periodic}) leads to 
\begin{align}
a_{2m}^{\dagger} & =\sqrt{\frac{2}{N}}\sum_{k=0}^{\pi}e^{ik2m}(U_{21}^{k}P_{k}^{\dagger}+U_{22}^{k}Q_{k}^{\dagger}),\;\;\;\;
a_{2m+1}^{\dagger}  =\sqrt{\frac{2}{N}}\sum_{k=0}^{\pi}e^{ik(2m+1)}(U_{11}^{k}P_{k}^{\dagger}+U_{12}^{k}Q_{k}^{\dagger}).
\label{eq:IFT}
\end{align}
If we define \textcolor{red}{$Q_{k+\pi}=P_{k}$} %$P_{k}=Q_{k+\pi}$ 
and 
\begin{equation}
Z_{k}^{m}=\begin{cases}
\sqrt{2}U_{22}^{k}, & \mathrm{if\ }m\ \mathrm{is\ even\ and}\ k<\pi;\\
\sqrt{2}U_{21}^{k-\pi}, & \mathrm{if\ }m\ \mathrm{is\ even\ and}\ k\geq\pi;\\
\sqrt{2}U_{12}^{k}, & \mathrm{if\ }m\ \mathrm{is\ odd\ and}\ k<\pi;\\
-\sqrt{2}U_{11}^{k-\pi}, & \mathrm{if\ }m\ \mathrm{is\ odd\ and}\ k\geq\pi,
\end{cases}
\label{eq:Zkm}
\end{equation}
the above inverse Fourier transformation \textcolor{red}{(\ref{eq:IFT})} can be rewritten as
\[
a_{m}^{\dagger}=\frac{1}{\sqrt{N}}\sum_{k=0}^{2\pi}Z_{k}^{m}Q_{k}^{\dagger}e^{ikm}.
\]
In the following, we will \textcolor{red}{simplify} %denote 
$\sum_{k=0}^{2\pi}$ as $\sum_{k}$
and $k$ should be understood as module $2\pi$. By assumption, all
the conserved quantity $\Tr(\rho Q_{k}^{\dagger}Q_{k})=\frac{1}{2}$
for $k\in[0,2\pi)$, thus
\begin{align*}
\textcolor{red}{[C_A]}_{ml} & =\frac{1}{2N}\sum_{k}Z_{k}^{m}Z_{k}^{l*}e^{ik(m-l)}+\frac{1}{2N}\sum_{k}e^{i\theta_{k}}Z_{k}^{m}Z_{k+\pi}^{l*}e^{ik(m-l)}e^{i\pi l}\\
 & =\frac{1}{2}\delta_{m,l}+\frac{1}{2N}\sum_{k}e^{i\theta_{k}}Z_{k}^{m}Z_{k+\pi}^{l*}e^{ik(m-l)}e^{i\pi l}
\end{align*}
where $\theta_{k+\pi}=-\theta_{k}$ by definition. In the last equality, we have
used the unitarity of $U^{k}$. Namely, if $m-l$ is odd:
\begin{align*}
\frac{1}{2N}\sum_{k}Z_{k}^{m}Z_{k}^{l*}e^{ik(m-l)} & =\frac{1}{4N}\sum_{k}[Z_{k}^{m}Z_{k}^{l*}e^{ik(m-l)}+Z_{k+\pi}^{m}Z_{k+\pi}^{l*}e^{i(k+\pi)(m-l)}]\\
 & =\frac{1}{4N}\sum_{k}(Z_{k}^{m}Z_{k}^{l*}-Z_{k+\pi}^{m}Z_{k+\pi}^{l*})e^{ik(m-l)}=0.
\end{align*}
A similar calculation \textcolor{red}{can be carried out for the case in which} %holds for 
$m-l$ is even. Therefore, we %still
have 
\[
\textcolor{red}{[X_A]}_{ml}=\frac{1}{N}\sum_{k}e^{i\theta_{k}}Z_{k}^{m}Z_{k+\pi}^{l*}e^{ik(m-l)}e^{i\pi l}.
\]
This expression is very similar to Eq. (\ref{eq:Expression_for_X_in_NNH})
except for the extra \textcolor{red}{factors} %terms 
$Z_{k}$. Nonetheless, we will show those extra
$Z_{k}$'s do %will 
not contribute %to 
\textcolor{red}{in the thermodynamic} limit. As a result,
the same Feynman rules and \textcolor{red}{dynamical Page curve} %entropy 
follows. 

\textcolor{red}{When evaluating a Feynman diagram} in the \textcolor{red}{thermodynamic} limit with %when 
$N_{A}$, $N$ both going to infinity,
we can first sum over the position indices. Introducing $s_{j}=k_{j}-k_{j-1}$, $k_{0}=k_{2n}$
and $m_{2n+1}=m_{1}$, we obtain
\begin{align*}
 & \sum_{m_{1,2\cdots2n}}\prod_{j=1}^{2n}e^{ik_{j}(m_{j}-m_{j+1})}e^{i\pi m_{j}}Z_{k_{j}}^{m_{j}}Z_{k_{j-1}+\pi}^{m_{j}*}\\
 & =\prod_{j=1}^{2n}e^{im_{j}s_{j}}e^{i\pi m_{j}}Z_{k_{j}}^{m_{j}}Z_{k_{j-1}+\pi}^{m_{j}*}\\
 & =\prod_{j=1}^{2n}\{\sum_{m_{j}:\mathrm{even}}e^{im_{j}(s_{j}+\pi)}Z_{k_{j}}^{0}Z_{k_{j-1}+\pi}^{0*}+\sum_{m_{j}:\mathrm{odd}}e^{im_{j}(s_{j}+\pi)}Z_{k_{j}}^{1}Z_{k_{j-1}+\pi}^{1*}\}\\
 & =\prod_{j=1}^{2n}\{\frac{1-e^{iN_{A}(s_{j}+\pi)}}{1-e^{2i(s_{j}+\pi)}}\}\prod_{j=1}^{2n}(Z_{k_{j}}^{0}Z_{k_{j-1}+\pi}^{0*}+e^{i(s_{j}+\pi)}Z_{k_{j}}^{1}Z_{k_{j-1}+\pi}^{1*})\\
 & =\frac{e^{i\frac{N_{A}}{2}\sum_{j=1}^{2n}(s_{j}+\pi)}}{e^{i\sum_{j=1}^{2n}(s_{j}+\pi)}}\prod_{j=1}^{2n}\{\frac{\sin\frac{N_{A}(s_{j}+\pi)}{2}}{\sin(s_{j}+\pi)}\}\prod_{j=1}^{2n}(Z_{k_{j}}^{0}Z_{k_{j-1}+\pi}^{0*}+e^{i(s_{j}+\pi)}Z_{k_{j}}^{1}Z_{k_{j-1}+\pi}^{1*})\\
 & =\prod_{j=1}^{2n}\{\frac{\sin\frac{N_{A}(s_{j}+\pi)}{2}}{\sin(s_{j}+\pi)}\}\prod_{j=1}^{2n}(Z_{k_{j}}^{0}Z_{k_{j-1}+\pi}^{0*}+e^{i(s_{j}+\pi)}Z_{k_{j}}^{1}Z_{k_{j-1}+\pi}^{1*})
\end{align*}
In the above calculation, we have assumed $N_{A}$ to be even for simplicity. We also emphasize that $s_{1}\cdots s_{2n}$ is not independent since $\sum_{i=1}^{2n}s_{i}=0$.
The factor $\frac{\sin\frac{N_{A}(s_{j}+\pi)}{2}}{\sin(s_{j}+\pi)}$
will be dominated by the contribution from $s_{j}=0$ or $s_{j}=\pi$
if $Z_{k}$ is smooth enough. This is due to the convergence of Fourier
series, see \citep{stein2003fourier}. In \citep{Kress1998}
the difference between the discrete sum over momenta and the integration is also upper bounded. However, if $s_{j}\textcolor{red}{\simeq}%\approx
0$, it will lead to 
\begin{equation}
Z_{k_{j}}^{0}Z_{k_{j-1}+\pi}^{0*}+e^{i(s_{j}+\pi)}Z_{k_{j}}^{1}Z_{k_{j-1}+\pi}^{1*}\textcolor{red}{\simeq}%=
Z_{k_{j}}^{0}Z_{k_{j}+\pi}^{0*}-Z_{k_{j}}^{1}Z_{k_{j}+\pi}^{1*}=0
\end{equation}
due to the unitarity of $U$. In \textcolor{red}{the end}, %conclusion, 
we obtain 
\begin{align}
\sum_{m_{1,2\cdots2n}}\prod_{j=1}^{2n}e^{ik_{j}(m_{j}-m_{j+1})}e^{i\pi m_{j}}Z_{k_{j}}^{m_{j}}Z_{k_{j-1}+\pi}^{m_{j}*} & \simeq(Z_{k}^{0}Z_{k}^{0*}+Z_{k}^{1}Z_{k}^{1*})^{2n}\prod_{j=1}^{2n}\{\frac{\sin\frac{N_{A}(s_{j}+\pi)}{2}}{\sin(s_{j}+\pi)}\}\,s_{j}\neq 0\nonumber \\
 & =2^{2n}\prod_{j=1}^{2n}\{\frac{\sin\frac{N_{A}(s_{j}+\pi)}{2}}{\sin(s_{j}+\pi)}\}\ s_{j}\neq 0\label{eq:Summing_Over_position}
\end{align}
Now we can see in the final expression Eq. (\ref{eq:Summing_Over_position}) that the model-dependent \textcolor{red}{factor} %term 
$Z$ disappears. Therefore, those Hamiltonians
satisfying the conditions in %main 
Theorem 2 \textcolor{red}{in the main text} will have the same Feynman
rules and \textcolor{red}{dynamical Page curve} %entropy 
as the minimal model.


In Fig. \ref{Taylor_completed_range3},
we plotted the dynamical Page curve for the Hamiltonian 
\begin{equation}
H=\sum_{i}a_{i}^{\dagger}a_{i+1}+0.3\sum_{i:\mathrm{even}}a_{i}^{\dagger}a_{i+3}-0.3\sum_{i:\mathrm{odd}}a_{i}^{\dagger}a_{i+3}+\mathrm{H.C.}.\label{eq:period2_range2_hamiltonian}
\end{equation}
This dynamical Page curve is nearly the same as the one of minimal
model in the main text.

\begin{figure}
\includegraphics[width=0.8\textwidth]{figs/period2range3Taylor_complete}

\caption{The dynamical Page curve for Hamiltonian (\ref{eq:period2_range2_hamiltonian})
(blue curve) and its comparison with the one for RFG-ensemble (red
curve) and the theoretical result (green curve). Here $N=200$. The
theoretical result is truncated up to order $\mathcal{O}(f^{4})$,
the same as in the main text. }

\label{Taylor_completed_range3}
\end{figure}

\subsection{Generalization to the %for 
atypical Page curves}

In this subsection, we %will 
further consider the case beyond the condition
in %main 
Theorem 2 \textcolor{red}{in the main text}, namely $\Tr(\rho Q_{k}^{\dagger}Q_{k})\neq\frac{1}{2}$
(\textcolor{red}{we recall that $Q_{k+\pi}=P_{k}$}). We denote $\textcolor{red}{n_{k}=}\Tr(\rho Q_{k}^{\dagger}Q_{k})$
and $\eta_{k}=\sqrt{n_{k}(1-n_{k})}$. Following the half filling condition,
we have 
\[
n_{k+\pi}=1-n_{k},\;\;\;\;\eta_{k+\pi}=\eta_{k}
\]
and still $a_{m}^{\dagger}=\frac{1}{\sqrt{N}}\sum_{k=0}^{2\pi}Z_{k}^{m}Q_{k}^{\dagger}e^{ikm}$
with $Z^m_{k}$ defined in \textcolor{red}{Eq.~(\ref{eq:Zkm})}. %previous subsection. 
The covariance matrix
can be calculated as
\begin{align*}
\textcolor{red}{[C_A]}_{ml} & =\frac{1}{N}\sum_{k_{1,2}}Z_{k_{1}}^{m}Z_{k_{2}}^{l*}e^{ik_{1}m}e^{-ik_{2}l}\Tr(\rho Q_{k_{1}}^{\dagger}Q_{k_{2}})\\
 & =\frac{1}{N}\sum_{k}Z_{k}^{m}Z_{k}^{l*}e^{ik(m-l)}n_{k}+\frac{1}{N}\sum_{k}e^{i\theta_{k}}Z_{k}^{m}Z_{k+\pi}^{l*}e^{ik(m-l)}e^{i\pi l}\eta_{k}.
\end{align*}
Since $n_{k}\neq\frac{1}{2}$, in general there is no simple expression
for $X_{A}$. 

Using the same techniques as in the previous subsection, we can still
establish the Feynman rules for this case. However, we have to distinguish
two kinds of legs, one \textcolor{red}{like} %represents 
$Z_{k_{j}}^{m_{j}}Z_{k_{j}}^{m_{j+1}*}e^{ik_{j}(m_{j}-m_{j+1})}n_{k_{j}}$
and the other \textcolor{red}{like} %represents 
$e^{i\theta_{k_{j}}}Z_{k_{j}}^{m_{j}}Z_{k_{j}+\pi}^{m_{j+1}*}e^{ik_{j}(m_{j}-m_{j+1})}e^{i\pi m_{j+1}}\eta_{k_{j}}$.
There is no dynamical phase $e^{i\theta_{k}}$ in the former, namely no delta functions associated with contraction. Also, there is no extra $e^{i\pi m_{j+1}}$ phase term in the former.
Due to the difference between these two kinds of legs, the rule is more complicated than the previous case. 

As an example, we can calculate the first three non-trivial terms to obtain:
\begin{align*}
\overline{\Tr C_{A}^{2}} & =\frac{N_{A}}{N}\sum_{k}n_{k}^{2}+\frac{N_{A}^{2}}{N^{2}}\sum_{k}\eta_{k}^{2},
\end{align*}
\[
\overline{\Tr C_{A}^{3}}=\frac{N_{A}}{N}\sum_{k}n_{k}^{3}+\frac{3N_{A}^{2}}{N^{2}}\sum_{k}n_{k}\eta_{k}^{2},
\]
\[
\overline{\Tr C_{A}^{4}}=\frac{N_{A}}{N}\sum_{k}n_{k}^{4}+4\frac{N_{A}^{2}}{N^{2}}\sum_{k}n_{k}^{2}\eta_{k}^{2}+2\frac{N_{A}^{2}}{N^{2}}\eta_{k}^{4}+\frac{2N_{A}^{3}}{N^{3}}\sum_{k}\eta_{k}^{4}-\frac{N_{A}^{4}}{N^{4}}\sum_{k}\eta_{k}^{4}.
\]
\textcolor{red}{Therefore,}  Up to $\overline{\Tr X_{A}^{4}}$, \footnote{Noting that in higher $n-$expansion of $\overline{\Tr X_{A}^{2n}}$, there
will also be contributions to the entropy density $\frac{\overline{S_A}}{N}$ at the order of $\mathcal{O}(f)$, like the term $\frac{N_{A}}{N}\frac{\sum_k n_{k}^{2n}}{N}$. Nonetheless, the convergence of expansion is guaranteed by $X_A^{2n}\leq X_A^{2n-2}$}
\[\overline{S_{A}}\simeq\frac{N_{A}(\ln2+\frac{3}{4})-\frac{N_{A}}{N}\sum_{k}(4n_{k}^{2}-\frac{8}{3}n_{k}^{3}+\frac{4}{3}n_{k}^{4})-\frac{N_{A}^{2}}{N^{2}}\sum_{k}(4\eta_{k}^{2}-8n_{k}\eta_{k}^{2}+\frac{16}{3}n_{k}^{2}\eta_{k}^{2}+\frac{8}{3}\eta_{k}^{4})-\frac{N_{A}^{3}}{N^{3}}\frac{8}{3}\textcolor{red}{\sum_{k}}\eta_{k}^{4}}{\ln2}.
\]

\section{Calculation of \textcolor{red}{Entanglement} Entropy in the Quasi-Particle Picture}

\textcolor{red}{In the quasi-particle picture, a nonequilibrium} %pre-quenched 
initial state is a source for generating quasi-particles
with opposite momenta, which travel ballistically through the system.
\textcolor{red}{Here} the main assumption is %quasi particle picture 
 those \textcolor{red}{quasi-particle} pairs  \textcolor{red}{generated} %generating
at different locations and times are incoherent. Therefore, the \textcolor{red}{entanglement} entropy
of subsystem $A$ is proportional to the number of pairs shared between
$A$ and its complement. Without loss of generality, we can assume
the subsystem $A$ is located in $[0,N_{A})$, $N_{A}\leq\frac{N}{2}$.
For a certain type of pairs with velocity $\pm v(k)$ ($v(k)>0$),
if the right-end of the pair is at position $0\leq x<N_{A}$, i.e.
within subsystem $A$, only when $x$ satisfies
\[
N_{A}-N\leq x-2v(k)t<0
\]
can this pair contribute to the \textcolor{red}{entanglement} entropy of $A$. Here the periodic
boundary condition is taken into account and $2v(k)t$ should be understood
as modulo $N$. The solutions of this inequality is a continuous range
$x\in[x_{\mathrm{min}},x_{\mathrm{max}})$, where $x_{\mathrm{min}}=\max\{0,2v(k)t+N_{A}-N\}$ \textcolor{red}{and} 
$x_{\mathrm{max}}=\min\{N_{A},2v(k)t\}$. \textcolor{red}{Accordingly}, we obtain
\[
\Delta x=x_{\mathrm{max}}-x_{\mathrm{min}}=\begin{cases}
2v(k)t, & 2v(k)t\leq N_{A};\\
N_{A}, & N_{A}<2v(k)t<N-N_{A};\\
N-2v(k)t, & 2v(k)t\geq N-N_{A}.
\end{cases}
\]
%The 
A similar argument holds if the left-end is in subsystem $A$. We
assume \textcolor{red}{that after a sufficiently long} %for long enough 
time, the quasi-particle pairs will distribute
uniformly among the system. \textcolor{red}{Hence,} %Thus, 
the contribution of %that kind of
quasi-particle pairs \textcolor{red}{with momentum $k$} to the \textcolor{red}{entanglement} entropy \textcolor{red}{upon the long-time average is given by} %after time average is
\[
%S_{A}=
\frac{S_{A}(k)}{N^{2}}\int_{0}^{N}d(2v(k)t)\Delta x=S_{A}(k)\left[\frac{N_{A}}{N}-\left(\frac{N_{A}}{N}\right)^{2}\right],
\]
\textcolor{red}{where the coefficient $S_A(k)$ is to be determined}. Summing over all types of pairs, we obtain 
\[
S^{\textcolor{red}{\rm qp}}_{A}=\left(\frac{N_{A}}{N}-\frac{N_{A}^{2}}{N^{2}}\right)\sum_{k}S_{A}(k).
\]
If $N_{A}\to0$, the limit $S_{A}^\mathrm{qp}\to N_{A}\sum_{k}\frac{H(n_{k})}{N}$
should hold \citep{Alba2018}. Therefore, $S_{A}(k)=H(n_{k})$. If
$n_{k}=\frac{1}{2}$ for all $k$s, the \textcolor{red}{entanglement} entropy for subsystem
$A$ is
\[
S^{\textcolor{red}{\rm qp}}_{A}=N_{A}-\frac{N_{A}^{2}}{N},
\]
which \textcolor{red}{deviates considerably from} %is in stark contrast with 
the dynamical Page curve %as 
discussed in the main text.

\end{document}