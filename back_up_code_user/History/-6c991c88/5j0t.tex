\documentclass[aps,onecolumn,superscriptaddress,notitlepage,longbibliography]{revtex4-1}
\usepackage{times}
\usepackage{graphicx}
\usepackage{feynmf}
\usepackage{tabularx}
\usepackage{amsmath}
\usepackage{amstext}
\usepackage{amssymb}
\usepackage{xfrac}
\usepackage[colorlinks,citecolor=blue]{hyperref}
\usepackage{graphicx}
\usepackage{amsmath}
\usepackage{amstext}
\usepackage{amssymb}
\usepackage{amsfonts}
\usepackage{longtable,booktabs}
\usepackage{hyperref}
\usepackage{url}
\usepackage{subfigure}
\usepackage{dsfont}
\usepackage{booktabs}
\usepackage{amsbsy}
\usepackage{dcolumn}
\usepackage{amsthm}
\usepackage{bm}
\usepackage{esint}
\usepackage{multirow}
\usepackage{hyperref}
\usepackage{cleveref}
\usepackage{mathrsfs}
\usepackage{amsfonts}
\usepackage{amsbsy}
\usepackage{dcolumn}
\usepackage{bm}
\usepackage{multirow}
\usepackage{color}
\usepackage{extarrows}
\usepackage{datetime}
\usepackage{comment}
\usepackage[super]{nth}
%%%%%%%%%% Start TeXmacs macros
\catcode`\<=\active \def<{
\fontencoding{T1}\selectfont\symbol{60}\fontencoding{\encodingdefault}}
\catcode`\>=\active \def>{
\fontencoding{T1}\selectfont\symbol{62}\fontencoding{\encodingdefault}}
\newcommand{\tmmathbf}[1]{\ensuremath{\boldsymbol{#1}}}
\newcommand{\tmop}[1]{\ensuremath{\operatorname{#1}}}
\newcommand{\tmfloatcontents}{}
\newlength{\tmfloatwidth}
\newcommand{\tmfloat}[5]{
  \renewcommand{\tmfloatcontents}{#4}
  \setlength{\tmfloatwidth}{\widthof{\tmfloatcontents}+1in}
  \ifthenelse{\equal{#2}{small}}
    {\setlength{\tmfloatwidth}{0.45\linewidth}}
    {\setlength{\tmfloatwidth}{\linewidth}}
  \begin{minipage}[#1]{\tmfloatwidth}
    \begin{center}
      \tmfloatcontents
      \captionof{#3}{#5}
    \end{center}
  \end{minipage}}
%%%%%%%%%% End TeXmacs macros

\begin{document}

\title{Supplementary Points}

\maketitle

\section{Phase Difference between Contours}

Previously we take the mean-field approximation
\begin{equation}
  a_{0 +}^{\dagger} = a_{0 -}^{\dagger} = a_{0 +} = a_{0 -} \approx \sqrt{N} .
\end{equation}
However, if the coupling between two contours is introduced, there may exist a
phase different between the two contours. Hence, we need to assume
\begin{equation}
  a_{0 -} = \sqrt{N} e^{i \theta_-}, a_{0 -}^{\dagger} = \sqrt{N} e^{- i
  \theta_-}, a_{0 +} = \sqrt{N} e^{i \theta_+}, a_{0 +}^{\dagger} = \sqrt{N}
  e^{- i \theta_+} .
\end{equation}
Hence, the mean-field action can be rewritten as
\begin{eqnarray}
  S & = & \int_{- \infty}^{\infty} d t \left[ \sum_{\tmmathbf{k}}
  (a^{\dagger}_{\tmmathbf{k}+} i \partial_t a_{\tmmathbf{k}+} -
  a^{\dagger}_{\tmmathbf{k}-} i \partial_t a_{\tmmathbf{k}-}) - H_+' + H'_- -
  4 i \gamma n \sum_{\tmmathbf{k}} a_{\tmmathbf{k}-}^{\dagger}
  a_{\tmmathbf{k}+} e^{i (\theta_+ - \theta_-)} \right] \nonumber\\
  &  & + \int_{- \infty}^{\infty} d t [i \gamma n N (1 - e^{2 i (\theta_+ -
  \theta_-)})], 
\end{eqnarray}
where
\begin{equation}
  H_{\alpha}' = \frac{U_R n}{2} N + \sum_{\tmmathbf{k}} \left[
  (\varepsilon_{\tmmathbf{k}} + U_R n - 2 i \alpha \gamma n) a_{\tmmathbf{k}
  \alpha}^{\dagger} a_{\tmmathbf{k} \alpha} + \frac{U^{\ast} n}{2}
  a_{-\tmmathbf{k} \alpha} a_{\tmmathbf{k} \alpha} e^{- 2 i \theta_{\alpha}} +
  \frac{U n}{2} a_{\tmmathbf{k} \alpha}^{\dagger} a_{-\tmmathbf{k}
  \alpha}^{\dagger} e^{2 i \theta_{\alpha}} \right] .
\end{equation}
To remove the phases in $H_{\alpha}$, we substitute the fields
$a_{\tmmathbf{k} \alpha}$ with $a_{\tmmathbf{k} \alpha} e^{i
\theta_{\alpha}}$. Then the action becomes
\begin{eqnarray}
  S & = & \int_{- \infty}^{\infty} d t \left[ \sum_{\tmmathbf{k}}
  (a^{\dagger}_{\tmmathbf{k}+} i \partial_t a_{\tmmathbf{k}+} -
  a^{\dagger}_{\tmmathbf{k}-} i \partial_t a_{\tmmathbf{k}-}) - H_+ + H_- - 4
  i \gamma n \sum_{\tmmathbf{k}} a_{\tmmathbf{k}-}^{\dagger} a_{\tmmathbf{k}+}
  e^{2 i (\theta_+ - \theta_-)} \right] \nonumber\\
  &  & + \int_{- \infty}^{\infty} d t [i \gamma n N (1 - e^{2 i (\theta_+ -
  \theta_-)})] . 
\end{eqnarray}
If there is no dissipation, we can always fix the gauge $\theta_+ - \theta_- =
0$. When the dissipation is exhibited, we can assume that $\theta_+ -
\theta_-$ is an analytic function of the dissipation $\gamma$. Hence, we can
see the difference of the phase brings a correction with the order $O
(\gamma^2 n^2)$, which is ignorable. In the following calculation we ignore
this phase difference.

We show this point in another way. The Heisenberg equation of the operator
$a_{\tmmathbf{k}}$ is given by
\begin{eqnarray}
  \frac{d a_{\tmmathbf{k}}}{d t} & = & - i [a_{\tmmathbf{k}}, H] -
  \frac{\gamma}{2 V} \sum_{\tmmathbf{p}, \tmmathbf{q}, \tmmathbf{l}} \{
  a_{\tmmathbf{k}}, a_{\tmmathbf{l}}^{\dagger} a^{\dagger}_{\tmmathbf{p}}
  a_{\tmmathbf{p -} \tmmathbf{q}} a_{\tmmathbf{l +} \tmmathbf{q}} \} +
  \frac{\gamma}{V} \sum_{\tmmathbf{p}, \tmmathbf{q}, \tmmathbf{l}}
  a_{\tmmathbf{l}}^{\dagger} a^{\dagger}_{\tmmathbf{p}} a_{\tmmathbf{k}}
  a_{\tmmathbf{p -} \tmmathbf{q}} a_{\tmmathbf{l +} \tmmathbf{q}} \nonumber\\
  & = & - i \varepsilon_{\tmmathbf{k}} a_{\tmmathbf{k}} - i \frac{U_R - i
  \gamma}{2 V} \sum_{\tmmathbf{p}, \tmmathbf{q}} (a^{\dagger}_{\tmmathbf{p}}
  a_{\tmmathbf{p -} \tmmathbf{q}} a_{\tmmathbf{k} \tmmathbf{+} \tmmathbf{q}} +
  a^{\dagger}_{\tmmathbf{p}} a_{\tmmathbf{k} \tmmathbf{-} \tmmathbf{q}}
  a_{\tmmathbf{p} \tmmathbf{+} \tmmathbf{q}}) \nonumber\\
  & = & - i \varepsilon_{\tmmathbf{k}} a_{\tmmathbf{k}} - i \frac{U_R - i
  \gamma}{V} \sum_{\tmmathbf{p}, \tmmathbf{q}} a^{\dagger}_{\tmmathbf{p}}
  a_{\tmmathbf{p -} \tmmathbf{q}} a_{\tmmathbf{k} \tmmathbf{+} \tmmathbf{q}} .
\end{eqnarray}
By definition of the phase of the condensate, we can remove the phase by the
transformation
\begin{equation}
  a_{\tmmathbf{k}} \rightarrow a_{\tmmathbf{k}} e^{i \theta} .
\end{equation}

\section{Green's Function}

Here we show the Green's function of the dissipative superfluids. The
correlation-function matrix takes the form of
\begin{eqnarray}
  G & = & \left\langle \left(\begin{array}{c}
    a_{k \tmmathbf{, +}}\\
    a^{\dagger}_{- k \tmmathbf{, +}}\\
    a_{k, -}\\
    a^{\dagger}_{- k, -}
  \end{array}\right) \left(\begin{array}{cccc}
    a^{\dagger}_{k, +} & a_{- k, +} & a^{\dagger}_{k, -} & a_{- k, -}
  \end{array}\right) \right\rangle \nonumber\\
  & = & \left(\begin{array}{cccc}
    - \frac{\varepsilon_{\tmmathbf{k}} + U_R n - 2 i \gamma n - \omega}{2} & -
    \frac{U n}{2} &  & \\
    - \frac{U^{\ast} n}{2} & - \frac{\varepsilon_{\tmmathbf{k}} + U_R n - 2 i
    \gamma n + \omega}{2} &  & - 2 i \gamma n\\
    - 2 i \gamma n &  & \frac{\varepsilon_{\tmmathbf{k}} + U_R n + 2 i \gamma
    n - \omega}{2} & \frac{U n}{2}\\
    &  & \frac{U^{\ast} n}{2} & \frac{\varepsilon_{\tmmathbf{k}} + U_R n + 2
    i \gamma n + \omega}{2}
  \end{array}\right)^{- 1} . 
\end{eqnarray}
By directly deriving the inverse matrix, we have the following Green's
function:
\begin{eqnarray}
  G^T & \equiv & \langle a_{k, +} a^{\dagger}_{k, +} \rangle \nonumber\\
  & = & \frac{2 (\omega + U_R n - 2 i \gamma n + \varepsilon_{\tmmathbf{k}})
  (\omega^2 - \varepsilon_{\tmmathbf{k}}^2 - 2 \varepsilon_{\tmmathbf{k}} (U_R
  n + 2 i \gamma n) + \gamma n (5 \gamma n - 4 i U_R n))}{\omega^4 + 2 (5
  \gamma^2 n^2 - \varepsilon_{\tmmathbf{k}} (\varepsilon_{\tmmathbf{k}} + 2
  U_R n)) \omega^2 + [\varepsilon_{\tmmathbf{k}} (\varepsilon_{\tmmathbf{k}} +
  2 U_R n) + 3 \gamma^2 n^2]^2}, \\
  G^{<} & \equiv & \langle a_{k, +} a^{\dagger}_{k, -} \rangle \nonumber\\
  & = & \frac{- 8 i \gamma n (U_R^2 n^2 + \gamma^2 n^2)}{\omega^4 + 2 (5
  \gamma^2 n^2 - \varepsilon_{\tmmathbf{k}} (\varepsilon_{\tmmathbf{k}} + 2
  U_R n)) \omega^2 + [\varepsilon_{\tmmathbf{k}} (\varepsilon_{\tmmathbf{k}} +
  2 U_R n) + 3 \gamma^2 n^2]^2}, \\
  G^{>} & = & \langle a_{k, -} a_{k, +}^{\dagger} \rangle \nonumber\\
  & = & \frac{- 8 i \gamma n ((\omega + \varepsilon_{\tmmathbf{k}} + U_R n)^2
  + 4 \gamma^2 n^2)}{\omega^4 + 2 (5 \gamma^2 n^2 - \varepsilon_{\tmmathbf{k}}
  (\varepsilon_{\tmmathbf{k}} + 2 U_R n)) \omega^2 +
  [\varepsilon_{\tmmathbf{k}} (\varepsilon_{\tmmathbf{k}} + 2 U_R n) + 3
  \gamma^2 n^2]^2}, \\
  G^{\tilde{T}} & = & \langle a_{k, -} a_{k, -}^{\dagger} \rangle \nonumber\\
  & = & \frac{2 (\omega + U_R n + 2 i \gamma n + \varepsilon_{\tmmathbf{k}})
  (- \omega^2 + \varepsilon_{\tmmathbf{k}}^2 + 2 \varepsilon_{\tmmathbf{k}}
  (U_R n - 2 i \gamma n) - \gamma n (5 \gamma n + 4 i U_R n))}{\omega^4 + 2 (5
  \gamma^2 n^2 - \varepsilon_{\tmmathbf{k}} (\varepsilon_{\tmmathbf{k}} + 2
  U_R n)) \omega^2 + [\varepsilon_{\tmmathbf{k}} (\varepsilon_{\tmmathbf{k}} +
  2 U_R n) + 3 \gamma^2 n^2]^2} . 
\end{eqnarray}
From these expression we can see the fundamental relation: $G^T = -
(G^{\tilde{T}})^{\ast}$. To observe the spectrum function, we need first
investigate the retarded Green's function, which is given by the relation
\begin{eqnarray}
  \left(\begin{array}{cc}
    G^K & G^R\\
    G^A & 0
  \end{array}\right) & = & \left(\begin{array}{cc}
    1 & 1\\
    1 & - 1
  \end{array}\right) \left(\begin{array}{cc}
    G^T & G^{<}\\
    G^{>} & G^{\tilde{T}}
  \end{array}\right) \left(\begin{array}{cc}
    1 & 1\\
    1 & - 1
  \end{array}\right) . 
\end{eqnarray}
Therefore, the retarded, advanced and Keldysh Green's functions are given by
\begin{eqnarray}
  G^K & = & \frac{- 8 i \gamma n ((\omega + \varepsilon_{\tmmathbf{k}} + U_R
  n)^2 + U_R^2 n^2 + 5 \gamma^2 n^2)}{\omega^4 + 2 (5 \gamma^2 n^2 -
  \varepsilon_{\tmmathbf{k}} (\varepsilon_{\tmmathbf{k}} + 2 U_R n)) \omega^2
  + [\varepsilon_{\tmmathbf{k}} (\varepsilon_{\tmmathbf{k}} + 2 U_R n) + 3
  \gamma^2 n^2]^2}, \\
  G^R & = & \frac{2 (\omega + \varepsilon_{\tmmathbf{k}} + U_R n + 2 i \gamma
  n)}{\omega^2 + 4 i \gamma n \omega - \varepsilon_{\tmmathbf{k}}^2 - 2 U_R n
  \varepsilon_{\tmmathbf{k}} - 3 \gamma^2 n^2}, \\
  G^A & = & \frac{2 (\omega + \varepsilon_{\tmmathbf{k}} + U_R n - 2 i \gamma
  n)}{\omega^2 - 4 i \gamma n \omega - \varepsilon_{\tmmathbf{k}}^2 - 2 U_R n
  \varepsilon_{\tmmathbf{k}} - 3 \gamma^2 n^2} = (G^R)^{\ast} . 
\end{eqnarray}
Based on the retarded Green's function, we obtain the spectrum function as
\begin{equation}
  A (\omega) = - 2 \tmop{Im} G^R (\omega) = \frac{8 \gamma n (3 \gamma^2 n^2 +
  (\omega + \varepsilon_{\tmmathbf{k}}) (\omega + \varepsilon_{\tmmathbf{k}} +
  2 U_R n))}{\omega^4 + 2 (5 \gamma^2 n^2 - \varepsilon_{\tmmathbf{k}}
  (\varepsilon_{\tmmathbf{k}} + 2 U_R n)) \omega^2 +
  [\varepsilon_{\tmmathbf{k}} (\varepsilon_{\tmmathbf{k}} + 2 U_R n) + 3
  \gamma^2 n^2]^2} .
\end{equation}
The diagrams of the spectrum Green's function are depicted below for some
given momenum $\tmmathbf{k}$. When $U_R = 0$, the spectrum function becomes
\begin{equation}
  A (\omega) = \frac{8 \gamma n (3 \gamma^2 n^2 + (\omega +
  \varepsilon_{\tmmathbf{k}})^2)}{\omega^4 + 2 (5 \gamma^2 n^2 -
  \varepsilon_{\tmmathbf{k}}^2) \omega^2 + [\varepsilon_{\tmmathbf{k}}^2 + 3
  \gamma^2 n^2]^2} = \frac{8}{\gamma n} \frac{3 + (\omega' +
  \varepsilon')^2}{{\omega'}^4 + 2 \left( {5 - \varepsilon'}^2 \right)
  {\omega'}^2 + \left( {\varepsilon'}^2 + 3 \right)^2},
\end{equation}
where $\omega' = \omega / \gamma n, \varepsilon' = \varepsilon_{\tmmathbf{k}}
/ \gamma n$.

\section{How many spectrums are there in the system?}

Previously, we have introduced four spectrums by figuring out the poles in
Green's function. However, they are not indepedent to each other. The poles
are given by the equation
\begin{equation}
  \omega^4 + 2 (5 \gamma^2 n^2 - \varepsilon_{\tmmathbf{k}}
  (\varepsilon_{\tmmathbf{k}} + 2 U_R n)) \omega^2 +
  [\varepsilon_{\tmmathbf{k}} (\varepsilon_{\tmmathbf{k}} + 2 U_R n) + 3
  \gamma^2 n^2]^2 = 0.
\end{equation}
Nevertheless, the polynomial on the L.H.S can be factorized as
\begin{equation}
  (\omega^2 + 4 i \gamma n \omega - \varepsilon_{\tmmathbf{k}}^2 - 2 U_R n
  \varepsilon_{\tmmathbf{k}} - 3 \gamma^2 n^2) (\omega^2 - 4 i \gamma n \omega
  - \varepsilon_{\tmmathbf{k}}^2 - 2 U_R n \varepsilon_{\tmmathbf{k}} - 3
  \gamma^2 n^2) . \label{eq:poles}
\end{equation}
The solutions to Eq. (19) can be expressed as
\begin{eqnarray}
  \omega_{1, 2} & = & - 2 i \gamma n \pm \sqrt{\varepsilon_{\tmmathbf{k}}
  (\varepsilon_{\tmmathbf{k}} + 2 U_R n) - \gamma^2 n^2}, \\
  \omega_{3, 4} & = & 2 i \gamma n \pm \sqrt{\varepsilon_{\tmmathbf{k}}
  (\varepsilon_{\tmmathbf{k}} + 2 U_R n) - \gamma^2 n^2} . 
\end{eqnarray}
For those momentum satisfying $\varepsilon_{\tmmathbf{k}}
(\varepsilon_{\tmmathbf{k}} + 2 U_R n) < \gamma^2 n^2$, the real parts of all
the spectrums are zero. For those momentum satisfying
$\varepsilon_{\tmmathbf{k}} (\varepsilon_{\tmmathbf{k}} + 2 U_R n) > \gamma^2
n^2$, the real parts of $\omega_{1 (2)}$ are equal to $\omega_{3 (4)}$. The
relations are given by $\omega_1 = \omega_3^{\ast}$ and $\omega_2 =
\omega_4^{\ast}$. Hence, we can reach the conclusion that
\begin{equation}
  \tmop{Re} [\omega_1] = \tmop{Re} [\omega_3] = - \tmop{Re} [\omega_2] = -
  \tmop{Re} [\omega_4] .
\end{equation}
There is only one nontrivial real part in four spectrums.

\section{Bogoliubov Diagonalization}

We begin with the Bogoliubov transformation in the closed quantum systems. The
mean-field Hamiltonian takes the form of
\begin{equation}
  H = \frac{1}{2} \left(\begin{array}{cc}
    a_k^{\dagger} & a_{- k}
  \end{array}\right) \left(\begin{array}{cc}
    \varepsilon_{\tmmathbf{k}} + U n & U n\\
    U n & \varepsilon_{\tmmathbf{k}} + U n
  \end{array}\right) \left(\begin{array}{c}
    a_k\\
    a_{- k}^{\dagger}
  \end{array}\right) .
\end{equation}


To derive the Bogoliubov diagonalization matrix, we first transform the matrix
as
\begin{equation}
  \left(\begin{array}{cc}
    \varepsilon_{\tmmathbf{k}} + U n & U n\\
    U n & \varepsilon_{\tmmathbf{k}} + U n
  \end{array}\right) \rightarrow \left(\begin{array}{cc}
    \varepsilon_{\tmmathbf{k}} + U n & U n\\
    U n & \varepsilon_{\tmmathbf{k}} + U n
  \end{array}\right) \left(\begin{array}{cc}
    1 & 0\\
    0 & - 1
  \end{array}\right) = \left(\begin{array}{cc}
    \varepsilon_{\tmmathbf{k}} + U n & - U n\\
    U n & - (\varepsilon_{\tmmathbf{k}} + U n)
  \end{array}\right)
\end{equation}
since the commutation relation differs for different indices in the matrix.
The eigenvectors of the matrix are
\begin{eqnarray}
  E_1 = \sqrt{\varepsilon_{\tmmathbf{k}} (\varepsilon_{\tmmathbf{k}} + 2 U n)}
  & , & v_1 = \left( 1 + x^2 + x \sqrt{x^2 + 2}, 1 \right)^T, \\
  E_2 = - \sqrt{\varepsilon_{\tmmathbf{k}} (\varepsilon_{\tmmathbf{k}} + 2 U
  n)} & , & v_2 = \left( 1 + x^2 - x \sqrt{x^2 + 2}, 1 \right)^T, 
\end{eqnarray}
where $x = \sqrt{\varepsilon_{\tmmathbf{k}} / U n}$. Hence, the similar
transformation can be written as
\begin{equation}
  U = \left(\begin{array}{cc}
    1 + x^2 + x \sqrt{x^2 + 2} & 1 + x^2 - x \sqrt{x^2 + 2}\\
    1 & 1
  \end{array}\right) \equiv \left(\begin{array}{cc}
    \frac{1}{\alpha} & \alpha\\
    1 & 1
  \end{array}\right),
\end{equation}
where $\alpha = 1 + x^2 - x \sqrt{x^2 + 2}$.

To recover the Bogoliubov transformation, we renormalize each eigenvector as
\begin{eqnarray}
  v_1 & \rightarrow & \beta_1 v_1 = \left( \frac{1}{\sqrt{1 - \alpha^2}},
  \frac{\alpha}{\sqrt{1 - \alpha^2}} \right)^T, \\
  v_2 & \rightarrow & \beta_2 v_2 = \left( \frac{\alpha}{\sqrt{1 - \alpha^2}},
  \frac{1}{\sqrt{1 - \alpha^2}} \right)^T, 
\end{eqnarray}
with $\beta_1 = \alpha / \sqrt{1 - \alpha^2}, \beta_2 = 1 / \sqrt{1 -
\alpha^2}$. In this case we obtain the Bogoliubov transformation matrix
\begin{equation}
  U = \left(\begin{array}{cc}
    \frac{1}{\sqrt{1 - \alpha^2}} & \frac{\alpha}{\sqrt{1 - \alpha^2}}\\
    \frac{\alpha}{\sqrt{1 - \alpha^2}} & \frac{1}{\sqrt{1 - \alpha^2}}
  \end{array}\right) . \label{Bogoliubov}
\end{equation}
In addition, we note that even though the original similar transformation is a
little different from the Bogoliubov transformation in \eqref{Bogoliubov}, it
can still be used to diagonalize the Hamiltonian.

For the open quantum system, we still transform the matrix $G^{- 1}$ by
\begin{equation}
  G^{- 1} \rightarrow G^{- 1} \left(\begin{array}{cccc}
    1 & 0 & 0 & 0\\
    0 & - 1 & 0 & 0\\
    0 & 0 & - 1 & 0\\
    0 & 0 & 0 & 1
  \end{array}\right) .
\end{equation}
Therefore, the Bogoliubov transformation matrix (unrenormalized one) is given
by

\begin{equation}
  \left(\begin{array}{cccc}
  \frac{(\gamma-i\text{UR})\left(-2i\gamma+\varepsilon+\sqrt{-5\gamma^{2}+\varepsilon^{2}-4\gamma\sqrt{\gamma^{2}-\varepsilon(\varepsilon+2\text{UR})}+2\varepsilon\text{UR}}+\text{UR}\right)}{4\sqrt{\gamma^{2}-\varepsilon(\varepsilon+2\text{UR})}\sqrt{-5\gamma^{2}-4\gamma\sqrt{\gamma^{2}-\varepsilon(\varepsilon+2\text{UR})}+\varepsilon(\varepsilon+2\text{UR})}} & \frac{i\left(\gamma^{2}+\text{UR}^{2}\right)}{4\sqrt{\gamma^{2}-\varepsilon(\varepsilon+2\text{UR})}\sqrt{-5\gamma^{2}-4\gamma\sqrt{\gamma^{2}-\varepsilon(\varepsilon+2\text{UR})}+\varepsilon(\varepsilon+2\text{UR})}} & \frac{(\gamma-i\text{UR})\left(i\gamma^{2}+\text{UR}\sqrt{\gamma^{2}-\varepsilon(\varepsilon+2\text{UR})}+\varepsilon\sqrt{\gamma^{2}-\varepsilon(\varepsilon+2\text{UR})}-i\varepsilon(\varepsilon+2\text{UR})\right)}{4\left(\gamma^{2}-\varepsilon(\varepsilon+2\text{UR})\right)\sqrt{-5\gamma^{2}-4\gamma\sqrt{\gamma^{2}-\varepsilon(\varepsilon+2\text{UR})}+\varepsilon(\varepsilon+2\text{UR})}} & \frac{\left(\gamma^{2}-i\text{UR}\sqrt{\gamma^{2}-\varepsilon(\varepsilon+2\text{UR})}-i\varepsilon\sqrt{\gamma^{2}-\varepsilon(\varepsilon+2\text{UR})}-\varepsilon(\varepsilon+2\text{UR})\right)\left(-2i\gamma-\varepsilon+\sqrt{-5\gamma^{2}+\varepsilon^{2}-4\gamma\sqrt{\gamma^{2}-\varepsilon(\varepsilon+2\text{UR})}+2\varepsilon\text{UR}}-\text{UR}\right)}{4\left(\gamma^{2}-\varepsilon(\varepsilon+2\text{UR})\right)\sqrt{-5\gamma^{2}-4\gamma\sqrt{\gamma^{2}-\varepsilon(\varepsilon+2\text{UR})}+\varepsilon(\varepsilon+2\text{UR})}}\\
  \frac{(\gamma-i\text{UR})\left(2i\gamma-\varepsilon+\sqrt{-5\gamma^{2}+\varepsilon^{2}-4\gamma\sqrt{\gamma^{2}-\varepsilon(\varepsilon+2\text{UR})}+2\varepsilon\text{UR}}-\text{UR}\right)}{4\sqrt{\gamma^{2}-\varepsilon(\varepsilon+2\text{UR})}\sqrt{-5\gamma^{2}-4\gamma\sqrt{\gamma^{2}-\varepsilon(\varepsilon+2\text{UR})}+\varepsilon(\varepsilon+2\text{UR})}} & -\frac{i\left(\gamma^{2}+\text{UR}^{2}\right)}{4\sqrt{\gamma^{2}-\varepsilon(\varepsilon+2\text{UR})}\sqrt{-5\gamma^{2}-4\gamma\sqrt{\gamma^{2}-\varepsilon(\varepsilon+2\text{UR})}+\varepsilon(\varepsilon+2\text{UR})}} & \frac{(\text{UR}+i\gamma)\left(-\gamma^{2}+\varepsilon^{2}+i\text{UR}\sqrt{\gamma^{2}-\varepsilon(\varepsilon+2\text{UR})}+i\varepsilon\sqrt{\gamma^{2}-\varepsilon(\varepsilon+2\text{UR})}+2\varepsilon\text{UR}\right)}{4\left(\gamma^{2}-\varepsilon(\varepsilon+2\text{UR})\right)\sqrt{-5\gamma^{2}-4\gamma\sqrt{\gamma^{2}-\varepsilon(\varepsilon+2\text{UR})}+\varepsilon(\varepsilon+2\text{UR})}} & \frac{\left(\gamma^{2}-i\text{UR}\sqrt{\gamma^{2}-\varepsilon(\varepsilon+2\text{UR})}-i\varepsilon\sqrt{\gamma^{2}-\varepsilon(\varepsilon+2\text{UR})}-\varepsilon(\varepsilon+2\text{UR})\right)\left(2i\gamma+\varepsilon+\sqrt{-5\gamma^{2}+\varepsilon^{2}-4\gamma\sqrt{\gamma^{2}-\varepsilon(\varepsilon+2\text{UR})}+2\varepsilon\text{UR}}+\text{UR}\right)}{4\left(\gamma^{2}-\varepsilon(\varepsilon+2\text{UR})\right)\sqrt{-5\gamma^{2}-4\gamma\sqrt{\gamma^{2}-\varepsilon(\varepsilon+2\text{UR})}+\varepsilon(\varepsilon+2\text{UR})}}\\
  -\frac{(\gamma-i\text{UR})\left(-2i\gamma+\varepsilon+\sqrt{-5\gamma^{2}+\varepsilon^{2}+4\gamma\sqrt{\gamma^{2}-\varepsilon(\varepsilon+2\text{UR})}+2\varepsilon\text{UR}}+\text{UR}\right)}{4\sqrt{\gamma^{2}-\varepsilon(\varepsilon+2\text{UR})}\sqrt{-5\gamma^{2}+4\gamma\sqrt{\gamma^{2}-\varepsilon(\varepsilon+2\text{UR})}+\varepsilon(\varepsilon+2\text{UR})}} & -\frac{i\left(\gamma^{2}+\text{UR}^{2}\right)}{4\sqrt{\gamma^{2}-\varepsilon(\varepsilon+2\text{UR})}\sqrt{-5\gamma^{2}+4\gamma\sqrt{\gamma^{2}-\varepsilon(\varepsilon+2\text{UR})}+\varepsilon(\varepsilon+2\text{UR})}} & \frac{(\text{UR}+i\gamma)\left(\gamma^{2}+i\text{UR}\sqrt{\gamma^{2}-\varepsilon(\varepsilon+2\text{UR})}+i\varepsilon\sqrt{\gamma^{2}-\varepsilon(\varepsilon+2\text{UR})}-\varepsilon(\varepsilon+2\text{UR})\right)}{4\left(\gamma^{2}-\varepsilon(\varepsilon+2\text{UR})\right)\sqrt{-5\gamma^{2}+4\gamma\sqrt{\gamma^{2}-\varepsilon(\varepsilon+2\text{UR})}+\varepsilon(\varepsilon+2\text{UR})}} & \frac{\left(\gamma^{2}+i\text{UR}\sqrt{\gamma^{2}-\varepsilon(\varepsilon+2\text{UR})}+i\varepsilon\sqrt{\gamma^{2}-\varepsilon(\varepsilon+2\text{UR})}-\varepsilon(\varepsilon+2\text{UR})\right)\left(-2i\gamma-\varepsilon+\sqrt{-5\gamma^{2}+\varepsilon^{2}+4\gamma\sqrt{\gamma^{2}-\varepsilon(\varepsilon+2\text{UR})}+2\varepsilon\text{UR}}-\text{UR}\right)}{4\left(\gamma^{2}-\varepsilon(\varepsilon+2\text{UR})\right)\sqrt{-5\gamma^{2}+4\gamma\sqrt{\gamma^{2}-\varepsilon(\varepsilon+2\text{UR})}+\varepsilon(\varepsilon+2\text{UR})}}\\
  \frac{(\gamma-i\text{UR})\left(-2i\gamma+\varepsilon-\sqrt{-5\gamma^{2}+\varepsilon^{2}+4\gamma\sqrt{\gamma^{2}-\varepsilon(\varepsilon+2\text{UR})}+2\varepsilon\text{UR}}+\text{UR}\right)}{4\sqrt{\gamma^{2}-\varepsilon(\varepsilon+2\text{UR})}\sqrt{-5\gamma^{2}+4\gamma\sqrt{\gamma^{2}-\varepsilon(\varepsilon+2\text{UR})}+\varepsilon(\varepsilon+2\text{UR})}} & \frac{i\left(\gamma^{2}+\text{UR}^{2}\right)}{4\sqrt{\gamma^{2}-\varepsilon(\varepsilon+2\text{UR})}\sqrt{-5\gamma^{2}+4\gamma\sqrt{\gamma^{2}-\varepsilon(\varepsilon+2\text{UR})}+\varepsilon(\varepsilon+2\text{UR})}} & \frac{(\gamma-i\text{UR})\left(-i\gamma^{2}+\text{UR}\sqrt{\gamma^{2}-\varepsilon(\varepsilon+2\text{UR})}+\varepsilon\sqrt{\gamma^{2}-\varepsilon(\varepsilon+2\text{UR})}+i\varepsilon(\varepsilon+2\text{UR})\right)}{4\left(\gamma^{2}-\varepsilon(\varepsilon+2\text{UR})\right)\sqrt{-5\gamma^{2}+4\gamma\sqrt{\gamma^{2}-\varepsilon(\varepsilon+2\text{UR})}+\varepsilon(\varepsilon+2\text{UR})}} & \frac{\left(\gamma^{2}+i\text{UR}\sqrt{\gamma^{2}-\varepsilon(\varepsilon+2\text{UR})}+i\varepsilon\sqrt{\gamma^{2}-\varepsilon(\varepsilon+2\text{UR})}-\varepsilon(\varepsilon+2\text{UR})\right)\left(2i\gamma+\varepsilon+\sqrt{-5\gamma^{2}+\varepsilon^{2}+4\gamma\sqrt{\gamma^{2}-\varepsilon(\varepsilon+2\text{UR})}+2\varepsilon\text{UR}}+\text{UR}\right)}{4\left(\gamma^{2}-\varepsilon(\varepsilon+2\text{UR})\right)\sqrt{-5\gamma^{2}+4\gamma\sqrt{\gamma^{2}-\varepsilon(\varepsilon+2\text{UR})}+\varepsilon(\varepsilon+2\text{UR})}}
  \end{array}\right)
  \end{equation}
  and the diagonal element is given by the spectrum of the system 
  \begin{align}
  \overline{G^{-1}} & =\frac{1}{2}\mathrm{diag}\{\omega-\sqrt{-5\gamma^{2}n^{2}+\epsilon_{k}(2U_{R}n+\epsilon_{k})-4\gamma n\sqrt{\gamma^{2}n^{2}-\epsilon_{k}(2U_{R}n+\epsilon_{k})}},-\omega-\sqrt{-5\gamma^{2}n^{2}+\epsilon_{k}(2U_{R}n+\epsilon_{k})-4\gamma n\sqrt{\gamma^{2}n^{2}-\epsilon_{k}(2U_{R}n+\epsilon_{k})}},\nonumber \\
   & -\omega+\sqrt{-5\gamma^{2}n^{2}+\epsilon_{k}(2U_{R}n+\epsilon_{k})+4\gamma n\sqrt{\gamma^{2}n^{2}-\epsilon_{k}(2U_{R}n+\epsilon_{k})}},\omega+\sqrt{-5\gamma^{2}n^{2}+\epsilon_{k}(2U_{R}n+\epsilon_{k})+4\gamma n\sqrt{\gamma^{2}n^{2}-\epsilon_{k}(2U_{R}n+\epsilon_{k})}}\}
  \end{align}

\section{Nambu-Goldstone Mode}

To observe the Nambu-Goldstone mode, we first transform the action into
another basis. By defining the retarded or advanced operators
$a_{\tmmathbf{k}, R} = \frac{1}{2} (a_{\tmmathbf{k}, +} + a_{\tmmathbf{k},
-})$ and $a_{\tmmathbf{k}, A} = a_{\tmmathbf{k}, +} - a_{\tmmathbf{k}, -}$.
The action becomes
\[ a^{\dagger}_{\tmmathbf{k}+} i \partial_t a_{\tmmathbf{k}+} -
   a^{\dagger}_{\tmmathbf{k}-} i \partial_t a_{\tmmathbf{k}-} =
   a^{\dagger}_{\tmmathbf{k}, A} i \partial_t a_{\tmmathbf{k}, R} -
   a^{\dagger}_{\tmmathbf{k}, R} i \partial_t a_{\tmmathbf{k}, A}, \]
\begin{eqnarray*}
  - H_+ + H_- - 4 i \gamma n \sum_{\tmmathbf{k}} a_{\tmmathbf{k}-}^{\dagger}
  a_{\tmmathbf{k}+} & = & \sum_{\tmmathbf{k}} - (\varepsilon_{\tmmathbf{k}} +
  U_R n - 2 i \gamma n) a_{\tmmathbf{k}, A}^{\dagger} a_{\tmmathbf{k}, R} -
  (\varepsilon_{\tmmathbf{k}} + U_R n + 2 i \gamma n) a_{\tmmathbf{k},
  R}^{\dagger} a_{\tmmathbf{k}, A}\\
  &  & - \frac{U^{\ast} n}{2} (a_{- \tmmathbf{k}, R} a_{\tmmathbf{k}, A} +
  a_{- \tmmathbf{k}, A} a_{\tmmathbf{k}, R}) - \frac{U n}{2} (a_{\tmmathbf{k},
  R}^{\dagger} a^{\dagger}_{- \tmmathbf{k}, A} + a^{\dagger}_{\tmmathbf{k}, A}
  a^{\dagger}_{- \tmmathbf{k}, R}),
\end{eqnarray*}
Hence, we can reorganize the action as
\begin{equation}
  S = \frac{1}{2} \sum_{\tmmathbf{k}} \int \frac{d \omega}{2 \pi}
  \left(\begin{array}{cccc}
    a_{\tmmathbf{k}, R}^{\dagger} & a_{- \tmmathbf{k}, R} & a_{\tmmathbf{k},
    A}^{\dagger} & a_{- \tmmathbf{k}, A}
  \end{array}\right) \left(\begin{array}{cc}
    O_{2 \times 2} & G\\
    G^{\dagger} & O_{2 \times 2}
  \end{array}\right) \left(\begin{array}{c}
    a_{\tmmathbf{k}, R}\\
    a_{- \tmmathbf{k}, R}^{\dagger}\\
    a_{\tmmathbf{k}, A}\\
    a_{- \tmmathbf{k}, A}^{\dagger}
  \end{array}\right),
\end{equation}
where
\begin{equation}
  G = \left(\begin{array}{cc}
    \omega - (\varepsilon_{\tmmathbf{k}} + U_R n + 2 i \gamma n) & - U n\\
    - U^{\ast} n & - \omega - (\varepsilon_{\tmmathbf{k}} + U_R n - 2 i \gamma
    n)
  \end{array}\right) .
\end{equation}
The poles of the Green's function is given by
\begin{equation}
  \det (G) = 0 \Rightarrow (\omega - (\varepsilon_{\tmmathbf{k}} + U_R n + 2 i
  \gamma n)) (\omega + (\varepsilon_{\tmmathbf{k}} + U_R n - 2 i \gamma n)) +
  | U |^2 n^2 = 0.
\end{equation}
\begin{equation}
  \det (G^{\dagger}) = 0 \Rightarrow (\omega - (\varepsilon_{\tmmathbf{k}} +
  U_R n - 2 i \gamma n)) (\omega + (\varepsilon_{\tmmathbf{k}} + U_R n + 2 i
  \gamma n)) + | U |^2 n^2 = 0.
\end{equation}
These two equations can be rewritten as
\begin{eqnarray}
  \omega^2 + 4 i \gamma n \omega - \varepsilon_{\tmmathbf{k}}^2 - 2 U_R n
  \varepsilon_{\tmmathbf{k}} - 3 \gamma^2 n^2 & = & 0, \\
  \omega^2 - 4 i \gamma n \omega - \varepsilon_{\tmmathbf{k}}^2 - 2 U_R n
  \varepsilon_{\tmmathbf{k}} - 3 \gamma^2 n^2 & = & 0, 
\end{eqnarray}
which are equivalent to \ Eq.\eqref{eq:poles}.

In this way, we diagonalize the matrix in another convenient way. Here we
define
\begin{equation}
  H = \left(\begin{array}{cc}
    (\varepsilon_{\tmmathbf{k}} + U_R n + 2 i \gamma n) & U n\\
    U^{\ast} n & (\varepsilon_{\tmmathbf{k}} + U_R n - 2 i \gamma n)
  \end{array}\right) .
\end{equation}
To diagonalize the matrix $H$, we follow the same procedure above and
transform $H$ as $H \sigma_z$. The problem is equivalent to diagonalize the
matrix
\begin{equation}
  \left(\begin{array}{cc}
    (\varepsilon_{\tmmathbf{k}} + U_R n + 2 i \gamma n) & - U n\\
    U^{\ast} n & - (\varepsilon_{\tmmathbf{k}} + U_R n - 2 i \gamma n)
  \end{array}\right) .
\end{equation}
The eigenvectors are given by
\begin{eqnarray}
  v_1 & = & \left( \frac{U_R n + \varepsilon_{\tmmathbf{k}} +
  \sqrt{\varepsilon_{\tmmathbf{k}} (\varepsilon_{\tmmathbf{k}} + 2 U_R n) -
  \gamma^2 n^2}}{U^{\ast} n}, 1 \right)^T \nonumber\\
  & \equiv & \left( \frac{1}{\bar{\alpha}}, 1 \right)^T, \\
  v_2 & = & \left( \frac{U_R n + \varepsilon_{\tmmathbf{k}} -
  \sqrt{\varepsilon_{\tmmathbf{k}} (\varepsilon_{\tmmathbf{k}} + 2 U_R n) -
  \gamma^2 n^2}}{U^{\ast} n}, 1 \right)^T \nonumber\\
  & \equiv & (\alpha, 1)^T, 
\end{eqnarray}
with $\alpha = \frac{U_R n + \varepsilon_{\tmmathbf{k}} -
\sqrt{\varepsilon_{\tmmathbf{k}} (\varepsilon_{\tmmathbf{k}} + 2 U_R n) -
\gamma^2 n^2}}{U^{\ast} n}, \bar{\alpha} = \frac{U_R n +
\varepsilon_{\tmmathbf{k}} - \sqrt{\varepsilon_{\tmmathbf{k}}
(\varepsilon_{\tmmathbf{k}} + 2 U_R n) - \gamma^2 n^2}}{U n}$. We note that
$\alpha^{\ast} = \bar{\alpha}$ only holds when $\varepsilon_{\tmmathbf{k}}
(\varepsilon_{\tmmathbf{k}} + 2 U_R n) - \gamma^2 n^2 \geqslant 0$ which
corresponds to a nontrivial real part. Therefore, the diagonalization matrix
is given by
\begin{equation}
  U = \left(\begin{array}{cc}
    \frac{1}{\bar{\alpha}} & \alpha\\
    1 & 1
  \end{array}\right) .
\end{equation}
By gauge transformation
\begin{eqnarray}
  v_1 & \rightarrow & \beta_1 v_1 = \left( \frac{1}{\sqrt{1 - \alpha
  \bar{\alpha}}}, \frac{\bar{\alpha}}{\sqrt{1 - \alpha \bar{\alpha}}}
  \right)^T, \\
  v_2 & \rightarrow & \beta_2 v_2 = \left( \frac{\alpha}{\sqrt{1 - \alpha
  \bar{\alpha}}}, \frac{1}{\sqrt{1 - \alpha \bar{\alpha}}} \right)^T, 
\end{eqnarray}
we obtain the Bogoliubov matrix for the retarded(classical) operators
\begin{equation}
  U = \left(\begin{array}{cc}
    \frac{1}{\sqrt{1 - \alpha \bar{\alpha}}} & \frac{\alpha}{\sqrt{1 - \alpha
    \bar{\alpha}}}\\
    \frac{\bar{\alpha}}{\sqrt{1 - \alpha \bar{\alpha}}} & \frac{1}{\sqrt{1 -
    \alpha \bar{\alpha}}}
  \end{array}\right), \left(\begin{array}{cc}\bar{\beta}_{\tmmathbf{k}, R} &
    \beta_{- \tmmathbf{k}, R}\end{array}\right) = \left(\begin{array}{cc}a^{\dagger}_{\tmmathbf{k}, R} &
    a_{- \tmmathbf{k}, R}\end{array}\right)\left(\begin{array}{cc}
    \frac{1}{\sqrt{1 - \alpha \bar{\alpha}}} & \frac{\alpha}{\sqrt{1 - \alpha
    \bar{\alpha}}}\\
    \frac{\bar{\alpha}}{\sqrt{1 - \alpha \bar{\alpha}}} & \frac{1}{\sqrt{1 -
    \alpha \bar{\alpha}}}
  \end{array}\right).
\end{equation}
and the Bogoliubov matrix for the advanced(quantum) operators
\begin{equation}
  U^{- 1} = \left(\begin{array}{cc}
    \frac{1}{\sqrt{1 - \alpha \bar{\alpha}}} & - \frac{\alpha}{\sqrt{1 -
    \alpha \bar{\alpha}}}\\
    - \frac{\bar{\alpha}}{\sqrt{1 - \alpha \bar{\alpha}}} & \frac{1}{\sqrt{1 -
    \alpha \bar{\alpha}}}
  \end{array}\right), \left(\begin{array}{c}
    \beta_{\tmmathbf{k}, A}\\
    \bar{\beta}_{- \tmmathbf{k}, A}
  \end{array}\right) = \left(\begin{array}{cc}
    \frac{1}{\sqrt{1 - \alpha \bar{\alpha}}} & \frac{\alpha}{\sqrt{1 - \alpha
    \bar{\alpha}}}\\
    \frac{\bar{\alpha}}{\sqrt{1 - \alpha \bar{\alpha}}} & \frac{1}{\sqrt{1 -
    \alpha \bar{\alpha}}}
  \end{array}\right) \left(\begin{array}{c}
    a_{\tmmathbf{k}, A}\\
    a^{\dagger}_{- \tmmathbf{k}, A}
  \end{array}\right) .
\end{equation}


Hence, the action becomes
\begin{equation}
  S = \frac{1}{2} \sum_{\tmmathbf{k}} \int \frac{d \omega}{2 \pi}
  \left(\begin{array}{cccc}
    \bar{\beta}_{\tmmathbf{k}, R} & \beta_{- \tmmathbf{k}, R} &
    \bar{\beta}_{\tmmathbf{k}, A} & \beta_{- \tmmathbf{k}, A}
  \end{array}\right) \left(\begin{array}{cc}
    O_{2 \times 2} & H'\\
    \bar{H'} & O_{2 \times 2}
  \end{array}\right) \left(\begin{array}{c}
    \beta_{\tmmathbf{k}, R}\\
    \bar{\beta}_{- \tmmathbf{k}, R}\\
    \beta_{\tmmathbf{k}, A}\\
    \bar{\beta}_{- \tmmathbf{k}, A}
  \end{array}\right),
\end{equation}
where
\begin{equation}
  H' = \left(\begin{array}{cc}
    i \partial_t - \left( - 2 i \gamma - \sqrt{\varepsilon_{\tmmathbf{k}}
    (\varepsilon_{\tmmathbf{k}} + 2 U_R n) - \gamma^2 n^2} \right) & 0\\
    0 & i \partial_t - \left( - 2 i \gamma + \sqrt{\varepsilon_{\tmmathbf{k}}
    (\varepsilon_{\tmmathbf{k}} + 2 U_R n) - \gamma^2 n^2} \right)
  \end{array}\right) ,
\end{equation}
and 
\begin{equation}
  \bar{H'} = \left(\begin{array}{cc}
    -i \partial_t - \left( 2 i \gamma - \sqrt{\varepsilon_{\tmmathbf{k}}
    (\varepsilon_{\tmmathbf{k}} + 2 U_R n) - \gamma^2 n^2} \right) & 0\\
    0 & -i \partial_t - \left(  2 i \gamma + \sqrt{\varepsilon_{\tmmathbf{k}}
    (\varepsilon_{\tmmathbf{k}} + 2 U_R n) - \gamma^2 n^2} \right)
  \end{array}\right) ,
\end{equation}
\left(\begin{array}{cc}\bar{\beta}_{\tmmathbf{k}, R} &
  \beta_{- \tmmathbf{k}, R}\end{array}\right) = \left(\begin{array}{cc}a^{\dagger}_{\tmmathbf{k}, R} &
  a_{- \tmmathbf{k}, R}\end{array}\right)\left(\begin{array}{cc}
  \frac{1}{\sqrt{1 - \alpha \bar{\alpha}}} & \frac{\alpha}{\sqrt{1 - \alpha
  \bar{\alpha}}}\\
  \frac{\bar{\alpha}}{\sqrt{1 - \alpha \bar{\alpha}}} & \frac{1}{\sqrt{1 -
  \alpha \bar{\alpha}}}
\end{array}\right).
\end{equation}
It shoule also be noted that the original action in the keldysh basis is Hermitian, therefore we can easily choose a Bogoliubov transformation which keeps $\bar{\beta}=\beta^\dagger$ by demanding that the matrix diagonalizing $G^\dagger$ and the one diagonalizing $G$ is related by Hermitian conjugating.



\end{document}
