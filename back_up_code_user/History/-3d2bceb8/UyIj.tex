\documentclass{article}
\usepackage[english]{babel}
\usepackage{geometry,amsmath,amssymb,bbm}
\geometry{a4paper}

%%%%%%%%%% Start TeXmacs macros
\newcommand{\nobracket}{}
\newcommand{\tmaffiliation}[1]{\\ #1}
\newcommand{\tmmathbf}[1]{\ensuremath{\boldsymbol{#1}}}
\newcommand{\tmop}[1]{\ensuremath{\operatorname{#1}}}
\newcommand{\tmstrong}[1]{\textbf{#1}}
%%%%%%%%%% End TeXmacs macros

\begin{document}

\title{Superfluidity in Open Quantum System}

\author{
  {\tmstrong{}}Hongchao Li
  \tmaffiliation{Department of Tokyo, University of Tokyo}
}

\date{April 17, 2023}

\maketitle

{\tableofcontents}

\

\section{Superfluidity in Fermionic System}

In the open quantum system for attractive Fermi-Hubbard model, we can use the
effective Hermitian Hamiltonian to describe the dynamics of fermions. On each
branch of Schwinger-Keldysh contour, we can write the effective Hamiltonian
as:
\begin{equation}
  H_{\tmop{eff}} = \sum_{\tmmathbf{k}} \Psi_{\tmmathbf{k}}^{\dagger}
  \left(\begin{array}{cc}
    \varepsilon_{\tmmathbf{k}} & \Delta\\
    \Delta^{\ast} & - \varepsilon_{-\tmmathbf{k}}
  \end{array}\right) \Psi_{\tmmathbf{k}},
\end{equation}
where $\Psi_{\tmmathbf{k}} = \left(\begin{array}{cc}
  C_{\tmmathbf{k} \uparrow} & C^{\dagger}_{-\tmmathbf{k} \downarrow}
\end{array}\right)^T$ and $\Delta = - \frac{U_R + i \gamma / 2}{N_0}
\sum_{\tmmathbf{k}} \langle C_{-\tmmathbf{k} \downarrow} C_{\tmmathbf{k}
\uparrow} \rangle \equiv - \frac{U}{N_0} \sum_{\tmmathbf{k}} \langle
C_{-\tmmathbf{k} \downarrow} C_{\tmmathbf{k} \uparrow} \rangle$.
Here according to dynamics, the particle number will decay with time.
\begin{equation}
  \frac{d N}{d t} = - N_0 \frac{2 \gamma | \Delta |^2}{| U |^2},
\end{equation}
and $\Delta = \Delta (t)$ is also a function of time. Due to the existence
of imaginary part induced by dissipation, it will also decay with time.
However, we will here prove that for fermionic superfluid system, the
superfluid density will not decay with time any more in zero-temperature case.

The superfluid density is defined by phase stiffness. The phase stiffness is
defined as:
\begin{equation}
  Q_{a b} = \frac{1}{V} \frac{\partial^2 F}{\partial A_a \partial A_b} |_{A =
  0} \nobracket, \label{Q}
\end{equation}
where $a$ and $b$ represents spatial indices. Here we use the Pierles
substitution with $\tmmathbf{k} \rightarrow \tmmathbf{k}- e\tmmathbf{A}$.
Since the Green's function in finite-temperature case is given by:
\begin{equation}
  G (k) = \left(\begin{array}{cc}
    i \omega_n - \varepsilon_{\tmmathbf{k}- e\tmmathbf{A}} & - \Delta (t)\\
    - \Delta^{\ast} (t) & i \omega_n + \varepsilon_{-\tmmathbf{k}-
    e\tmmathbf{A}}
  \end{array}\right)^{- 1} = (i \omega_n - h_{\tmmathbf{k}- e\tmmathbf{A}
  \sigma_z})^{- 1},
\end{equation}
where $h_{\tmmathbf{k}} = \varepsilon_{\tmmathbf{k}}
\sigma_z + \tmop{Re} \Delta \sigma_x - \tmop{Im} \Delta \sigma_y$. Hence, the
free energy is given by:
\begin{equation}
  F = - \frac{1}{\beta} \sum_{\tmmathbf{k}, i \omega_n} \tmop{Tr} \ln
  [\varepsilon_{\tmmathbf{k}- e\tmmathbf{A} \sigma_z} \sigma_z + \tmop{Re}
  \Delta \sigma_x - \tmop{Im} \Delta \sigma_y - i \omega_n] +
  \frac{\Delta^2}{U}
\end{equation}
Therefore, we can substitude this in right hand side of (\ref{Q}) and have:
\begin{equation}
  Q_{a b} = \frac{e^2}{\beta V} \sum_k (\partial_{k_a} \partial_{k_b}
  \varepsilon_{\tmmathbf{k}} \tmop{Tr} [\sigma_z G (k)] + (\partial_{k_a}
  \varepsilon_{\tmmathbf{k}} \partial_{k_b} \varepsilon_{\tmmathbf{k}})
  \tmop{Tr} [G (k) G (k)])
\end{equation}
After simplification, it gives the result as:
\begin{equation}
  Q_{a b} = \frac{4 e^2}{\beta V} \sum_k \partial_{k_a}
  \varepsilon_{\tmmathbf{k}} \partial_{k_b} \varepsilon_{\tmmathbf{k}} \frac{|
  \Delta |^2 (t)}{[\omega_n^2 + \varepsilon_{\tmmathbf{k}}^2 + | \Delta |^2
  (t)]}
\end{equation}
By assuming $Q_{a b} = Q \delta_{a b}$, we have:
\begin{equation}
  Q (T) = \frac{n e^2}{m} \frac{1}{\beta} \sum_n \int_{- \infty}^{\infty} d
  \varepsilon \frac{2 | \Delta |^2}{(\varepsilon^2 + \omega_n^2 + | \Delta
  |^2)^2} = \left( \frac{n e^2}{m} \right) \pi \frac{1}{\beta} \sum_n \frac{|
  \Delta |^2}{(\omega_n^2 + | \Delta |^2)^2}.
\end{equation}
Here we use $\frac{1}{3} N (0) v_F^2 = \frac{n}{m}$. In finite-temperature
case, it will decay with time. However, in zero-temperature case we find it
can be transformed into an integral.
\begin{equation}
  Q (0) = \left( \frac{n e^2}{m} \right) \int_{- \infty}^{\infty} \frac{d
  \omega}{2} \frac{d}{d \omega} \left( \frac{\omega}{(\omega^2 + | \Delta
  |^2)^{1 / 2}} \right) = \frac{n e^2}{m}
\end{equation}
What can be figured out is $Q (0)$ is propotional to the number density. With
phase stiffness, one can express the superfluid density as:
\begin{equation}
  Q = \left( \frac{2 e}{\hbar} \right)^2 \rho_s
\end{equation}
Hence, the superfluid density will take the value $\frac{n \hbar^2}{4 m}$. We
can see the superfluid density decay as the same speed with number of
fermions.

\section{Superfluidity in Bosonic System}

\subsection{Construction of Hamiltonian}

Here we consider a Hermitian bosonic system with weakly interatcion. The
Hamiltonian reads as:
\begin{eqnarray}
  H & = & \sum_{\tmmathbf{k}} \varepsilon_{\tmmathbf{k}}
  a_{\tmmathbf{k}}^{\dagger} a_{\tmmathbf{k}} + \frac{U_R}{2} \sum_i
  (a_i^{\dagger})^2 (a_i)^2 \nonumber\\
  & = & \sum_{\tmmathbf{k}} \varepsilon_{\tmmathbf{k}}
  a_{\tmmathbf{k}}^{\dagger} a_{\tmmathbf{k}} + \frac{U_R}{2 V}
  \sum_{\tmmathbf{k}, \tmmathbf{q}, \tmmathbf{p}} a_{\tmmathbf{k}}^{\dagger}
  a_{\tmmathbf{p}}^{\dagger} a_{\tmmathbf{p}-\tmmathbf{q}}
  a_{\tmmathbf{k}+\tmmathbf{q}} \label{Hamiltonian} 
\end{eqnarray}
Here we generalize the model in a continuous case. By adding dissipation to
the system, we turn it into an open quantum system. This open quantum system
is described by the Lindblad equation:
\begin{equation}
  \frac{d \rho}{d t} =\mathcal{L} \rho = - i [H, \rho] - \frac{\gamma}{2}
  \sum_i (\{ L_i^{\dagger} L_i, \rho \} - 2 L_i^{\dagger} \rho L_i)
  \label{Lindbland}
\end{equation}
Here we take the Lindblad operator as $L_i = a_i^2$. Then we consider
generalize it to closed-time-contour path integral. On Schwinger-Keldysh
contour, the action is written as:
\begin{equation}
  S = \int_{- \infty}^{\infty} d t \left[ \sum_{\tmmathbf{k}}
  (a^{\dagger}_{\tmmathbf{k}+} i \partial_t a_{\tmmathbf{k}+} -
  a^{\dagger}_{\tmmathbf{k}-} i \partial_t a_{\tmmathbf{k}-}) - H_+ + H_- +
  \frac{i \gamma}{2} \sum_i (L_{i +}^{\dagger} L_{i +} + L_{i -}^{\dagger}
  L_{i -} - 2 L_{i +} L_{i -}^{\dagger}) \right],
\end{equation}
where $H_{\alpha} = \sum_{\tmmathbf{k}} \varepsilon_{\tmmathbf{k}}
a_{\tmmathbf{k} \alpha}^{\dagger} a_{\tmmathbf{k} \alpha} + \frac{U_R}{2 V}
\sum_{\tmmathbf{k}, \tmmathbf{q}, \tmmathbf{p}} a_{\tmmathbf{k}
\alpha}^{\dagger} a_{\tmmathbf{p} \alpha}^{\dagger}
a_{\tmmathbf{p}-\tmmathbf{q}, \alpha} a_{\tmmathbf{k}+\tmmathbf{q}, \alpha}
(\alpha = +, -)$. \ By applying Fourier transformation to the dissipation
part, the Schwinger-Keldysh action is given by:
\begin{eqnarray}
  S & = & \int_{- \infty}^{\infty} d t \left[ \sum_{\tmmathbf{k}}
  (a^{\dagger}_{\tmmathbf{k}+} (i \partial_t - \varepsilon_{\tmmathbf{k}})
  a_{\tmmathbf{k}+} - a^{\dagger}_{\tmmathbf{k}-} (i \partial_t -
  \varepsilon_{\tmmathbf{k}}) a_{\tmmathbf{k}-}) - \frac{U}{2 V}
  \sum_{\tmmathbf{k}, \tmmathbf{q}, \tmmathbf{p}} a_{\tmmathbf{k}+}^{\dagger}
  a_{\tmmathbf{p}+}^{\dagger} a_{\tmmathbf{p}-\tmmathbf{q}, +}
  a_{\tmmathbf{k}+\tmmathbf{q}, +} \right. \nonumber\\
  &  & + \frac{U^{\ast}}{2 V} \sum_{\tmmathbf{k}, \tmmathbf{q}, \tmmathbf{p}}
  a_{\tmmathbf{k}-}^{\dagger} a_{\tmmathbf{p}-}^{\dagger}
  a_{\tmmathbf{p}-\tmmathbf{q}, -} a_{\tmmathbf{k}+\tmmathbf{q}, -} - i
  \frac{\gamma}{V} \sum_{\tmmathbf{k}, \tmmathbf{q}, \tmmathbf{p}}
  a_{\tmmathbf{k}-}^{\dagger} a_{\tmmathbf{p}-}^{\dagger}
  a_{\tmmathbf{p}-\tmmathbf{q}, +} a_{\tmmathbf{k}+\tmmathbf{q}, +},
\end{eqnarray}
where $U = U_R - i \gamma$. Under the mean-field approximation, we seperate
the operators into the condensate parts and non-condensate parts. Assumed that
most of bosons in the system form the condensate, we have:
\begin{equation}
  a_0^{\dagger} a_0 \approx N, \sum_{\tmmathbf{k}} a_{\tmmathbf{k}}^{\dagger}
  a_{\tmmathbf{k}} \ll N
\end{equation}
Here $N$ is the total number of electrons in the system. Since $N \gg 1$, we
can equivalently assume that for condensate we have
\begin{equation}
  a_{0 +}^{\dagger} = a_{0 -}^{\dagger} = a_{0 +} = a_{0 -} \approx \sqrt{N}
\end{equation}
By applying the approximations, we simplify the interaction term as:
\begin{equation}
  \sum_{\tmmathbf{k}, \tmmathbf{q}, \tmmathbf{p}} a_{\tmmathbf{k}+}^{\dagger}
  a_{\tmmathbf{p}+}^{\dagger} a_{\tmmathbf{p}-\tmmathbf{q}, +}
  a_{\tmmathbf{k}+\tmmathbf{q}, +} \approx N^2 + N \sum_{\tmmathbf{k}}
  a_{-\tmmathbf{k}, +} a_{\tmmathbf{k}+} + N \sum_{\tmmathbf{k}}
  a_{\tmmathbf{k}+}^{\dagger} a_{-\tmmathbf{k}, +}^{\dagger} + 4 N
  \sum_{\tmmathbf{k}} a_{\tmmathbf{k}+}^{\dagger} a_{\tmmathbf{k}+}
  \label{mean1}
\end{equation}

\begin{equation}
  \sum_{\tmmathbf{k}, \tmmathbf{q}, \tmmathbf{p}} a_{\tmmathbf{k}-}^{\dagger}
  a_{\tmmathbf{p}-}^{\dagger} a_{\tmmathbf{p}-\tmmathbf{q}, -}
  a_{\tmmathbf{k}+\tmmathbf{q}, -} \approx N^2 + N \sum_{\tmmathbf{k}}
  a_{-\tmmathbf{k}, -} a_{\tmmathbf{k}-} + N \sum_{\tmmathbf{k}}
  a_{\tmmathbf{k}-}^{\dagger} a_{-\tmmathbf{k}, -}^{\dagger} + 4 N
  \sum_{\tmmathbf{k}} a_{\tmmathbf{k}-}^{\dagger} a_{\tmmathbf{k}-}
\end{equation}
\begin{equation}
  \sum_{\tmmathbf{k}, \tmmathbf{q}, \tmmathbf{p}} a_{\tmmathbf{k}-}^{\dagger}
  a_{\tmmathbf{p}-}^{\dagger} a_{\tmmathbf{p}-\tmmathbf{q}, +}
  a_{\tmmathbf{k}+\tmmathbf{q}, +} \approx N^2 + N \sum_{\tmmathbf{k}}
  a_{-\tmmathbf{k}, +} a_{\tmmathbf{k}+} + N \sum_{\tmmathbf{k}}
  a_{\tmmathbf{k}-}^{\dagger} a_{-\tmmathbf{k}, -}^{\dagger} + 4 N
  \sum_{\tmmathbf{k}} a_{\tmmathbf{k}-}^{\dagger} a_{\tmmathbf{k}+}
  \label{mean3}
\end{equation}
Then the action is simplified as:
\begin{eqnarray}
  S & = & \int_{- \infty}^{\infty} d t [\nobracket \sum_{\tmmathbf{k}}
  (a^{\dagger}_{\tmmathbf{k}+} (i \partial_t - \varepsilon_{\tmmathbf{k}})
  a_{\tmmathbf{k}+} - a^{\dagger}_{\tmmathbf{k}-} (i \partial_t -
  \varepsilon_{\tmmathbf{k}}) a_{\tmmathbf{k}-}) - \frac{U_R}{2 V} a_{0
  +}^{\dagger} a_{0 +}^{\dagger} a_{0, +} a_{0, +} + \frac{U_R}{2 V} a_{0
  -}^{\dagger} a_{0 -}^{\dagger} a_{0, -} a_{0, -} \nonumber\\
  &  & - \frac{U^{\ast} n}{2} \sum_{\tmmathbf{k}} a_{-\tmmathbf{k}, +}
  a_{\tmmathbf{k}+} - \frac{U n}{2} \sum_{\tmmathbf{k}}
  a_{\tmmathbf{k}+}^{\dagger} a_{-\tmmathbf{k}, +}^{\dagger} - 2 U n
  \sum_{\tmmathbf{k}} a_{\tmmathbf{k}+}^{\dagger} a_{\tmmathbf{k}+} +
  \frac{U^{\ast} n}{2} \sum_{\tmmathbf{k}} a_{-\tmmathbf{k}, -}
  a_{\tmmathbf{k}-} \nonumber\\
  &  & + \frac{U n}{2} \sum_{\tmmathbf{k}} a_{\tmmathbf{k}-}^{\dagger}
  a_{-\tmmathbf{k}, -}^{\dagger} + 2 U^{\ast} n \sum_{\tmmathbf{k}}
  a_{\tmmathbf{k}-}^{\dagger} a_{\tmmathbf{k}-} - 4 i \gamma n
  \sum_{\tmmathbf{k}} a_{\tmmathbf{k}-}^{\dagger} a_{\tmmathbf{k}+}
  \large{\large{}] \nobracket} \\
  & = & \int_{- \infty}^{\infty} d t \left[ \sum_{\tmmathbf{k}}
  (a^{\dagger}_{\tmmathbf{k}+} i \partial_t a_{\tmmathbf{k}+} -
  a^{\dagger}_{\tmmathbf{k}-} i \partial_t a_{\tmmathbf{k}-}) - H_+ + H_- - 4
  i \gamma n \sum_{\tmmathbf{k}} a_{\tmmathbf{k}-}^{\dagger} a_{\tmmathbf{k}+}
  \right], \label{action} 
\end{eqnarray}
where:
\begin{eqnarray}
  H_{\alpha} & = & \frac{U_R}{2 V} a_{0 \alpha}^{\dagger} a_{0
  \alpha}^{\dagger} a_{0 \alpha} a_{0 \alpha} + \sum_{\tmmathbf{k}} \left[
  (\varepsilon_{\tmmathbf{k}} + 2 U_R n - 2 i \alpha \gamma n) a_{\tmmathbf{k}
  \alpha}^{\dagger} a_{\tmmathbf{k} \alpha} + \frac{U^{\ast} n}{2}
  a_{-\tmmathbf{k} \alpha} a_{\tmmathbf{k} \alpha} + \frac{U n}{2}
  a_{\tmmathbf{k} \alpha}^{\dagger} a_{-\tmmathbf{k} \alpha}^{\dagger} \right]
  \nonumber\\
  & = & \frac{U_R n}{2} N + \sum_{\tmmathbf{k}} \left[
  (\varepsilon_{\tmmathbf{k}} + U_R n - 2 i \alpha \gamma n) a_{\tmmathbf{k}
  \alpha}^{\dagger} a_{\tmmathbf{k} \alpha} + \frac{U^{\ast} n}{2}
  a_{-\tmmathbf{k} \alpha} a_{\tmmathbf{k} \alpha} + \frac{U n}{2}
  a_{\tmmathbf{k} \alpha}^{\dagger} a_{-\tmmathbf{k} \alpha}^{\dagger} \right]
  \label{H} 
\end{eqnarray}
However, the most difficult part in action $\left( \ref{action} \right)$ is
the term at the end. Without this term, the effective Hamitonian is written in
the formula $\left( \ref{H} \right)$. We can see this is an non-Hermitian
Hamiltonian indeed. The dynamics of density matrix is given by:
\begin{equation}
  \frac{d \rho_{\alpha}}{d t} = - i [H_{\alpha}, \rho_{\alpha}]
\end{equation}
Here $[A, B] = A B - B A^{\dagger}$. Then let's diagonalize this Hamiltonian.
Firstly we rewrite the Hamiltonian in the matrix form:
\begin{equation}
  H_{\alpha} = \frac{U_R n}{2} N + \sum_{\tmmathbf{k}} \left(\begin{array}{cc}
    a_{\tmmathbf{k} \alpha}^{\dagger} & a_{-\tmmathbf{k} \alpha}
  \end{array}\right) \left(\begin{array}{cc}
    \frac{1}{2} (\varepsilon_{\tmmathbf{k}} + U_R n - 2 i \alpha \gamma n) &
    \frac{U n}{2}\\
    \frac{U^{\ast} n}{2} & \frac{1}{2} (\varepsilon_{\tmmathbf{k}} + U_R n + 2
    i \alpha \gamma n)
  \end{array}\right) \left(\begin{array}{c}
    a_{\tmmathbf{k} \alpha}\\
    a_{-\tmmathbf{k} \alpha}^{\dagger}
  \end{array}\right)
\end{equation}
Since two copies are just Hermitian conjugate to each other, we only consider
the $+$ branch for example. The exication spectrum of forward branch

is given by:
\begin{equation}
  (\varepsilon_{\tmmathbf{k}} + U_R n - 2 i \gamma n)^2 - \omega^2 - | U |^2
  n^2 = 0, | U |^2 = \sqrt{U^2 + \gamma^2}
\end{equation}
By solving the equation, we have:
\begin{equation}
  \omega = \sqrt{(\varepsilon_{\tmmathbf{k}} + U_R n - 2 i \gamma n)^2 - | U
  |^2 n^2} = \sqrt{\varepsilon_{\tmmathbf{k}}^2 + 2 \varepsilon_{\tmmathbf{k}}
  (U_R n - 2 i \gamma n) - 4 i \gamma U_R n^2 - 5 \gamma^2 n^2} \label{ES}
\end{equation}
We surprisingly find that the real part of this spectrum of bosons is gapped.
In the limit $\gamma \rightarrow 0$, it will become gapless again.

\subsection{Complete Excitation Spectrum in Bosonic System}

The exact excitation spectrum is given by the poles of Green's function.
However, there exists coupling between forward and backward branches. We can
no longer analyze the system as above. The Green's function should contain
both branches. Hence, we define the action in the energy space like:
\begin{equation}
  S = \int_{- \infty}^{\infty} d t \sum_{\tmmathbf{k}}
  \mathcal{L}_{\tmmathbf{k}} (t)
\end{equation}
\begin{equation}
  \mathcal{L}= \Psi_{\tmmathbf{k}}^{\dagger} (t) \left(\begin{array}{cccc}
    - \frac{\varepsilon_{\tmmathbf{k}} + U_R n - 2 i \gamma n - i
    \partial_t}{2} & - \frac{U n}{2} &  & \\
    - \frac{U^{\ast} n}{2} & - \frac{\varepsilon_{\tmmathbf{k}} + U_R n - 2 i
    \gamma n + i \partial_t}{2} &  & - 2 i \gamma n\\
    - 2 i \gamma n &  & \frac{\varepsilon_{\tmmathbf{k}} + U_R n + 2 i \gamma
    n - i \partial_t}{2} & \frac{U n}{2}\\
    &  & \frac{U^{\ast} n}{2} & \frac{\varepsilon_{\tmmathbf{k}} + U_R n + 2
    i \gamma n + i \partial_t}{2}
  \end{array}\right) \Psi_{\tmmathbf{k}} (t),
\end{equation}
where the spinor is defined by:
\begin{equation}
  \Psi_{\tmmathbf{k}} (t) = (a_{\tmmathbf{k}+} (t),
  a^{\dagger}_{-\tmmathbf{k}+} (t), a_{\tmmathbf{k}-} (t),
  a^{\dagger}_{-\tmmathbf{k}, -} (t))^T
\end{equation}
Under Fourier transformation, it can be simplified as:
\begin{equation}
  \mathcal{L}= \Psi_{\tmmathbf{k}}^{\dagger} \left(\begin{array}{cccc}
    - \frac{\varepsilon_{\tmmathbf{k}} + U_R n - 2 i \gamma n - \omega}{2} & -
    \frac{U n}{2} &  & \\
    - \frac{U^{\ast} n}{2} & - \frac{\varepsilon_{\tmmathbf{k}} + U_R n - 2 i
    \gamma n + \omega}{2} &  & - 2 i \gamma n\\
    - 2 i \gamma n &  & \frac{\varepsilon_{\tmmathbf{k}} + U_R n + 2 i \gamma
    n - \omega}{2} & \frac{U n}{2}\\
    &  & \frac{U^{\ast} n}{2} & \frac{\varepsilon_{\tmmathbf{k}} + U_R n + 2
    i \gamma n + \omega}{2}
  \end{array}\right) \Psi_{\tmmathbf{k}}
\end{equation}
Then the Green's function of this Lagrangian is given by the inverse of the
matrix:
\begin{equation}
  G (\tmmathbf{k}, \omega) = \left(\begin{array}{cccc}
    - \frac{\varepsilon_{\tmmathbf{k}} + U_R n - 2 i \gamma n - \omega}{2} & -
    \frac{U n}{2} &  & \\
    - \frac{U^{\ast} n}{2} & - \frac{\varepsilon_{\tmmathbf{k}} + U_R n - 2 i
    \gamma n + \omega}{2} &  & - 2 i \gamma n\\
    - 2 i \gamma n &  & \frac{\varepsilon_{\tmmathbf{k}} + U_R n + 2 i \gamma
    n - \omega}{2} & \frac{U n}{2}\\
    &  & \frac{U^{\ast} n}{2} & \frac{\varepsilon_{\tmmathbf{k}} + U_R n + 2
    i \gamma n + \omega}{2}
  \end{array}\right)^{- 1}
\end{equation}
The poles of this Green's function gives the form of excitation spectrum of
this open bosonic system. This can be given by:
\begin{equation}
  \omega^4 - a \omega^2 + b = 0,
\end{equation}
where
\begin{eqnarray}
  a & = & (\varepsilon_{\tmmathbf{k}} + U_R n - 2 i \gamma n)^2 +
  (\varepsilon_{\tmmathbf{k}} + U_R n + 2 i \gamma n)^2 - 2 | U |^2 n^2 \\
  b & = & | \varepsilon_{\tmmathbf{k}} + U_R n - 2 i \gamma n |^4 + (| U |
  n)^4 - 16 \gamma^2 n^2 | U |^2 n^2 - | U |^2 n^2 (\varepsilon_{\tmmathbf{k}}
  + U_R n - 2 i \gamma n)^2 \nonumber\\
  &  & - | U |^2 n^2 (\varepsilon_{\tmmathbf{k}} + U_R n - 2 i \gamma n)^2
  \nonumber\\
  & = & [(\varepsilon_{\tmmathbf{k}} + U_R n - 2 i \gamma n)^2 - | U |^2 n^2]
  [(\varepsilon_{\tmmathbf{k}} + U_R n + 2 i \gamma n)^2 - | U |^2 n^2] - 16
  \gamma^2 n^2 | U |^2 n^2 
\end{eqnarray}
From observation to these two expressions, we find that both are real
parameters. Then the energy spectrum is given by:
\begin{eqnarray}
  \omega & = & \pm \sqrt{\frac{M_{\tmmathbf{k}} + M_{\tmmathbf{k}}^{\ast} \pm
  \sqrt{(M_{\tmmathbf{k}} - M_{\tmmathbf{k}}^{\ast})^2 + 64 \gamma^2 n^2 | U
  |^2 n^2}}{2}} \nonumber\\
  & = & \pm \sqrt{\varepsilon_{\tmmathbf{k}} (\varepsilon_{\tmmathbf{k}} + 2
  U_R n) - 5 \gamma^2 n^2 \pm 4 \gamma n \sqrt{\gamma^2 n^2 -
  \varepsilon_{\tmmathbf{k}} (\varepsilon_{\tmmathbf{k}} + 2 U_R n)}},
\end{eqnarray}
where $M_{\tmmathbf{k}} = (\varepsilon_{\tmmathbf{k}} + U_R n - 2 i \gamma
n)^2 - | U |^2 n^2$. From this argument we can see no matter how small
$\gamma$ is, the boson condensates will decay. This can be seen from the
$\tmmathbf{k}= 0$ condensate component. The excitation spectrum reads as:
\begin{equation}
  \omega = \pm \sqrt{- 5 \gamma^2 n^2 \pm 4 \gamma^2 n^2} = \pm i \gamma n /
  \pm 3 i \gamma n
\end{equation}
We can see the decay of the condensate part no matter how little the
dissipation is. In the following part, we are going to use the Green's
function to consider the superfluid density in the system.

\section{Evolution of Superfluid Component in the System}

In this section we introduce the evolution of superfluid density of bosons.
Since the system is generally in dynamics and difficult to solve, we only
assue a case that $n \gamma \ll 1$ which represents quasi-steady state for the
interacting bosonic systems. Due to the complexity in Schwinger-Keldysh
action, we here turn to calculate the phase stiffness perturbed by velocity
introduced. In other words, since bosons are neutral in electricity, we
replace the Pierles substitution in electromagnetic field with $\tmmathbf{k}
\rightarrow \tmmathbf{k}- m\tmmathbf{v}$. The physical picture is the
dissipative bosonic fluid is contained in a cylinder moving with velocity
$\tmmathbf{v}$. In the comoving coordinate system, bosons flow with the
momentum $\tmmathbf{k}- m\tmmathbf{v}$. \ In this way we can still define the
phase stiffness with
\begin{equation}
  Q_{a b} = \frac{1}{2 V} \frac{\partial^2 F}{\partial v_a \partial v_b} |_{v
  = 0} \nobracket . \label{Qv}
\end{equation}
In this case, one can express the superfluid density as:
\begin{equation}
  Q = m \rho_s,
\end{equation}
which is similar with the previous problem for fermions. This is because we
consider the perturbation to the Schwinger-Keldysh theory as $\tmmathbf{k}
\rightarrow \tmmathbf{k}- m\tmmathbf{v}$ for the forward contour and
$\tmmathbf{k} \rightarrow \tmmathbf{k}+ m\tmmathbf{v}$ for the backward
contour. In this sense, the Schwinger-Keldysh action is transformed as
\begin{equation}
  S \rightarrow S -\tmmathbf{v} \cdot (\tmmathbf{J}_+ +\tmmathbf{J}_-),
\end{equation}
where $\tmmathbf{J}_+ = \int d \tau d\tmmathbf{r} \hat{\tmmathbf{j}}_+,
\tmmathbf{J}_+ = \int d \tau d\tmmathbf{r} \hat{\tmmathbf{j}}_-$ with
$\hat{\tmmathbf{j}}_{\alpha} = m\tmmathbf{v}a_{\tmmathbf{k} \alpha}^{\dagger}
a_{\tmmathbf{k} \alpha}$ for $\alpha = \pm$ which represents the mass current
operator. Therefore, the averaged mass current can be substituted as
\begin{equation}
  \langle J_a \rangle = - \frac{1}{2 V} \frac{\partial F}{\partial v_a},
\end{equation}
where $F = - T \log Z$ is the free energy of the system for the quasi-steady
state. This is because
\begin{equation}
  - \frac{1}{V} \frac{\partial F}{\partial \tmmathbf{v}} = \frac{1}{\beta V}
  \frac{1}{Z} \frac{\partial Z}{\partial \tmmathbf{v}} = \frac{1}{\beta V}
  \frac{1}{Z} \int D \phi e^{- S (\phi) -\tmmathbf{v} \cdot (\tmmathbf{J}_+
  +\tmmathbf{J}_-)} (\tmmathbf{J}_+ +\tmmathbf{J}_-) |_{v = 0} \nobracket =
  \langle \hat{\tmmathbf{j}}_+ + \hat{\tmmathbf{j}}_- \rangle = 2 \langle
  \tmmathbf{J} \rangle .
\end{equation}
Here we define $\hat{\tmmathbf{J}} = \frac{1}{2} (\hat{\tmmathbf{j}}_+ +
\hat{\tmmathbf{j}}_-)$ as the classical current operator. Indeed for closed
system, we have the exact relation: $\langle \tmmathbf{j}_+ \rangle = \langle
\hat{\tmmathbf{j}}_- \rangle$ and hence
\begin{equation}
  \langle \tmmathbf{j}_+ \rangle = \langle \tmmathbf{J} \rangle = - \frac{1}{2
  V} \frac{\partial F}{\partial \tmmathbf{v}} . \label{Jopen}
\end{equation}
This is a little different from $\langle \tmmathbf{J} \rangle = - \frac{1}{V}
\frac{\partial F}{\partial \tmmathbf{v}}$ in the closed system since in
\eqref{Jopen} the free energy contains two parts: forward and backward and
they are the same for the closed system. Then we move to consider how $Q$
evolves with time. We begin from calculating the free energy of bosonic
systems with Green's function on the Schwinger-Keldysh contour in imaginary
time. Since partition function is given by:
\begin{eqnarray}
  Z & = & \int D [a_{\tmmathbf{k}+} (\tau), a^{\dagger}_{-\tmmathbf{k}+}
  (\tau), a_{\tmmathbf{k}-} (\tau), a^{\dagger}_{-\tmmathbf{k}, -} (\tau)]
  e^{- S (a_{\tmmathbf{k}+} (\tau), a^{\dagger}_{-\tmmathbf{k}+} (\tau),
  a_{\tmmathbf{k}-} (\tau), a^{\dagger}_{-\tmmathbf{k}, -} (\tau))}
  \nonumber\\
  & = & \int D [a_{\tmmathbf{k}+} (i \omega_n), a^{\dagger}_{-\tmmathbf{k}+}
  (i \omega_n), a_{\tmmathbf{k}-} (i \omega_n), a^{\dagger}_{-\tmmathbf{k}, -}
  (i \omega_n)] e^{\frac{1}{2} \Psi_{\tmmathbf{k}}^{\dagger} (i \omega_n) G^{-
  1} (\tmmathbf{k}, i \omega_n) \Psi_{\tmmathbf{k}} (i \omega_n)}, 
\end{eqnarray}
where we define
\begin{equation}
  \Psi_{\tmmathbf{k}} (i \omega_n) = (a_{\tmmathbf{k}+} (i \omega_n),
  a^{\dagger}_{-\tmmathbf{k}+} (i \omega_n), a_{\tmmathbf{k}-} (i \omega_n),
  a^{\dagger}_{-\tmmathbf{k}, -} (i \omega_n))^T .
\end{equation}
Here under the frame of imaginary time, we rewrite the action as:
\begin{eqnarray}
  S & = & \int_0^{\beta} d \tau [\nobracket \sum_{\tmmathbf{k}}
  (a^{\dagger}_{\tmmathbf{k}+} (\partial_{\tau} + \varepsilon_{\tmmathbf{k}})
  a_{\tmmathbf{k}+} - a^{\dagger}_{\tmmathbf{k}-} (\partial_{\tau} +
  \varepsilon_{\tmmathbf{k}}) a_{\tmmathbf{k}-}) + \frac{U_R}{2 V} a_{0
  +}^{\dagger} a_{0 +}^{\dagger} a_{0, +} a_{0, +} - \frac{U_R}{2 V} a_{0
  -}^{\dagger} a_{0 -}^{\dagger} a_{0, -} a_{0, -} \nonumber\\
  &  & + \frac{U^{\ast} n}{2} \sum_{\tmmathbf{k}} a_{-\tmmathbf{k}, +}
  a_{\tmmathbf{k}+} + \frac{U n}{2} \sum_{\tmmathbf{k}}
  a_{\tmmathbf{k}+}^{\dagger} a_{-\tmmathbf{k}, +}^{\dagger} + 2 U n
  \sum_{\tmmathbf{k}} a_{\tmmathbf{k}+}^{\dagger} a_{\tmmathbf{k}+} -
  \frac{U^{\ast} n}{2} \sum_{\tmmathbf{k}} a_{-\tmmathbf{k}, -}
  a_{\tmmathbf{k}-} \nonumber\\
  &  & - \frac{U n}{2} \sum_{\tmmathbf{k}} a_{\tmmathbf{k}-}^{\dagger}
  a_{-\tmmathbf{k}, -}^{\dagger} - 2 U^{\ast} n \sum_{\tmmathbf{k}}
  a_{\tmmathbf{k}-}^{\dagger} a_{\tmmathbf{k}-} + 4 i \gamma n
  \sum_{\tmmathbf{k}} a_{\tmmathbf{k}-}^{\dagger} a_{\tmmathbf{k}+}
  \large{\large{}] \nobracket} \\
  & = & \int_{- \infty}^{\infty} d t \left[ \sum_{\tmmathbf{k}}
  (a^{\dagger}_{\tmmathbf{k}+} \partial_{\tau} a_{\tmmathbf{k}+} -
  a^{\dagger}_{\tmmathbf{k}-} \partial_{\tau} a_{\tmmathbf{k}-}) + H_+ - H_- +
  4 i \gamma n \sum_{\tmmathbf{k}} a_{\tmmathbf{k}-}^{\dagger}
  a_{\tmmathbf{k}+} \right], 
\end{eqnarray}
where we define the $H_+$ and $H_-$ above in (\ref{H}) and
\begin{equation}
  G (\tmmathbf{k}, i \omega_n) = \left(\begin{array}{cccc}
    a_1 & b & 0 & 0\\
    b^{\ast} & a_2 & 0 & c\\
    c & 0 & - a_2^{\ast} & - b\\
    0 & 0 & - b^{\ast} & - a_1^{\ast}
  \end{array}\right)^{- 1} \label{Green2}
\end{equation}
, where $a_1 = - (\varepsilon_{\tmmathbf{k}- m\tmmathbf{v}} + U_R n - 2 i
\gamma n - i \omega_n), b = - U n, a_2 = - (\varepsilon_{\tmmathbf{k}+
m\tmmathbf{v}} + U_R n - 2 i \gamma n + i \omega_n), c = - 4 i \gamma n$.

In this case, we integrate the all bosonic degrees of freedom and obtain the
effective action:
\begin{equation}
  Z = e^{- S_{\tmop{eff}}} = e^{- \beta F} \Rightarrow F = - \frac{1}{\beta}
  S_{\tmop{eff}}
\end{equation}
Hence, the free energy takes the form of:
\begin{equation}
  F = - \frac{1}{\beta} \sum_k \tmop{Tr} \ln [- G (\tmmathbf{k}, i
  \omega_n)^{- 1}]
\end{equation}
. Here $\sum_k = \sum_{\tmmathbf{k}} \sum_{i \omega_n}$.

Then we move to calculate $\left( \ref{Qv} \right)$. The first derivative
gives the expression for the bosonic current.
\begin{equation}
  - \langle J_a \rangle = \frac{1}{2 V} \frac{\partial F}{\partial v_a} = -
  \frac{m}{2 \beta V} \sum_k \text{Tr} [(\sigma_z \otimes \sigma_z) \nabla_a
  \varepsilon_{\tmmathbf{k}- m\tmmathbf{v} \sigma_0 \otimes \sigma_z} G
  (\tmmathbf{k}, i \omega_n)]
\end{equation}
In this case the phase stiffness is given by:
\begin{eqnarray}
  Q_{a b} & = & \frac{- 1}{2 V} \frac{\partial^2 F}{\partial v_a \partial v_b}
  |_{v = 0} \nobracket \nonumber\\
  & = & - \frac{m^2}{2 \beta V} \sum_k \text{Tr} [(\sigma_z \otimes \sigma_0)
  \nabla_{a b}^2 \varepsilon_{\tmmathbf{k}} G (\tmmathbf{k}, i \omega_n)] +
  \frac{m^2}{2 \beta V} \sum_k \nabla_a \varepsilon_k \nabla_b \varepsilon_k
  \text{Tr} [(\sigma_z \otimes \sigma_z) G (\sigma_z \otimes \sigma_z) G] 
\end{eqnarray}
Here we apply the equality: $\nabla_b G = - G  \nabla_b G^{- 1} G$. Also we
use the part integral to the first part and obtain that:
\begin{equation}
  Q_{a b} = - \frac{m^2}{2 \beta V} \sum_k \nabla_a \varepsilon_k \nabla_b
  \varepsilon_k \left\{ \text{Tr} [(\sigma_z \otimes \sigma_z) G (\sigma_z
  \otimes \sigma_z) G] - \text{Tr} [(\sigma_z \otimes \sigma_0) G (\sigma_z
  \otimes \sigma_0) G] \right\} \label{Qab}
\end{equation}
Let's firstly consider a trace: $\tmop{Tr} [A G A G]$. We transform it into:
\begin{eqnarray}
  \tmop{Tr} [A G A G] & = & \tmop{Tr} [A [G, A] G] + \tmop{Tr} [A^2 G^2]
  \nonumber\\
  & = & \tmop{Tr} [[G, A] [G, A]] + \tmop{Tr} [[G, A] A G] + \tmop{Tr} [A^2
  G^2] \nonumber\\
  & = & \tmop{Tr} [[G, A] [G, A]] + 2 \tmop{Tr} [A^2 G^2] - \tmop{Tr} [A G A
  G] 
\end{eqnarray}
Hence, we reach a conclusion that:
\begin{equation}
  \tmop{Tr} [A G A G] = \tmop{Tr} [A^2 G^2] + \frac{1}{2} \tmop{Tr} [[G, A]^2]
\end{equation}
In the expression $\left( \ref{Qab} \right)$, we can replace $A$ with
$\sigma_z \otimes \sigma_z$ and $\sigma_z \otimes \sigma_0$. In this case,
both of the matrices satisfy that $A^2 =\mathbbm{1}$. Therefore, we simplify
the formula into:
\begin{equation}
  Q_{a b} = - \frac{m^2}{4 \beta V} \sum_k \nabla_a \varepsilon_k \nabla_b
  \varepsilon_k \left\{ \text{Tr} [[G, \sigma_z \otimes \sigma_z]^2] -
  \text{Tr} [[G, \sigma_z \otimes \sigma_0]^2] \right\} \label{Qab2}
\end{equation}
In this way, we firstly rewrite the Green's function into:
\begin{equation}
  G = \left(\begin{array}{cc}
    G_{11} & G_{12}\\
    G_{21} & G_{22}
  \end{array}\right)
\end{equation}
With the help of Green's function before (Eq. (\ref{Green2})), we have:
\begin{eqnarray}
  G_{11} & = & \frac{1}{| G^{- 1} |} \left(\begin{array}{cc}
    a_2 (a_1^{\ast} a_2^{\ast} - | b |^2) & - b (a_1^{\ast} a_2^{\ast} - | b
    |^2)\\
    b^{\ast} (| b |^2 + c^2 - a_1^{\ast} a_2^{\ast}) & a_1 (a_1^{\ast}
    a_2^{\ast} - | b |^2)
  \end{array}\right) \\
  G_{12} & = & \frac{1}{| G^{- 1} |} \left(\begin{array}{cc}
    | b |^2 c & - a_2^{\ast} b c\\
    - a_1 b^{\ast} c & a_1 a_2^{\ast} c
  \end{array}\right) \\
  G_{21} & = & \frac{1}{| G^{- 1} |} \left(\begin{array}{cc}
    a_2 a_1^{\ast} c & - a_1^{\ast} b c\\
    - a_2 b^{\ast} c & | b |^2 c
  \end{array}\right) \\
  G_{22} & = & \frac{1}{| G^{- 1} |} \left(\begin{array}{cc}
    - a_1^{\ast} (a_1 a_2 - | b |^2) & - b (| b |^2 + c^2 - a_1 a_2)\\
    b^{\ast} (a_1 a_2 - | b |^2) & - a_2^{\ast} (a_1 a_2 - | b |^2)
  \end{array}\right) \,, 
\end{eqnarray}
where we follow the notation above. Here we define the determinant of the
Green's function as $| G |$ with the form of:
\begin{equation}
  | G^{- 1} | = | a_1 |^2 | a_2 |^2 - a_1 a_2 | b |^2 + | b |^2 c^2 -
  a_1^{\ast} a_2^{\ast} | b |^2 + | b |^4 \label{deter}
\end{equation}
Then let's focus on the commutators in the phase stiffness$\left( \ref{Qab2}
\right)$.
\begin{eqnarray}
  \tmop{Tr} [G, \sigma_z \otimes \sigma_z]^2 & = & \tmop{Tr} \left[
  \left(\begin{array}{cc}
    G_{11} & G_{12}\\
    G_{21} & G_{22}
  \end{array}\right), \sigma_z \otimes \sigma_z \right]^2 \nonumber\\
  & = & \tmop{Tr} [G_{11}, \sigma_z]^2 + \tmop{Tr} [G_{22}, \sigma_z]^2 - 2
  \tmop{Tr} [\{ G_{12}, \sigma_z \} \{ G_{21}, \sigma_z \}] \\
  \text{Tr} [[G, \sigma_z \otimes \sigma_0]^2] & = & \tmop{Tr} \left[
  \left(\begin{array}{cc}
    G_{11} & G_{12}\\
    G_{21} & G_{22}
  \end{array}\right), \sigma_z \otimes \sigma_0 \right]^2 \nonumber\\
  & = & - 4 \tmop{Tr} [\{ G_{12}, G_{21} \}] \nonumber\\
  & = & - 8 \tmop{Tr} [G_{12} G_{21}] 
\end{eqnarray}
Here we define the anti-commutator $\{ A, B \} = A B + B A$. The first trace
has four components to calculate:
\begin{eqnarray}
  \tmop{Tr} [G_{11}, \sigma_z]^2 & = & \frac{1}{| G^{- 1} |^2} 8 | b |^2
  (a_1^{\ast} a_2^{\ast} - | b |^2) (| b |^2 + c^2 - a_1^{\ast} a_2^{\ast})
  \label{G11} \\
  \tmop{Tr} [G_{22}, \sigma_z]^2 & = & \frac{1}{| G^{- 1} |^2} 8 | b |^2 (a_1
  a_2 - | b |^2) (| b |^2 + c^2 - a_1 a_2) \\
  \tmop{Tr} [\{ G_{12}, \sigma_z \} \{ G_{21}, \sigma_z \}] & = & \frac{1}{|
  G^{- 1} |^2} 4 | b |^2 c^2 (a_1^{\ast} a_2 + a_1 a_2^{\ast}) \\
  \tmop{Tr} [G_{12} G_{21}] & = & \frac{1}{| G^{- 1} |^2} | b |^2 c^2 [a_2
  a_1^{\ast} + | a_2 |^2 + | a_1 |^2 + a_1 a_2^{\ast}] \label{G12} 
\end{eqnarray}
With the help of $\left( \ref{G11} - \ref{G12} \right)$, we have:


\begin{equation}
  Q_{a b} = \frac{4 m^2}{\beta V} \sum_k \nabla_a \varepsilon_k \nabla_b
  \varepsilon_k \frac{- | b |^2}{2 | G^{- 1} |^2} \left[ (a_1^{\ast}
  a_2^{\ast} - | b |^2) (| b |^2 + c^2 - a_1^{\ast} a_2^{\ast}) + \left( a_1
  a_2 - | b |^2 \right) (| b |^2 + c^2 - a_1 a_2) - c^2 (| a_2 |^2 + | a_1
  |^2) \right]
\end{equation}
Since we here only concern about the diagonal part, we rewrite the $Q_{a b}$
into $Q (T) \delta_{a b}$. In this case, we obtain that:
\begin{equation}
  Q = \frac{4}{3 \beta} \sum_n \int_{- \infty}^{\infty} k^4 d k \frac{- | b
  |^2}{4 \pi^2 | G^{- 1} |^2} \left[ (a_1^{\ast} a_2^{\ast} - | b |^2) (| b
  |^2 + c^2 - a_1^{\ast} a_2^{\ast}) + \left( a_1 a_2 - | b |^2 \right) (| b
  |^2 + c^2 - a_1 a_2) - c^2 (| a_2 |^2 + | a_1 |^2) \right] \,,
  \label{Qboson}
\end{equation}
where we substitute the variable $\varepsilon = \varepsilon_{\tmmathbf{k}} +
U_R n$. We firstly deal with the determinant $| G |$. With the help of Eq.
$\left(  \ref{deter} \right)$, we have:
\begin{eqnarray}
  | G^{- 1} | & = & \varepsilon^4 + 2 (4 \gamma^2 n^2 + \omega_n^2 - | U |^2
  n^2) \varepsilon^2 + (4 \gamma^2 n^2 - \omega_n^2)^2 - 2 (4 \gamma^2 n^2 +
  \omega_n^2) | U |^2 n^2 + | U |^4 n^4 \nonumber\\
  & \equiv & \omega_n^4 + 2 k_1 \omega_n^2 + k_2 
\end{eqnarray}
where $k_1 = (\varepsilon^2 - 4 \gamma^2 n^2 - | U |^2 n^2), k_2 =
(\varepsilon^2 + 4 \gamma^2 n^2 - | U |^2 n^2)^2$. The numerator of it can be
shown by:
\begin{eqnarray}
  &  & - [(\varepsilon + 2 i \gamma n)^2 + \omega_n^2 - | U |^2 n^2]^2 -
  [(\varepsilon - 2 i \gamma n)^2 + \omega_n^2 - | U |^2 n^2]^2 + 32 \gamma^2
  n^2 (| U |^2 n^2 + 8 \gamma^2 n^2) \nonumber\\
  & = & - 2 [\omega_n^4 + 2 k_1 \omega_n^2 + k_1^2 - 16 \gamma^2 n^2
  \varepsilon^2] + 32 \gamma^2 n^2 (| U |^2 n^2 + 8 \gamma^2 n^2) 
\end{eqnarray}
Let's now consider the case with $\gamma = 0$ where there is no dissipation
and the system is closed. In such case, the quantities $a_1, a_2, b, c$ become
\begin{equation}
  a_1 = i \omega_n - \varepsilon_{\tmmathbf{k}} - U n, a_2 = - i \omega_n -
  \varepsilon_{\tmmathbf{k}} - U n, b = - U n, c = 0
\end{equation}
Then we substitute these quantities into superfluid density and obtain
\begin{equation}
  Q = \frac{4}{3 \beta} \sum_n \int_{- \infty}^{\infty} \frac{k^4}{2 \pi^2} d
  k \frac{U^2 n^2}{[\omega_n^2 + \varepsilon_{\tmmathbf{k}}^2 + 2
  \varepsilon_{\tmmathbf{k}} U n]^2} \label{Qzero}
\end{equation}
With the similar procedure, we can also calculate $Q'$ in the closed system
without using Schwinger-Keldysh Green's function and find its relation with
$Q$ as $Q' = Q$. The proof is as below.

The Green's function of the forward contour can be written as:
\begin{equation}
  G = - \left(\begin{array}{cc}
    \varepsilon_{\tmmathbf{k}- m\tmmathbf{v}} + U n - i \omega_n & U n\\
    U n & \varepsilon_{\tmmathbf{k}+ m\tmmathbf{v}} + U n + i \omega_n
  \end{array}\right)^{- 1} = - (\varepsilon_{\tmmathbf{k}- m\tmmathbf{v}
  \sigma_z} + U n + U n \sigma_x - i \omega_n \sigma_z)^{- 1}
\end{equation}
Then we still consider the second derivative to free energy. The first
derivative gives the expression for the bosonic current.
\begin{equation}
  - \langle J_a \rangle = \frac{1}{V} \frac{\partial F}{\partial v_a} = -
  \frac{m}{\beta V} \sum_k \text{Tr} [\sigma_z \nabla_a
  \varepsilon_{\tmmathbf{k}- m\tmmathbf{v} \sigma_z} G (\tmmathbf{k}, i
  \omega_n)]
\end{equation}
In this case the phase stiffness is also given by:
\begin{eqnarray}
  Q'_{a b} & = & - \frac{1}{V} \frac{\partial^2 F}{\partial v_a \partial v_b}
  |_{v = 0} \nobracket \nonumber\\
  & = & - \frac{m^2}{\beta V} \sum_k \text{Tr} [\nabla_{a b}^2
  \varepsilon_{\tmmathbf{k}} G (\tmmathbf{k}, i \omega_n)] + \frac{m^2}{\beta
  V} \sum_k \nabla_a \varepsilon_k \nabla_b \varepsilon_k \text{Tr} [\sigma_z
  G \sigma_z G] \\
  & = & - \frac{m^2}{\beta V} \sum_k \nabla_a \varepsilon_k \nabla_b
  \varepsilon_k \text{Tr} [\sigma_z G \sigma_z G - G^2] \nonumber\\
  & = & - \frac{m^2}{2 \beta V} \sum_k \nabla_a \varepsilon_k \nabla_b
  \varepsilon_k \text{Tr} [\sigma_z, G]^2 
\end{eqnarray}
By explicitly writing down the form of the Green's function, we have
\begin{equation}
  G = \frac{1}{(\omega_n^2 + \varepsilon_{\tmmathbf{k}}^2 + 2
  \varepsilon_{\tmmathbf{k}} U n)} \left(\begin{array}{cc}
    - \varepsilon_{\tmmathbf{k}} - U n - i \omega_n & U n\\
    U n & - \varepsilon_{\tmmathbf{k}} - U n + i \omega_n
  \end{array}\right)
\end{equation}
and the commutator has the form of
\begin{equation}
  [\sigma_z, G] = \frac{1}{(\omega_n^2 + \varepsilon_{\tmmathbf{k}}^2 + 2
  \varepsilon_{\tmmathbf{k}} U n)} \left(\begin{array}{cc}
    0 & - 2 U n\\
    2 U n & 0
  \end{array}\right)
\end{equation}
Therefore, the behaviour of $Q'$ is given by
\begin{equation}
  Q' = \frac{4 m^2}{\beta V} \sum_k \nabla_a \varepsilon_k \nabla_b
  \varepsilon_k \frac{U^2 n^2}{(\omega_n^2 + \varepsilon_{\tmmathbf{k}}^2 + 2
  \varepsilon_{\tmmathbf{k}} U n)^2}
\end{equation}
By replacing the energy $\varepsilon_{\tmmathbf{k}}$ with
$\varepsilon_{\tmmathbf{k}} = k^2 / 2 m$, we have:
\begin{equation}
  Q' = \frac{4 m^2}{\beta} \sum_n \int \frac{d^3 k}{(2 \pi)^3} \frac{k_a
  k_b}{m^2} \frac{U^2 n^2}{(\omega_n^2 + \varepsilon_{\tmmathbf{k}}^2 + 2
  \varepsilon_{\tmmathbf{k}} U n)^2} = \frac{4}{3 \beta} \sum_n \int \frac{k^4
  d k}{2 \pi^2} \frac{U^2 n^2}{(\omega_n^2 + \varepsilon_{\tmmathbf{k}}^2 + 2
  \varepsilon_{\tmmathbf{k}} U n)^2}
\end{equation}
By comparing with (\ref{Qzero}), we have the conclusion
\begin{equation}
  Q' = Q.
\end{equation}
In the zero-temperature limit, the superfluid becomes
\begin{eqnarray}
  Q' & = & \frac{4}{3} \int \frac{d \omega}{2 \pi} \int \frac{k^4 d k}{2
  \pi^2} \frac{U^2 n^2}{(\omega_n^2 + \varepsilon_{\tmmathbf{k}}^2 + 2
  \varepsilon_{\tmmathbf{k}} U n)^2} \nonumber\\
  & = & \frac{2}{3} \int_0^{\infty} \frac{k^4 d k}{2 \pi^2} \frac{U^2
  n^2}{[\varepsilon_{\tmmathbf{k}}^2 + 2 \varepsilon_{\tmmathbf{k}} U n]^{3 /
  2}} \nonumber\\
  & = & \frac{1}{3 \pi^2} \sqrt{n^3 U^3} m^{5 / 2} \nonumber\\
  & \propto & (U n)^{3 / 2} \label{QQ} 
\end{eqnarray}
This is consistent with the result calculated by linear response theory.$(Q =
m \rho_s)$
\begin{equation}
  \rho_s = \frac{1}{3 \pi^2} \sqrt{(n U)^3 m^3}
\end{equation}
Then we return back to the phase stiffness.
\begin{equation}
  Q = \frac{2}{3 \beta} \sum_n \int_{- \infty}^{\infty} k^4 d k \frac{- | b
  |^2}{2 \pi^2 | G^{- 1} |^2} [\alpha^{\ast} + \alpha - c^2 (| a_2 |^2 + | a_1
  |^2)],
\end{equation}
where $\alpha = (a_1 a_2 - | b |^2) (| b |^2 + c^2 - a_1 a_2)$. General
solution to this phase stiffness is difficult to solve. However, We can also
consider another special case for this quantity: $U_R = 0$. In this way, we
have
\begin{equation}
  a_1 = - (\varepsilon_{\tmmathbf{k}} - 2 i \gamma n - i \omega_n), b = i
  \gamma n, a_2 = - (\varepsilon_{\tmmathbf{k}} - 2 i \gamma n + i \omega_n),
  c = - 4 i \gamma n
\end{equation}
The numerator becomes
\begin{equation}
  - 2 [\omega_n^4 + 2 k_1 \omega_n^2 + k_1^2 - 16 \gamma^2 n^2
  \varepsilon^2_{\tmmathbf{k}}] + 288 \gamma^4 n^4
\end{equation}
, where $k_1 = (\varepsilon^2_{\tmmathbf{k}} - 5 \gamma^2 n^2), k_2 =
(\varepsilon^2_{\tmmathbf{k}} + 3 \gamma^2 n^2)^2$. Besides, the denominator
becomes
\begin{equation}
  | G^{- 1} |^2 = (\omega_n^4 + 2 k_1 \omega_n^2 + k_2)^2
\end{equation}
Hence, the phase stiffness can be rewritten as
\begin{equation}
  Q = \frac{(2 m)^{5 / 2} \gamma^2 n^2}{3 \pi^2} \int_{- \infty}^{\infty} d
  \omega \int_0^{\infty} \varepsilon_{\tmmathbf{k}}^{3 / 2} d
  \varepsilon_{\tmmathbf{k}} \frac{[\omega^4 + 2 k_1 \omega^2 + k_1^2 - 16
  \gamma^2 n^2 \varepsilon^2_{\tmmathbf{k}}] - 144 \gamma^4 n^4}{(\omega^4 + 2
  k_1 \omega^2 + k_2)^2} \,,
\end{equation}
By substitude the integrated variables as $\omega \rightarrow \gamma n \omega,
\varepsilon_{\tmmathbf{k}} \rightarrow \gamma n \varepsilon_{\tmmathbf{k}}$,
we have
\begin{equation}
  Q = \frac{(2 m)^{5 / 2}}{3 \pi^2} (\gamma n)^{3 / 2} \times A \label{QU=0}
\end{equation}
, where $A$ takes the form of
\begin{equation}
  A = \int_{- \infty}^{\infty} d x \int_0^{\infty} y^{3 / 2} d y \frac{[x^4 +
  2 (y^2 - 5) x^2 + (y^2 - 5)^2 - 16 y^2] - 144}{(x^4 + 2 (y^2 - 5) x^2 + (y^2
  + 3)^2)^2} \neq 0
\end{equation}
From the expression $\left( \ref{QU=0} \right)$, we can see when $U_R = 0$,
the density of superfluid in quasi-steady state is propotional to $(\gamma
n)^{3 / 2}$ with $n (t) \simeq n (0) / (1 + 2 \gamma n (0) t)$. Compared with
the expression (\ref{QQ}), we can see dissipation here can be also considered
as a kind of effective interaction to the system. However, to admit such a
quasi-steady state, we have to consider the condition $n (0) \gamma \ll 1$.

\section{Dephasing Operator}

We still use the interacting-boson Hamiltonian (\ref{Hamiltonian}) and
Lindblad equation (\ref{Lindbland}) to describe the system. In this section we
replace the Lindblad operator with $L_i = a_i^{\dagger} a_i$ and observe the
excitation spectrum and superfluidity. In this case the Schwinger-Keldysh
action in the momentum and time space is given by
\begin{eqnarray}
  S & = & \int_{- \infty}^{\infty} d t \left[ \sum_{\tmmathbf{k}}
  (a^{\dagger}_{\tmmathbf{k}+} (i \partial_t - \varepsilon_{\tmmathbf{k}})
  a_{\tmmathbf{k}+} - a^{\dagger}_{\tmmathbf{k}-} (i \partial_t -
  \varepsilon_{\tmmathbf{k}}) a_{\tmmathbf{k}-}) - \frac{U_R}{2 V}
  \sum_{\tmmathbf{k}, \tmmathbf{q}, \tmmathbf{p}} a_{\tmmathbf{k}+}^{\dagger}
  a_{\tmmathbf{p}+}^{\dagger} a_{\tmmathbf{p}-\tmmathbf{q}, +}
  a_{\tmmathbf{k}+\tmmathbf{q}, +} \right. \nonumber\\
  &  & + \frac{U_R}{2 V} \sum_{\tmmathbf{k}, \tmmathbf{q}, \tmmathbf{p}}
  a_{\tmmathbf{k}-}^{\dagger} a_{\tmmathbf{p}-}^{\dagger}
  a_{\tmmathbf{p}-\tmmathbf{q}, -} a_{\tmmathbf{k}+\tmmathbf{q}, -} + \frac{i
  \gamma}{2 V} \sum_{\tmmathbf{k}, \tmmathbf{q}, \tmmathbf{p}}
  a_{\tmmathbf{k}+}^{\dagger} a_{\tmmathbf{p}-\tmmathbf{q}, +}
  a_{\tmmathbf{p}+}^{\dagger} a_{\tmmathbf{k}+\tmmathbf{q}, +} \nonumber\\
  &  & + \frac{i \gamma}{2 V} \sum_{\tmmathbf{k}, \tmmathbf{q}, \tmmathbf{p}}
  a_{\tmmathbf{k}-}^{\dagger} a_{\tmmathbf{p}-\tmmathbf{q}, -}
  a_{\tmmathbf{p}-}^{\dagger} a_{\tmmathbf{k}+\tmmathbf{q}, -} - i
  \frac{\gamma}{V} \left. \sum_{\tmmathbf{k}, \tmmathbf{q}, \tmmathbf{p}}
  a_{\tmmathbf{k}+}^{\dagger} a_{\tmmathbf{p}-\tmmathbf{q}, +}
  a_{\tmmathbf{p}-}^{\dagger} a_{\tmmathbf{k}+\tmmathbf{q}, -} \right] 
\end{eqnarray}
Then we apply the mean-field approximation
\begin{equation}
  a_{0 +}^{\dagger} = a_{0 -}^{\dagger} = a_{0 +} = a_{0 -} \approx \sqrt{N} .
\end{equation}
With similar analysis in (\ref{mean1}-\ref{mean3}), we have
\begin{eqnarray}
  S & = & \int_{- \infty}^{\infty} d t \left[ \sum_{\tmmathbf{k}}
  (a^{\dagger}_{\tmmathbf{k}+} (i \partial_t - \varepsilon_{\tmmathbf{k}} - U
  n) a_{\tmmathbf{k}+} - a^{\dagger}_{\tmmathbf{k}-} (i \partial_t -
  \varepsilon_{\tmmathbf{k}} - U^{\ast} n) a_{\tmmathbf{k}-}) \right.
  \nonumber\\
  &  & - \frac{U n}{2} \sum_{\tmmathbf{k}} a_{\tmmathbf{k}+}^{\dagger}
  a_{-\tmmathbf{k}, +}^{\dagger} - \frac{U n}{2} \sum_{\tmmathbf{k}}
  a_{\tmmathbf{k}+} a_{-\tmmathbf{k}, +} + \frac{U^{\ast} n}{2}
  \sum_{\tmmathbf{k}} a_{\tmmathbf{k}-}^{\dagger} a_{-\tmmathbf{k},
  -}^{\dagger} + \frac{U^{\ast} n}{2} \sum_{\tmmathbf{k}} a_{\tmmathbf{k}-}
  a_{-\tmmathbf{k}, -} \nonumber\\
  &  & - i \gamma n \left. \sum_{\tmmathbf{k}} (a_{\tmmathbf{k}+}
  a_{-\tmmathbf{k}, -} + a_{\tmmathbf{k}+}^{\dagger} a_{-\tmmathbf{k},
  -}^{\dagger} + a_{\tmmathbf{k}-}^{\dagger} a_{\tmmathbf{k}+} +
  a_{\tmmathbf{k}+}^{\dagger} a_{\tmmathbf{k}-}) \right] \\
  & = & \int_{- \infty}^{\infty} d t \left[ \sum_{\tmmathbf{k}}
  (a^{\dagger}_{\tmmathbf{k}+} i \partial_t a_{\tmmathbf{k}+} -
  a^{\dagger}_{\tmmathbf{k}-} i \partial_t a_{\tmmathbf{k}-}) - H_+ + H_-
  \right. \nonumber\\
  &  & - i \gamma n \left. \sum_{\tmmathbf{k}} (a_{\tmmathbf{k}+}
  a_{-\tmmathbf{k}, -} + a_{\tmmathbf{k}+}^{\dagger} a_{-\tmmathbf{k},
  -}^{\dagger} + a_{\tmmathbf{k}-}^{\dagger} a_{\tmmathbf{k}+} +
  a_{\tmmathbf{k}+}^{\dagger} a_{\tmmathbf{k}-}) \right], 
\end{eqnarray}
where the Hamiltonians $H_{\alpha}, \alpha = \pm$ on different contours are
defined by
\begin{equation}
  H_{\alpha} = (\varepsilon_{\tmmathbf{k}} + U_R n - i \alpha \gamma n)
  a^{\dagger}_{\tmmathbf{k} \alpha} a_{\tmmathbf{k} \alpha} + \frac{(U_R - i
  \alpha \gamma) n}{2} \sum_{\tmmathbf{k}} (a_{\tmmathbf{k} \alpha}^{\dagger}
  a_{-\tmmathbf{k}, \alpha}^{\dagger} + a_{\tmmathbf{k} \alpha}
  a_{-\tmmathbf{k}, \alpha}) .
\end{equation}
Also by considering the spinor form and applying the Fourier transformation,
we obtain the action in momentum and energy space as
\begin{equation}
  S = \int \frac{d \omega}{2 \pi} \int \frac{d^3 k}{(2 \pi)^3} \Psi^{\dagger}
  (k, \omega) G^{- 1} (k, \omega) \Psi (k, \omega),
\end{equation}
where
\begin{equation}
  G^{- 1} (k, \omega) = - \left(\begin{array}{cccc}
    \frac{\varepsilon_{\tmmathbf{k}} + U n - \omega}{2} & \frac{U n}{2} &
    \frac{i \gamma n}{2} & \frac{i \gamma n}{2}\\
    \frac{U n}{2} & \frac{\varepsilon_{\tmmathbf{k}} + U n + \omega}{2} &
    \frac{i \gamma n}{2} & \frac{i \gamma n}{2}\\
    \frac{i \gamma n}{2} & \frac{i \gamma n}{2} & \frac{\omega -
    \varepsilon_{\tmmathbf{k}} - U^{\ast} n}{2} & - \frac{U^{\ast} n}{2}\\
    \frac{i \gamma n}{2} & \frac{i \gamma n}{2} & - \frac{U^{\ast} n}{2} & -
    \frac{\varepsilon_{\tmmathbf{k}} + U^{\ast} n + \omega}{2}
  \end{array}\right)
\end{equation}
We firstly consider the poles of the Green's function. The condition is given
by
\begin{equation}
  M (\omega) = \omega^4 - 2 \varepsilon_{\tmmathbf{k}}
  (\varepsilon_{\tmmathbf{k}} + 2 U_R n) \omega^2 +
  \varepsilon_{\tmmathbf{k}}^4 + 4 \varepsilon_{\tmmathbf{k}}^3 U_R n + 4
  \varepsilon_{\tmmathbf{k}}^2 (U_R n)^2 = 0
\end{equation}
The solution is given by $\omega = \pm \sqrt{\varepsilon_{\tmmathbf{k}}
(\varepsilon_{\tmmathbf{k}} + 2 U_R n)}$, which conincides with the excitation
spectrum of interacting bosons without dissipation. Hence, we can see no
matter how large the parameter $\gamma$ is, it will never influence the
excitation.

We further consider the superfluid density in the system. By applying the
Wick rotation, we obtain the action with imaginary energy as
\begin{equation}
  Z = e^{- S}, S = \frac{1}{2} \int \frac{d \omega}{2 \pi} \int \frac{d^3
  k}{(2 \pi)^3} \Psi^{\dagger} (k, i \omega) G^{- 1} (k, i \omega) \Psi (k, i
  \omega),
\end{equation}
where $G$ is given by
\begin{equation}
  G (k, \omega) = \left(\begin{array}{cccc}
    \varepsilon_{\tmmathbf{k}} + U n - i \omega & U n & i \gamma n & i \gamma
    n\\
    U n & \varepsilon_{\tmmathbf{k}} + U n + i \omega & i \gamma n & i \gamma
    n\\
    i \gamma n & i \gamma n & \frac{i \omega - \varepsilon_{\tmmathbf{k}} -
    U^{\ast} n}{2} & - \frac{U^{\ast} n}{2}\\
    i \gamma n & i \gamma n & - \frac{U^{\ast} n}{2} & -
    \frac{\varepsilon_{\tmmathbf{k}} + U^{\ast} n + i \omega}{2}
  \end{array}\right)^{- 1} .
\end{equation}
Still we consider the phase stiffness $Q_{a b}$ in (\ref{Qab}).
\begin{eqnarray}
  Q_{a b} & = & - \frac{m^2}{2 \beta V} \sum_k \nabla_a \varepsilon_k \nabla_b
  \varepsilon_k \left\{ \text{Tr} [(\sigma_z \otimes \sigma_z) G (\sigma_z
  \otimes \sigma_z) G] - \text{Tr} [(\sigma_z \otimes \sigma_0) G (\sigma_z
  \otimes \sigma_0) G] \right\} \nonumber\\
  & = & - \frac{1}{6} \delta_{a b} \int \frac{d \omega}{2 \pi} \frac{d^3
  k}{(2 \pi)^3} k^2 \left\{ \text{Tr} [(\sigma_z \otimes \sigma_z) G (\sigma_z
  \otimes \sigma_z) G] - \text{Tr} [(\sigma_z \otimes \sigma_0) G (\sigma_z
  \otimes \sigma_0) G] \right\} . 
\end{eqnarray}
Following the same calculation skills, we obtain the trace part as
\begin{eqnarray}
  \tmop{Tr} [\cdots] & = & \frac{1}{M (\omega)} (- 8
  (\varepsilon_{\tmmathbf{k}}^2 \gamma n + \omega^2 \gamma n)^2 - 4 (i
  \varepsilon_{\tmmathbf{k}}^2 \gamma n - \varepsilon_{\tmmathbf{k}}^2 U_R n -
  2 \varepsilon_{\tmmathbf{k}} U_R n + i \gamma n \omega^2 - U_R n \omega^2)^2
  \nobracket \nonumber\\
  &  & - 4 (i \varepsilon_{\tmmathbf{k}}^2 \gamma n -
  \varepsilon_{\tmmathbf{k}}^2 U_R n - 2 \varepsilon_{\tmmathbf{k}} U_R n + i
  \gamma n \omega^2 - U_R n \omega^2)^2 \nobracket) \nonumber\\
  &  & = - \frac{8 U_R^2 n^2}{(\varepsilon_{\tmmathbf{k}}^2 + 2
  \varepsilon_{\tmmathbf{k}} U_R n + \omega^2)^2} . 
\end{eqnarray}
Hence, we simplify it as
\begin{equation}
  Q_{a b} = \frac{1}{3} \delta_{a b} \int \frac{d \omega}{2 \pi} \frac{d^3
  k}{(2 \pi)^3} k^2 \frac{4 U_R^2 n^2}{(\varepsilon_{\tmmathbf{k}}^2 + 2
  \varepsilon_{\tmmathbf{k}} U_R n + \omega^2)^2},
\end{equation}
which is exactly the same with (\ref{Qzero}). We can see the dephasing
operator contributes to the superfluidity by decreasing the number of bosons
with zero momentum. The final superfluid density is propotional to $n_0^{3 /
2}$. Hence we can see dephasing operator can only destroy the superfluidity.
This is because only the two-body loss can solely induce the effective
interaction which leads to the emergence of superfluidity. If there is at most
one boson on each site for a state, the two-body loss will vanish if we act
the Lindbladian operator on the state. However, the dephasing operator still
remains finite after acting the operator on the state with at most one
particle on each site. The dephasing operator cannot induce an effective
interaction among bosons.

\


\[ \  \]


\end{document}
